%File: ~/OOP/renderer/ColorMap.tex
%What: "@(#) ColorMap.tex, revA"

THIS INTERFACE IS GONNA CHANGE. 1) MOVABLE\_OBJECT SO CAN RENDER IN
PARALLEL, 2) SET\_MINMAX() and 3) ONLY ONE CALL TO GET RGB VALUES, i.e.
getRGB(value, \&r, \&g, \&b) \\

\noindent {\bf Files}   \\
\indent \#include $<\tilde{ }$/earthquake/ColorMap.h$>$  \\

\noindent {\bf Class Declaration}  \\
\indent class ColorMap   \\

\noindent {\bf Class Hierarchy} \\
\indent {\bf ColorMap} \\

\noindent {\bf Description} \\ 
\indent 
The ColorMap is an abstract class, it defines the interface all
concrete subclasses must provide. A ColorMap object is used to
determine the mapping between scalar quantities to be displayed in an
image and the rgb values that are displayed. \\

\noindent {\bf Class Interface} \\
\indent // Constructor \\ 
\indent {\em ColorMap();}\\ \\
\indent // Destructor \\ 
\indent {\em virtual $\tilde{ }$ColorMap();}\\  \\
\indent // Public Methods\\ 
\indent {\em virtual float getRed(float value) =0;}\\
\indent {\em virtual float getGreen(float value) =0;}\\
\indent {\em virtual float getBlue(float value) =0;}\\

\noindent {\bf Constructor} \\ 
\indent {\em ColorMap();}\\ 
Does nothing. \\

\noindent {\bf Destructor} \\
\indent {\em virtual $\tilde{ }$ColorMap();}\\  
Does Noting. \\

\noindent {\bf Public Methods} \\
\indent {\em virtual float getRed(float value) =0;}\\
To return the red intensity of the rgb triple for the scalar quantity
{\em value}. \\

\indent {\em virtual float getGreen(float value) =0;}\\
To return the green intensity of the rgb triple for the scalar quantity
{\em value}. \\

\indent {\em virtual float getBlue(float value) =0;}\\
To return the blue intensity of the rgb triple for the scalar quantity
{\em value}. \\