%File: ~/OOP/system_of_eqn/linearSOE/sparseGEN/SparseGenColLinSolver.tex
%What: "@(#) SparseGenColLinSolver.tex, revA"

\noindent {\bf Files}   \\
\indent \#include $<\tilde{ }$/system\_of\_eqn/linearSOE/SparseGen/SparseGenColLinSolver.h$>$  \\

\noindent {\bf Class Declaration}  \\
\indent class SparseGenColLinSolver: public LinearSOESolver  \\

\noindent {\bf Class Hierarchy} \\
\indent MovableObject \\
\indent\indent  Solver \\
\indent\indent\indent LinearSOESolver \\
\indent\indent\indent\indent {\bf SparseGenColLinSolver} \\

\noindent {\bf Description}  \\
\indent SparseGenColLinSolver is an abstract class.  The SparseGenColLinSolver
class provides access for each subclass to the SparseGenColLinSOE object
through the pointer {\em theSOE}, which is a protected pointer. \\

\noindent {\bf Interface}  \\
\indent\indent // Constructor \\
\indent\indent {\em SparseGenColLinSolver(int classTag);}  \\ \\
\indent\indent // Destructor \\
\indent\indent {\em virtual~ $\tilde{}$SparseGenColLinSolver();}\\  \\
\indent\indent // Public Methods \\
\indent\indent {\em virtual int setLinearSOE(SparseGenColLinSOE \&theSOE);} \\

\noindent {\bf Constructor}  \\
\indent {\em SparseGenColLinSolver(int classTag);}  \\
The integer {\em classTag} is passed to the LinearSOESolver classes
constructor. \\ 

\noindent {\bf Destructor} \\
\indent {\em virtual~ $\tilde{}$SparseGenColLinSolver();}\\ 
Does nothing, provided so the subclasses destructor will be called. \\

\noindent {\bf Public Methods }  \\
\indent {\em virtual int setLinearSOE(SparseGenColLinSOE \&theSOE);} \\
Sets the link to the SparseGenColLinSOE object {\em theSOE}. This is the
object on which the solver will perform the numerical computations. \\




