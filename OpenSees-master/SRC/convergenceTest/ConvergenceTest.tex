%File: ~/OOP/convergenceTest/ConvergenceTest.tex
%What: "@(#) ConvergenceTest.tex, revA"

\noindent {\bf Files}   \\
\indent \#include $<\tilde{}$/convergenceTest/ConvergenceTest.h$>$  \\

\noindent {\bf Class Declaration}  \\
\indent class ConvergenceTest: public MovableObject  \\

\noindent {\bf Class Hierarchy} \\
\indent MovableObject \\
\indent\indent {\bf ConvergenceTest} \\

\noindent {\bf Description}  \\
\indent A ConvergenceTest object is an object which can be used in an
algorithmic class to test if convergence has been achieved for an 
iteration. The ConvergenceTest class is an abstract class, defining
the interface that all subclasses must provide. \\

\noindent {\bf Class Interface} \\
\indent\indent {// Constructors}  \\ 
\indent\indent {\em ConvergenceTest(int classTag);}  \\ \\
\indent\indent {// Destructor}  \\ 
\indent\indent {\em virtual $\tilde{}$ConvergenceTest();} \\ \\
\indent\indent {// Public Methods}  \\ 
\indent\indent {\em virtual int setEquiSolnAlgo(EquiSolnAlgo \&theAlgo) =0;} \\
\indent\indent {\em virtual int start(void) =0;} \\
\indent\indent {\em virtual int test(void) =0;} \\

\noindent {\bf Constructors}  \\
\indent {\em ConvergenceTest();}  \\
The integer {\em classTag} is passed to the MovableObject constructor. \\

\noindent {\bf Destructor} \\
\indent {\em virtual $\tilde{}$ConvergenceTest();} \\ 
Does nothing. \\

\noindent {\bf Public Methods }  \\
\indent {\em virtual int setEquiSolnAlgo(EquiSolnAlgo \&theAlgo) =0;} \\
To set the corresponding EquiSolnAlgo class. \\

\indent {\em virtual int test(void) =0;} \\
To return a positive number if the convergence criteria defined for the
object has been satisfied, the positibe number equal to the number of times 
since {\em start} that {\em test()} has been invoked. Otherwise a negative number 
is to be returned. A {\em -2} 
is returned if the test fails to meet the criteria and no more tests are to be 
performed due to limits set, i.e. the maximum number of iterations, otherwise a
{\em -1} is to be returned.  \\

\indent {\em virtual int start(void) =0;} \\
This is invoked at the start of each iteration. To return {\em 0} if successful, i.e 
that testing can proceed, a negative number if not. \\