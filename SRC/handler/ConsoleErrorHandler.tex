%File: ~/handler/ConsoleErrorHandler.tex
%What: "@(#) ConsoleErrorHandler.tex, revA"

\noindent {\bf Files}   \\
\indent \#include $<\tilde{ }$/handler/ConsoleErrorHandler.h$>$  \\

\noindent {\bf Class Declaration}  \\
\indent class ConsoleErrorHandler \\

\noindent {\bf Class Hierarchy} \\
\indent ErrorHandler \\
\indent\indent {\bf ConsoleErrorHandler} \\

\noindent {\bf Description}  \\
\indent The ConsoleErrorHandler class is a concrete subclass of error
handler which sends the error messages to the opserr stream. \\

\noindent {\bf Class Interface} \\
\indent // Constructor \\
\indent {\em ConsoleErrorHandler();}\\ \\
\indent // Destructor \\
\indent {\em $\tilde{ }$ConsoleErrorHandler();}\\ \\
\indent // Public Methods  \\
\indent {\em void warning(const char *msg, ...);}\\
\indent {\em void fatal(const char *msg, ...);}\\ 

\noindent {\bf Constructor}  \\
\indent {\em ConsoleErrorHandler();}  \\ 
Does nothing.\\

\noindent {\bf Destructor} \\
\indent {\em ~ $\tilde{}$ConsoleErrorHandler();}\\ 
Does nothing. \\

\noindent {\bf Public Methods }  \\
\indent {\em void warning(const char *msg, ...) =0;}\\
Creates a va\_list using {\em va\_start()} on the ellipses arguments
and invokes the {\em outputMessage(opserr, msg, va\_list)} routine in
the parent class. It then invokes  {\em va\_end()} on this va\_list
and returns.\\ 

\indent {\em void fatal(const char *msg, ...) =0;}\\
Creates a va\_list using {\em va\_start()} on the ellipses arguments
and invokes the {\em outputMessage(opserr, msg, va\_list)} routine in
the parent class. It then invokes {\em va\_end()} on this va\_list,
and finally terminates the program with an {\em exit()}. 
