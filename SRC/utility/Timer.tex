%File: ~/OOP/timer/Timer.tex
%What: "@(#) Timer.tex, revA"

\noindent {\bf Files}   \\
\indent \#include $<\tilde{}$/Timer/Timer.h$>$  \\

\noindent {\bf Class Declaration}  \\
\indent class Timer: public MovableObject  \\

\noindent {\bf Class Hierarchy} \\
\indent MovableObject \\
\indent\indent {\bf Timer} \\

\noindent {\bf Description}  \\
\indent A Timer object is an object which can be used to measure
system resources, i.e. cpu time and memory usage. Currently for Unix
systems only. COMPILE FLAG NEEDED.\\

\noindent {\bf Class Interface} \\
\indent {// Constructors}  \\ 
\indent {\em Timer(int classTag);}  \\ \\
\indent {// Destructor}  \\ 
\indent {\em virtual $\tilde{}$Timer();} \\ \\
\indent {// Public Methods}  \\ 
\indent {\em  void start(void);}\\
\indent {\em  void pause(void);}\\
\indent {\em  double getReal(void) const;}\\
\indent {\em  double getCPU(void) const;}\\
\indent {\em  int getNumPageFaults(void) const;}\\
\indent {\em  virtual void Print(ostream \&s) const; }\\
\indent {\em  friend ostream \&operator$<<$(ostream \&s, const Timer \&E); }\\

\noindent {\bf Constructors}  \\
\indent {\em Timer();}  \\
Does nothing. \\

\noindent {\bf Destructor} \\
\indent {\em virtual $\tilde{}$Timer();} \\ 
Does nothing. \\

\noindent {\bf Public Methods }  \\
\indent {\em  void start(void);}\\
Sets the accounting variables to mark the start of accounting period
using the unix functions {\em times()} and {\em getrusage}.\\

\indent {\em  void pause(void);}\\
Sets the accounting variables to mark the end of accounting period
using the unix functions {\em times()} and {\em getrusage}.\\


\indent {\em  double getReal(void) const;}\
Uses the difference between the starting and ending accounting
variables to determine the elapsed real time between the last calls to
{\em start()} and {\em pause()}. Returns this value in units of seconds.\\

\indent {\em  double getCPU(void) const;}\\
Uses the difference between the starting and ending accounting
variables to determine the CPU time allocated the process between the
last calls to {\em start()} and {\em pause()}. Returns this value in
units of seconds.\\

\indent {\em  int getNumPageFaults(void) const;}\\
Uses the difference between the starting and ending accounting
variables to determine the number of page faults that required reading
of pages from disk between the last calls to {\em start()} and {\em
pause()}. Returns this value. \\


\indent {\em virtual void Print(ostream \&s) const; }\\
Uses the difference between the starting and ending accounting
variables to determine the real time, CPU time, operating system time
allocate the process, total number of page faults, number of page
faults that required reading of pages from memory, and number of page
faults that required no reading from disk between the last calls to
{\em start()} and {\em pause()}. Send these values to {\em s}. \\

\indent {\em friend ostream \&operator$<<$(ostream \&s, const Timer \&E); }\\
Invokes {\em Print(s)} on the Timer object {\em E}. 

