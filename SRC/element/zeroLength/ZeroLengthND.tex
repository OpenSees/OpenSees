%File: ~/OOP/element/ZeroLength/ZeroLengthND.tex
%What: "@(#) ZeroLengthND.tex, revA"

\noindent {\bf Files}   \\
\indent \#include $<\tilde{ }$/element/zeroLength/ZeroLengthND.h$>$  \\

\noindent {\bf Class Declaration}  \\
\indent class ZeroLengthND : public Element \\

\noindent {\bf Class Hierarchy} \\
\indent TaggedObject \\
\indent MovableObject \\
\indent\indent DomainComponent \\
\indent\indent\indent Element \\
\indent\indent\indent\indent {\bf ZeroLengthND} \\

\noindent {\bf Description}  \\
\indent 
The ZeroLengthND class represents an element defined by two nodes at the same geometric
location, hence it has zero length.
The nodes are connected by an NDMaterial object of order 2 or 3, which
represents the force-deformation relationship for the element. If the order
is 2, a UniaxialMaterial object may be used to define the third mode of
force-deformation response.

ZeroLengthND elements are constructed with a {\em tag} in a domain of {\em dimension} 2 or 3,
connected by nodes {\em Nd1} and {\em Nd2}. 
The vector {\em x} defines the local x-axis for the element and the vector {\em yprime}
lies in the local x-y plane for the element.  The local z-axis is the cross product between 
{\em x} and {\em yprime}, and the local y-axis is the cross product between the local z-axis
and {\em x}.
\\

\noindent {\bf Class Interface} \\
\indent // Constructors \\
\indent {\em    ZeroLengthND(int tag, 			      
	       int dimension,
	       int Nd1, int Nd2, 
	       const Vector\& x,
	       const Vector\& yprime,
	       NDMaterial\& theNDMaterial);} \\
\indent {\em    ZeroLengthND(int tag, 			      
	       int dimension,
	       int Nd1, int Nd2, 
	       const Vector\& x,
	       const Vector\& yprime,
	       NDMaterial\& theNDMaterial,
		UniaxialMaterial\& the1DMaterial);} \\
\indent {\em    ZeroLengthND();} \\ \\
\indent // Destructor \\
\indent {\em    ~ZeroLengthND();} \\ \\
\indent    // public methods to obtain inforrmation about dof \& connectivity \\
\indent {\em    int getNumExternalNodes(void) const;} \\
\indent {\em    const ID \&getExternalNodes(void);} \\
\indent {\em    int getNumDOF(void);} \\	
\indent {\em    void setDomain(Domain *theDomain);} \\

\indent    // public methods to set the state of the element    \\
\indent {\em    int commitState(void);} \\
\indent {\em    int revertToLastCommit(void);} \\        
\indent {\em    int revertToStart(void);} \\        

\indent    // public methods to obtain stiffness, mass, damping and residual information    \\
\indent {\em    const Matrix \&getTangentStiff(void);} \\
\indent {\em    const Matrix \&getSecantStiff(void);} \\    
\indent {\em    const Matrix \&getDamp(void);} \\    
\indent {\em    const Matrix \&getMass(void);} \\    

\indent {\em    void zeroLoad(void);} \\	
\indent {\em    int addLoad(const Vector \&addP);} \\
\indent {\em    int addInertiaLoadToUnbalance(const Vector \&accel);} \\    
\indent {\em    const Vector \&getResistingForce(void);} \\
\indent {\em    const Vector \&getResistingForceIncInertia(void);} \\            

\indent    // public methods for element output \\
\indent {\em    int sendSelf(int commitTag, Channel \&theChannel);} \\
\indent {\em    int recvSelf(int commitTag, Channel \&theChannel, FEM\_ObjectBroker \&theBroker);} \\
\indent {\em    int displaySelf(Renderer \&theViewer, int displayMode, float fact);} \\    
\indent {\em    void Print(ostream \&s, int flag =0);} \\    

\indent {\em    int setResponse(char **argv, int argc, Information \&eleInformation);} \\
\indent {\em    int getResponse(int responseID, Information \&eleInformation);} \\
    


\noindent {\bf Constructors}  \\
\indent {\em    ZeroLengthND(int tag, 			      
	       int dimension,
	       int Nd1, int Nd2, 
	       const Vector\& x,
	       const Vector\& yprime,
	       NDMaterial\& theNDMaterial);} \\
Construct a ZeroLengthND element with {\em tag} .
The force-deformation relationship for the element is obtained by invoking
{\em getCopy()} on the {\bf NDMaterial} pointer {\em theNDMaterial}.
The NDMaterial object may be of order 2 or 3.
The material model acts in the local space defined by the {\em x} and
{\em yprime} vectors. If the material model is of order 2, it acts in the
element local x-y plane. \\

\indent {\em    ZeroLengthND(int tag, 			      
	       int dimension,
	       int Nd1, int Nd2, 
	       const Vector\& x,
	       const Vector\& yprime,
	       NDMaterial\& theNDMaterial,
		UniaxialMaterial\& the1DMaterial);} \\
Construct a ZeroLengthND element with {\em tag} .
The force-deformation relationship for the element is obtained by invoking
{\em getCopy()} on the {\bf NDMaterial} pointer {\em theNDMaterial}.
The NDMaterial object must be of order 2. The third mode of force-deformation
response is obtained by invoking {\em getCopy()} on the {\bf UniaxialMaterial}
pointer {\em the1DMaterial}.
The material models act in the local space defined by the {\em x} and
{\em yprime} vectors. The NDMaterial acts in the element x-y plane, while
the UniaxialMaterial acts orthogonal to the x-y plane. \\

\indent {\em    ZeroLengthND();} \\ 
This is the constructor invoked by an {\bf FEM\_ObjectBroker} object. It
constructs an empty ZeroLengthND element with two nodes.
The {\em recvSelf()} method is
invoked on the object for it to set the internal data. 
\\

\noindent {\bf Destructor} \\
\indent {\em    ~ZeroLengthND();} \\ 
Element destructor deletes memory for storing the material model pointer(s). 
\\

\noindent {\bf Public Methods }  \\
\indent {\em    int getNumExternalNodes(void) const;} \\
Returns 2.
\\

\indent {\em    const ID \&getExternalNodes(void);} \\
Return {\bf ID} of size $2$ with the node tags defining the element.
\\

\indent {\em    int getNumDOF(void);} \\	
Return the number of DOF for the element, which depends on the dimension of the problem
and the number of DOF associated with each node.
\\

\indent {\em    void setDomain(Domain *theDomain);} \\
Initialize element and define data structures.  Sets up the element
transformation matrix, $A$, which defines the kinematic relationship between
nodal displacements and material deformations.
\\

\indent {\em    int commitState(void);} \\
Commit state of element by commiting state of the material(s).
Return 0 if successful, !0 otherwise.
\\

\indent {\em    int revertToLastCommit(void);} \\        
Revert state of element to last commit by reverting to last committed state of the material(s).
Return 0 if successful, !0 otherwise.
\\

\indent {\em    int revertToStart(void);} \\        
Revert state of element to initial sate by reverting to initial state of the material(s).
Return 0 if successful, !0 otherwise.
\\

\indent {\em    const Matrix \&getTangentStiff(void);} \\
Return tangent stiffness matrix for element.  The basic element stiffness,
$k_b$, is a block diagonal matrix containing the NDMaterial tangent and
UniaxialMaterial tangent (if present). The element tangent is computed from
the basic stiffness as $K_e = A^T k_b A$.  The material tangents are 
obtained by calling {\em getTangent()}.
\\

\indent {\em    const Matrix \&getSecantStiff(void);} \\    
Returns the tangent stiffness matrix for the element as the secant stiffness
is not defined for NDMaterial objects.
\\

\indent {\em    const Matrix \&getDamp(void);} \\    
Return a zero damping matrix.
\\

\indent {\em    const Matrix \&getMass(void);} \\    
Return a zero mass matrix.
\\

\indent {\em    void zeroLoad(void);} \\	
The element has no loads, so this operation has no effect.
\\

\indent {\em    int addLoad(const Vector \&addP);} \\
The element has no loads, so this operation has no effect and returns 0.
\\

\indent {\em    int addInertiaLoadToUnbalance(const Vector \&accel);} \\    
The element has no mass, so this operation has no effect and returns 0.
\\

\indent {\em    const Vector \&getResistingForce(void);} \\
Return resisting force vector for element.  The basic element forces, $q$,
is computed from the material stresses, $s$, a concatenation of the
NDMaterial stress vector and the UniaxialMaterial stress (if present). The
element resisting force is computed from the basic forces as $P_e = A^T q$. 
The material stresses are obtained by calling {\em getStress()}.
\\

\indent {\em    const Vector \&getResistingForceIncInertia(void);} \\            
Returns the result of {\em getResistingForce()} as there is no element mass.
\\

\indent {\em    int sendSelf(int commitTag, Channel \&theChannel);} \\
Send information about element and the material(s) over a channel.
\\

\indent {\em    int recvSelf(int commitTag, Channel \&theChannel, FEM\_ObjectBroker \&theBroker);} \\
Receive information about element and material(s) from a channel.
\\

\indent {\em    int displaySelf(Renderer \&theViewer, int displayMode, float fact);} \\    
Display element.
\\

\indent {\em    void Print(ostream \&s, int flag =0);} \\    
Prints the element node tags and material model(s) to the stream {em s}.
\\

\indent {\em    int setResponse(char **argv, int argc, Information \&eleInformation);} \\
Currently returns -1.
\\

\indent {\em    int getResponse(int responseID, Information \&eleInformation);} \\
Currently returns -1.
\\






