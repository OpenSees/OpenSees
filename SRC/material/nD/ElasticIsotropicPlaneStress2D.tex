%File: ~/OOP/material/nD/ElasticIsotropicPlaneStress2D.tex
%What: "@(#) ElasticIsotropicPlaneStress2D.tex, revA"

\noindent {\bf Files}   \\
\indent \#include $<\tilde{ }$/material/nD/ElasticIsotropicPlaneStress2D.h$>$  \\

\noindent {\bf Class Declaration}  \\
\indent class ElasticIsotropicPlaneStress2D : public ElasticIsotropicMaterial \\

\noindent {\bf Class Hierarchy} \\
\indent TaggedObject \\
\indent MovableObject \\
\indent\indent Material \\
\indent\indent\indent NDMaterial \\
\indent\indent\indent\indent ElasticIsotropicMaterial \\
\indent\indent\indent\indent\indent {\bf ElasticIsotropicPlaneStress2D} \\

\noindent {\bf Description}  \\
ElasticIsotropicPlaneStress2D provides the implementation of an
elastic isotropic material which exhibits plane stress behavior in two
dimensions. \\

\noindent {\bf Class Interface} \\
\indent // Constructor \\
\indent {\em ElasticIsotropicPlaneStress2D (int tag, double E,
double v);}  \\ \\
\indent // Destructor \\
\indent {\em $\tilde{ }$ElasticIsotropicPlaneStress2D ();}\\ \\
\indent // Public Methods \\
\indent {\em int setTrialStrain (const Vector \&strain); } \\
\indent {\em const Vector \&getStress (void); } \\
\indent {\em const Matrix \&getTangent (void); } \\
\indent {\em int commitState (void); } \\
\indent {\em int revertToLastCommit (void); } \\
\indent {\em int revertToStart (void); } \\
\indent {\em NDMaterial *getCopy (void); } \\

\noindent {\bf Constructor}  \\
\indent {\em ElasticIsotropicPlaneStress2D (int tag, double E, double v);}  \\
To construct an ElasticIsotropicPlaneStress2D whose unique integer tag
among NDMaterials in the domain is given by {\em tag}.  The material
model have Young's modulus {\em E} and Poisson's ratio {\em v}. \\

\noindent {\bf Destructor} \\
\indent {\em $\tilde{ }$ElasticIsotropicPlaneStress2D ();} \\
Does nothing. \\ 

\noindent {\bf Public Methods} \\
\indent {\em int setTrialStrain (const Vector \&strain); }  \\
Sets the value of the current trial strain vector, $\myepsilon$,
to be {\em strain}. Returns $0$. \\

\begin{displaymath}
\myepsilon := \left[
   \begin{array}{c}
       \epsilon_{xx} \\
       \epsilon_{yy}   \\
       2 \gamma_{xy}   
   \end{array} 
 \right]
\end{displaymath}

\indent {\em const Vector \&getStress (void); } \\
Returns the material stress vector, $\mysigma$, for the current
trial strain. \\

\begin{displaymath}
\mysigma := \left[
   \begin{array}{c}
       \sigma_{xx} \\
       \sigma_{yy}   \\
       \tau_{xy}   
   \end{array} 
 \right]
\end{displaymath}

\indent {\em const Matrix \&getTangent (void); } \\
Returns the material tangent stiffness matrix, $\D$. \\

\begin{displaymath}
\D := \frac{E}{1-\nu^2} \left[
   \begin{array}{ccc}
         1 & \nu &     0 \\
       \nu &   1 &     0 \\
         0 &   0 & 1-\nu
   \end{array} 
 \right]
\end{displaymath}

\indent {\em int commitState (void); } \\
Returns $0$. \\

\indent {\em int revertToLastCommit (void); } \\
Returns $0$. \\

\indent {\em int revertToStart (void); } \\
Returns $0$. \\

\indent {\em NDMaterial *getCopy (void); } \\
Returns a pointer to a new ElasticIsotropicPlaneStress2D, with the
same values for {\em tag}, {\em E}, and $\nu$. It is up to the
caller to ensure that the destructor is invoked. \\