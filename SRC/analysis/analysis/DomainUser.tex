%File: ~/OOP/analysis/analysis/DomainUser.tex
%What: "@(#) DomainUser.tex, revA"

\noindent {\bf DomainUser} \\

\noindent {\bf Files}   \\
\indent \#include $<$DomainUser.h$>$  \\

\noindent {\bf Class Decleration}  \\
\indent class DomainUser;  \\

\noindent {\bf Class Hierarchy} \\
\indent {\bf DomainUser} \\
\indent\indent Analysis \\
\indent\indent Design \\
\indent\indent DomainDisplay \\

\noindent {\bf Description} \\ 
\indent The DomainUser class is an abstract base class. Its purpose is
to define the interface common among all subclasses.  A DomainUser is
a user of the domain, example subtypes being Analysis and Design. The
class defines the pure virtual function {\em domainChange()}: it is
this method that is invoked by the domain on all domain users once the
domain has changed, i.e. the connectivity has changed. \\

\noindent {\bf Constructors} 
\\ \indent {\em DomainUser(theDomain \&theDomain);}\\ 
All DomainUser are associated with a single domain, this constructor
sets up the link between the DomainUser and the domain, setting its link
with theDomain. The constructor invokes {\em addDomainUser(*this)} on
the domain. \\  

\noindent {\bf Destructor} \\
\indent {\em virtual~ $\tilde{}$DomainUser();}\\ 
All DomainUser are associated with a single domain, the destructor
removes the link in the domain by invoking {\em removeDomainUser(*this)}
on the domain. \\

\noindent {\bf Pure Virtual Public Member Functions}\\
\indent {\em virtual void domainChange(void) = 0;} \\
Invoked by the associated domain to inform the domainUser that the
connectivity of the domain has changed. \\


\noindent {\bf Protected Member Functions}  \\
\indent {\em Domain \&getDomain(void) const;} \\
A const method which returns a pointer to the Domain object on which
the DomainUser performs its DomainUser.




