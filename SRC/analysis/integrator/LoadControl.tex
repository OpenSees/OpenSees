%File: ~/OOP/analysis/integrator/LoadControl.tex
%What: "@(#) LoadControl.tex, revA"

\noindent {\bf Files}   \\
\indent \#include $<\tilde{ }$/analysis/integrator/LoadControl.h$>$  \\

\noindent {\bf Class Declaration}  \\
\indent class LoadControl: public StaticIntegrator  \\

\noindent {\bf Class Hierarchy} \\
\indent MovableObject \\
\indent\indent Integrator \\
\indent\indent\indent IncrementalIntegrator \\
\indent\indent\indent\indent StaticIntegrator \\
\indent\indent\indent\indent\indent {\bf LoadControl} \\

\noindent {\bf Description} \\ 
\indent LoadControl is a subclass of StaticIntegrator, it is
used to when performing a static analysis on the FE\_Model using the
load control method. In the load control method, the following
constraint equation is added to equation~\ref{staticFormTaylor} of the
StaticIntegrator class: 

\[ 
\lambda_n^{(i)} - \lambda_{n-1} = \delta \lambda_n
\]

\noindent where $\delta \lambda_n$ depends on $\delta \lambda_{n-1}$,
the load increment at the previous time step, $J_{n-1}$,
the number of iterations required to achieve convergence in the
previous load step, and $Jd$, the desired number of iterations. $\delta
\lambda_n$ is bounded by $\delta \lambda_{min}$  and $\delta \lambda_{max}$. \\


\[ 
\delta \lambda_n = max \left( \delta \lambda_{min}, min \left(
\frac{Jd}{J_{n-1}} \delta \lambda_{n-1}, \delta \lambda_{max} \right) \right)
\]

Knowing $\lambda_n^{(i)}$ prior to each iteration, the $N+1$ unknowns
in equation~\ref{staticFormTaylor}, is reduced to $N$ unknowns and
results in the following equation:

\begin{equation} 
\R(\U_{n}) = \lambda_n^{(i)} \P 
 - \f_{R}\left(\U_{n}^{(i)} \right) - 
\K_n^{(i)} 
(\U_{n} - \U_{n}^{(i)})  
\label{staticFormLoadControl}
\end{equation} 

\noindent {\bf Class Interface} \\
\indent // Constructors \\
\indent {\em LoadControl(double $\delta \lambda_1$, int Jd, 
double $\delta \lambda_{min}$, double $\delta \lambda_{max}$);}\\ \\
\indent // Destructor \\
\indent {\em $\tilde{ }$LoadControl();}\\  \\
\indent // Public Methods \\
\indent {\em int newStep(void);} \\
\indent {\em int update(const Vector \&$\Delta U$);} \\
\indent {\em int setDeltaLambda(double $\delta
\lambda$;)} \\ \\
\indent // Public Methods for Output\\
\indent {\em int sendSelf(int commitTag, Channel \&theChannel);}\\ 
\indent {\em int recvSelf(int commitTag, Channel \&theChannel,
FEM\_ObjectBroker \&theBroker);}\\ 
\indent {\em int Print(OPS_Stream \&s, int flag = 0);}\\

\noindent {\bf Constructors} \\
\indent {\em LoadControl(double $\delta \lambda_1$, int Jd, 
double $\delta \lambda_{min}$, double $\delta \lambda_{max}$);}\\ \\
The integer INTEGRATOR\_TAGS\_LoadControl (defined in
$<$classTags.h$>$) is passed to the StaticIntegrator classes
constructor. $\delta \lambda_1$ is the load factor used in the first
step. The arguments $Jd$, $\delta \lambda_{min}$, and $\delta
\lambda_{max}$ are used in the determination of the increment in the
load factor at each step. \\



\noindent {\bf Destructor} \\
\indent {\em $\tilde{ }$LoadControl();}\\ 
Does nothing. \\

\noindent {\bf Public Methods}\\

{\em int newStep(void);} \\
The object obtains the current value of $\lambda$ from the AnalysisModel
object. It increments this by $\delta \lambda_n $.

\[ 
\delta \lambda_n = max \left( \delta \lambda_{min}, min \left(
\frac{Jd}{J_{n-1}} \delta \lambda_{n-1}, \delta \lambda_{max} \right) \right)
\]

\noindent It will then invoke
{\em applyLoadDomain(0.0, $\lambda$)} on the AnalysisModel
object. Returns $0$ if successful. A warning message is printed and a
$-1$ is returned if no AnalysisModel is associated with the object. \\

{\em int update(const Vector \&$\Delta U$);} \\
Invoked this causes the object to first increment the DOF\_Group
displacements with $\Delta U$, by invoking {\em incrDisp($\Delta U)$}
on the AnalysisModel, and then to update the domain, by invoking {\em
updateDomain()} on the AnalysisModel. Returns $0$ if successful, a
warning message and a $-1$ is returned if no AnalysisModel is
associated with the object. \\


\indent {\em int setDeltaLambda(double $\delta \lambda$;)} \\
Sets the value of the load increment in {\em newStep()} to be $\delta
\lambda$. Returns $0$.\\

{\em int sendSelf(int commitTag, Channel \&theChannel); } \\ 
Places in a vector if size 5 the value of $\delta \lambda_{n-1}$,
$Jd$, $J_{n-1}$, $\delta \lambda_{min}$ and $\delta \lambda_{max}$)
and then sends the Vector. Returns $0$ if successful, a warning
message is printed and a $-1$ is returned if {\em theChannel} fails to
send the Vector. \\ 

{\em int recvSelf(int commitTag, Channel \&theChannel, 
FEM\_ObjectBroker \&theBroker); } \\ 
Receives in a Vector of size 5 the data that was sent in {\em sendSelf()}.
Returns $0$ if successful, a warning message is printed, $\delta
\lambda$ is set to $0$, and a $-1$ is returned if {\em theChannel} 
fails to receive the Vector.\\

{\em int Print(OPS_Stream \&s, int flag = 0);}\\
The object sends to $s$ its type, the current value of $\lambda$, and
$\delta \lambda$. 