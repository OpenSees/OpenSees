% File: ~/OOP/analysis/fe_ele/penalty/PenaltySP_FE.tex 
%What: "@(#) PenaltySP_FE.tex, revA"

\noindent {\bf Files}   \\
\indent \#include $<\tilde{ }$/analysis/fe\_ele/penalty/PenaltySP\_FE.h$>$  \\

\noindent {\bf Class Declaration}  \\
\indent class PenaltySP\_FE: public FE\_Element ;  \\

\noindent {\bf Class Hierarchy}  \\
\indent FE\_Element \\
\indent\indent {\bf PenaltySP\_FE} \\ 

\noindent {\bf Description}  \\
\indent PenaltySP\_FE is a subclass of FE\_Element used to enforce a
single point constraint. It does this by adding $\alpha$ to the
tangent and $\alpha * (U\_s - U\_t)$ to the residual at the locations
corresponding to the constrained degree-of-freedom specified by the
SP\_Constraint, where $U_s$ is the specified value of the constraint
and $U_t$ the current trial displacement at the node corresponding to
the constraint.\\


\noindent {\bf Class Interface}  \\
\indent\indent // Constructor  \\
\indent\indent {\em PenaltySP\_FE(Domain \&theDomain, SP\_Constraint
\&theSP, double alpha = 1.0e8);} \\ \\
\indent\indent // Destructor  \\
\indent\indent {\em virtual~ $\tilde{}$PenaltySP\_FE();}  \\ \\
\indent\indent // Public Methods \\
\indent\indent {\em virtual void setID(void);} \\ 
\indent\indent {\em virtual const Matrix \&getTangent(Integrator
*theIntegrator);} \\  
\indent\indent {\em virtual const Vector \&getResidual(Integrator
*theIntegrator);} \\ 
\indent\indent {\em virtual const Vector \&getTangForce(const Vector
\&disp, double fact = 1.0);    }\\

\noindent {\bf Constructor}  \\
\indent {\em PenaltySP\_FE(Domain \&theDomain, SP\_Constraint \&theSP,
double alpha = 1.0e8);}\\
To construct a PenaltySP\_FE element to enforce the constraint
specified by the SP\_Constraint {\em theSP} using a value for
$\alpha$ of {\em alpha} (which, if none is specified, defaults to
$1.0e8$). The FE\_Element class constructor is called with 
the integers $1$ and $1$. A Matrix and a Vector object of order $1$
are created to return the tangent and residual contributions, with the
tangent entry being set at $\alpha$. A link to the Node in the  Domain
corresponding to the SP\_Constraint is also set. A warning message is
printed and program terminates if there is not enough memory or no
Node associated with the SP\_Constraint exists in the Domain.\\

\noindent {\bf Destructor}  \\
\indent {\em virtual~ $\tilde{}$PenaltySP\_FE();}  \\
Invokes the destructor on the Matrix and Vector objects created in the
constructor. \\

\noindent {\bf Public Methods}  \\
\indent {\em virtual void setID(void);} \\
Causes the PenaltySP\_FE to determine the mapping between it's equation
numbers and the degrees-of-freedom. From the Node object link, created
in the constructor, the DOF\_group corresponding to the Node
associated with the constraint is determined. From this {\em
DOF\_Group} object the mapping for the constrained degree of freedom
is determined and the ID in the base class is set. Returns $0$ if
successful. Prints a warning message and returns a negative number if
an error occurs: $-2$ if the
Node has no associated DOF\_Group, $-3$ if the constrained DOF
specified is invalid for this Node and $-4$ if the ID in the
DOF\_Group is too small for the Node. \\ 

\indent {\em virtual const Matrix \&getTangent(Integrator *theIntegrator);} \\
Returns the tangent Matrix created in the constructor. \\

\indent {\em virtual const Vector \&getResidual(Integrator *theIntegrator);} \\
Sets the FE\_Elements contribution to the residual to be
$\alpha * (U_s - U_t)$, where $U_s$ is the specified value of the
constraint and $U_t$ the current trial displacement at the node
corresponding to constrained degree-of-freedom. Prints a warning
message and sets this contribution to $0$ if the specified constrained
degree-of-freedom is invalid. Returns this residual Vector set.


{\em virtual const Vector \&getTangForce(const Vector \&disp, double
fact = 1.0);    }\\
Sets the FE\_Elements contribution to the residual to be
$\alpha * (U\_s - disp\_t)$, where $U\_s$ is the specified value of the
constraint and $disp\_t$ the value in {\em disp}
corresponding to constrained degree-of-freedom. Prints a warning
message and sets this contribution to $0$ if the mapping, determined in
{\em setID()}, for the the specified constrained degree-of-freedom lies 
outside {\em disp}. \\  










