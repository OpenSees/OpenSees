% File: ~/domain/node/DummyNode.tex 

\noindent {\bf DummyNode} \\

\noindent {\bf Files}   \\
\indent \#include $<$DummyNode.h$>$  \\

\noindent {\bf Class Decleration}  \\
\indent class DummyNode: public DomainComponent  \\



\noindent {\bf Class Hierarchy} \\
\indent DomainComponent \\
\indent\indent {\bf DummyNode} \\

\noindent {\bf Description}  \\
\indent DummyNodes are a type of node created and used by Subdomains for their
exterior nodes. They reference a real node and most methods invoked on
them are in turn invoked by the dummy node on the real node. The calls
asking the real node to change its current state are ignored. The
calls involving DOF\_Group are handled by the dummy node. \\


\noindent {\bf Constructors}  \\
\indent {\em DummyNode(Node &theRealNode);}  \\
To construct a dummy node which is associated with the node pointed to
by {\em theRealNode}. \\

\noindent {\bf Destructor}  \\
\indent {\em virtual~$\tilde{}$ DummyNode();} \\

%%%%%%%% Public Member Functions - In addition to DOMAIN
\noindent {\bf Public Member Functions }  \\
\indent {\em virtual void setDOF\_GroupPtr(DOF\_Group *theDOF\_Grp);} \\
Each DummyNode, when involved with a StaticCondensationAnalysis
analysis, will be associated with a DOF\_Group object. It is the
DOF\_Group that contains the ID of equation numbers. When invoked this
method sets that link for the DummyNode object. \\ 


{\em virtual DOF\_Group *getDOF\_GroupPtr(void);} \\
Method which returns a pointer to the DOF\_Group object that was set
using {\em setDOF\_GroupPtr}. If no pointer has been set a $0$ is
returned. \\


{\em virtual int getNumberDOF(void) const;}\\
Returns the result of invoking {\em getNumberDOF()} on its associated
Node object. \\ 

{\em virtual Vector \&getMass(void) const;}\\
Returns the result of invoking {\em getMass()} on its associated
Node object. \\ 


{\em virtual void setMass(Vector \&mass);} \\
Invokes {\em getNumberDOF()} on its associated
Node object. \\ 


{\em virtual const Vector \&getCrds(void) const;}  \\
Returns the result of invoking {\em getCrds()} on its associated
Node object. \\ 

{\em  virtual const Vector \&getDisp(void) const;}  \\
Returns the result of invoking {\em getDisp()} on its associated
Node object. \\ 


{\em  virtual const Vector \&getVel(void) const;}  \\
Returns the result of invoking {\em getVel()} on its associated
Node object. \\ 


{\em  virtual const Vector \&getAccel(void) const;}  \\
Returns the result of invoking {\em getAccel()} on its associated
Node object. \\ 


{\em  virtual const Vector \&getTrialDisp(void) const;}  \\
Returns the result of invoking {\em getTrialDisp()} on its associated
Node object. \\ 


{\em  virtual const Vector \&getTrialVel(void) const;}  \\
Returns the result of invoking {\em getTrialVel()} on its associated
Node object. \\ 


{\em  virtual const Vector \&getTrialAccel(void) const;}  \\
Returns the result of invoking {\em getTrialAccel()} on its associated
Node object. \\ 


{\em void addUnbalancedLoad(const Vector \&additionalLoad);} \\
Returns the result of invoking {\em addUnbalancedLoad()} on its associated
Node object. \\ 

{\em virtual const Vector \&getUnbalancedLoad(void) const;} \\
Returns the result of invoking {\em getUnbalancedLoad()} on its associated
Node object. \\ 



The following commands do nothing, they just return. \\
\indent{\em  void setTrialDisp(const Vector \&newTrialDisp);} \\
\indent{\em  void setTrialVel(const Vector \&newTrialVel);} \\
\indent{\em  void setTrialAccel(const Vector \&newTrialAccel);} \\
\indent{\em  void setIncrTrialDisp(const Vector \&trialIncrDisp);} \\
\indent{\em  void setIncrTrialVel(const Vector \&trialIncrVel);} \\
\indent{\em  void setIncrTrialAccel(const Vector \&trialIncrAccel);} \\
\indent{\em void zeroUnbalancedLoad(void);} \\
\indent{\em void commitState(void);} \\
