\documentstyle[epsf,fullpage]{article}
%\setlength{\textheight}{9true in}
%\setlength{\textwidth}{6.25true in}
%\setlength{\topmargin}{-.43true in}
%\setlength{\headheight}{.17true in}
%\setlength{\headsep}{.2true in}
%\setlength{\footheight}{.17true in}
%\setlength{\footskip}{.42true in}
%\setlength{\parskip}{0.0true in}
%\setlength{\topskip}{10pt}
%\setlength{\oddsidemargin}{0.25true in}
%\if@twoside\setlength{\evensidemargin}{0true in}
%        \else\setlength{\evensidemargin}{0.25true in}\fi

\begin{document}


\def\A{\mathop{\mbox{\huge \rm A}}\limits}
\def\I{\mathop{\cal {\bf I}}}
\def\U{{\bf U}}
\def\Ud{  \dot{\bf U}}
\def\Udd{  \ddot{\bf U}}
\def\F{{\bf F}}
\def\R{{\bf R}}
\def\K{{\bf K}}
\def\P{{\bf P}}
\def\B{{\bf B}}
\def\M{{\bf M}}
\def\f{{\bf F}}
\def\C{{\bf C}}
\def\D{{\bf D}}
\def\T{{\bf T}}
\def\X{{\bf X}}
\def\Q{{\bf Q}}
\def\V{{\bf V}}
\def\mylambda{{\bf \lambda}}
\def\myLambda{{\bf \Lambda}}
\def\myalpha{{\bf \alpha}}


\def\myPhi{{\bf \Phi}}
\def\zero{{\bf 0}}

\begin{titlepage}
{.}\vspace{2.0in}
\begin{center}

{\Huge G 3}

\vspace{0.5in}
{\bf {\Large C l a s s \hspace{0.1in} I n t e r f a c e  \hspace{0.1in} S p
e c i f i c a t i o n} }

\vspace{1.0in}
{\bf Version 0.1 - Preliminary Draft} 

\vspace{1.0in}
{\bf  December 20, 1999} 

\vspace{1.0in}
{\bf  Frank McKenna and Gregory L. Fenves} 

{\bf  PEER, University of California at Berkeley}

\end{center}
\end{titlepage}

\noindent {\bf \Large {\bf Introduction}} \\
This document outlines the class interfaces currently provided by G3.
The main categories of classes are:
\begin{enumerate} 

\item Matrix Classes: These include the classes Matrix,
Vector and ID (integer array). These classes are used in the framework
for passing information between objects in a safe manner, and for
small scale numerical calculations in element formulation.

\item Domain Classes: These classes describe the finite element model
and store the results of an analysis on the model. The classes
include Domain, Element, Node, Load, SP\_Constraint, MP\_Constraint,
and their subclasses.   

\item Analysis Classes: These classes perform the analysis of the
finite element model. The classes include the Analysis,
ConstraintHandler, DOF\_Numberer, SolutionAlgorithm, Integrator,
FE\_Element, DOF\_Group and AnalysisModel classes, and their
subclasses.  

\item Modeling Classes: These include the abstract class
ModelBuilder, and concrete subclasses of this class. An analyst will
interact with a ModelBuilder object, to create the Element, Node, Load
and Constraint objects that define the model. 

\item Numerical Classes: These include the abstract SystemOfEquation
and Solver classes, and subclasses of these classes. These classes are
provided for the solving of large scale systems of linear and eigenvalue
equations. 

\item{Data Storage} These are classes used to store data. There are
two abstract classes TaggedObjectStorage and FE\_Datastore. Objects of
type TaggedObjectStorage are used as containers to store and provide
access to the TaggedObjects in memory during program
execution. FE\_Datastore objects are used to store/retrieve
information from databases, containers which can permanently hold
program data. 

\item {Visualization Classes} These are classes used to generate images
of the model for the analyst. These classes include Renderer,
ColorMap, and their subclasses.

\item{Graph Classes} These are classes used to provide information
about nodal and elemental connectivity and sparsity of systems of
equations. The classes include Graph, Vertex, GraphNumberer, 
GraphPartitioner, and their subclasses. There is no Edge class provided
at present. In current design each Vertex stores in an ID the tag of
all it's adjacent Vertices. For graph numbering and partitioning this
has proved sufficient.  

\item{Parallel Classes}
To facilitate the development of parallel object-oriented finite
element programs, classes are provided for parallel programming. The
classes in the framework support the aggregate programming model. The
classes include Actor, Shadow, Message, MachineBroker,
FEM\_ObjectBroker, Channel, and their subclasses.

\end{enumerate}

As the design is very modular and most of the classes that are
provided can be subclassed, this allows for the contribution from many
researchers in different areas. The design will allow for
contributions in the fields of: \begin{itemize}
\item Element and material types.
\item Solution algorithms, integration procedures and constraint
handling techniques.
\item Model generation.
\item Numerical analysis for solution of linear and eigenvalue
problems. 
\item Graph theory for numbering and partitioning graphs.
\item Data structures for container classes and database.
\item Graphics.
\item Message passing systems and load balancing in parallel environments.
\end{itemize}

\pagebreak
\tableofcontents

\pagebreak
\section{Matrix Classes}
Numerical classes are used to pass numerical information between
objects and to handle the numerical operations in the solution
procedure. The classes provided include Matrix, Vector, and ID.
The abstractions provided by the Matrix and Vector classes are
obvious. The ID class provides the abstraction of an integer array. 


\pagebreak
\subsection{Matrix}

\pagebreak
\subsection{Vector}
%File: ~/OOP/matrix/Vector.tex
%What: "@(#) Vector.tex, revA"

\noindent {\bf Files}   \\
\indent \#include $<\tilde{}$/matrix/Vector.h$>$  \\

\noindent {\bf Class Declaration}  \\
\indent class Vector:  \\

\noindent {\bf Class Hierarchy} \\
\indent {\bf Vector} \\

\noindent {\bf Description}  \\
\indent The Vector class provides the vector abstraction. A vector of
order {\em size} is an ordered 1d array of {\em size} numbers. For
example a vector of order 5: \\

\indent\indent $ x = [x_0$ $x_1$ $x_2$  $x_3$ $x_4]$ \\


In the Vector class the data is stored in a 1d double array of length
equal to the order of the Vector.  At present time none of the methods
are declared as being virtual. THIS MAY CHANGE FOR PARALLEL. \\

\noindent {\bf Class Interface} \\
\indent {// Constructors}  \\ 
\indent {\em Vector();}  \\
\indent {\em  Vector(int size);}  \\
\indent {\em  Vector(double *data, int size)}  \\
\indent {\em  Vector(const Vector \&M); }  \\ \\
\indent {// Destructor}  \\ 
\indent {\em $\tilde{}$Vector();} \\ \\
\indent {// Public Methods}  \\ 
\indent {\em int Size() const;} \\
\indent {\em void Zero();} \\
\indent {\em Assemble(const Vector \&M, const ID \&loc, double
fact = 1.0);} \\ 
\indent {\em int addVector(const Vector \&other, double fact =
1.0);} \\ 
\indent {\em int addMatrixVector(const Matrix \&m, Vector \&v,
double fact = 1.0);} \\  
\indent {\em double Norm(void)} \\ \\
\indent {// Overloaded Operator Functions}  \\
\indent {\em double \&operator()(int x) const;} \\
\indent {\em double \&operator()(int x);} \\
\indent {\em double \&operator[](int x) const;} \\
\indent {\em double \&operator[](int x);} \\
\indent {\em Vector operator()(const ID \&loc) const;}\\ 
\indent {\em Vector \&operator=(const Vector \&M);}\\
\indent {\em  Vector \&operator+=(double fact);} \\
\indent {\em Vector \&operator-=(double fact);} \\
\indent {\em Vector \&operator*=(double fact);} \\
\indent {\em Vector \&operator/=(double fact); } \\
\indent {\em Vector operator+(double fact) const;} \\
\indent {\em Vector operator-(double fact) const;} \\
\indent {\em  Vector operator*(double fact) const;} \\
\indent {\em  Vector operator/(double fact) const;} \\
\indent {\em  Vector \&operator+=(const Vector \&V);} \\
\indent {\em  Vector \&operator-=(const Vector \&V);} \\
\indent {\em  Vector operator+(const Vector \&V);} \\
\indent {\em  Vector operator-(const Vector \&V);} \\
\indent {\em  double operator { }$\hat{ }$(const Vector \&V) const;   } \\
\indent {\em Vector operator/(const Matrix \&M) const; } \\
\indent {\em friend OPS_Stream \&operator$<<$(OPS_Stream \&s, const
Vector \&V);} \\ 
\indent {\em friend istream \&operator$>>$(istream \&s, const
Vector \&V);} \\ 


\noindent {\bf Constructors}  \\
\indent {\em Vector();}  \\
To construct a Vector of order $0$. \\

\indent {\em  Vector(int size);}  \\
To construct a Vector of order {\em size}. The constructor creates an
array to store the data and zeroes this array. If not enough memory is
available a warning message is printed and a Vector of order $0$ is 
returned. The {\em Zero()}  method is invoked on the new Vector before
it is returned.\\ 

\indent {\em  Vector(double *data, int size)}  \\
To construct a Vector of order {\em size} whose data will be stored in the
array pointed to by {\em data}. The array pointed to by data is not set to
zero by the constructor. Note that delete will not be called on this array
in the destructor. It is up to the user to ensure that the array pointed to
by {\em data} is at least as large as {\em size}, if this is not the case
erroneous results or a segmentation fault may occur.\\

\indent {\em  Vector(const Vector \&other); }  \\
To construct a Vector using another Vector. The new Vector will be
identical to the Vector {\em other}. The constructor creates an array
to store the data and zeroes this array. If not enough memory is available
a warning message is printed and a Vector of order $0$ is returned. The
contents of the array are then set equal to the contents of {\em
other}.\\ 

\noindent {\bf Destructor} \\
\indent {\em virtual~ $\tilde{}$Vector();}\\ 
Will delete any space allocated in the constructors. If the array is
passed in the constructor, the space is not deallocated.\\

\noindent {\bf Public Member Functions }  \\
\indent {\em int Size() const;} \\
Returns the order of the Vector, {\em size}. \\

\indent {\em void Zero();} \\
Zeros out the Vector, i.e. sets all the components of the Vector to
$0$. \\

\indent {\em int Assemble(const Vector \&V, 
const ID \&loc,
double fact = 1.0);} \\
Assembles into the current Vector the Vector {\em V}. The contents of the
current Vector at location ({\em loc(i)}) is set equal to the current
value plus {\em fact} times the value of the Vector {\em V} at
location ({\em i}). returns $0$ if successful. A warning message is
printed for each invalid location in the current Vector or {\em V} and a
$-1$ is returned. \\ 

\indent {\em void addVector(const Vector \&other, double fact = 1.0);} \\
To add a factor {\em fact} times the Vector {\em other} to the current
Vector. returns $0$ if successful. An error message is printed and
$-1$ is returned if Vectors are not of the same size. Checks are made
to see if the number of operations can be reduced if {\em fact} is $0$
or $1$. \\ 

\indent {\em void addMatrixVector(const Matrix \&m, Vector \&v, double
fact = 1.0);} \\ 
To add a factor {\em fact} times the Vector formed by the product of
the matrix {\em m} and the Vector {\em v} to the current Vector. No
temporary Vector is created. Returns $0$ if successful. Prints a
warning message and returns $-1$ if sizes are incompatible. Checks are
made to see if the number of operations can be reduced if {\em fact}
is $0$ or $1$. \\ 


\indent {\em double Norm(void)} \\
Returns the 2 norm of the Vector. Returns the {\em sqrt()} of the
result of invoking the $\hat{ }$ operator on the current Vector with
the current Vector as the argument. \\ 

\noindent {\bf Overloaded Operator Functions}  \\
\indent {\em double \&operator()(int x) const;} \\
Returns the data at location {\em x} in the Vector. Assumes ({\em x}) 
is a valid location in the Vector, i.e. $0 <= x $ order, a
segmentation fault or erroneous results can occur if this is not the 
case. \\ 

\indent {\em double \&operator()(int x);} \\
Used to set the data at location({\em x}) in the Vector. Assumes ({\em x})
is a valid location in the Vector, i.e. $0 <= x < $ order, a
segmentation fault or erroneous results can occur if this is not the
case. \\ 

\indent {\em double \&operator[](int x) const;} \\
To safely return the data at location {\em x} in the Vector. Checks to
ensure {\em x} is a valid location, i.e. $0 <= x $ order. If {\em x}
is not a valid location a warning message is printed and
VECTOR\_NOT\_VALID\_ENTRY (a static class variable) is returned. This
is a slower but safer version of {\em () const}.\\ 

\indent {\em double \&operator[](int x);} \\
Used to safely set the data at location({\em x}) in the Vector. Checks
to ensure {\em x} is a valid location, i.e. $0 <= x $ order. If {\em
x} is not a valid location a warning message is printed and
VECTOR\_NOT\_VALID\_ENTRY (a static class variable) is modified. This
is a slower but safer version of {\em ()}.\\ 

\indent {\em Vector operator()(const ID \&loc) const;}\\ 
Returns a Vector of order {\em loc.Size()}. The contents of the new
Vector are given by the contents of the current Vector at the
locations given by the {\em loc}. For example the contents of the new
Vector at location $i$ are equal to the contents of the current Vector
at location {\em loc(i)}. Creates a new Vector, copies the data from
the current Vector and returns the new Vector. For each invalid
location specified in {\em loc} for the current Vector, a warning
message is printed.\\ 

\indent {\em Vector \&operator=(const Vector \&other);}\\
Sets the current Vector to be equal to the Vector given by {\em
other}. If the Vectors are of different sizes, the current data, if
allocated in a constructor, is deallocated and more space allocated
before the contents are copied. If not enough memory is available a
warning message is printed and the order of the current Vector is set
to $0$.\\ 

\indent {\em  Vector \&operator+=(double fact);} \\
A method to add {\em fact} to each component of the current Vector. \\

\indent {\em Vector \&operator-=(double fact);} \\
A method to subtract {\em fact} from each component of the current Vector. \\

\indent {\em Vector \&operator*=(double fact);} \\
A method to multiply each component of the current Vector by fact. \\

\indent {\em Vector \&operator/=(double fact); } \\
A method which will divide each component of the current Vector by
{\em fact}. If {\em fact} is equal to zero an warning message is printed
and the components of the Vector are set to VECTOR\_VERY\_LARGE\_VALUE
(defined in $<$Vector.h$>$). \\

\indent {\em Vector operator+(double fact) const;} \\
A method to return a new Vector whose components are equal to the
components of the current Vector plus the value {\em fact}. A new Vector
is constructed using the current Vector as an argument to the
constructor; before the new matrix is returned, the {\em +=} operator
is invoked on the matrix with {\em fact}. If the new Vector and
current Vector are of different size, i.e. constructor fails to get
enough memory, a warning message is printed. \\ 

\indent {\em Vector operator-(double fact) const;} \\
A method to return a new Vector whose components are equal to the
components of the current Vector minus the value {\em fact}.  A new Vector
is constructed using the current Vector as an argument to the
constructor; before the new matrix is returned, the {\em -=} operator
is invoked on the matrix with {\em fact}. If the new Vector and
current Vector are of different size, i.e. constructor fails to get
enough memory, a warning message is printed. \\ 


\indent {\em  Vector operator*(double fact) const;} \\
A method to return a new Vector whose components are equal to the
components of the current Vector times the value {\em fact}.  A new Vector
is constructed using the current Vector as an argument to the
constructor; before the new matrix is returned, the {\em *=} operator
is invoked on the matrix with {\em fact}. If the new Vector and
current Vector are of different sizes, a warning message is printed. \\


\indent {\em  Vector operator/(double fact) const;} \\
A method to return a new Vector whose components are equal to the
components of the current Vector divided the value {\em fact}. A new
Vector is constructed using the current Vector as an argument to the
constructor; before the new matrix is returned, the {\em /=} operator
is invoked on the matrix with {\em fact}. Warning messages are printed
if {\em fact} is equal to $0$ or if the new Vector and current Vector
are of different sizes. \\ 

\indent {\em  Vector \&operator+=(const Vector \&V);} \\
A method to add the contents of the Vector {\em V} to the current
Vector. If Vectors are not of same order a warning message is printed
and nothing is done.  \\ 

\indent {\em Vector \&operator-=(const Vector \&V);} \\
A method to subtract the contents of the Vector {\em V} from the
current Vector. If Vectors are not of same order a warning message is
printed and nothing is done.  \\ 


\indent {\em  Vector operator+(const Vector \&V);} \\
A method to return a new Vector which is equal to the sum of the
the current Vector and the Vector {\em V}. A new Vector is constructed
using the current Vector as an argument to the constructor; before the
new matrix is returned, the {\em +=} operator is invoked on the matrix
with {\em V}. If the current Vector and {\em V} are not of the same size,
a warning message is printed and a copy of the current Vector is
returned. A warning message is also returned if the new Vector is not
of the correct size, i.e. ran out of memory. \\ 


\indent {\em  Vector operator-(const Vector \&V);} \\
A method to return a new Vector which is equal to the the current
Vector minus the Vector {\em V}. A new Vector is constructed using the
current Vector as an argument to the constructor; before the new
matrix is returned, the {\em -=} operator is invoked on the matrix
with {\em V}. If the current Vector and {\em V} are not of the same size,
a warning message is printed and a copy of the current Vector is
returned. A warning message is also returned if the new Vector is not
of the correct size, i.e. ran out of memory. \\ 


\indent {\em double operator{ }$ \hat{ }$(const Vector \&V) const;   } \\
A method to return the dot product of the current Vector and the
Vector {\em V}. If the current Vector and {\em V} are not of the same
size, a warning message is printed and $0$ returned. \\ 

\indent {\em Vector operator/(const Matrix \&M) const; } \\
A method to return a new Vector, $x$, equal to the solution of the
matrix equation $Mx=$ the current Vector. A new Vector is created for
the return of size {\em M.noRows()}. A new Matrix is created of order
{\em M.noRows()} x {\em M.noRows()} and set equal to {\em M} if {\em
M} is square, or $M^tM$ if {\em M} is not square. The new Vector is
then set equal to the result of invoking {\em Solve(*this)} on the new
Matrix. \\ 

{\em friend OPS_Stream \&operator$<<$(OPS_Stream \&s, const Vector \&V);} \\
A function to print out the contents of the Vector {\em V} to the
output stream {\em s}. prints out the contents of the Vector in the
stream and then prints the newline character. \\

{\em friend istream \&operator$>>$(istream \&s, const Vector \&V);} \\
A function to read the contents of the Vector {\em V} from the input
stream {\em s}. Sets the components of {\em V} equal to the next {\em
V.Size()} entries in the stream.\\ 







\pagebreak
\subsection{ID}
%File: ~/OOP/matrix/ID.tex
%What: "@(#) ID.tex, revA"

\noindent {\bf Files}   \\
\indent \#include $<\tilde{ }$/matrix/ID.h$>$  \\

\noindent {\bf Class Declaration}  \\
\indent class ID:  \\

\noindent {\bf Class Hierarchy} \\
\indent {\bf ID} \\

\noindent {\bf Description}  \\
\indent The ID class provides the abstraction for an integer Vector. 
The class is introduced in addition to the Vector class, to save memory
and casting when integer arrays are required. An ID of order {\em
size} is an ordered 1d array of {\em size} integer values. For example
an ID id of order 5:  \\

\indent\indent $id = [id_0$ $id_1$ $id_2$ $id_3$ $id_4]$ \\

In the ID class, the data is stored in a 1d integer array of length
equal to arraySize, where order <= arraySize. Creating an ID with
storage capacity greater than that required allows the ID object to
grow without the need to deallocate and allocate more memory. At
present time none of the methods are declared as being virtual. THIS
MAY CHANGE. \\

\noindent {\bf Class Interface} \\
\indent {// Constructors}  \\ 
\indent {\em ID();}  \\
\indent {\em  ID(int idSize);}  \\
\indent {\em  ID(int idSize, int arraySize);}  \\
\indent {\em  ID(const ID \&M); }  \\ \\
\indent {// Destructor} \\
\indent {\em virtual~ $\tilde{}$ID();}\\  \\
\indent {// Public Methods }  \\
\indent {\em virtual int Size() const;} \\
\indent {\em virtual void Zero();} \\
\indent {\em virtual int getLocation(int x) const;} \\
\indent {\em virtual int removeValue(int x);} \\ \\
\indent {// Overloaded Operator Functions}  \\
\indent {\em virtual double \&operator()(int x);} \\
\indent {\em virtual double \&operator[](int x);} \\
\indent {\em virtual ID \&operator=(const ID \&M);}\\
\indent {\em friend OPS_Stream \&operator$<<$(OPS_Stream \&s, const ID
\&id);} \\  
\indent {\em friend istream \&operator$<<$(istream \&s, const ID
\&id);} \\ 



\noindent {\bf Constructors}  \\
\indent {\em ID();}  \\
To construct an ID of size $0$. No memory is allocated for the storage
of the data. \\

\indent {\em  ID(int idSize);}  \\
To construct a ID of size {\em idSize}. The constructor creates an
integer array of size {\em idSize} to store the data and zeroes this
array by invoking {\em Zero()} on itself. If not enough memory is
available an error message is printed and an ID of size $0$ is
returned. \\  

\indent {\em  ID(int idSize, int arraySize);}  \\
To construct a ID of size {\em idSize}. The constructor creates an
integer array of size {\em arraySize} to store the data and zeroes
this array. If {\em arraySize} is less than {\em idSize}, the new {\em
arraySize} is set equal to {\em idSize}. If not enough memory is
available an error message is printed and the program is
terminated. This constructor is provided to allow an ID to grow. \\  

\indent {\em  \indent  ID(const ID \&M); }  \\
To construct an ID using another ID {\em M}. The new ID will be
identical to the ID $M$, same order and same size of array to hold the
integer values. If not enough memory is available a
warning message is printed and both the order and arraySize of the ID
are set to $0$. \\

\noindent {\bf Destructor} \\
\indent {\em virtual~ $\tilde{}$ID();}\\ 
Will invoke delete on the integer array used to store the components. \\

\noindent {\bf Public Member Functions }  \\
\indent {\em virtual int Size() const;} \\
Returns the order of the ID. \\

\indent {\em virtual void Zero();} \\
Zeros out the ID, i.e. sets all the components of the ID to
$0$. This is accomplished by zeroing the first {\em this.Size()}
components of the array. \\

\indent {\em virtual int getLocation(int x) const;} \\
Will return the location the first location in the ID of the integer
{\em x}. If {\em x} is not in the ID a $-1$ is returned. \\

\indent {\em virtual int removeValue(int x);} \\
Will return the last location in the ID of the integer
{\em x}. If {\em x} is not in the ID a $-1$ is returned. All the integer
components {\em x} are removed from the ID and the length of the ID is
reduced by the number of the removed components. The {\em arraySize}
remains unchanged. \\ 


\noindent {\bf Overloaded Operator Functions}  \\
\indent {\em virtual virtual double \&operator()(int x) const;} \\
Returns the data at location {\em x} in the ID. Assumes ({\em x})
is a valid location in the ID, a segmentation fault or erroneous
results can occur if this is not the case. \\

\indent {\em virtual double \&operator()(int x);} \\
Used to set the data at location({\em x}) in the ID. Assumes ({\em x})
is a valid location in the ID, a segmentation fault or erroneous
results can occur if this is not the case. \\

\indent {\em virtual double \&operator[](int x);} \\
Used to set the data at location({\em x}) in the ID. If {\em x} is outside
the order of the ID the ID is order of the ID is enlarged to {\em
x+1}. When increasing the order, a check is first made to see if the
current array is large enough; if it is the components between the old
end and the new component are set to $0$ and the order of the ID is set
to {\em x+1}, if not a new array is created. The size of this array is
max($2*$old array size, x). A copy of the components of the old array
into the new array is made, with any new components set to $0$. If not
enough space is available or {\em x} is less than $0$, a warning
message is printed and the contents of ID\_ERROR returned. \\


\indent {\em virtual ID \&operator=(const ID \&M);}\\
Sets the current ID to be equal to the ID given by {\em M}. If the
IDs are of different sizes, the current data is deallocated and
more space allocated before the contents are copied. If not enough
memory is available, the order and {\em arraySize} of the current ID
is set to $0$ and the ID is returned without copying the components. \\


{\em friend OPS_Stream \&operator$<<$(OPS_Stream \&s, const ID \&id);} \\
A function to print out the contents of the ID {\em id} to the output stream
{\em s}. Prints out the components into the stream and then sends a
newline character. \\

{\em friend istream \&operator$>>$(istream \&s, const ID \&id);} \\
A function to read the contents of the ID {\em id} from the input
stream {\em s}. Sets the components of {\em id} equal to the next {\em
id.Size()} entries in the stream.\\ 









\pagebreak
\section{Domain Classes}
These are the classes that are used to describe the finite element
model and to store the results of an analysis on this model. The
classes provide include Domain, Element, Node, Load, Constraint and
their subclasses. Typically, a Domain object is used as a container
object to store and provide access to the  Node, Element, Load and
Constraint objects created by the ModelBuilder object.

\pagebreak

\subsection{{\bf DomainComponent}}
% File: ~/domain/component/DomainComponent.tex 
% What: "@(#) DomainComponent.tex, revA"

\noindent {\bf Files}   \\
\indent \#include $<\tilde{ }$/domain/component/DomainComponent.h$>$  \\

\noindent {\bf Class Declaration}  \\
\indent class DomainComponent \\

\noindent {\bf Class Hierarchy} \\
\indent  TaggedObject \\
\indent  MovableObject \\
\indent\indent  {\bf DomainComponent} \\
\indent\indent\indent  Element \\
\indent\indent\indent  Node \\
\indent\indent\indent  NodalLoad \\
\indent\indent\indent  ElementalLoad \\
\indent\indent\indent  SP\_Constraint \\
\indent\indent\indent  MP\_Constraint \\

\noindent {\bf Description}  \\
\indent The DomainComponent class is an abstract class, example
subclasses include Element, Node, SP\_Constraint, MP\_Constraint,
NodalLoad, ElementalLoad. Each object of these types is a component of
an enclosing Domain object. The DomainComponent class provides methods
to set and retrieve a pointer to the enclosing Domain object. \\

\noindent {\bf Class Interface}  \\
\indent\indent // Constructor  \\
\indent\indent {\em DomainComponent(int tag, int classTag);}  \\ \\
\indent\indent // Destructor  \\
\indent\indent {\em virtual~ $\tilde{}$DomainComponent();}  \\ \\
\indent\indent // Public Methods  \\
\indent\indent {\em virtual void setDomain(Domain *theDomain);} \\
\indent\indent {\em virtual Domain *getDomain(void) const;}\\
\indent\indent {\em virtual int displaySelf(Renderer \&theRenderer, int
displayMode, float fact); } \\

\noindent {\bf Constructor}  \\
\indent {\em DomainComponent(int tag, int classTag);}  \\
Constructs a DomainComponent with a tag given by {\em tag}, whose
class tag is given by {\em classTag}. The tag of
a component is some unique means of identifying the component among
like components, i.e. the tag of a node would be its unique node
number. The {\em classTag} is a means of identifying the class of the object.
No domain is associated with the object. The integer {\em tag} is
passed to the TaggedObject constructor and the integer {\em classTag}
is passed to the MovableObject constructor. \\

\noindent {\bf Destructor}  \\
\indent {\em virtual~ $\tilde{}$DomainComponent();}  \\
Does nothing. Provided so subclasses destructor's will always be
called. \\

\noindent {\bf Public Methods}  \\
\indent {\em virtual void setDomain(Domain *theDomain);} \\
Sets the encompassing domain of the component to that given by {\em
theDomain}. This method is invoked by {\em theDomain} when the component is
being added to the domain, in an {\em addDomain..} invocation (see
interface for Domain). \\

\indent {\em virtual Domain *getDomain(void) const;}\\
Returns a pointer to the Domain to which the component was added,
or $0$ if the {\em setDomain()} command was never called on the
object. \\

\indent {\em virtual int displaySelf(Renderer \&theRenderer, int
displayMode, float fact); } \\ 
To cause the object to display itself using {\em theRenderer}. The
integer {\em displayMode} is used to indicate what is to be displayed
and the float {\em fact} is used to factor the nodal displacements. 
To return $0$ if successful, a negative number if not. This base
class simply returns $0$. Its up to the subclasses to override this
method if the objects are to be rendered.



\pagebreak
\subsection{{\bf Element}}
%File: ~/OOP/element/Element.tex
%What: "@(#) Element.tex, revA"

NEED TO ADD ADD\_INERTIA\_LOAD TO INTERFACE .. SEE EARTHQUAKE\_PATTERN
CLASS.\\

\noindent {\bf Files}   \\
\indent \#include $<\tilde{ }$/element/Element.h$>$  \\

\noindent {\bf Class Declaration}  \\
\indent class Element: public DomainComponent \\

\noindent {\bf Class Hierarchy} \\
\indent TaggedObject \\
\indent MovableObject \\
\indent\indent DomainComponent \\
\indent\indent\indent {\bf Element} \\

\noindent {\bf Description}  \\
\indent Element is an abstract class, i.e. no instances of Element
will exist. The element class provides the interface that all element
writers must provide when introducing new element classes. \\


\noindent {\bf Class Interface} \\
\indent // Constructor \\
\indent {\em Element(int tag, int classTag);}  \\ \\
\indent // Destructor \\
\indent {\em virtual~ $\tilde{}$Element();}\\ \\
\indent // Public Methods dealing with Nodes and dof\\
\indent {\em virtual int getNumExternalNodes(void) const =0;} \\
\indent {\em virtual const ID \&getExternalNodes(void) =0;} \\
\indent {\em virtual int getNumDOF(void) =0;}\\ \\
\indent // Public Methods dealing with State\\
\indent{\em virtual int commitState(void) =0;} \\
\indent{\em virtual int revertToLastCommit(void) =0;} \\ 
\indent{\em virtual int revertToStart(void) =0;} \\
\indent{\em virtual int update(void);} \\
\indent {\em virtual bool isSubdomain(void);} \\ \\
\indent // Public Methods for obtaining Linearized Stiffness,
Mass and Damping Matrices\\
\indent {\em virtual Matrix \&getTangentStiff(void) =0;} \\
\indent {\em virtual Matrix \&getSecantStiff(void) =0;} \\
\indent {\em virtual Matrix \&getDamp(void) =0;} \\
\indent {\em virtual Matrix \&getMass(void) =0;} \\ \\
\indent // Public Methods for obtaining Resisting Forces \\
\indent {\em virtual void zeroLoad(void) =0;	}\\
\indent {\em virtual Vector \&getResistingForce(void) =0;} \\
\indent {\em virtual Vector \&getResistingForceIncInertia(void) =0;} \\ \\
\indent // methods for obtaining information specific to an element \\
\indent {\em virtual int setResponse(char **argv, int argc, Information \&eleInformation);}\\
\indent {\em virtual int getResponse(int responseID, Information \&eleInformation);}\\

\noindent {\bf Constructor}  \\
\indent {\em Element(int tag, int classTag);}  \\
To construct an element whose unique integer among elements in the
domain is given by {\em tag}, and whose class identifier is given
by {\em classTag}. Both of these integers are passed to the
DomainComponent constructor. \\

\noindent {\bf Destructor} \\
\indent {\em virtual~ $\tilde{}$Element();}\\ 
The destructor. Declared as virtual to allow subclass destructors to
be invoked. \\

\noindent {\bf Public Methods }  \\
\indent {\em virtual int getNumExternalNodes(void) const =0;} \\
To return the number of external nodes associated with the element. \\

\indent {\em virtual const ID \&getExternalNodes(void) =0;} \\
To return an ID containing the tags of the external nodes for the element. \\

\indent {\em virtual int getNumDOF(void) =0;}\\
To return the number of dof associated with the element. This should
equal the sum of the dofs at each of the external nodes. To ensure
this, each subclass can overwrite the DomainComponent classes {\em
setDomain()} method.\\ 

{\em virtual int commitState(void) =0;} \\
The element is to commit its current state. To return $0$ if
sucessfull, a negative number if not. \\

{\em virtual int revertToLastCommit(void) =0;} \\
The element is to set it's current state to the last committed
state. To return $0$ if sucessfull, a negative number if not. \\

{\em virtual int revertToStart(void) =0;} \\
The element is to set it's current state to the state it was at before
the analysis started. To return $0$ if sucessfull, a negative number
if not. \\ 

{\em virtual int update(void);} \\
This method is invoked after the response quantities have been updated
in the Domain, but not necessarily committed, e.g. during a
non-linear Newton-Raphson solution algorithm. To return $0$ if
successful, a negative number if not. This base class implementation returns
$0$. \\

{\em virtual bool isSubdomain(void);} \\
The element is to return true if the element is of type (or subtype)
Subdomain, else the element should return false. This base class
implementation returns $false$. \\


{\em virtual Matrix \&getTangentStiff(void) =0;} \\
To return the tangent stiffness matrix. The element is to compute its
stiffness matrix based on the original location of the nodes and the
current trial displacement at the nodes. \\ 

$$ 
\K_e = {\frac{\partial \f_{R_i}}{\partial \U}
\vert}_{\U_{trial}}
$$

{\em virtual Matrix \&getSecantStiff(void) =0;} \\
To return the elements secant stiffness matrix. The element is to
compute its stiffness matrix based on the original location of the
nodes and the current trial displacement at the nodes. THIS SECANT MAY
BE REMOVED. \\

{\em virtual Matrix \&getDamp(void) =0;} \\
To return the damping matrix. The element is to compute its
damping matrix based on the original location of the nodes and the
current trial response quantities at the nodes. \\ 

$$ 
\C_e = {\frac{\partial \f_{R_i}}{\partial \dot \U}
\vert}_{\U_{trial}}
$$

{\em virtual Matrix \&getMass(void) =0;} \\
To return the mass matrix. The element is to compute its
mass matrix based on the original location of the nodes and the
current trial response quantities at the nodes. \\ 

$$ 
\M_e  = {\frac{\partial \f_{I_i}}{\partial \ddot \U}
\vert}_{\U_{trial}}
$$

{\em virtual void zeroLoad(void) =0;}\\
This is a method invoked to zero the element load contributions to the
residual, i.e. $\P_e = \zero$ \\ 

{\em virtual Vector \&getResistingForce(void) =0;} \\
Returns the resisting force vector for the element. This is equal to
the applied load due to element loads minus the loads at the nodes due
to internal stresses in the element due to the current trial
displacement, i.e. 
$$
\R_e = 
\P_{e} - \f_{R_e}(\U_{trial}) 
$$

{\em virtual Vector \&getResistingForceIncInertia(void) =0;} \\
Returns the resisting force vector for the element with inertia forces
included. This is equal to the applied load due to element loads
(loads set using {\em addLoad()}, minus the loads at the nodes due to
internal stresses in the element due to the current trial response
quantities, i.e.
$$
\R_e = 
\P_e -  \f_{I_e} (\ddot \U_{trial}) - \f_{R_e}(\dot
\U_{trial}, \U_{trial})
$$


\indent {\em virtual int setResponse(char **argv, int argc, Information \&eleInformation);}\\
{\em setResponse()} is a method invoked to determine if the element
will respond to a request for a certain of information. The
information requested of the element is passed in the array of char
pointers {\em argv} of length {em argc}. If the element does not
respond to the request a $-1$ is returned. If it does respond, an
integer value greater than or equal to $0$ is returned. This is the
{\em responseID} passed in the {\em getResponse()} method. In addition
the Element object is responsible for setting the Information object
{\em eleInformation} with the type of the return, i.e. {\em IntType,
DoubleType, MatrixType, VectorType, IDType}, and for creating a Matrix,
Vector or ID object for the Information object, if the information to
be returned is of any of these types. The information object is
responsible for invoking the destructor on these objects. The base
class responds to no requests and will always return $-1$. \\

\indent {\em virtual int getResponse(int responseID, Information
\&eleInformation);}\\ 
getResponse is a method invoked to obtain information from an
analysis. The method is invoked with the integer argument returned and
the Information object that was prepared in a successfull {\em
setResponse()} method invocation. To return $0$ if successfull, a
negative number if not. The base class implementation will always
return $-1$. 


\pagebreak
\subsection{Node}
%File: ~/OOP/element/Node.tex
%What: "@(#) Node.tex, revA"

NEW METHOD POSSIBLY NEEDED TO SPECIFY THE NUMBER OF PREVIOUSLY
COMMITTED RESPONSES TO KEEP .. NEEDED FOR EXPLICIT DYNAMIC INTEGRATORS\\

\noindent {\bf Files}   \\
\indent \#include $<$/domain/node/Node.h$>$  \\

\noindent {\bf Class Declaration}  \\
\indent class Node: public DomainComponent  \\

\noindent {\bf Class Hierarchy} \\
\indent TaggedObject \\
\indent MovableObject \\
\indent\indent DomainComponent \\
\indent\indent\indent {\bf Node} \\

\noindent {\bf Description}  \\
\indent Nodes are points in space connected by the elements. Nodes
have original position, trial displacement, velocity and acceleration, 
and committed displacement, velocity and acceleration (the last
committed trial quantities). Nodes also store information about any
load acting on the node, nodal mass and the nodal participation
matrix. In addition, each Node object keeps track of it's associated
DOF\_Group object. The Node interface provides methods to set and
retrieve these quantities.\\  


\noindent {\bf Class Interface}  \\
\indent\indent // Constructors  \\
\indent\indent {\em Node(int classTag);}  \\
\indent\indent {\em Node(int tag, int classTag);}  \\
\indent\indent {\em Node(int tag, int ndof, double Crd1);}  \\
\indent\indent {\em Node(int tag, int ndof, double Crd1, double Crd2);}  \\
\indent\indent {\em Node(int tag, int ndof, double Crd1, double Crd2, double Crd3);}\\
\indent\indent {\em Node(const Node \&theCopy);}\\ \\
\indent\indent // Destructor  \\
\indent\indent {\em virtual $\tilde{ }$ Node();} \\ \\
\indent\indent // Public Methods dealing with DOF at the Node\\
\indent\indent {\em virtual int  getNumberDOF(void) const;}\\
\indent\indent {\em virtual void setDOF\_GroupPtr(DOF\_Group *theDOF\_Grp);} \\
\indent\indent {\em virtual DOF\_Group *getDOF\_GroupPtr(void);} \\ \\
\indent\indent // Public Method for obtaining nodal coordinates\\
\indent\indent {\em virtual const Vector \&getCrds(void) const;}  \\ \\
\indent\indent // Public Method for obtaining committed and trial responses\\ 
\indent\indent {\em  virtual const Vector \&getDisp(void) ;}  \\
\indent\indent {\em  virtual const Vector \&getVel(void) ;}  \\
\indent\indent {\em  virtual const Vector \&getAccel(void) ;}  \\
\indent\indent {\em  virtual const Vector \&getTrialDisp(void) ;}  \\
\indent\indent {\em  virtual const Vector \&getTrialVel(void) ;}  \\
\indent\indent {\em  virtual const Vector \&getTrialAccel(void) ;}  \\ 
\indent\indent {\em  virtual const Vector \&getIncrDisp(void) ;}  \\\\
\indent\indent // Public Method for updating trial responses\\
\indent\indent {\em virtual int setTrialDisp(const Vector \&newTrialDisp);} \\
\indent\indent {\em virtual int setTrialVel(const Vector \&newTrialVel);} \\
\indent\indent {\em virtual int setTrialAccel(const Vector \&newTrialAccel);} \\
\indent\indent {\em virtual int incrTrialDisp(const Vector \&trialIncrDisp);} \\
\indent\indent {\em virtual int incrTrialVel(const Vector \&trialIncrVel);} \\
\indent\indent {\em virtual int incrTrialAccel(const Vector \&trialIncrAccel);} \\ \\
\indent\indent // Public Method for setting and obtaining unbalanced load\\
\indent\indent {\em virtual void zeroUnbalancedLoad(void);} \\
\indent\indent {\em virtual int addUnbalancedLoad(const Vector
\&additionalLoad, double fact = 1.0);} \\ 
\indent\indent {\em virtual int addInertiaLoadToUnbalance(const Vector
\&accel, double fact = 1.0);} \\ 
\indent\indent {\em virtual const Vector \&getUnbalancedLoad(void);} \\
\indent\indent {\em virtual const Vector \&getUnbalancedLoadIncInertia(void);} \\ \\
\indent\indent // Public Method for setting state \\
\indent\indent {\em virtual int commitState(void);} \\
\indent\indent {\em virtual int revertToLastCommit(void);} \\
\indent\indent {\em virtual int revertToStart(void);} \\ \\
\indent\indent // Public Method for dynamic and modal analysis\\
\indent\indent {\em virtual const Matrix \&getMass(void) ;}\\
\indent\indent {\em virtual int setMass(const Matrix \&mass);} \\
\indent\indent {\em virtual int setNumColR(int numCol);} \\
\indent\indent {\em virtual int setR(int row, int col, double Value);} \\
\indent\indent {\em virtual const Vector \& getRV(const Vector \&V);}\\ \\
\indent\indent // Public Method for Output \\
\indent\indent {\em virtual int sendSelf(int commitTag, Channel
\&theChannel);} \\
\indent\indent {\em virtual int recvSelf(int commitTag, Channel \&theChannel,
FEM\_ObjectBroker \&theBroker);} \\ 
\indent\indent {\em void Print(OPS_Stream \&s, int flag = 0);} \\
\indent\indent {\em int displaySelf(Renderer \&theRenderer, int flag,
float fact);} \\



\noindent {\bf Constructors}  \\
\indent {\em Node(int classTag);}  \\
To construct a node which has no data, other than the {\em classTag}
identifier; $0$ and {\em classTag} are passed to the DomainComponent
constructor. This is the constructor called by an
FEM\_ObjectBroker. The data must be filled in subsequently by a call
to {\em recvSelf()}. \\ 

\indent {\em Node(int tag, int classTag);}  \\
To construct a node whose unique integer among nodes in the
domain is given by {\em tag} and whose classTag is given by {\em
classTag}. This constructor can be used by 
subclasses who wish to handle their own data management. \\ 

\indent {\em Node(int tag, int ndof, double Crd1);}  \\
To construct a node for 1d problems whose unique integer among nodes in the
domain is given by {\em tag} and whose original position in 1d space
is given by (Crd1). With the node is associated {\em ndof} number
of degrees of freedom. The class tag is NOD\_TAG\_Node (defined in
classTags.h). A Vector object is created to hold the coordinates. No
storage objects are created to hold the trial and committed response
quantities, mass, load quantities; these are created as needed to
reduce the memory demands on the system in certain situations. \\ 

\indent {\em Node(int tag, int ndof, double Crd1, double Crd2);}  \\
To construct a node for 2d problems whose unique integer among nodes in the
domain is given by {\em tag} and whose original position in 2d space
is given by (Crd1,Crd2). With the node is associated {\em ndof} number
of degrees of freedom. The class tag is NOD\_TAG\_Node. A Vector object
is created to hold the coordinates. No
storage objects are created to hold the trial and committed response
quantities, mass, load quantities; these are created as needed to
reduce the memory demands on the system in certain situations. \\ 

\indent {\em Node(int tag, int ndof, double Crd1, double Crd2, double Crd3);}\\
To construct a node for 3d problems whose unique integer among nodes in the
domain is given by {\em tag} and whose original position in 3d space
is given by (Crd1,Crd2,Crd3). With the node is associated {\em ndof} number
of degrees of freedom. The class tag is NOD\_TAG\_Node. A Vector object
is created to hold the coordinates. No
storage objects are created to hold the trial and committed response
quantities, mass, load quantities; these are created as needed to
reduce the memory demands on the system in certain situations. \\ 



\indent {\em Node(const Node \&theCopy);}\\
To construct a node which is an exact copy of {\em theCopy}. \\

\noindent {\bf Destructor}  \\
\indent {\em virtual~$\tilde{}$ Node();} \\
Invokes the destructor on all the storage objects created to hold the coordinates,
response quantities, mass and load quantities. \\

\noindent {\bf Public Member Functions }  \\
\indent {\em virtual int  getNumberDOF(void) const;}\\
Returns the number of degrees-of-freedom, {\em ndof}, associated with
the node. \\

\indent {\em virtual void setDOF\_GroupPtr(DOF\_Group *theDOF\_Grp);} \\
Each node, when involved with an analysis, will be associated with a
DOF\_Group object. It is the DOF\_Group that contains the ID of equation
numbers. When invoked this method sets the pointer to that DOF\_Group object. \\

{\em virtual DOF\_Group *getDOF\_GroupPtr(void);} \\
Method which returns a pointer to the DOF\_Group object that was set
using {\em setDOF\_GroupPtr}. If no pointer has been set a $0$ is
returned. \\

{\em virtual const Vector \&getCrds(void) const;}  \\
Returns the original coordinates in a Vector. The size of the vector
is 2 if node object was created for a 2d problem and the size is 3 if
created for a 3d problem. \\

{\em  virtual const Vector \&getDisp(void) ;}  \\
Returns the last committed displacement as a Vector, the vector of
size {\em ndof}. If no Vector has yet been allocated, two Vector
objects are created to store the committed and trial response
quantities created; if not enough space is available an error message
is printed and program terminated. \\ 

{\em  virtual const Vector \&getVel(void) ;}  \\
Returns the last committed velocity as a Vector, the vector of size
{\em ndof}. If no Vector has yet been allocated, two Vector
objects are created to store the committed and trial response
quantities created; if not enough space is available an error message
is printed and program terminated. \\ 


{\em  virtual const Vector \&getAccel(void) ;}  \\
Returns the last committed acceleration as a Vector, the vector of
size {\em ndof}. If no Vector has yet been allocated, two Vector 
objects are created to store the committed and trial response
quantities created; if not enough space is available an error message
is printed and program terminated. \\ 

{\em  virtual const Vector \&getTrialDisp(void) ;}  \\
Returns the current trial displacements as a Vector, the vector of size {\em ndof}.
If no Vector has yet been allocated, a new Vector is created and returned;
if not enough space is available an error message is printed and the
program is terminated. \\ 


{\em  virtual const Vector \&getTrialVel(void) ;}  \\
Returns the current trial velocities as a Vector, the vector of size {\em ndof}.
If no Vector has yet been allocated, a new Vector is created and returned;
if not enough space is available an error message is printed and the
program is terminated. \\ 

{\em  virtual const Vector \&getTrialAccel(void) ;}  \\
Returns the current trial accelerations as a Vector, the vector of size {\em ndof}.
If no Vector has yet been allocated, a new Vector is created and returned;
if not enough space is available an error message is printed and the
program is terminated. \\ 


{\em  virtual const Vector \&getIncrDisp(void) ;}  \\
Returns the incremental displacement as a Vector. The incremental displacement is 
equal to the difference between the current trial displacement and committed
displacement (trial - committed).
If no Vector has yet been allocated, three Vector 
objects are created to store the committed, trial and incremental response
quantities; if not enough space is available an error message
is printed and program terminated. \\ 


{\em virtual int setTrialDisp(const Vector \&newTrialDisp);} \\
Sets the current trial displacement to be that given by
{\em newTrialDisp}. Sets th incremental displacement to be trial $-$ committtd.
If no space has yet been allocated for
the trial displacements, two Vector objects are now created to store
the trial and committed response quantities; if not enough memory 
is available on the heap to create these new Vectors an error message
is printed and the program is terminated. Returns $0$ if successful,
an error message is printed and a $-2$ is returned if {\em
newTrialDisp} is not of size {\em ndof}. \\


{\em virtual int setTrialVel(const Vector \&newTrialVel);} \\
Sets the current trial velocity to be that given by
{\em newTrialVel}. If no space has yet been allocated for
the trial velocities, two Vector objects are now created to store
the trial and committed response quantities; if not enough memory 
is available on the heap to create these new Vectors an error message
is printed and the program is terminated. Returns $0$ if successful,
an error message is printed and a $-2$ is returned if {\em
newTrialVel} is not of size {\em ndof}. \\

{\em virtual int setTrialAccel(const Vector \&newTrialAccel);} \\
Sets the current trial acceleration to be that given by
{\em newTrialAccel}. If no space has yet been allocated for
the trial accelerations, two Vector objects are now created to store
the trial and committed response quantities; if not enough memory 
is available on the heap to create these new Vectors an error message
is printed and the program is terminated. Returns $0$ if successful,
an error message is printed and a $-2$ is returned if {\em
newTrialAccel} is not of size {\em ndof}. \\

{\em virtual int incrTrialDisp(const Vector \&trialIncrDisp);} \\
Sets the current trial displacement to be that given by the addition
of the last trial displacement, assumed $0$ if not yet set, and {\em
trialIncrDisp}. Increments the incremental displacement by {\em trialIncrDisp}.
If no space has yet been allocated for the displacements, three Vector objects 
are now created to store
the trial, committed and incremental response quantities; if not enough memory 
is available on the heap to create these new Vectors an error message
is printed and the program is terminated. Returns $0$ if successful,
an error message is printed and a $-2$ is returned if {\em
trialIncrDisp} is not of size {\em ndof}. \\

{\em virtual int incrTrialVel(const Vector \&trialIncrVel);} \\
Sets the current trial velocity to be that given by the addition
of the last trial velocity, assumed $0$ if not yet set, and {\em
trialIncrVel}. If no space has yet been allocated for
the trial velocities, two Vector objects are now created to store
the trial and committed response quantities; if not enough memory 
is available on the heap to create these new Vectors an error message
is printed and the program is terminated. Returns $0$ if successful,
an error message is printed and a $-2$ is returned if {\em
trialIncrVel} is not of size {\em ndof}. \\

{\em virtual int incrTrialAccel(const Vector \&trialIncrAccel);} \\
Sets the current trial Acceleration to be that given by the addition
of the last trial Acceleration, assumed $0$ if not yet set, and
{\em trialIncrAccel}.  If no space has yet been allocated for
the trial accelerations, two Vector objects are now created to store
the trial and committed response quantities; if not enough memory 
is available on the heap to create these new Vectors an error message
is printed and the program is terminated. Returns $0$ if successful,
an error message is printed and a $-2$ is returned if {\em
trialIncrAccel} is not of size {\em ndof}. \\

{\em virtual void zeroUnbalancedLoad(void);} \\
Causes the node to zero out its unbalanced load vector. \\

{\em virtual int addUnbalancedLoad(const Vector \&additionalLoad, double fact);} \\
The Node is responsible for adding {\em fact} times {\em
additionalLoad} to the current unbalanced load at the Node. If {\em
additionalLoad} is not of size {\em ndof} no load is added, an error
message is printed and a $-1$ is returned. If no space has yet been
allocated for the unbalanced load a new Vector is now created;  
if not enough space is available for this Vector an error message is
printed and the program is terminated. Returns $0$ if successful.  \\ 

\indent {\em virtual int addInertiaLoadToUnbalance(const Vector
\&accel, double fact = 1.0);} \\ 
To add {\bf minus} {\em fact} times the product $M * R * accel$ to the
current unbalanced load. Nothing is done if no mass or R matrix have
been set. Prints a warning and returns a $-1$ if the size of accel and
the number of columns in $R$ are not the same. If no space has yet been
allocated for the unbalanced load a new Vector is now created;  
if not enough space is available for this Vector an error message is
printed and the program is terminated. Returns $0$ if successful.  \\ 

{\em virtual const Vector \&getUnbalancedLoad(void);} \\
Returns the current unbalanced load. If no space has yet been
allocated for the unbalanced load a new Vector of size {\em numDOF} is
now created; if not enough space is available for this Vector an error
message is printed and the program is terminated. \\ 

{\em virtual const Vector \&getUnbalancedLoadIncInertia(void);} \\
Returns the current unbalanced load Vector, as defined above, MINUS the product
of the nodes mass matrix and the trial nodal accelerations. The result
is saved in another vector which is returned. If no space has yet been
allocated for this new Vector, a Vector of size {\em numDOF} is now
created; if not enough space is available for this Vector an error
message is printed and the program is terminated. \\ 

{\em virtual int commitState(void);} \\
Causes the node to set the committed model displacements, velocities and
accelerations to be equal to the current trial displacements, velocities and
accelerations. The incremental displacement is set to $0$. No assignment is done 
for any of the quantities for
which no memory has been allocated. Returns $0$.\\

{\em virtual int revertToLastCommit(void);} \\
Causes the node to set the trial nodal displacements, velocities and
accelerations to be equal to the current committed displacements, velocities and
accelerations. The incremental displacement is set to $0$. No assignment is done 
for any of the trial quantities for which no memory has been allocated. Returns $0$.\\

{\em virtual int revertToStart(void);} \\
Causes the node to set the trial and committed nodal displacements,
velocities and accelerations to zero. No assignment is done for any of
the trial quantities for which no memory has been allocated. Returns $0$.\\


{\em virtual const Matrix \&getMass(void) ;}\\
Returns the mass matrix set for the node, which is a matrix of size
{\em ndof,ndof}. This matrix is equal to that set in {\em setMass()}
or zero if {\em setMass()} has not been called. If no storage space
has been allocated for the mass, a matrix is now created. An error
message is printed and the program terminated if no space is available
on the heap for this matrix.\\

{\em virtual int setMass(const Matrix \&mass);} \\
Sets the value of the mass at the node. A check is made to ensure that
the {\em mass} has the same dimensions of the mass matrix associated with the
Node; if incompatible size an error message is printed and -1 returned. If no
mass matrix yet allocated, one is created; if no space is available an
error message is printed and the program terminated. Returns 0 if successful. \\

\indent {\em virtual int setNumColR(int numCol);} \\
Creates a Matrix to store the R matrix. The matrix is of dimension
{\em ndof} and {\em numCol}. If not enough space is available for this
matrix an error message is printed and the program is terminated. Zeros the
matrix R and returns $0$ if successful. \\

\indent {\em virtual int setR(int row, int col, double Value);} \\
Sets the {\em \(row,col\)} entry of R to be equal to {\em Value}. If
no matrix R has been specified or the position in R is out of range a
warning message is printed and a $-1$ is returned. Returns $0$ if
successful. \\


\indent {\em virtual const Vector \& getRV(const Vector \&V);} \\
This is a method provided for Element objects, the Node object returns
the product of the matrix $R$ and the vector $V$. If the matrix 
and vector are of inappropriate size a warning message is printed and
a zero vector is returned. \\

{\em virtual int sendSelf(int commitTag, Channel \&theChannel);} \\
Causes the Node object to send the data needed to init itself on a
remote machine to the Channel object {\em theChannel}. 
The data sent includes the tag, number of dof, coordinates, committed
response quantities, unbalanced load, mass and participation matrix. 
To do this the Node creates an ID object into which it stores its tag,
the {\em ndof} and a flag indicating whether any additional
information, i.e. mass, response quantities also need to be sent. In
addition four database tags are also included in this ID object. The
database tags, if not already obtained, are requested from the Channel
object (these are needed as each object can only store a single object
of a particular size using it's own database tags -- additional tags
are needed when multiple objects of the same size are needed.
The objects that have been created are then sent. \\

{\em virtual int recvSelf(int commitTag, Channel \&theChannel,
FEM\_ObjectBroker \&theBroker);} \\ 
Invoked on a remote machine to read its data that was sent by a node
object in another actor when {\em sendSelf()} was invoked. As in {\em
sendSelf()}, the Node object creates an ID object. It asks the Channel
object to fill this object with data. Based on the data it creates
Matrix and Vector objects to store the Nodes data and asks the Channel
object to fill these with data. The data placed here by the Channel
object correspond to the data put there by the sending Node object.\\


{\em void Print(OPS_Stream \&s, int flag = 0);} \\
Causes the Node to print out its tag, mass matrix, and committed
response quantities. \\ 

\indent {\em int displaySelf(Renderer \&theRenderer, int flag,
float fact);} \\
Causes the Node to display itself. If {\em flag} is $1$ the Node will
cause its tag to be printed to the display. 





\pagebreak
\subsection{{\bf Load}}
% File: ~/domain/loadcase/Load.tex 

\noindent {\bf Files}   \\
\indent \#include $<\tilde{ }$/domain/load/Load.h$>$  \\

\noindent {\bf Class Declaration}  \\
\indent class Load: public DomainComponent  \\

\noindent {\bf Class Hierarchy} \\
\indent TaggedObject \\
\indent MovableObject \\
\indent\indent DomainComponent \\
\indent\indent\indent {\bf Load} \\

\noindent {\bf Description}  \\
\indent Load is an abstract base class. A Load object is used to add
load to the domain. The Load class defines one method in its interface
{\em applyLoad()}, a method all subclasses must implement. \\


\noindent {\bf Class Interface}  \\
\indent // Constructor  \\
\indent {\em Load(tag, int classTag);}  \\ \\
\indent // Destructor  \\
\indent {\em virtual $\tilde{ }$ Load();} \\ \\
\indent // Public Methods   \\
\indent {\em virtual void applyLoad(loadFactor) = 0;} \\
\indent {\em virtual void setLoadPatternTag(int loadPaternTag);}\\
\indent {\em virtual int  getLoadPatternTag(void) const;}\\

\noindent {\bf Constructor}  \\
\indent {\em Load(tag, int classTag);}  \\
Constructs a load with a tag given by {\em tag} and a class tag is
given by {\em classTag}. These are passed to the DomainComponent constructor. \\

\noindent {\bf Destructor}  \\
\indent {\em virtual~$\tilde{}$ Load();} \\

\noindent {\bf Public Methods }  \\
\indent {\em virtual void applyLoad(double loadFactor) = 0;} \\
The load object is to add {\em loadFactor} times the load to the
corresponding residual value at its associated element(s) or node(s). \\

\indent {\em virtual void setLoadPatternTag(int loadPaternTag);}\\
To set the tag of the enclosing load pattern for the load to be 
{\em loadPatternTag}. \\

\indent {\em virtual int  getLoadPatternTag(void) const;}\\
To return the current load pattern tag associated with the load. If no
load pattern tag has been set $-1$ is returned.



\pagebreak
\subsubsection{NodalLoad}
%File: ~/OOP/element/node/NodalLoad.tex
%What: "@(#) NodalLoad.tex, revA"

\noindent {\bf Files}   \\
\indent \#include $<\tilde{ }$/domain/node/NodalLoad.h$>$  \\

\noindent {\bf Class Declaration}  \\
\indent class NodalLoad: public Load  \\

\noindent {\bf Class Hierarchy} \\
\indent TaggedObject \\
\indent MovableObject \\
\indent\indent DomainComponent \\
\indent\indent\indent Load \\
\indent\indent\indent\indent {\bf NodalLoad} \\

\noindent {\bf Description}  \\
\indent NodalLoads are loads acting on Nodes. The public methods are
all declared as virtual to allow subclasses to be introduced for the
provision of time varying loads. Each NodalLoad object is associated
with a single Node object and has a Vector object corresponding to the
load acting on this Node object as a result of the NodalLoad. \\

\noindent {\bf Class Interface}  \\
\indent\indent // Constructors  \\
\indent\indent {\em NodalLoad(int tag, int node, const Vector \&load, bool
isLoadConstant = false);}  \\ 
\indent\indent {\em NodalLoad(int tag, int node, int classTag);}  \\
\indent\indent {\em NodalLoad(int classTag);}  \\ \\
\indent\indent // Destructor  \\
\indent\indent {\em virtual $\tilde{ }$ NodalLoad();} \\ \\
\indent\indent // Public Methods  \\
\indent\indent {\em virtual void setDomain(Domain *newDomain);} \\
\indent\indent {\em virtual int getNodeTag(void) const;} \\
\indent\indent {\em virtual void applyLoad(double loadFactor);} \\ \\
\indent\indent // Public Methods for Output \\
\indent\indent {\em virtual int sendSelf(int commitTag, Channel \&theChannel); } \\
\indent\indent {\em virtual int recvSelf(int commitTag, Channel \&theChannel,
FEM\_ObjectBroker \&theBroker); } \\
\indent\indent {\em virtual void Print(ostream \&s, int flag = 0);} \\


\noindent {\bf Constructors}  \\
\indent {\em NodalLoad(int tag, int node, const Vector \&load, bool
isLoadConstant = false);}  \\
To create a NodalLoad object with tag {\em tag} acting on Node {\em node} with a
reference load given by {\em load}. {\em tag} and {\em LOAD\_TAG\_NodalLoad} (defined in
$<$classTags.h$>$)are passed to the Load constructor.  A new vector object is created using
the vector {\em load} as the argument. (THIS MAY CHANGE - may associate
the load Vector with {\em load}, allowing identical loads on diff
nodes to be created without the need for each to have its own
vector) If no enough memory is available an error message is printed
and the program terminates. The boolean {\em isLoadConstant} is used
to indicate whether the value of the load applies to the Node is
independent of the load factor. \\ 

\indent {\em NodalLoad(int tag, int node, int classTag);}  \\
Provided for subclasses, which may which to provide the abstraction of
time varying nodal loads. The integers {\em tag} and {\em classTag}
are passed to the Load constructor. \\

\indent {\em NodalLoad(int classTag);}  \\
Provided so that a FEM\_ObjectBroker can construct an object. $0$ and
{\em classTag} are passed to the Load classes constructor. The data
for the object is filled in when {\em recvSelf()} is invoked on the
object.\\

\noindent {\bf Destructor}  \\
\indent {\em virtual $\tilde{ }$ NodalLoad();} \\ 
If a Vector for the load was constructed, the destructor invokes
delete on this Vector object. \\


\noindent {\bf Public Methods}  \\
\indent {\em virtual void setDomain(Domain *newDomain);} \\
To set the associated Domain object. First checks to ensure that a
Node object with the tag exists in the Domain. It sets the pointer for
it's associated Node object to point to this object, and then 
invokes the DomainComponent classes {\em setDomain()} method. If the
Node does not exist in the Domain a warning message and {\em
setDomain()} is not invoked. \\

\indent {\em virtual int getNodeTag(void) const;} \\
Returns the integer {\em node} passed in the constructor. \\

\indent {\em virtual void applyLoad(double timeStep = 0.0, double
loadFactor = 1.0);} \\
To it's associated Node it invokes {\em addUnbalancedLoad()} with it's
copy of the Vector object {\em load} and a factor of {\em loadFactor}
if {\em isLoadConstant} was specified as {\em false} in the constructor or
$1$ if it was specified as {\em true}. Warning
messages are printed, if no Domain has been associated with the
NodalLoad object or no Node with a tag {\em node} exists in the
Domain. For efficiency reasons, the NodalLoad object keeps a pointer
to it's associated Node object. The time {\em timeStep} has no
influence on the load added. \\

\indent {\em virtual int sendSelf(int commitTag, Channel \&theChannel); } \\
Determines its database tag. The object then sends it's tag, {\em
node} and size of load Vector to the Channel in an ID object. Then, if
{\em load} is not NULL, it sends it's copy of the {\em load}
Vector. Returns $0$ if successful, a negative number if the Channel
failed to send the data. \\  

\indent {\em virtual int recvSelf(int commitTag, Channel \&theChannel,
FEM\_ObjectBroker \&theBroker); } \\
The object first determines its database tag. It then invokes receives
an ID object from the Channel containing it's tag, {\em node} and size
of load Vector. If size of {\em load} is not $0$ it then receives it's
copy of the {\em load} Vector. Returns $0$ if successful, a negative 
number if the Channel failed to receive the data. \\ 

\indent {\em virtual void Print(ostream \&s, int flag = 0);} \\
Prints it's {\em node} and then prints the load Vector. \\



\pagebreak
\subsubsection{\bf ElementalLoad}
%File: ~/OOP/element/ElementalLoad.tex
%What: "@(#) ElementalLoad.tex, revA"

\noindent {\bf Files}   \\
\indent \#include $<\tilde{ }$/element/ElementalLoad.h$>$  \\

\noindent {\bf Class Declaration}  \\
\indent class ElementalLoad: public Load \\

\noindent {\bf Class Hierarchy} \\
\indent TaggedObject \\
\indent MovableObject \\
\indent\indent DomainComponent \\
\indent\indent\indent Load \\
\indent\indent\indent\indent {\bf ElementalLoad} \\

\noindent {\bf Description}  \\
\indent ElementalLoad is an abstract class, i.e. no instances of
ElementalLoad will exist. The ElementalLoad class provides the
interface that all ElementalLoad writers must provide when
introducing new ElementalLoad classes. \\ 

\noindent {\bf Class Interface} \\
\indent\indent // Constructors \\
\indent\indent {\em ElementalLoad(int elementTag, int tag, int classTag);}  \\ 
\indent\indent {\em ElementalLoad(int classTag);}  \\ \\
\indent\indent // Destructor \\
\indent\indent {\em virtual~ $\tilde{}$ElementalLoad();}\\ \\
\indent\indent // Public Methods  \\
\indent\indent {\em virtual int getElementTag(void) const;} \\



\noindent {\bf Constructor}  \\
\indent {\em ElementalLoad(int elementTag, int tag, int classTag);}  \\
Provided to allow subclasses to construct an ElementalLoad object
associated with the Element whose unique identifier in the Domain will
be {\em elementTag}. The integers {\em tag} and and {\em classTags}
are passed to the Load constructor. \\ 

\indent {\em ElementalLoad(int classTag);}  \\
Provided so that a FEM\_ObjectBroker can construct an object. $0$ and
{\em classTag} are passed to the Load classes constructor. The data
for the object is filled in when {\em recvSelf()} is invoked on the
object.\\

\noindent {\bf Destructor} \\
\indent {\em virtual~ $\tilde{}$ElementalLoad();}\\ 
Does nothing. Provided so that the ElementalLoad subclasses destructor
will be called. \\

\noindent {\bf Public Methods }  \\
\indent\indent {\em virtual int getElementTag(void) const;} \\
Returns the integer {\em elementTag} passed in the constructor. 


\pagebreak
\subsection{SP\_Constraint}
% File: ~/domain/constraints/SP\_Constraint.tex 

\noindent {\bf Files}   \\
\indent \#include $<\tilde{ }$/domain/constraints/SP\_Constraint.h$>$  \\

\noindent {\bf Class Declaration}  \\
\indent class SP\_Constraint: public DomainComponent \\

\noindent {\bf Class Hierarchy} \\
\indent TaggedObject \\
\indent MovableObject \\
\indent\indent DomainComponent \\
\indent\indent\indent {\bf SP\_Constraint} \\

\noindent {\bf Description}  \\
\indent An SP\_Constraint represents a single point constraint in the
domain. A single point constraint specifies the response of a particular
degree-of-freedom at a node. The declaration that all methods are
virtual allows for time varying constraints to be introduced. \\

\noindent {\bf Class Interface}  \\
\indent\indent // Constructors  \\
\indent\indent {\em SP\_Constraint(int tag, int nodeTag, int ndof,
double value, bool isConstant = true);} \\
\indent\indent {\em SP\_Constraint(int tag, int nodeTag, int ndof, int classTag);}\\ 
\indent\indent {\em SP\_Constraint(int classTag);} \\ \\
\indent\indent // Destructor  \\
\indent\indent {\em virtual $\tilde{ }$ SP\_Constraint();} \\ \\
\indent\indent // Public Methods  \\
\indent\indent {\em virtual int getNodeTag(void) const;} \\
\indent\indent {\em virtual int getDOF\_Number(void) const;} \\
\indent\indent {\em virtual int applyConstraint(double loadFactor)
const;} \\ 
\indent\indent {\em virtual double getValue(void) const;} \\
\indent\indent {\em virtual bool isHomogeneous(void) const;}\\ 
\indent\indent {\em virtual void setLoadPatternTag(int loadPaternTag);}\\
\indent\indent {\em virtual int  getLoadPatternTag(void) const;}\\ \\
\indent\indent // Public Methods for Output \\
\indent\indent {\em virtual int sendSelf(int commitTag, Channel \&theChannel);} \\ 
\indent\indent {\em virtual int recvSelf(int commitTag, Channel \&theChannel,
FEM\_ObjectBroker \&theBroker);} \\ 
\indent\indent {\em virtual void Print(OPS_Stream \&s, int flag = 0);} \\


\noindent {\bf Constructors}  \\
\indent {\em SP\_Constraint(int tag, int nodeTag, int ndof, double value);} \\
To construct a single point constraint to constrain the trial
displacement of the {\em ndof}'th dof at node {\em node} to the value
given by {\em value}. The integer value {\em tag} is used to identify
the SP\_Constraint among all other SP\_Constraints. If
{\em value} is equal to $0.0$ the method {\em isHomogeneous()} will
always return {\em true}, otherwise {\em false}. \\ 

\indent {\em SP\_Constraint(int tag, int node, int ndof, int classTag);} \\
Provided for subclasses to use. The subclasses can vary the value of the
imposed displacement when {\em getValue()} is invoked. If this
constructor is used the {\em isHomogeneous()} method will always
return {\em false}. The integer value {\em tag} is used to identify
the SP\_Constraint among all other SP\_Constraints. \\


\indent {\em SP\_Constraint(int classTag);} \\
Provided for the FEM\_ObjectBroker to be able to instantiate an
object; the data for this object will be read from a Channel object
when {\em recvSelf()} is invoked. $0$ and {\em classTag} are passed to
the DomainComponent constructor. \\


\noindent {\bf Destructor}  \\
\indent {\em virtual $\tilde{ }$ SP\_Constraint();} \\
Does nothing. Provided so that a subclasses destructor can be
invoked. \\


\noindent {\bf Public Methods }  \\
\indent {\em virtual int getNodeTag(void) const;} \\
Returns the value of {\em node} passed in the constructor, this should be 
the tag of the node that is being constrained. \\

{\em virtual int getDOF\_Number(void) const;} \\
Returns the value of {\em ndof} that was passed in the constructor,
this identifies the dof number corresponding to the constraint. \\

\indent {\em virtual int applyConstraint(double loadFactor);}\\ 
To set the value of the constraint for the load factor given by {\em
loadFactor}. The constraint is set equal to {\em loadFactor} * {\em
value} if the constraint is not constant, or {\em value} if the
constraint was identified as constant in the constructor. \\

\indent {\em virtual bool isHomogeneous(void) const;}\\
To return a boolean indicating whether or not the constraint is a
homogeneous constraint. A homogeneous constraint is one where the value
of the constraint, {\em value}, is always $0$. This information can be used by the
ConstraintHandler to reduce the number of equations in the system. \\

{\em virtual double getValue(void) const;} \\
To return the value of the constraint determined in the last call to
{\em applyConstraint()}. This base class returns {\em value} passed in
the constructor.  \\ 

\indent {\em virtual void setLoadPatternTag(int loadPaternTag);}\\
To set the LoadPattern tag associated with the object to be {\em
loadPatternTag}. \\

\indent {\em virtual int  getLoadPatternTag(void) const;}\\ 
To return the load pattern tag associated with the load. \\

{\em virtual int sendSelf(int commitTag, Channel \&theChannel);} \\ 
Creates a Vector, and stores the SP\_Constraints tag, nodeTag, ndof and value in
the Vector. It then passes the Vector as an argument to {\em
theChannel} objects {\em sendVector()} method, along with the objects 
database tag and {\em commitTag}. Subclasses must invoke this method
in their implementation of {\em sendSelf()}, so that the {\em node}
and {\em ndof} values in remote object can be set. Returns $0$ if
successful, a negative number if the Channel object, {\em theChannel},
failed to send the data. \\ 

{\em virtual int recvSelf(int commitTag, Channel \&theChannel, FEM\_ObjectBroker
\&theBroker);} \\ 
Creates a Vector, and receives the Vector from the channel object
using the {\em recvVector()} method call and the objects own database
tag and {\em commitTag}. Using the information contained in the Vector, the 
SP\_Constraints tag, nodeTag, ndof and value are set. Subclasses must
invoke this method in their implementation of {\em recvSelf()}, so
that the {\em node} and {\em ndof} values can be set. Returns $0$ if
successful, a negative number if the Channel object, {\em
theChannel}, failed to receive the data. \\   

{\em virtual void Print(OPS_Stream \&s, int flag = 0) const;} \\
Prints out the SP\_Constraints tag, then {\em node}, {\em ndof} and
{\em value}. 



\pagebreak
\subsection{MP\_Constraint}
% File: ~/domain/constraints/MP\_Constraint.tex 

\noindent {\bf Files}   \\
\indent \#include $<\tilde{ }$/domain/constraints/MP\_Constraint.h$>$  \\

\noindent {\bf Class Declaration}  \\
\indent class MP\_Constraint: public DomainComponent  \\

\noindent {\bf Class Hierarchy} \\
\indent TaggedObject \\
\indent MovableObject \\
\indent\indent DomainComponent \\
\indent\indent\indent {\bf MP\_Constraint} \\

\noindent {\bf Description}  \\
\indent An MP\_Constraint represents a multiple point constraint in the
domain. A multiple point constraint imposes a relationship between the
displacement for certain dof at two nodes in the model, typically called
the {\em retained} node and the {\em constrained} node: $U_c = C_{cr} U_r$


An MP\_Constraint is responsible for providing information on the
relationship between the dof, this is in the form of a constraint
Matrix, $C_{cr}$, and two ID objects, {\em retainedID} and {\em
constrainedID} indicating the dof's at the nodes 
represented by $C_{cr}$. For example, for the following constraint
imposing a relationship between the displacements at node $1$, the 
constrained node, with the displacements at node $2$, the retained
node in a problem where the x,y,z components are identified as the
0,1,2 degrees-of-freedom:

$$ u_{1,x} = 2 u_{2,x} + u_{2,z} $$
$$ u_{1,y} = 3 u_{2,z}$$

the constraint matrix is:
$$ C_{cr} =
\left[
\begin{array}{cc}
2 & 1  \\
0 & 3  \\
\end{array}
\right] 
$$

{\em constrainedID} = $[0$ $1]$ and {\em retainedID} $= [0$ $2]$. \\

\noindent {\bf Class Interface}  \\
\indent\indent  // Constructors  \\
\indent\indent {\em MP\_Constraint(int tag, int nodeRetain, int nodeConstr,
Matrix \&constraint, \\
\indent\indent\indent\indent\indent\indent ID \&constrainedDOF, ID
\&retainedDOF);}\\ 
\indent\indent {\em MP\_Constraint(int tag, int nodeRetain, int nodeConstr, \\
\indent\indent\indent\indent\indent\indent ID \&constrainedDOF, ID
\&retainedDOF, int classTag);}\\
\indent\indent {\em MP\_Constraint(int classTag);}\\ \\
\indent\indent  // Destructor  \\
\indent\indent {\em virtual $\tilde{ }$ MP\_Constraint();} \\ \\
\indent\indent // Public Methods  \\
\indent\indent {\em virtual int getNodeRetained(void) const; } \\
\indent\indent {\em virtual int getNodeConstrained(void) const; }\\
\indent\indent {\em virtual const ID \& getConstrainedDOFs(void) const;}\\
\indent\indent {\em virtual const ID \& getRetainedDOFs(void) const;}\\
\indent\indent {\em virtual int applyConstraint(double timeStamp);} \\
\indent\indent {\em virtual bool isTimeVarying(void) const;} \\
\indent\indent {\em virtual const Matrix \&getConstraint(void) const;}\\\\
\indent\indent // Public Methods for Output \\
\indent\indent {\em virtual int sendSelf(int commitTag, Channel \&theChannel);} \\ 
\indent\indent {\em virtual int recvSelf(int commitTag, Channel \&theChannel, FEM\_ObjectBroker
\&theBroker);} \\ 
\indent\indent {\em virtual void Print(ostream \&s, int flag = 0);} \\

\noindent {\bf Constructors}  \\
\indent {\em MP\_Constraint(int tag, int nodeRetain, int nodeConstr,
Matrix \&constraint, \\
\indent\indent\indent\indent\indent\indent
 ID \&constrainedDOF, ID \&retainedDOF);}\\ 
To construct a multiple point constraint where the constrained node is
given by {\em nodeConstr}, the retained node by {\em nodeRetain}, the
{\em constrainedID} by {\em constrainedDOF}, the {\em retainedID} by
{\em retainedDOF} and $C_{cr}$ by {\em constraint}. 
The integers {\em tag} and  CNSTRNT\_TAG\_MP\_Constraint
are passed to the DomainComponent classes constructor. New Matrix and ID objects are
created to hold the information. \\

\indent {\em MP\_Constraint(int tag, int nodeRetain, int nodeConstr, \\
\indent\indent\indent\indent\indent\indent ID \&constrainedDOF, ID
\&retainedDOF, int classTag);}\\
For the subclasses to use. The subclasses can vary the contents of the
Matrix returned when {\em getConstraint()} is invoked. 
The integers {\em tag} {\em classTag} are
passed to the DomainComponent classes constructor. New ID objects are
created to hold the information. \\

\indent {\em MP\_Constraint(int classTag);} \\
Provided for the FEM\_ObjectBroker to construct a blank object. The
data for the object is filled in when {\em recvSelf()} is invoked on
the object. $0$ and {\em classTag} are passed to the DomainComponent
constructor. \\ 

\noindent {\bf Destructor}  \\
\indent {\em virtual~$\tilde{}$ MP\_Constraint();} \\
Invokes the destructor on both the ID and the Matrix object, if a
Matrix object is passed in the constructor. \\

\noindent {\bf Public Methods}  \\
\indent {\em virtual int getNodeRetained(void) const; } \\
Returns the value of {\em nodeRetain} passed in the constructor,
i.e. the tag of the retained node. \\

{\em virtual int getNodeConstrained(void) const;    }\\
Returns the value of {\em nodeConstr} passed in the constructor, i.e. the
tag of the constrained node. \\

{\em virtual const ID \&getConstrainedDOFs(void) const;   }\\     
Returns, as a const, the {\em constrainedID} formed in the constructor. \\

{\em virtual const ID \&getRetainedDOFs(void) const;   }\\     
Returns, as a const, the {\em retainedID} formed in the
constructor. \\

{\em virtual int applyConstraint(double timeStamp)}\\     
A method to invoked to inform the MP\_Constraint to determine $C_{cr}$,
for the time {\em timeStamp}. {\bf The base class will do nothing, as
Matrix is assumed to be constant.} \\

{\em virtual const Matrix \&getConstraint(void) const;} \\
Returns the current constraint Matrix, that determined in the last
call to {\em applyConstraint()}. For the MP\_Constraint class, $C_{cr}$
determined in the constructor is returned. \\

{\em virtual int sendSelf(int commitTag, Channel \&theChannel);} \\
Creates a Vector, stores the MP\_Constraints tag, nodeRetain,
nodeConstrained and value in the Vector, and sends the Vector to the
Channel using the objects own database tag and {\em commitTag}. It then
sends the {\em participatingDOF} ID and the {\em constraint}
Matrix, again using the objects database tag and {\em
commitTag}. Returns $0$ if successful, a negative number if the
Channel object, {\em theChannel}, failed to send the data. \\ 

{\em virtual int recvSelf(int commitTag, Channel \&theChannel,
FEM\_ObjectBroker \&theBroker);} \\ 
Creates a Vector, receives the Vector from the Channel using {\em
commitTag} and the objects database tag, and sets theMP\_Constraints
tag, nodeRetain, nodeConstrained from the the Vector. Creates a
Vector and a Matrix, and then receives the {\em participatingDOF} ID
and the {\em constraint} Matrix into these objects. Returns $0$ if
successful, a negative number if the Channel object, {\em
theChannel}, failed to receive the data.\\ 

{\em virtual void Print(ostream \&s, int flag = 0);} \\
Prints out the MP\_Constraints tag, then the tags of the constrained
and retained nodes, then the two ID's and finally the constraint Matrix.\\








\pagebreak
\subsection {\bf TimeSeries}
%File: ~/OOP/domain/pattern/TimeSeries.tex
%What: "@(#) TimeSeries.tex, revA"

\noindent {\bf Files}   \\
\indent \#include $<\tilde{ }$domain/pattern/TimeSeries.h$>$  \\

\noindent {\bf Class Declaration}  \\
\indent class TimeSeries: public DomainComponent  \\

\noindent {\bf Class Hierarchy} \\
\indent MovableObject \\
\indent\indent {\bf TimeSeries} \\

\noindent {\bf Description} \\ 
\indent The TimeSeries class is an abstract base class. A
TimeSeries object is used in a LoadPattern to determine the current
load factor to be applied to the loads and constraints for the time
specified. \\ 

\noindent {\bf Class Interface} \\
\indent // Constructor \\ 
\indent {\em TimeSeries(int classTag);}\\ \\
\indent // Destructor \\ 
\indent {\em virtual $\tilde{ }$TimeSeries();}\\  \\
\indent // Pure Virtual Public Methods \\ 
\indent {\em  virtual double getFactor(double pseudoTime) =0;}\\
\indent {\em  virtual void Print(ostream \&s, int flag =0) =0;}\\

\noindent {\bf Constructor} \\ 
\indent {\em TimeSeries(int tag);}\\ 
The integer {\em classTag} is passed to the MovableObject classes
constructor. \\

\noindent {\bf Destructor} \\
\indent {\em virtual $\tilde{ }$TimeSeries();}\\ 
Does nothing. \\

\noindent {\bf Public Methods} \\
\indent {\em  virtual double getFactor(double pseudoTime) =0;}\\
To return the current load factor for the given value of {\em
pseudoTime} to be applied to the loads and single-point constraints in
a LoadPattern based on the value of {\em pseudoTime}. \\

\indent {\em  virtual void Print(ostream \&s, int flag =0) =0;}\\
To print to the stream {\em s} output based on the value of {\em flag}.

\pagebreak
\subsubsection {LinearSeries}
%File: ~/OOP/domain/pattern/LinearSeries.tex
%What: "@(#) LinearSeries.tex, revA"

\noindent {\bf Files}   \\
\indent \#include $<\tilde{ }$domain/pattern/LinearSeries.h$>$  \\

\noindent {\bf Class Declaration}  \\
\indent class LinearSeries: public DomainComponent  \\

\noindent {\bf Class Hierarchy} \\
\indent MovableObject \\
\indent\indent TimeSeries \\
\indent\indent\indent {\bf LinearSeries} \\

\noindent {\bf Description} \\ 
\indent The LinearSeries class is a concrete subclass of TimeSeries.
The relationship between the pseudo time and the load factor is linear
for objects of this class. \\


\noindent {\bf Class Interface} \\
\indent // Constructor \\ 
\indent {\em LinearSeries(double factor = 1.0);}\\ \\
\indent // Destructor \\ 
\indent {\em virtual $\tilde{ }$LinearSeries();}\\  \\
\indent // Public Methods \\ 
\indent {\em  virtual double getFactor(double pseudoTime);}\\
\indent {\em  virtual int sendSelf(int commitTag, Channel \&theChannel);}\\
\indent {\em  virtual int recvSelf(int commitTag, Channel \&theChannel,
FEM\_ObjectBroker \&theBroker);}\\
\indent {\em  virtual void Print(OPS_Stream \&s, int flag =0);}\\

\noindent {\bf Constructor} \\ 
\indent {\em LinearSeries(double cFactor = 1.0);}\\ 
The tag TSERIES\_TAG\_LinearSeries is passed to the TimeSeries
constructor. Sets the constant factor used in the relation to {\em
cFactor}. \\

\noindent {\bf Destructor} \\
\indent {\em virtual $\tilde{ }$LinearSeries();}\\ 
Does nothing. \\

\noindent {\bf Public Methods} \\
\indent {\em  virtual double getFactor(double pseudoTime);}\\
Returns the product of {\em cFactor} and {\em pseudoTime}. \\

\indent {\em  virtual int sendSelf(int commitTag, Channel
\&theChannel);}\\
Creates a vector of size 1 into which it places {\em cFactor} and
invokes {\em sendVector()} on the Channel object. \\

\indent {\em  virtual int recvSelf(int commitTag, Channel \&theChannel,
FEM\_ObjectBroker \&theBroker);}\\
Does the mirror image of {\em sendSelf()}. \\

\indent {\em  virtual void Print(OPS_Stream \&s, int flag =0) =0;}\\
Prints the string 'LinearSeries' and the factor{\em cFactor}.

\pagebreak
\subsubsection {RectangularSeries}
%File: ~/OOP/domain/pattern/RectangularSeries.tex
%What: "@(#) RectangularSeries.tex, revA"

\noindent {\bf Files}   \\
\indent \#include $<\tilde{ }$domain/pattern/RectangularSeries.h$>$  \\

\noindent {\bf Class Declaration}  \\
\indent class RectangularSeries: public DomainComponent  \\

\noindent {\bf Class Hierarchy} \\
\indent MovableObject \\
\indent\indent TimeSeries \\
\indent\indent\indent {\bf RectangularSeries} \\

\noindent {\bf Description} \\ 
\indent The RectangularSeries class is a concrete subclass of TimeSeries.
The relationship between the pseudo time and the load factor is
defined by the simple relationship: factor $=$ cFactor when tStart $<=$
pseudo time $<=$ tFinish, otherwise factor $ =0.0$. \\


\noindent {\bf Class Interface} \\
\indent // Constructor \\ 
\indent {\em RectangularSeries(double tStart, double tFinish, double factor = 1.0);}\\ \\
\indent // Destructor \\ 
\indent {\em virtual $\tilde{ }$RectangularSeries();}\\  \\
\indent // Public Methods \\ 
\indent {\em  virtual double getFactor(double pseudoTime);}\\
\indent {\em  virtual int sendSelf(int commitTag, Channel \&theChannel);}\\
\indent {\em  virtual int recvSelf(int commitTag, Channel \&theChannel,
FEM\_ObjectBroker \&theBroker);}\\
\indent {\em  virtual void Print(OPS_Stream \&s, int flag =0);}\\

\noindent {\bf Constructor} \\ 
\indent {\em RectangularSeries(double tStart, double tFinish, double cFactor = 1.0);}\\ 
The tag TSERIES\_TAG\_RectangularSeries is passed to the TimeSeries
constructor. Saves the values {\em tStart}, {\em tFinish} and {\em
cfactor}. \\


\noindent {\bf Destructor} \\
\indent {\em virtual $\tilde{ }$RectangularSeries();}\\ 
Does nothing. \\

\noindent {\bf Public Methods} \\
\indent {\em  virtual double getFactor(double pseudoTime);}\\
Returns cFactor if tStart $<=$ pseudo time $<=$ tFinish, otherwise
returns $0.0$. \\ 

\indent {\em  virtual int sendSelf(int commitTag, Channel
\&theChannel);}\\
Creates a vector of size 3 into which it places {\em tStart}, {\em
tFinish} and {\em cFactor}, and it then invokes {\em sendVector()} on
the Channel object. \\ 

\indent {\em  virtual int recvSelf(int commitTag, Channel \&theChannel,
FEM\_ObjectBroker \&theBroker);}\\
Does the mirror image of {\em sendSelf()}. \\

\indent {\em  virtual void Print(OPS_Stream \&s, int flag =0) =0;}\\
Prints the string 'RectangularSeries' and the values {\em tStart}, {\em
tFinish} and {\em factor}. 

\pagebreak
\subsubsection {PathSeries}
%File: ~/OOP/domain/pattern/PathSeries.tex
%What: "@(#) PathSeries.tex, revA"

\noindent {\bf Files}   \\
\indent \#include $<\tilde{ }$domain/pattern/PathSeries.h$>$  \\

\noindent {\bf Class Declaration}  \\
\indent class PathSeries: public DomainComponent  \\

\noindent {\bf Class Hierarchy} \\
\indent MovableObject \\
\indent\indent TimeSeries \\
\indent\indent\indent {\bf PathSeries} \\

\noindent {\bf Description} \\ 
\indent The PathSeries class is a concrete subclass of TimeSeries.
The relationship between the pseudo time and the load factor follows
a user specified path. The path points are specified at constant time
intervals. For a pseudo time not at a path point, linear interpolation
is performed to determine the load factor. If the time specified is
beyond the last path point a load factor of $0.0$ will be returned.\\

\noindent {\bf Class Interface} \\
\indent // Constructors \\ 
\indent {\em PathSeries(Vector \&thePath, double dT; double cFactor);}\\ 
\indent {\em PathSeries(char *fileName, double dT; double cFactor);}\\ 
\indent {\em PathSeries();}\\ \\
\indent // Destructor \\ 
\indent {\em virtual $\tilde{ }$PathSeries();}\\  \\
\indent // Public Methods \\ 
\indent {\em  virtual double getFactor(double pseudoTime);}\\
\indent {\em  virtual int sendSelf(int commitTag, Channel \&theChannel);}\\
\indent {\em  virtual int recvSelf(int commitTag, Channel \&theChannel,
FEM\_ObjectBroker \&theBroker);}\\
\indent {\em  virtual void Print(ostream \&s, int flag =0);}\\

\noindent {\bf Constructor} \\ 
\indent {\em PathSeries(Vector \&thePath, double dT; double cFactor);}\\ 
Used to construct a PathSeries when the data points are specified in a
Vector. The tag TSERIES\_TAG\_PathSeries is passed to the TimeSeries
The tag TSERIES\_TAG\_PathSeries is passed to the TimeSeries
constructor. Sets the constant factor used in the relation to {\em
cFactor}. Constructs a new Vector equal to {\em thePath} containing the
data points which are specified at {\em dT} time intervals. Prints a
warning message if no space is available for the Vector.\\

\indent {\em PathSeries(char *fileName, double dT; double cFactor);}\\ 
Used to construct a PathSeries when the data points are specified in a
file. The tag TSERIES\_TAG\_PathSeries is passed to the TimeSeries
constructor. Sets the constant factor used in the relation to {\em
cFactor}. Opens the file containing and reads in the data points into
a new Vector which are specified at {\em dT} time intervals. Prints a
warning message if no space is available for the Vector or if the file
could not be found.\\


\indent {\em PathSeries();}\\ 
For a FEM\_ObjectBroker to instantiate an empty PathSeries, recvSelf()
must be invoked on this object. The tag TSERIES\_TAG\_PathSeries is
passed to the TimeSeries constructor. \\


\noindent {\bf Destructor} \\
\indent {\em virtual $\tilde{ }$PathSeries();}\\ 
Invokes the destructor on the Vector created to hold the data
points. \\


\noindent {\bf Public Methods} \\
\indent {\em  virtual double getFactor(double pseudoTime);}\\
Determines the load factor based on the {\em pseudoTime} and the data
points. Returns $0.0$ if {\em pseudoTime} is greater than time of last
data point, otherwise returns a linear interpolation of the data
points times the factor {\em cFactor}. \\

\indent {\em  virtual int sendSelf(int commitTag, Channel
\&theChannel);}\\
Creates a vector of size 4 into which it places {\em cFactor}, {\em
dT}, the size of {\em thePath} and another database tag for {\em
thaPath} Vector.  Invokes {\em sendVector()} on the Channel with this
newly created Vector object, and the {\em sendVEctor()} on {\em
thePath}. \\

\indent {\em  virtual int recvSelf(int commitTag, Channel \&theChannel,
FEM\_ObjectBroker \&theBroker);}\\
Does the mirror image of {\em sendSelf()}. \\

\indent {\em  virtual void Print(ostream \&s, int flag =0) =0;}\\
Prints the string 'PathSeries', the factor{\em cFactor}, and the time
increment {\em dT}. If {\em flag} is equal to $1.0$ the load path
Vector is also printed.




\pagebreak
\subsubsection {PathTimeSeries}
%File: ~/OOP/domain/pattern/PathTimeSeries.tex
%What: "@(#) PathTimeSeries.tex, revA"

\noindent {\bf Files}   \\
\indent \#include $<\tilde{ }$domain/pattern/PathTimeSeries.h$>$  \\

\noindent {\bf Class Declaration}  \\
\indent class PathTimeSeries: public DomainComponent  \\

\noindent {\bf Class Hierarchy} \\
\indent MovableObject \\
\indent\indent TimeSeries \\
\indent\indent\indent {\bf PathTimeSeries} \\

\noindent {\bf Description} \\ 
\indent The PathTimeSeries class is a concrete subclass of TimeSeries.
The relationship between the pseudo time and the load factor follows
a user specified path. The path points are specified at user specified
time values. For a pseudo time not at a path point, linear interpolation
is performed to determine the load factor. If the time specified is
beyond the last path point a load factor of $0.0$ will be returned.\\

\noindent {\bf Class Interface} \\
\indent // Constructors \\ 
\indent {\em PathTimeSeries(Vector \&thePath, Vector \&time; double cFactor);}\\ 
\indent {\em PathTimeSeries(char *fileName, Vector \&time; double cFactor);}\\ 
\indent {\em PathTimeSeries();}\\ \\
\indent // Destructor \\ 
\indent {\em virtual $\tilde{ }$PathTimeSeries();}\\  \\
\indent // Public Methods \\ 
\indent {\em  virtual double getFactor(double pseudoTime);}\\
\indent {\em  virtual int sendSelf(int commitTag, Channel \&theChannel);}\\
\indent {\em  virtual int recvSelf(int commitTag, Channel \&theChannel,
FEM\_ObjectBroker \&theBroker);}\\
\indent {\em  virtual void Print(ostream \&s, int flag =0);}\\

\noindent {\bf Constructor} \\ 
\indent {\em PathTimeSeries(Vector \&thePath, Vector \&time; 
double cFactor);}\\ 
Used to construct a PathTimeSeries when the data points and time
values are specified in a Vectors. The tag
TSERIES\_TAG\_PathTimeSeries is passed to the TimeSeries 
constructor. Sets the constant factor used in the relation to {\em
cFactor}. Constructs two new Vectors equal to {\em thePath} and {\em
time}. Prints a warning message if no space is available for the
Vectors or if the two Vectors are not of the same size.\\ 

\indent {\em PathTimeSeries(char *filePath, file *timeName; double
cFactor);}\\  
Used to construct a PathTimeSeries when the data points and time
values are specified in files. The tag TSERIES\_TAG\_PathTimeSeries is
passed to the TimeSeries constructor. Sets the constant factor used in
the relation to {\em cFactor}. Opens the two files and counts the
number of entries in each, if different prints a warning message and
returns. Constructs the two vectors for the data and reads the data
from the files into the two vectors. Prints a warning message if no
space is available for the Vectors.\\

\indent {\em PathTimeSeries();}\\ \\
For a FEM\_ObjectBroker to instantiate an empty PathTimeSeries, recvSelf()
must be invoked on this object. The tag TSERIES\_TAG\_PathTimeSeries is
passed to the TimeSeries constructor. \\


\noindent {\bf Destructor} \\
\indent {\em virtual $\tilde{ }$PathTimeSeries();}\\ 
Invokes the destructor on the two Vectors created to hold the data. \\

\noindent {\bf Public Methods} \\
\indent {\em  virtual double getFactor(double pseudoTime);}\\
Determines the load factor based on the {\em pseudoTime} and the data
points. Returns $0.0$ if {\em pseudoTime} is greater than time of last
data point, otherwise returns a linear interpolation of the data
points times the factor {\em cFactor}. \\

\indent {\em  virtual int sendSelf(int commitTag, Channel
\&theChannel);}\\
Creates a vector of size 4 into which it places {\em cFactor}, the
size of {\em thePath} and two  database tag for {\em 
thaPath} and {\em time} Vectors.  Invokes {\em sendVector()} on the
Channel with this newly created Vector object, and then again with the
{\em time} and {\em thePath} Vectors.\\

\indent {\em  virtual int recvSelf(int commitTag, Channel \&theChannel,
FEM\_ObjectBroker \&theBroker);}\\
Does the mirror image of {\em sendSelf()}. \\

\indent {\em  virtual void Print(ostream \&s, int flag =0) =0;}\\
Prints the string 'PathTimeSeries', the factor{\em cFactor}, and the time
increment {\em dT}. If {\em flag} is equal to $1.0$ {\em
thePath} and {\em time} Vector are also printed\\

\pagebreak
\subsection {LoadPattern}
%File: ~/OOP/domain/pattern/LoadPattern.tex
%What: "@(#) LoadPattern.tex, revA"

\noindent {\bf Files}   \\
\indent \#include $<\tilde{ }$domain/pattern/LoadPattern.h$>$  \\

\noindent {\bf Class Declaration}  \\
\indent class LoadPattern: public DomainComponent  \\

\noindent {\bf Class Hierarchy} \\
\indent TaggedObject \\
\indent MovableObject \\
\indent\indent DomainComponent \\
\indent\indent\indent {\bf LoadPattern} \\

\noindent {\bf Description} \\ 
\indent The LoadPattern class is a concrete base class. A
LoadPattern is a container class for Load and SP\_Constraint
objects. Each LoadPattern object is assocaited with a TimeSeries
object which, for a given pseudo time, will return the appropriate
load factor to be applied to th load in the LoadPattern. \\

\noindent {\bf Class Interface} \\
\indent // Constructors \\ 
\indent {\em LoadPattern(int tag);}\\ 
\indent {\em LoadPattern(int tag, int classTag);}\\ \\
\indent // Destructor \\ 
\indent {\em virtual $\tilde{ }$LoadPattern();}\\  \\
\indent // Public Methods \\ 
\indent {\em virtual void setTimeSeries(TimeSeries *theSeries);}\\
\indent {\em virtual void setDomain(Domain *theDomain);}\\ \\
\indent {\em  // Public Methods to add loads}\\
\indent {\em  virtual bool addNodalLoad(NodalLoad *);}\\
\indent {\em  virtual bool addElementalLoad(ElementalLoad *);}\\
\indent {\em  virtual bool addSP\_Constraint(SP\_Constraint *);}\\ 
\indent {\em  virtual NodalLoadIter     \&getNodalLoads(void);}\\
\indent {\em  virtual ElementalLoadIter \&getElementalLoads(void);}\\
\indent {\em  virtual SP\_ConstraintIter \&getSPs(void);}\\ \\
\indent {\em  // Public Methods to remove loads}\\
\indent {\em  virtual NodalLoad *removeNodalLoad(int tag);}\\
\indent {\em  virtual ElementalLoad *removeElementalLoad(int tag);}\\
\indent {\em  virtual SP\_Constraint *removeSP\_Constraint(int tag);}\\ \\
\indent {\em  // Public Methods to apply loads}\\
\indent {\em  virtual void applyLoad(double pseudoTime = 0.0);}\\
\indent {\em  virtual void setLoadConstant(void);}\\ \\
\indent {\em  // Public Methods for o/p}\\
\indent {\em  virtual int sendSelf(int commitTag, Channel \&theChannel);}\\
\indent {\em  virtual int recvSelf(int commitTag, Channel \&theChannel,
FEM\_ObjectBroker \&theBroker);}\\
\indent {\em  virtual void Print(ostream \&s, int flag =0);}\\


\noindent {\bf Constructor} \\ 
\indent {\em LoadPattern(int tag);}\\ 
The integer {\em tag} is passed to the DomainComponent classes
constructor. Creates three ArrayOftaggedObjects objects to store
pointers to the NodalLoad, ElementalLoad and SP\_Constraints and three
iters. If not enough memory is available for these objects an error
message is printed and the program is terminated. \\  

\indent {\em LoadPattern(int tag, classTag);}\\ 
This is the constructor provided for subclasses to use. The integers
{\em tag} and {\em classTag} are passed to the DomainComponent classes
constructor. \\ 

\noindent {\bf Destructor} \\
\indent {\em virtual $\tilde{ }$LoadPattern();}\\ 
Invokes the destructor on the TimeSeries object. Also invokes the
destructor on any objects created in the constructor for storage of
the pointers and for iters. It does not invoke the destructor on these
objects, the Domain object is responsible for doing this.\\

\noindent {\bf Public Methods} \\
\indent {\em virtual void setTimeSeries(TimeSeries *theSeries);}\\
If a TimeSeries object is associated with the pattern, the destructor
is invoked on that TimeSeries object. It then sets the TimeSeries object
associated with the LoadPattern to {\em theSeries}. \\

\indent {\em virtual void setDomain(Domain *theDomain);}\\ 
If any loads or constraint objects exist in the LoadPattern, the
LoadPattern will invoke {\em setDomain()} on those objects. Finally
invokes the DomainComponent classes {\em setDomain()} method. \\

\indent {\em  virtual bool addNodalLoad(NodalLoad *theLoad);}\\
Adds the NodalLoad object pointed to by {\em theLoad} to the
LoadPattern. If the LoadPattern could add the pointer to its storage
object for nodal loads, it will invoke {\em setDomain()}
and {\em setLoadPattern()} on the load object if a Domain has been set. \\

\indent {\em  virtual bool addElementalLoad(ElementalLoad *theLoad);}\\
Adds the ElementalLoad pointed to by {\em theLoad} to the
LoadPattern. If the LoadPattern could add the pointer to its storage
object for elemental loads, it will invoke {\em setDomain()}
and {\em setLoadPattern()} on the load object if a Domain has been set. \\

\indent {\em  virtual bool addSP\_Constraint(SP\_Constraint *theSp);}\\ 
Adds the SP\_Constraint pointed to by {\em theSp} to the
LoadPattern. If the LoadPattern could add the pointer to its storage
object for single-point constraints, it will invoke {\em setDomain()}
and {\em setLoadPattern()} on the constraint object if a Domain has
been set. \\

\indent {\em  virtual NodalLoadIter     \&getNodalLoads(void);}\\
Returns an iter to the nodal loads in the LoadPattern. \\

\indent {\em  virtual ElementalLoadIter \&getElementalLoads(void);}\\
Returns an iter to the elemental loads in the LoadPattern. \\

\indent {\em  virtual SP\_ConstraintIter \&getSPs(void);}\\ 
Returns an iter to the single-point constraints in the LoadPattern. \\
    
\indent {\em  virtual NodalLoad *removeNodalLoad(int tag);}\\
To remove the nodal load whose identifier is given by {\em tag} from
the LoadPattern and sets the laods associated Domain object to
$0$. Returns a pointer to the load if succesfully removed, otherwise
$0$ is returned. \\ 

\indent {\em  virtual ElementalLoad *removeElementalLoad(int tag);}\\
To remove the elemental load whose identifier is given by {\em tag} from
the LoadPattern and set the loads associated Domain object to
$0$. Returns a pointer to the load if succesfully removed, otherwise
$0$ is returned. \\ 

\indent {\em  virtual SP\_Constraint *removeSP\_Constraint(int tag);}\\ 
To remove the single-point constraint whose identifier is given by {\em tag} from
the LoadPattern and st its associated Doman object to $0$. Returns a
pointer to the load if succesfully removed, otherwise $0$ is
returned. \\ 

\indent {\em  virtual void applyLoad(double pseudoTime = 0.0);}\\
To apply the load for the pseudo time {\em pseudoTime}. From the
associated TimeSeries object the LoadPattern will obtain a current
load factor for the pseudo time. It will then invoke {\em
applyLoad(load factor)} on the loads and {\em applyConstraint(load
factor)} on the single-point constraints in the LoadPattern. If {\em
setLoadConstant()} has been invoked, the saved load factor is used and
no call is made to the TimeSeries object. If no TimeSeries is
associated with the object a load factor of $0.0$ is used. \\

\indent {\em  virtual void setLoadConstant(void);}\\ 
Marks the LoadPattern as being constant. Subsequent calls to {\em
applyLoad()} will use the current load factor. \\

\indent {\em  virtual int sendSelf(int commitTag, Channel
\&theChannel);}\\
Creates a vector of size 4 into which it places the current load
factor, the mark indicating whether LoadPattern is constant, and the
database tag and class tag of the TimeSeris object. Invokes {\em
sendVector()} on the Channel object and {\em sendSelf()} on the
TimeSeries object. \\

\indent {\em  virtual int recvSelf(int commitTag, Channel \&theChannel,
FEM\_ObjectBroker \&theBroker);}\\
Does the mirror image of {\em sendSelf()}. \\

\indent {\em  virtual void Print(ostream \&s, int flag =0);}\\
Invokes {\em Print(s, flag)} on the TimeSeries, NodalLoads,
ElementalLoads and SP\_Constraints. 


\pagebreak
\subsubsection{\bf EarthquakePattern}
%File: ~/OOP/earthquake/EarthquakePattern.tex
%What: "@(#) EarthquakePattern.tex, revA"

\noindent {\bf Files}   \\
\indent \#include $<\tilde{ }$/domain/pattern/EarthquakePattern.h$>$  \\

NEEDS MODIFICATION TO ALLOW MULTIPLE EARTHQUAKE PATTERNS, SO DON''T
HAVE TO SET R IN NODES EACH APPLYLOAD.\\

\noindent {\bf Class Declaration}  \\
\indent class EarthquakePattern: public LoadPattern  \\

\noindent {\bf Class Hierarchy} \\
\indent TaggedObject \\
\indent MovableObject \\
\indent\indent DomainComponent \\
\indent\indent\indent LoadPattern \\
\indent\indent\indent\indent {\bf EarthquakePattern} \\

\noindent {\bf Description} \\ 
\indent The EarthquakePattern class is an abstract class. An
EarthquakePattern is an object which adds earthquake loads to
models. This abstract class keeps track of the GroundMotion objects
and implements the {\em applyLoad()} method. It is up to the concrete
subclasses to set the appropriate values of {\em R} at each node in
the model.\\

\noindent {\bf Class Interface} \\
\indent // Constructor \\ 
\indent {\em EarthquakePattern(int tag, int classTag);}\\ \\
\indent // Destructor \\ 
\indent {\em virtual $\tilde{ }$EarthquakePattern();}\\  \\
\indent // Public Methods \\ 
\indent {\em virtual void applyLoad(double time);} \\ 
\indent // Protected Methods \\ \\
\indent {\em int addMotion(GroundMotion \&theMotion)} \\

\noindent {\bf Constructor} \\ 
\indent {\em EarthquakePattern(int tag, int classTag);}\\ 
The integers {\em tag} and {\em classTag} are passed to the
LoadPattern classes constructor. \\

\noindent {\bf Destructor} \\
\indent {\em virtual~ $\tilde{}$EarthquakePattern();}\\ 
Invokes the destructor on all GroundMotions added to the
Earthquakepattern. It then invokes the destructor on the array holding
pointers to the GroundMotion objects.\\

\indent {\em virtual void applyLoad(double time, double loadFactor =
1.0) = 0;} \\ 
Obtains from each GroundMotion, the velocity and acceleration for the
time specified. These values are placed in two Vectors of size equal
to the number of GroundMotion objects. For each node in the Domain
{\em addInertiaLoadToUnbalance()} is invoked with the acceleration Vector
objects. SIMILAR OPERATION WITH VEL and ACCEL NEEDS TO BE INVOKED ON
ELEMENTS .. NEED TO MODIFY ELEMENT INTERFACE \\

\indent {\em int addMotion(GroundMotion \&theMotion)} \\
Adds the GroundMotion, {\em theMotion} to the list of GroundMotion
objects. \\

\pagebreak
\subsubsection{UniformExcitation}
%File: ~/OOP/domain/pattern/UniformExcitation.tex
%What: "@(#) UniformExcitation.tex, revA"

NEEDS FINISHING. \\

\noindent {\bf Files}   \\
\indent \#include $<\tilde{ }$/domain/pattern/UniformExcitation.h$>$  \\

\noindent {\bf Class Declaration}  \\
\indent class UniformExcitation: public EarthquakeLoad  \\

\noindent {\bf Class Hierarchy} \\
\indent TaggedObject \\
\indent MovableObject \\
\indent\indent DomainComponent \\
\indent\indent\indent LoadPattern \\
\indent\indent\indent\indent EarthquakePattern \\
\indent\indent\indent\indent {\bf UniformExcitation} \\

\noindent {\bf Description} \\ 
\indent A UniformExcitation is an object which adds the loads imposed
by a single ground excitation to the model. For a UniformExcitation
this means that the {\em R} matrix at each node will have $1$ column
and all entries but those corresponding to the degree of freedom
direction will be set to $0$, the value for the degree of freedom
direction will be set to $1$.\\

\noindent {\bf Class Interface} \\
\indent // Constructor \\ 
\indent {\em UniformExcitation(GroundMotion \&theMotion, int dof, int tag);}\\ \\
\indent // Destructor \\ 
\indent {\em $\tilde{ }$UniformExcitation();}\\  \\
\indent // Public Methods \\ 
\indent {\em void applyLoad(double time);} \\ 
\indent {\em void Print(ostream \&s, int flag);}\\
\indent {\em int sendSelf(int commitTag, Channel \&theChannel);}\\
\indent {\em int recvSelf(int commitTag, Channel \&theChannel, 
FEM\_ObjectBroker \&theBroker);}\\

\noindent {\bf Constructor} \\ 
\indent {\em UniformExcitation(GroundMotion \&theMotion, int dof, int tag);}\\ \\
The integers {\em LOAD\_TAG\_UniformExcitation} and {\em classTag} are
passed to the Load classes constructor. \\

\noindent {\bf Destructor} \\
\indent {\em virtual~ $\tilde{}$UniformExcitation();}\\ 
Does nothing. \\

\indent {\em void applyLoad(double time);} \\ 
Checks to see if the number of nodes in the domain has changed, if
there has been a change it invokes {\em setNumColR(1)} and then 
{\em setR(theDof, 0, 1.0)} on each Node. It then invokes the base classes {\em
applyLoad()} method. THIS SHOULD BE CHANGED TO USE LATEST domainChanged().\\

\indent {\em void Print(ostream \&s, int flag);}\\
Does nothing. NEEDS WORK. \\

\indent {\em int sendSelf(int commitTag, Channel \&theChannel);}\\
Does nothing. NEEDS WORK. \\

\indent {\em int recvSelf(int commitTag, Channel \&theChannel, 
FEM\_ObjectBroker \&theBroker);}\\
Does nothing. NEEDS WORK. 



\pagebreak
\subsection{Domain}
% File: ~/domain/domain/Domain.tex 
% What: "@(#) Domain.tex, revA"

\noindent {\bf Files}   \\
\indent \#include $<$/domain/domain/Domain.h$>$  \\

\noindent {\bf Class Declaration}  \\
\indent class Domain  \\

\noindent {\bf Class Hierarchy} \\
\indent  {\bf Domain} \\

\noindent {\bf Description}  \\
\indent Domain is container class for storing and providing access to
the components of a domain, i.e. nodes, elements, boundary conditions,
and load patterns. 
A Domain is a container class which contains the  all elements,
nodes, load cases, single point constraints and multiple point
constraints that the model builder uses to create the model of the
structure. It provides operations for the following: 
\begin{itemize}
\item Population: Methods so that the DomainComponents can be addled
to the Domain. 
\item Depopulation: Methods so that the DomainComponents can be removed
from the Domain.
\item Access: Methods so that other classes, i.e. analysis and design,
can access the DomainComponents.
\item Query: Methods for determining the state of the domain.
\item Update: Methods for updating the state of the Domain
\item Analysis: Methods added for the Analysis class.
\item Output: Methods added for outputting information.
\end{itemize}
The Domain class stores each type of object, i.e. Nodes, Elements,
SP\_Constraints, MP\_Constraints, NodalLoads and ElementalLoads, in a
container object. Currently these container objects are a subtype of
TaggedObjectStorage (templates are not used as yet due to present
difficulties in porting code which uses templates). \\

\noindent {\bf Class Interface}  \\
\indent // Constructors  \\
\indent {\em Domain();}  \\
\indent {\em Domain(int numNodes, int numElements, int numNodeLoads,
int numEleLoads, int numSPs, int numMPs);}  \\
\indent {\em Domain(TaggedObjectStorage \&theNodesStorage, \\
\indent\indent\indent           TaggedObjectStorage \&theElementsStorage, \\
\indent\indent\indent           TaggedObjectStorage \&theMPsStorage,\\
\indent\indent\indent           TaggedObjectStorage \&theSPsStorage, \\
\indent\indent\indent           TaggedObjectStorage \&theLoadPatternsStorage);}\\ 
\indent {\em Domain(TaggedObjectStorage \&theStorageType);}\\ \\
\indent // Destructor  \\
\indent {\em virtual~ $\tilde{}$Domain();}  \\ \\
\indent // Public Methods to Populate the Domain  \\
\indent {\em virtual bool addElement(Element *theElementPtr);}  \\ 
\indent {\em virtual bool addNode(Node *theNodePtr);} \\  
\indent {\em virtual bool addSP\_Constraint(SP\_Constraint
*theSPptr);} \\
\indent {\em virtual bool addMP(MP\_Constraint *theMPptr);}  \\  
\indent {\em virtual bool addLoadPattern(LoadPattern *thePattern);}\\ \\
\indent // Public Methods to Populate the LoadPatterns  \\
\indent {\em virtual bool addSP(SP\_Constraint *theSPptr, int pattern);}  \\   
\indent {\em virtual bool addNodalLoad(NodalLoad *theLd, int pattern);}  \\   
\indent {\em virtual bool addElementalLoad(ElementalLoad *theLd, int pattern);}\\\\
\indent // Public Methods to Depopulate the Domain  \\
\indent {\em virtual void clearAll(void);}\\
\indent {\em virtual Element *removeElement(int tag);}\\
\indent {\em virtual Node *removeNode(int tag);}\\    
\indent {\em virtual SP\_Constraint *removeSP\_Constraint(int tag);}\\
\indent {\em virtual MP\_Constraint *removeMP\_Constraint(int tag);} \\
\indent {\em virtual LoadPattern *removeLoadPattern(int patternTag);}\\ \\
\indent // Public Methods to Access the Components of the Domain  \\
\indent {\em virtual ElementIter \&getElements(void);} \\
\indent {\em virtual NodeIter \&getNodes(void);} \\
\indent {\em virtual SP\_ConstraintIter \&getSPs(void);} \\
\indent {\em virtual MP\_ConstraintIter \&getMPs(void);} \\
\indent {\em virtual  LoadPatternIter \&getLoadPatterns(void);} \\ 
\indent {\em virtual  Element *getElement(int tag);}\\
\indent {\em virtual  Node *getNode(int tag);}\\
\indent {\em virtual  SP\_Constraint *getSP\_ConstraintPtr(int tag);}\\
\indent {\em virtual  MP\_Constraint *getMP\_ConstraintPtr(int tag);}\\ 
\indent {\em virtual  LoadPattern *getLoadPattern(int tag);} \\ \\
\indent // Public Methods to Query the State of the Domain \\
\indent {\em virtual double getCurrentTime(void) const;} \\
\indent {\em virtual double getCurrentLoadFactor(void) const;} \\
\indent {\em virtual int getNumElements(void) const;}\\ 
\indent {\em virtual int getNumNodes(void) const;}\\
\indent {\em virtual int getNumSPs(void) const;}\\
\indent {\em virtual int getNumMPs(void) const;}\\
\indent {\em virtual int getNumLoadPatterns(void) const;}\\
\indent {\em virtual const Vector \&getPhysicalBounds(void);} \\
\indent {\em virtual Graph \&getElementGraph(void);} \\
\indent {\em virtual Graph \&getNodeGraph(void);} \\ \\
\indent // Public Methods to Update the Domain  \\
\indent {\em virtual void setCommitTag(int newTag);} \\
\indent {\em virtual void setCurrentTime(double newTime);} \\
\indent {\em virtual void setCommittedTime(double newTime);} \\
\indent {\em virtual void applyLoad(double pseudoTime);} \\
\indent {\em virtual void setLoadConstant(void);} \\ 
\indent {\em virtual void commit(void);} \\ 
\indent {\em virtual void revertToLastCommit(void);} \\ 
\indent {\em virtual void update(void);} \\ 
\indent {\em virtual int hasDomainChanged(void);} \\ 
\indent {\em virtual void setDomainChangeStamp(int newStamp);}\\\\
\indent // Public Methods for Output \\
\indent {\em virtual  int  addRecorder(Recorder \&theRecorder);}\\
\indent {\em virtual  int  playback(int cTag);}\\
\indent {\em virtual  void Print(OPS_Stream \&s, int flag =0);}\\
\indent {\em friend OPS_Stream \&operator<<(OPS_Stream \&s, Domain \&M); }\\  
\indent // Protected Member Functions  \\
\indent {\em virtual void domainChange(void) }\\
\indent {\em virtual int buildEleGraph(Graph \&theEleGraph)} \\
\indent {\em virtual int buildNodeGraph(Graph \&theNodeGraph)} \\

\noindent {\bf Constructors}  \\
\indent {\em Domain();}  \\
Constructs an empty domain. The storage for the DomainComponents uses
ArrayOfTaggedObjects objects for each type of object to be stored. The
initial sizes specified for these objects are 4096 for the Elements
and Nodes, 256 SP\_Constraints and MP\_Constraints, and 32 for the
container for the LoadPatterns. A check is made to ensure that memory
is allocated for these objects, if not the {\em fatal()} method of the
global ErrorHandler is invoked.\\   

\indent {\em Domain(int numNodes, int numElements, int numSPs, int
numMPs, int numLoadPatterns);}  \\ 
Constructs an empty Domain. The storage for the DomainComponents uses
ArrayOfTaggedObjects objects for each type of object to be stored. The
initial sizes specified for these objects are as given in the
arguments for this constructor, i.e. {\em numElements} for the
Elements and {\em numNodes} for the Nodes. A size of 32 is used for
the LoadPatterns. A check is made to ensure that memory
is allocated for these objects, if not the {\em fatal()} method of the
global ErrorHandler is invoked.\\   


\indent {\em Domain(TaggedObjectStorage \&theNodesStorage, \\
\indent\indent\indent           TaggedObjectStorage \&theElementsStorage, \\
\indent\indent\indent           TaggedObjectStorage \&theMPsStorage, \\
\indent\indent\indent           TaggedObjectStorage \&theSPsStorage, \\
\indent\indent\indent           TaggedObjectStorage \&theLoadPatternsStorage); } \\
Constructs a Domain where the Nodes, Elements, MP\_Constraints,
SP\_Constraint and LoadPattern objects will be
stored in the storage objects provided. A check is made to ensure
these container objects are empty is made; if not empty the {\em warning()}
method of the global ErrorHandler is invoked and the objects are
cleared. \\

\indent {\em Domain(TaggedObjectStorage \&theStorageType); } \\
Constructs a Domain where the Nodes, Elements, MP\_Constraints,
SP\_Constraint and LoadPattern objects will be stored in the storage
objects obtained by invoking {\em getEmptyCopy()} on the {\em
theStorageType} object. A check is made to ensure that memory is
allocated for these objects, if not the {\em fatal()} method of the
global ErrorHandler is invoked.\\    

\noindent {\bf Destructor}  \\
\indent {\em virtual~ $\tilde{}$Domain();}  \\
Invokes delete on all the storage objects. This means that, if the two
latter constructors have been called, the container objects must have 
been created using {\em new} and that at no other point in the program
is the destructor invoked on these objects. It should be noted, that
the objects in the Domain, i.e. the DomainComponents, are not
destroyed. To clean up these objects {\em clearAll()} should be
invoked before the destructor is called. \\

%%%%%%%% Public Member Functions - POPULATION
\noindent {\bf Public Methods to Populate the Domain}  \\
\indent {\em virtual bool addElement(Element *theElementPtr);}  \\
To add the element pointed to by theElementPtr to the domain. 
If {\em \_DEBUG} is defined the domain checks to
see that: 1) that all external nodes for the element exists in the
domain. and 2) that the sum of the dof at all the nodes equals the num
of dof at the element. In addition the domain always checks to ensure
that no other element with a similar tag exists in the domain.
If the checks are successful, the element is added to
the domain by the domain invoking {\em addComponent(theElePtr)} on
the container for the elements. The domain then invokes {\em
setDomain(this)} on the element and {\em domainChange()} on
itself if the element is successfully added. The call returns {\em
true} if the element is added, otherwise the {\em warning()} method of
the global ErrorHandler is invoked and {\em false} is returned.\\

{\em virtual bool addNode(Node *theNodePtr = false);}  \\
To add the node pointed to by {\em theNodePtr} to the domain. 
The domain first checks that no other node with a similar tag,
node number, has been previously added to the domain. The
domain will then add the node to it's node container object, by
invoking {\em addComponent(theNodePtr)}. If successful, the Domain
invokes {\em setDomain(this)} on the node, {\em domainChange()} on
itself, and checks the coordinates of the domain to see if they effect
the box encompassing the Domain. The call returns {\em true} if the
node was added, otherwise the {\em warning()} method of
the global ErrorHandler is invoked and {\em false} is returned.\\


{\em virtual bool addSP(SP\_Constraint *theSPptr = false);}  \\
To add the single point constraint pointed to by theSPptr to the
domain. If {\em \_DEBUG} is defined the domain is responsible for
checking to see that 1) the constrained node exists in the domain and
2) that the node has the dof that is to be constrained. 
In addition the domain always checks to ensure that no other
constraint with a similar tag exists in the domain. If the checks are
successful, the constraint is added to domain by the domain invoking {\em
addComponent(theSPptr)} on the container for the SP\_Constraints. The
domain then invokes {\em setDomain(this)} on the 
constraint and {\em domainChange()} on itself. The call returns {\em
true} if the constraint was added, otherwise the {\em warning()} method of
the global ErrorHandler is invoked and {\em false} is returned.\\



{\em virtual bool addMP(MP\_Constraint *theMPptr = false);}  \\
To add the multiple point constraint pointed to by theMPptr, to the
domain. If {\em \_DEBUG} is defined the domain first
checks to see that the retained and the constrained node both exist
in the model and that the matrix is of proper dimensions (THIS LAST
PART NOT YET IMPLEMENTED). 
In addition the domain always checks to ensure that no other
MP\_Constraint with a similar tag exists in the domain. If the checks are
successful, the constraint is added to domain by the domain invoking {\em
addComponent(theMPptr)} on the container for the MP\_Constraints. The
domain then invokes {\em setDomain(this)} on the 
constraint and {\em domainChange()} on itself. The call returns {\em
true} if the constraint was added, otherwise the {\em warning()} method of
the global ErrorHandler is invoked and {\em false} is returned.\\


{\em virtual bool addLoadPattern(LoadPattern *thePattern);}  \\ 
To add the LoadPattern {\em thePattern} to the domain.
The load is added to domain by the domain invoking {\em
addComponent(theLd)} on the container for the LoadPatterns. The domain
is responsible for invoking {\em setDomain(this)} on the load. The
call returns {\em true} if the load was added, otherwise the {\em
warning()} method of the global ErrorHandler is invoked and {\em
false} is returned.\\ 

{\em virtual bool addNodalLoad(NodalLoad *theLd, int loadPatternTag);}  \\ 
To add the nodal load {\em theld} to the LoadPattern in the domain
whose tag is given by {\em loadPatternTag}.
If {\em \_DEBUG} is defines, checks to see that corresponding node
exists in the domain. A pointer to the LoadPattern is obtained from
the LoadPattern container and the load is added to LoadPattern by
invoking {\em addNodalLoad(theLd)} on the LoadPattern object. The
domain is responsible for invoking {\em setDomain(this)} on the
load. The call returns {\em true} if the load was added, otherwise the
{\em warning()} method on the global ErrorHandler is invoked and {\em
false} is returned.\\ 


{\em virtual bool addElementalLoad(ElementalLoad *theLd, int loadPatternTag);}  \\ 
To add the elemental load {\em theld} to the LoadPattern in the domain
whose tag is given by {\em loadPatternTag}.
If {\em \_DEBUG} is defines, checks to see that corresponding element
exists in the domain. A pointer to the LoadPattern is obtained from
the LoadPattern container and the load is added to LoadPattern by
invoking {\em addElementalLoad(theLd)} on the LoadPattern object. The
domain is responsible for invoking {\em setDomain(this)} on the
load. The call returns {\em true} if the load was added, otherwise the
{\em warning()} method on the global ErrorHandler is invoked and {\em
false} is returned.\\ 


{\em virtual bool addSP\_Constraint(SP\_Constraint *theConstraint, int
loadPatternTag);}  \\  
To add the elemental load {\em theConstraint} to the LoadPattern in the domain
whose tag is given by {\em loadPatternTag}.
If {\em \_DEBUG} is defines, checks to see that corresponding node
exists in the domain. A pointer to the LoadPattern is obtained from
the LoadPattern container and the load is added to LoadPattern by
invoking {\em addSP\_Constraint(theConstraint)} on the LoadPattern object. The
domain is responsible for invoking {\em setDomain(this)} on the
constraint. The call returns {\em true} if the load was added, otherwise the
{\em warning()} method on the global ErrorHandler is invoked and {\em
false} is returned.\\ 


\noindent {\bf Public Methods to Depopulate the Domain}  \\
\indent{\em virtual void clearAll(void);}\\
To remove all the components from the model and invoke the destructor
on these objects. First {\em clearAll()} is invoked on all the
LoadPattern objects. Then the domain object invokes {\em
clearAll()} on its container objects. In addition the destructor for
each Recorder object that has been added to the domain is invoked. In
addition the current time and load factor are set to $0$, and the box
bounding the domain is set to the box enclosing the origin. \\


\indent{\em virtual Element *removeElement(int tag);}\\
To remove the element whose tag is given by {\em tag} from the
domain. The domain achieves this by invoking {\em
removeComponent(tag)} on the container for the elements. 
Returns $0$ if no such element exists in the domain. Otherwise 
the domain invokes {\em setDomain(0)} on the element (using a cast to
go from a TaggedObject to an Element, which is safe as only an
Element objects are added to this container) and {\em
domainChange()} on itself before a pointer to the element is returned. \\

{\em virtual Node *removeNode(int tag);}\\    
To remove the node whose tag is given by {\em tag} from the domain. 
The domain achieves this by invoking {\em
removeComponent(tag)} on the container for the nodes. 
Returns $0$ if no such node exists in the domain. If the node is to be
removed the domain invokes {\em setDomain(0)} on the node and {\em
domainChange()} on itself before a pointer to the Node is returned.\\ 

{\em virtual SP\_Constraint *removeSP\_Constraint(int tag);}\\
To remove the SP\_Constraint whose tag is given by {\em tag} from the
domain. The domain achieves this by invoking {\em
removeComponent(tag)} on the container for the single point
constraints. Returns $0$ if the constraint was not in the domain,
otherwise the domain invokes {\em setDomain(0)} on the constraint and
{\em domainChange()} on itself before a pointer to the constraint is
returned. Note this will only remove SP\_Constraints which have been
added to the domain and not directly to LoadPatterns.\\

{\em virtual MP\_Constraint *removeMP\_Constraint(int tag);} \\
To remove the MP\_Constraint whose tag is given by {\em tag} from the
domain. The domain achieves this by invoking {\em
removeComponent(tag)} on the container for the multi point
constraints. Returns $0$ if the constraint was not in the domain,
otherwise the domain invokes {\em setDomain(0)} on the constraint and
{\em domainChange()} on itself before a pointer to the constraint is
returned.\\   

{\em virtual LoadPattern *removeLoadPattern(int tag);}\\         
To remove the LoadPattern whose tag is given by {\em tag} from the
domain. The domain achieves this by invoking {\em
removeComponent(tag)} on the container for the load patterns. 
If the LoadPattern exists, the domain then iterates through the loads
and constraints of the LoadPattern invoking {\em setDomain(0)} on
these objects. Returns
$0$ if the load was not in the domain, otherwise returns a pointer to
the load that was removed. Invokes {\em setDomain(0)} on the load case
before it is returned.\\ 


%%%%%%%% Public Member Functions - ACCESS
\noindent {\bf Public Methods to Access the Components of the Domain}  \\
\indent {\em virtual ElementIter \&getElements(void);} \\
To return an iter for the Elements in the domain. It returns an {\em
ElementIter} for the elements of the domain that were added using {\em
addElement()}. \\  

{\em virtual NodeIter \&getNodes(void);} \\
It returns a {\em NodeIter} for the nodes which have been added to the
domain. \\ 

{\em virtual SP\_ConstraintIter \&getSPs(void);} \\
To return an {\em SP\_ConstraintIter} for the single point constraints
which have been added to the domain. \\ 

{\em virtual MP\_ConstraintIter \&getMPs(void);} \\
To return an {\em MP\_ConstraintIter} for the multiple point
constraints which have been added to the domain. \\ 

{\em virtual LoadPatternIter \&getLoadPatterns(void);} \\
To return an {\em LoadPatternIter} for the LoadPattern
objects which have been added to the domain. \\ 

{\em virtual  Element *getElement(int tag);}\\
To return a pointer to the element {\em tag}. If no such element
exists $0$ is returned. It does this by invoking {em
getComponentPtr(tag)} on the element container and performing a cast
to an element if the object exists. \\

{\em virtual  Node *getNode(int tag);}\\
To return a pointer to the node whose tag is given by {\em tag}. If
no such node exists $0$ is returned. It does this by invoking {em
getComponentPtr(tag)} on the node container and performing a cast
to a node if the object exists. \\

{\em virtual  SP\_Constraint *getSP\_ConstraintPtr(int tag);}\\
To return a pointer to the SP\_Constraint whose tag is given by {\em tag}. If
no such SP\_Constraint exists $0$ is returned. It does this by invoking {em
getComponentPtr(tag)} on the single-point constraint container and
performing a cast to an SP\_Constraint if the object exists. \\


{\em virtual  MP\_Constraint *getMP\_ConstraintPtr(int tag);}\\
To return a pointer to the MP\_Constraint whose tag is given by {\em tag}. If
no such MP\_Constraint exists $0$ is returned. It does this by invoking {em
getComponentPtr(tag)} on the multi-point constraint container and
performing a cast to an MP\_Constraint if the object exists. \\


{\em virtual  ElementalLoad *getLoadPattern(int tag);}\\
To return a pointer to the LoadPattern whose tag is given by {\em tag}. If
no such LoadPattern exists $0$ is returned. It does this by invoking {em
getComponentPtr(tag)} on the elemental load container and
performing a cast to a LoadPattern if the object exists. \\


%%%%%%%% Public Member Functions - QUERY
\noindent {\bf Public Methods to Query the Domain} \\
\indent {\em virtual int getNumElements(void) const;}\\
To return the number of elements in the domain. It does this by
invoking {\em getNumComponents()} on the container for the elements. \\

{\em virtual int getNumNodes(void) const;}\\
To return the number of nodes in the domain. It does this by
invoking {\em getNumComponents()} on the container for the
nodes. \\

{\em virtual int getNumSPs(void) const;}\\
To return the number of single point constraints in the domain. It
does this by invoking {\em getNumComponents()} on the container for
the single point constraints. \\

{\em virtual int getNumMPs(void) const;}\\
To return the number of multi point constraints in the domain. It
does this by invoking {\em getNumComponents()} on the container for
the multi point constraints. \\

{\em virtual int getNumLoadPatterns(void) const;}\\
To return the number of load patterns in the domain. It
does this by invoking {\em getNumComponents()} on the container for
the load patterns. \\

{\em virtual const Vector \&getPhysicalBounds(void);} \\
To return the bounding rectangle for the Domain. The information is
contained in a Vector of size 6 containing in the following order
\{xmin, ymin, zmin, xmax, ymax, zmax\}. This information is built up
as nodes are added to the domain, initially all are set to $0$ in the
constructor. \\

{\em virtual Graph \&getElementGraph(void);} \\
Returns the current element graph (the connectivity of the elements
in the domain). If the {\em eleChangeFlag} has been set
to {\em true} the method will invoke {\em buildEleGraph(theEleGraph)}
on itself before returning the graph. The vertices in the element
graph are to be labeled $0$ through $numEle-1$. The Vertices references
contain the elemental tags. \\ 

{\em virtual Graph \&getNodeGraph(void);} \\
Returns the current node graph (the connectivity of the nodes in
the domain). If the {\em nodeChangeFlag} has been set to {\em true} the
will invoke {\em buildNodeGraph(theNodeGraph)} on itself before
returning the graph. The vertices in the node graph are to be labeled
$0$ through $numNode-1$. The Vertices references contain the nodal tags. \\ 


%%%%%%%% Public Member Functions - UPDATE
\noindent {\bf Public Methods to Update the Domain}  \\
\indent {\em virtual setCommitTag(int newTag);} \\
To set the current commitTag to {\em newTag}. \\

\indent {\em virtual setCurrentTime(double newTime);} \\
To set the current time to {\em newTime}. \\

\indent {\em virtual setCommittedTime(double newTime);} \\
To set the committed time to {\em newTime}. \\

\indent {\em virtual void applyLoad(double pseudoTime);} \\ 
To apply the loads for the given time {\em pseudoTime}. The domain
first clears the current load at all nodes and elements, done by
invoking {\em zeroUnbalancedLoad()} on each node and {\em zeroLoad()}
on each element. The domain then invokes {\em applyLoad(pseudoTime)}
on all LoadPatterns which have been added to the domain. The domain
will then invoke {\em applyConstraint(pseudoTime)} on all the
constraints in the single and multi point constraint
containers. Finally the domain sets its current time to be {\em
pseudoTime}.\\  

\indent {\em virtual setLoadConstant(void);} \\
To set the loads in the LoadPatterns to be constant. The domain
achieves this by invoking {\em setLoadConstant()} on all the
LoadPatterns which have been added to the domain. Note that
LoadPatterns added after this method has been invoked will not be
constant until this method is invoked again. \\


{\em virtual void commit(void);} \\
To commit the state of the domain , that is to accept the current
state as being ion the solution path. The domain invokes {\em
commit()} on all nodes in the domain and then {\em 
commit()} on all elements of the domain. These are calls for the nodes
and elements to set there committed state as given by their current
state. The domain will then set its committed time variable to be
equal to the current time and lastly increments its commit tag by $1$. \\ 


{\em virtual int revertToLastCommit(void);} \\
To return the domain to the state it was in at the last commit. The
domain invokes {\em revertToLastCommit()} on all nodes and elements in
the domain. The domain sets its current loadFactor and time
stamp to be equal to the last committed values. The domain decrements
the current commitTag by $1$. Finally it invokes {\em applyLoad()}
on itself with the current time.\\

{\em virtual void update(void);} \\
Called by the AnalysisModel to update the state of the
domain. Iterates over all the elements of the Domain and invokes {\em
update()}. \\

\indent {\em virtual void setDomainChangeStamp(int newStamp);}\\
To set the domain stamp to be {\em newStamp}. Domain stamp is the
integer returned by {\em hasDomainChanged()}. \\

\noindent {\bf Public Methods for the Analysis} \\
{\em virtual int hasDomainChanged(void);} \\
To return an integer stamp indicating the state of the
domain. Initially $0$, this integer is incremented by $1$ if  {\em
domainChange()} has been invoked since the last invocation of the
method. If the domain has changed it marks the element and node graph
flags as not having been built.\\  

\noindent {\bf Public Methods for Output} \\
\indent {\em  virtual int  addRecorder(Recorder \&theRecorder);}\\
To add a recorder object {\em theRecorder} to the domain. {\em
record(commitTag)} is invoked on each {\em commit()}. The pointers to
the recorders are stored in an array which is resized on each
invocation of this method.\\  

\indent {\em int playback(int commitTag);}\\
The domain will invoke {\em playback(commitTag)} on all recorder
objects which have been added to the domain.\\

\indent {\em virtual void Print(OPS_Stream \&s, int flag =0);}\\
To print the state of the domain. The domain invokes {\em Print(s,flag)} on
all it's container objects. \\


\indent {\em friend OPS_Stream \&operator$<<$(OPS_Stream \&s, Domain \&M); }\\  
This function allows domain objects to be printed to streams. The
function invokes $M.Print(s)$ before returning $s$. \\

\noindent {\bf Protected Methods}  \\
\indent{\em virtual void domainChange(void) }\\
Sets a flag indicating that the integer returned in the next call to 
{\em hasDomainChanged()} must be incremented by $1$. This method is
invoked whenever a Node, Element or Constraint object is added to the
domain. \\ 

{\em virtual int buildEleGraph(Graph \&theEleGraph)} \\
A method which will cause the domain to discard the current element
graph and build a new one based on the element connectivity. Returns
$0$ if successful otherwise $-1$ is returned along with an error
message to opserr. \\

{\em virtual int buildNodeGraph(Graph \&theNodeGraph)} \\
A method which will cause the domain to discard the current node
graph and build a new one based on the node connectivity. Returns
$0$ if successful otherwise $-1$ is returned along with an error
message to opserr. \\







%\pagebreak
%\subsubsection{Subdomain}
%% File: ~/domain/domain/subdomain/Subdomain.tex 
% What: "@(#) Subdomain.tex, revA"

UNDER CONSTRUCTION

\noindent {\bf Files}   \\
\indent \#include $<$/domain/subdomain/Subdomain.h$>$  \\

\noindent {\bf Class Decleration}  \\
\indent class Subdomain: public Domain, public Element  \\

\noindent {\bf Class Hierarchy} \\
\indent  Domain  \\ 
\indent MovableObject \\
\indent TaggedObject \\
\indent\indent DomainComponent \\
\indent\indent\indent  Element \\
\indent\indent\indent\indent {\bf Subdomain} \\

\noindent {\bf Description}  \\
\indent  A Subdomain is a Domain that can be an Element in an
enclosing Domain. For this reason, it inherits from both Domain and
Subdomain. It also extends the Domain interface to deal with the
distintion between whether Nodes in the Subdomain are internal to the
Subdomain, or external. An external Node is a Node shared by two
Elements residing in different Subdomains. \\


\noindent {\bf Class Interface}  \\
\indent {\bf Constructors}  \\
\indent {\em Subdomain(int tag);}  \\
\indent {\em Subdomain(int tag, \\
\indent\indent\indent\indent TaggedObjectStorage \&theInternalNodeStorage,\\
\indent\indent\indent\indent TaggedObjectStorage \&theExternalNodeStorage,\\
\indent\indent\indent\indent TaggedObjectStorage \&theElementsStorage, \\
\indent\indent\indent\indent TaggedObjectStorage \&theLCsStorage, \\
\indent\indent\indent\indent TaggedObjectStorage \&theMPsStorage, \\
\indent\indent\indent\indent TaggedObjectStorage \&theSPsStorage);} \\ \\
\indent // Destructor  \\
\indent {\em virtual~ $\tilde{}$Subdomain();}  \\ \\
\indent // Public Methods modified from Domain class}  \\
\indent {\em virtual bool addNode(Node *theNodePtr);} \\
\indent {\em virtual Node *removeNode(int tag); } \\
\indent {\em virtual Node *getNode(int tag); } \\
\indent {\em virtual int getNumNodes(void); } \\
\indent{\em virtual NodeIter \&getNodes();} \\
\indent {\em virtual void setAnalysis(DomainDecompositionAnalysis *theAnalysis);}\\
\indent {\em void commit(void);} \\ \\
\indent // Public Methods inherited from Element now implemented \\
\indent {\em virtual int getNumExternalNodes(void) const;    }\\
\indent {\em virtual const ID \&getExternalNodes(void);}\\
\indent {\em virtual int getNumDOF(void);}\\
\indent {\em virtual void commitState(void);}\\ 
\indent {\em virtual const Matrix \&getStiff(void);}\\
\indent {\em virtual const Matrix \&getMass(void);    }\\
\indent {\em virtual const Matrix \&getDamp(void);    }\\
\indent {\em virtual const Vector \&getResistingForce(void);}\\
\indent {\em virtual const Vector \&getResistingForceWithInertia(void);}\\
\indent {\em virtual bool isSubdomain(void);}\\
\indent {\em int sendSelf(Channel \&theChannel, FEM\_ObjectBroker
\&theBroker);}\\ 
\indent {\em int recvSelf(Channel \&theChannel, FEM\_ObjectBroker
\&theBroker);}\\ \\
\indent  // Public Methods introduced for the Subdomain class \\
\indent {\em virtual bool addExternalNode(Node *theNodePtr);} \\
\indent{\em virtual NodeIter \&getInternalNodeIter();} \\
\indent{\em virtual const ID \&getExternalNodeIter();} \\
\indent {\em virtual int computeTang(void);}\\
\indent {\em virtual int computeResidual(void);}\\
\indent {\em virtual const Matrix \&getTang(void);    }\\
\indent {\em void setFE\_ElementPtr(FE\_Element *theFEelePtr);}\\
\indent {\em const Vector \&getLastExternalSysResponse(void);}\\
\indent {\em virtual int computeNodalResponse(void);}\\
\indent {\em double getCost(void); }\\ \\
\indent // Protected Methods  \\
\indent{\em FE\_Element *getFE\_ElementPtr(void); }\\



\noindent {\bf Constructors}  \\
\indent {\em Subdomain(int tag);}  \\
Constructs an empty Subdomain with a number given by {\em tag}. The
storage of the Elements, constraints and loads are handled by the base
Domain class, the storage of the internal and external nodes are
handle by the Subdomain class. For storage of the Nodes objects of
type ArrayOfTaggedObjects are used. The constructors for these objects
take initial sizes of the arrays, the sizes used are:
InternalNodeArray = 8024, ExternalNodeArray = 256. The initial sizes
are not upperbounds on the number of components that can be added. \\


\indent {\em Subdomain(int tag, \\
\indent\indent\indent\indent TaggedObjectStorage \&theInternalNodeStorage,\\
\indent\indent\indent\indent TaggedObjectStorage \&theExternalNodeStorage,\\
\indent\indent\indent\indent TaggedObjectStorage \&theElementsStorage, \\
\indent\indent\indent\indent TaggedObjectStorage \&theLCsStorage, \\
\indent\indent\indent\indent TaggedObjectStorage \&theMPsStorage, \\
\indent\indent\indent\indent TaggedObjectStorage \&theSPsStorage);} \\
Constructs an empty Subdomain with a number given by {\em tag}. The
components of the subdomain are stored in the storage objects given in
the argument for the constructor. \\ 

\noindent {\bf Destructor}  \\
\indent {\em virtual~ $\tilde{}$Subdomain();}  \\
Invokes the destructor on the storage objects for both the internal
and external nodes. \\ 


\noindent {\bf Public Methods modified from Domain class}  \\
\indent {\em virtual bool addNode(Node *theNodePtr);} \\
A Method to add the node pointed to by {\em theNoedPtr} to the
Subdomain. This node is an internal node to the domain and is
returned in calls to {\em getNodes()} and {\em getInternalNodes()}. 
Invokes {\em setDomain(this)} on the Node and {\em domainChanged()} on
itself.\\ 

{\em virtual Node *removeNode(int tag); } \\
To remove a Node whose tag is given by {\em tag} from the
Subdomain. Returns $0$ if the Node is not in the Subdoamin, otherwise
it removes the pointer to the node from the storage object, invokes
{\em domainChange()} on itself, and returns the pointer to the
Node. If the Node is external, the destructor is called on the
DummyNode. Invokes {\em domainChanged()} on itself. \\ 

{\em virtual Node *getNode(int tag); } \\
To return a pointer to the Node whose tag is given by {\em tag} from
the Subdomain. Returns $0$ if the Node is not in the Subdoamin,
otherwise returns a pointer to the Node, if external a pointer to the
DummyNode is returned. \\ 

{\em virtual int getNumNodes(void); } \\
Returns the number of external and internal Nodes. \\

\indent{\em virtual NodeIter \&getNodes();} \\
Returns an iter to all nodes that have been added to the subdomain. \\

{\em virtual void setAnalysis(DomainDecompositionAnalysis *theAnalysis);}\\
Sets the corresponding DomainDecompositionAnalysis object to be {\em
theAnalysis}. \\

{\em void commit(void);} \\
invokes the base Domain classes {\em commit()} method. It then goes through
all the Nodes in the Subdomain, invoking {\em commitState()} on the Nodes. \\


\noindent {\bf Public Methods inherited from Element now implemented} \\
\indent {\em virtual int getNumExternalNodes(void) const;    }\\
Returns the number of external nodes that have been successfully added
to the subdomain as external nodes and have yet to be removed from the
subdomain. \\ 

{\em virtual const ID \&getExternalNodes(void);}\\
Returns an ID containing the tags of all nodes added to the subdomain
as external nodes and have yet to be removed from the subdomain. \\

{\em virtual int getNumDOF(void);}\\
Returns the num of external dof associated with the subdomain, the number
returned is the result of invoking {\em getNumExternalEqn()} on
the DomainDecompAnalysis object assocaited with the Subdomain. If
no Analysis yet associated with the Subdomain $0$ is returned. \\

{\em virtual void commitState(void);}\\ 
Invokes {\em commit()} on itself. \\
    
{\em virtual const Matrix \&getStiff(void);}\\
For this class does nothing but print an error message. Subtypes may
provide a condensed stiffness matrix, $T^tKT$ corresponding to
external nodes. Returns a zero matrix of dimensions (1x1). \\

{\em virtual const Matrix \&getMass(void);    }\\
For this class does nothing but print an error message. Subtypes may
provide a condensed mass matrix, $T^tMT$ or a mass matrix with zero
diag elements. Returns a zero matrix of dimensions (1x1). \\

{\em virtual const Matrix \&getDamp(void);    }\\
For this class does nothing but print an error message. Subtypes may
provide a condensed damping matrix, $T^tDT$ or a damping matrix
corresponding to some comination of the condensed stifffness and mass
matrices. Returns a zero matrix of dimensions (1x1). \\

{\em virtual const Vector \&getResistingForce(void);}\\
Returns the Vector obtained from invoking {\em getCondensedRHS()} on
the DomainDecompositionAnalysis object. \\

{\em virtual bool isSubdomain(void);}\\
Returns {\em true}. Subtypes can change this. The result of changing
this will be that the corresponding FE\_Elements will ask for the
stiffness, mass and damping matrices to form the tangent and will ask
for the residual to form the residual for the system of equations. \\

{\em int sendSelf(Channel \&theChannel, FEM\_ObjectBroker \&theBroker);}\\
Sends an ID of size 2 to the channel, sending {\em theAnalysis-$>$getClassTag()} and the integer $0$,
if a link exists to a DomainDecompAnalyssi object. 
{\em sendSelf()} is then invoked on theDomainDecompAnalysis object if the link exists. \\

{\em int recvSelf(Channel \&theChannel, FEM\_ObjectBroker \&theBroker);}\\
Receives an ID of size 2 from the channel. If the integer $0$ is sent {\em theBroker}
is asked to construct a new DomainDecompositionAnalysis object using the class tag also sent.
Returns resultt of invoking {\em recvSelf()} on this object. \\

\noindent {\bf Public Methods introduced for the Subdomain class} \\
\indent {\em virtual bool addExternalNode(Node *theNodePtr);} \\
A Method to add the node pointed to by {\em theNoedPtr} to the
Subdomain. This node is an external node to the subdomain and is
returned in calls to {\em getExternalNodes()} and {\em
getNodes()}. A DummyNode is created and added to the external nodes storage object. 
Invokes {\em setDomain(this)} on the DummyNode and {\em domainChanged()} on itself.\\

\indent{\em virtual NodeIter \&getInternalNodeIter();} \\
Returns an iter to the internal nodes of the subdomain, nodes that are
added using the {\em addNode()} command.\\

\indent{\em virtual const ID \&getExternalNodeIter();} \\
Returns an ID identifyng the node numbers of all nodes that have been
added using the {\em addExternalNode(Node *)} method. \\


{\em virtual int computeTang(void);}\\
The method first starts a Timer object running. {\em formTang()}, 
is then invoked on the DomainDecompositionAnalysis object. The
Timer is then stopped and the real time is added to the {\em realCost}. 
Returns the result of invoking {\em formTang}. \\

{\em virtual int computeResidual(void);}\\
The method first starts a Timer object running. {\em formResidual()}, 
is then invoked on the DomainDecompositionAnalysis object. The
Timer is then stopped and the real time is added to the {\em realCost}. 
Returns the result of invoking {\em formResidual}. \\

{\em virtual const Matrix \&getTang(void);    }\\
Returns the Matrix obtained from invoking {\em getTangent()} on
the DomainDecompositionAnalysis object. \\


{\em void setFE\_ElementPtr(FE\_Element *theFEelePtr);}\\
Sets the corresponding {\em FE\_Element} to be that poited to by {\em
theFEelePtr} . \\

{\em const Vector \&getLastExternalSysResponse(void);}\\
Returns the Vector obtained by calling {\em getLastSysResponse()} on
the associated FE\_Element. \\

{\em virtual int computeNodalResponse(void);}\\
To set the nodal responses for the nodes in the subdomain. Returns the
result of invoking {\em computeInternalResponse()} on the DomainDecomposition
analysis object associated with the subdomain.\\ 

{\em double getCost(void); }\\
Returns the current value of {\em realCost}, restting the value of {\em realCost} to be 0.0. The
value of {\em realCost} is added to when {\em computeTang()} and {\em computeREsidual} are invoked on the
Subdomain. \\

\noindent {\bf Protected Member Functions}  \\
\indent{\em FE\_Element *getFE\_ElementPtr(void); }\\
Returns a pointer to the last FE\_Element set using {\em
setFE\_ElementPtr}. If no FE\_Element has been set $0$ is returned. \\








%\pagebreak
%\subsubsection{PartitionedDomain}
%% File: ~/domain/domain/partitioned/PartitionedDomain.tex 
% What: "@(#) PartitionedDomain.tex, revA"

UNDER CONSTRUCTION

\noindent {\bf Files}   \\
\indent \#include $<$/domain/domain/partitioned/PartitionedDomain.h$>$  \\

\noindent {\bf Class Declaration}  \\
\indent class PartitionedDomain: public Domain  \\

\noindent {\bf Class Hierarchy} \\
\indent  Domain \\
\indent\indent {\bf PartitionedDomain} \\


\noindent {\bf Description}  \\
\indent PartitionedDomain is an extension of Domain. A partitioned
domain is an aggregation of subdomains. All elements, nodes, loadcases
are added to the PartitionedDomain. The components can be moved among 
subdomains (keeping in mind that subdomains are themselves subclasses
of domain and therefore must obey the requirements for adding and
removing elements specified in the interface for Domain) by invoking
the {\em remove..()} and {\em add...()} methods on the subdomain. \\


\noindent {\bf Class Interface}  \\
\indent\indent // Constructors  \\
\indent\indent {\em PartitionedDomain(DomainPartitioner \&thePartitioner); } \\
\indent\indent {\em PartitionedDomain(int numNodes, int numElements, int
numLCs, int numSPs, int numMPs, \\ 
\indent\indent\indent\indent int numSubdomains, DomainPartitioner
\&thePartitioner); } \\  \\
\indent\indent //  Destructor  \\
\indent\indent {\em virtual~ $\tilde{}$PartitionedDomain();}  \\ \\
\indent\indent // Public Member Functions - which extend the Domain class \\
\indent\indent {\em virtual int partition(int numPartitions);}\\
\indent\indent {\em virtual bool addSubdomain(Subdomain *theSubdomainPtr);}  \\
\indent\indent {\em virtual int getNumSubdomains(void);}\\
\indent\indent {\em virtual Subdomain *getSubdomainPtr(int tag);}\\
\indent\indent {\em virtual SubdomainIter \&getSubdomains(void);} \\
\indent\indent {\em Node *removeExternalNode(int tag); } \\
\indent\indent {\em Graph \&getSubdomainGraph(void); } \\ \\
\indent\indent // Public Member Functions - inherited from Domain but
rewritten \\
\indent\indent {\em virtual bool addElement(Element *theElementPtr, bool
check = false);}  \\
\indent\indent {\em virtual bool addNode(Node *theNodePtr, bool check = false);}  \\
\indent\indent {\em virtual bool addSP\_Constraint(SP\_Constraint *theSPptr, bool
check = false);}  \\
\indent\indent {\em virtual bool addMP(MP\_Constraint *theMPptr, bool check = false);}  \\
\indent\indent {\em virtual bool addLoadCase(LoadCase *theLCptr);}  \\
\indent\indent {\em virtual ElementIter \&getElements(void);} \\
\indent\indent {\em virtual  Element *getElement(int tag) const;}\\
\indent\indent {\em virtual  Node *getNode(int tag) const;}\\
\indent\indent {\em virtual  LoadCase *getLoadCasePtr(int tag) const;}\\
\indent\indent {\em virtual bool setCurrentLoadCase(int LCtag);} \\
\indent\indent {\em virtual void applyLoad(double time = 0.0,
double loadFactor = 1.0);} \\
\indent\indent {\em virtual void linearize(void);} \\
\indent\indent {\em virtual void commit(void);} \\
\indent\indent {\em virtual int getCurrentLoadCase(void);} \\
\indent\indent {\em virtual int getNumElements(void) const;}\\
\indent\indent {\em virtual int getNumNodes(void) const = 0;}\\
\indent\indent {\em virtual int getNumSPs(void) const;}\\
\indent\indent {\em virtual int getNumMPs(void) const;}\\
\indent\indent {\em virtual int getNumLCs(void) const;}\\    
\indent\indent {\em virtual Domain *getEmptyDomainCopy(void);}\\
\indent\indent {\em virtual Element *removeElement(int tag);}\\
\indent\indent {\em virtual Node *removeNode(int tag, bool checkNeeded
= true);}\\ 
\indent\indent {\em virtual LoadCase *removeLoadCase(int tag);}\\        
\indent\indent {\em virtual SP\_Constraint *removeSP\_Constraint(int
tag);}\\ 
\indent\indent {\em virtual MP\_Constraint *removeMP\_Constraint(int
tag);} \\ \\
\indent\indent // Protected Methods   \\
\indent {\em DomainPartitioner \&getPartitioner(void) const;} \\
{\em virtual int buildEleGraph(Graph \&theEleGraph)} \\



\noindent {\bf Constructors}  \\
\indent {\em PartitionedDomain(DomainPartitioner \&thePartitioner); } \\
Constructs an empty PartitionedDomain. A link with the domain
partitioner {\em thePartitioner} is set up. The {\em thePartitioner}
is used by the domain to partition and load balance the partitioned domain. \\

\indent {\em PartitionedDomain(int numNodes, int numElements, int
numLCs, int numSPs, int numMPs, \\ 
\indent\indent\indent int numSubdomains, DomainPartitioner
\&thePartitioner); } \\ 
Constructs an empty PartitionedDomain, storage is allocated for the components
that are to be added using the estimated number of components passed
as arguments. A link with the domain partitioner {\em thePartitioner}
is set up. The {\em thePartitioner} is used by the domain to partition
and load balance the partitioned domain. \\


\noindent {\bf Destructor}  \\
\indent {\em virtual~ $\tilde{}$PartitionedDomain();}  \\
Deletes the storage components. \\

\noindent {\bf Public Member Functions - which extend the Domain class}  \\
\indent {\em virtual int partition(int numPartitions);}\\
Method which first checks that subdomains with tags 1 through {\em numPartitions} exist in the 
PartitionedDomain. Then it invokes {\em setPartitionedDomain(*this)} on the DomainPartitioner
and finally it returns the result of invoking {\em partition(numPartitions} on
the DomainPartitioner, which will return 0 if successful, a negative number if not. \\

{\em virtual bool addSubdomain(Subdomain *theSubdomainPtr);}  \\
Adds the subdomain pointed to by theSubdomainPtr to the domain. The domain
is responsible for checking that no other subdomain with a similar tag,
has been previously added to the domain. If successful
the domain is responsible for invoking {\em setDomain(this)} on the
Subdomain. The domain is also responsible for invoking {\em
domainChange()}. The call returns {\em false} if the subdomain was not added, 
otherwise {\em true} is returned. \\  

{\em virtual int getNumSubdomains(void);}\\
Method which returns the number of Subdomains (partitions). \\

{\em virtual Subdomain *getSubdomainPtr(int tag);}\\
Returns the Subdomain whose tag is given by {\em tag}. \\

{\em virtual SubdomainIter \&getSubdomains(void);} \\
Returns an iter for the Subdomains of the PartitionedDomain. \\


{\em Node *removeExternalNode(int tag); } \\
A method to remove a Node whose tag is given by {\em tag} from the PartitionedDomain, 
but will not remove the Node from any Subdomains. \\

{\em Graph \&getSubdomainGraph(void); } \\
This will create a new graph each time it is invoked; deleting the old graph. THIS WILL
BE CHANGED.  A vertex is created for each Subdomain, with an edge to each Subdomain the 
Subdomain is connected to, a tag equal to the Subdomain tag,  and a weight equal to the 
result of invoking {\em getCost()} on the Subdomain. \\

\noindent {\bf Public Member Functions - inherited from Domain but rewritten}\\
\indent {\em virtual bool addElement(Element *theElementPtr, bool
check = false);}  \\
To add the element pointed to by theElementPtr to the domain. If {\em check}
is {\em true} the domain is responsible for checking to see that: 1)
no other element with a similar tag, element number, exists in any of
the subdomains. If check is successful the partitioned domain
attempts to add the element to the storage arrey. The call returns
{\em false} if the element was not added, otherwise {\em true} is 
returned.\\   

{\em virtual bool addNode(Node *theNodePtr, bool check = false);}  \\
Adds the node pointed to by theNodePtr to the domain. If {\em check}
is {\em true} the domain is responsible for checking that no other
node with a similar tag, node number, exists in any of the
subdomains. If successful the partition domain attempts to add the
node by invoking {\em Domain::addNode}. The call returns {\em false} if
the node was not added, otherwise {\em true} is returned. \\  


{\em virtual bool addSP\_Constraint(SP\_Constraint *theSPptr, bool
check = false);}  \\
Adds the single point constraint pointed to by theSPptr to the 
domain. The domain performs some checks is {\em check} is true. If
successful the domain adds the constraint using {\em
Domain::addSP\_Constraint()}. The call returns {\em false} if
the constraint was not added, otherwise {\em true} is returned. \\  

{\em virtual bool addMP(MP\_Constraint *theMPptr, bool check = false);}  \\
Adds the multiple point constraint pointed to by theMPptr, to the
domain. The domain performs some checks is {\em check} is true. If
successful the domain adds the constraint using {\em
Domain::addMP\_Constraint()}. The call returns {\em false} if
the constraint was not added, otherwise {\em true} is returned. \\  


{\em virtual bool addLoadCase(LoadCase *theLCptr);}  \\
\\

%%%%%%%% Public Member Functions - ACCESS
\noindent {\bf Pure Virtual Public Member Functions - Access}  \\
\indent {\em virtual ElementIter \&getElements(void);} \\
It returns an {\em PartitionedDomEleIter} for the elements of the domain. This
is an iter which goes through all the subdomains, invoking {\em
getElements()} on the subdomain to get an ElementIter. The
PartitionedDomEleIter uses this iter to go through the elements of the
subdomain until it begins returning $0$; at which point it goes on to
the next subdomain. \\

{\em virtual  Element *getElement(int tag) const;}\\
Returns a pointer to the element whose tag is given by {\em tag}. If
no such element exists $0$ is returned. This is done by invoking {\em
getElement(tag)} on the subdomains until the element is found or no
more subdomains exist; in which case a $0$ is returned. \\

{\em virtual  Node *getNode(int tag) const;}\\
Returns a pointer to the node whose tag is given by {\em tag}. If
no such node exists $0$ is returned. This is done by invoking {\em
getNode(tag)} on the subdomains until the element is found or no
more subdomains exist; in which case a $0$ is returned. \\

{\em virtual  LoadCase *getLoadCasePtr(int tag) const;}\\
Returns a pointer to the element whose tag is given by {\em tag}. If
no such load case exists $0$ is returned. \\

\indent {\em virtual bool setCurrentLoadCase(int LCtag);} \\
Sets the current load case of the domain to be that whose tag is given
by LCtag. It iterates through all the subdomains invoking the same
operation on them. Returns {\em false} if no such load case exists, otherwise
returns {\em true}. \\

\indent {\em virtual void setCurrentTime(double newTime);} \\
Sets the current load case of the domain to be that whose tag is given
by {\em newTime}. It iterates through all the subdomains invoking the
same operation on them. \\

{\em virtual void applyLoad(double time = 0.0, double loadFactor
= 1.0);} \\
The partitioned domain iterates through all the subdomains invoking {\em
applyLoad(double timeStamp)} on them. \\

{\em virtual void linearize(void);} \\
The partitioned domain iterates through all the subdomains invoking {\em
linearize()} on them. \\


{\em virtual void commit(void);} \\
The partitioned domain iterates through all the subdomains invoking {\em
commit()} on them. \\

%%%%%%%% Public Member Functions - QUERY
\noindent {\bf Public Member Functions - Query}  \\
\indent {\em virtual int getCurrentLoadCase(void);} \\
Returns the tag of the current load case set for the domain. If no
load case is set $-1$ is returned. \\

\indent {\em virtual double getCurrentTime(void);} \\
Returns the currentTime set for the domain. If no load case is set $0$
is returned. \\ 

\noindent {\bf Pure Virtual Public Member Functions - Query}  \\
\indent {\em virtual int getNumElements(void) const;}\\
Returns the number of elements in the domain. This number is obtained
by summing the contributions from each subdomain. \\

{\em virtual int getNumNodes(void) const = 0;}\\
Returns the number of nodes in the domain.
This number is obtained by summing the contributions from each subdomain. \\

{\em virtual int getNumSPs(void) const;}\\
Returns the number of SP\_Constraints in the domain. 
This number is obtained by summing the contributions from each subdomain. \\

{\em virtual int getNumMPs(void) const;}\\
Returns the number of MP\_Constraints in the domain. 
This number is obtained by summing the contributions from each subdomain. \\

{\em virtual int getNumLCs(void) const;}\\    
Returns the number of LoadCases in the domain. \
This number is obtained by summing the contributions from each subdomain. \\

{\em virtual Domain *getEmptyDomainCopy(void);}\\
Returns an empty copy of the actual domain.\\

{\em virtual Element *removeElement(int tag);}\\
To remove the element whose tag is given by {\em tag} from the
domain. The method Returns $0$ if no such element exists in the domain. Otherwise 
the domain invokes {\em setDomain(0)} on the element and {\em
setDomainChange(true,true,false)} on itself before a pointer to the
element is returned. \\

{\em virtual Node *removeNode(int tag, bool checkNeeded = true);}\\    
To remove the node whose tag is given by {\em tag} from the domain. 
Returns $0$ if no such node exists in the domain. Otherwise
if the {\em checkNeeded} is {\em true} before the node is removed a
check is made to see that the node is not referenced by any element,
constraint or load. If it is referenced the Node will not be
removed and $0$ is returned. If the node is to be removed the domain
invokes {\em setDomain(0)} on the node and {\em
setDomainChange(true,false,true)} on itself before a pointer to the
Node is returned.\\

{\em virtual LoadCase *removeLoadCase(int tag);}\\        
To remove the load case whose tag is given by {\em tag} from the domain. 
Returns $0$ if the load case was not in the domain, otherwise
returns a pointer to the load case that was removed. Invokes {\em
setDomain(0)} on the load case before it is returned.\\

{\em virtual SP\_Constraint *removeSP\_Constraint(int tag);}\\
To remove the SP\_Constraint whose tag is given by {\em tag} from the
domain. Returns $0$ if the constraint was not in the domain, otherwise
the domain invokes {\em setDomain(0)} on the constraint and {\em
setDomainChange(true,false,false)} on itself before a pointer to the
constraint is returned.\\  

{\em virtual MP\_Constraint *removeMP\_Constraint(int tag);} \\
To remove the MP\_Constraint whose tag is given by {\em tag} from the
domain. Returns $0$ if the constraint was not in the domain, otherwise
the domain invokes {\em setDomain(0)} on the constraint and {\em
setDomainChange(true,false,false)} on itself before a pointer to the
constraint is returned.\\  




\noindent {\bf Protected Member Functions}  \\
\indent {\em DomainPartitioner \&getPartitioner(void) const;} \\
Will return a pointer to the DomainPartitioner object associated with the
PartitionedDomain. \\

{\em virtual int buildEleGraph(Graph \&theEleGraph)} \\
A method which will cause the domain to discard the current element
graph and build a new one based on the element connectivity. Returns
$0$ if successful otherwise $-1$ is returned along with an error
message to opserr. \\


%\pagebreak
%\subsection{DomainPartitioner}
%% File: ~/domain/domain/partitioner/DomainPartitioner.tex 
% What: "@(#) DomainPartitioner.tex, revA"

UNDER CONSTRUCTION

\noindent {\bf Files}   \\
\indent \#include $<\tilde{ }$/domain/partitioner/DomainPartitioner.h$>$  \\

\noindent {\bf Class Decleration}  \\
\indent class DomainPartitioner \\

\noindent {\bf Class Hierarchy} \\
\indent {\bf DomainPartitioner} \\


\noindent {\bf Description}  \\
\indent A DomainPartitioner is an object used to partition and load balance
a PartitionedDomain. The DomainPartitioner uses the element graph of
the domain to partition and load balance. Derived types can use the
node graph of the domain. The partitioner uses a GraphPartitioner and
a LoadBalancingAlgo to partition and load balance the domain. \\

\noindent {\bf Class Interface}  \\
\indent  // Constructors  \\
\indent {\em DomainPartitioner(GraphPartitioner \&theGraphPartitioner,\\
\indent\indent\indent	LoadBalancer \&theLoadBalancer);} \\
\indent {\em DomainPartitioner(GraphPartitioner
\&theGraphPartitioner);}\\ \\
\indent Destructor  \\
\indent {\em virtual~ $\tilde{}$DomainPartitioner();}  \\ \\
\indent // Public Methods  \\
\indent{\em virtual void setPartitionedDomain(PartitionedDomain
\&theDomain);}\\
\indent {\em virtual int partition(int numParts);}\\
\indent {\em virtual void balance(Graph \&theWeightedGraph)} \\ \\
\indent // Public Methods Used by the LoadBalancer  \\
\indent {\em virtual Graph \&getNumPartitions(void)} \\
\indent {\em virtual Graph \&getPartitionGraph(void)} \\
\indent {\em virtual Graph \&getColoredGraph(void); } \\
\indent {\em virtual void swapVertex(int from, int to, int vertexTag,
bool notAdjacentOther = true); } \\ 
\indent {\em virtual void swapBoundary(int from, int to, bool
notAdjacentOther = true); } \\ 
\indent {\em virtual int releaseVertex(int from, \\
\indent\indent\indent\indent\indent\indent int vertexTag, \\
\indent\indent\indent\indent\indent\indent Graph \&theWeightedPartitionGraph, \\
\indent\indent\indent\indent\indent\indent double factorGreater, \\
\indent\indent\indent\indent\indent\indent bool adjacentVertexOonly); } \\
\indent {\em virtual int releaseBoundary(int from, \\
\indent\indent\indent\indent\indent\indent Graph \&theWeightedPartitionGraph, \\
\indent\indent\indent\indent\indent\indent double factorGreater, \\
\indent\indent\indent\indent\indent\indent bool adjacentVertexOonly); } \\



\noindent {\bf Constructors}  \\
\indent {\em DomainPartitioner(GraphPartitioner \&theGraphPartitioner,\\
\indent\indent\indent	LoadBalancer \&theLoadBalancer);} \\
Constructs a DomainPartitioner which will use {\em
theGraphPartitioner} to initially partition the PartitionedDomain
using the element graph and the {\em theLoadBalancer} to load balance 
the PartitionedDomain. The max number of subdomains that the Domain can be 
partitioned is currently set at 8.\\

\indent {\em DomainPartitioner(GraphPartitioner \&theGraphPartitioner);}\\
Constructs a DomainPartitioner which will use {\em
theGraphPartitioner} to initially partition the PartitionedDomain. The
max number of subdomains that the Domain can be partitioned is
currently set at 8.\\ 

\noindent {\bf Destructor}  \\
\indent {\em virtual~ $\tilde{}$DomainPartitioner();}  \\


\noindent {\bf Public Member Functions}  \\
\indent{\em virtual void setPartitionedDomain(PartitionedDomain
\&theDomain);}\\
Sets the link with the PartitionedDomain that is to be partitioned. \\

{\em virtual int partition(int numParts);}\\
Method invoked to partition the Domain. It first checks to see that
the PartitionedDomain has at least {\em numParts} Subdomains, with tags
1 through {\em numParts}; if not prints an error message and returns -1. 
It then asks the domain for the element graph. This graph is then partitioned 
using the GraphPartitioner into {\em numParts}; if partitioning fails an error 
message is printed and  $-10 +$ number returned from GraphPartitioner is
returned. If successfull the domain is partitioned according to the
following rules: \begin{itemize}
\item All nodes which are internal to a partition are added using the
{\em addNode()} method of the Subdomain. These nodes are removed from
the PartionedDomain using {\em removeNode()}. 
\item External nodes (these are nodes shared across partitions as a
result of element connectivity or MP\_Constraints are added to those
Subdomains whose elements reference them. They are added using the
{\em addExternalNode()} command. 
\item SP\_Constraints whose node is interior to a Subdomain are removed
from the PartitionedDomain and added to the Subdomain. 
\item MP\_Constraints whose two nodes are interior to a Subdomain are
removed from the PartitionedDomain and added to the Subdomain.
\item The elements are sent to the partition whose tag is given by the
color of the vertex in the partitioned (colored) element graph. The
elements are removed from the PartitionedDomain using {\em
removeElement()} and added to the Subdomain using {\em addElement()}.
\item For the loads, a check is made to ensure that each Subdomain has
a LoadCase with a tag equal to the tags in the LoadCases that have
been added to the PartitionedDomain; if not new LoadCases are created
and added to the Subdomain. It then iterates through all the
NodalLoads in the LoadCases in the PartionedDomain, if the
corresponding node is external the NodalLoad is removed and added to
the corresponding LoadCase in the Subdomain. ELEMENTAL LOADS are not
yet dealt with. 
\end{itemize}

The DomainPartitioner invokes {\em hasDomainChanged()} on each Subdomain; if the Subdomain 
has changed {\em invokeChangeOnAnalysis()}. Finally {\em
hasDomainChanged()} is invoked on the PartitionedDomain; if it has
changed {\em invokeChangeOnAnalysis()}. {\em partitionFlag} is set to
true. \\ 


{\em virtual void balance(Graph \&theWeightedGraph)} \\
Checks first to see that the {\em partitionFlag} has been set; if it
hasn'nt an error message is printed and a $-1$ is returned. If a
LoadBalancer was passed in the constructor {\em balance()} is invoked
on this object; if no LoadBalancer was passed nothing is done and
method returns$0$. If balancing is performed, the DomainPartitioner
invokes {\em hasDomainChanged()} on each Subdomain; if the Subdomain
has changed {\em invokeChangeOnAnalysis()}. Finally {\em
hasDomainChanged()} is invoked on the PartitionedDomain; if it has
changed {\em invokeChangeOnAnalysis()}. {\em partitionFlag} is set to
true. \\ 

Method which invokes {\em setPartitioner(this)} on the
LoadBalancingAlgo. It then invokes {\em balance(load)} on this
object, where {\em load} is vector of size {\em numParts}
containing the load of each subdomain. \\ 

\noindent {\bf Public Member Functions Used by the Balancing Algorithm}  \\
\indent {\em virtual Graph \&getNumPartitions(void)} \\
Returns the number of partitions in the PartitionedDomain. \\

\indent {\em virtual Graph \&getPartitionGraph(void)} \\
Method which returns the partition graph. This is a graph with a
vertex for every partition and an edge between partitions if there
exists an element in one partition which is connected to an element
in the other partition. \\

{\em virtual Graph \&getColoredGraph(void); } \\
A method which returns the current colored graph representing the
partitioning of the elements in the subdomains. Does this by invoking
{\em getElementGraph()} on the PartitionedDomain. Note that this is
the same graph that was colored by the DomainPartitioner in
partitioning the PartitionedDomain. \\ 

{\em virtual void swapVertex(int from, int to, int vertexTag, bool
notAdjacentOther = true); } \\ 
Method which will take the element given by vertex reference of the
vertex whose tag is given by {\em vertexTag} from subdomain {\em from}
and place it in subdomain {\em to}. If {\em notAdjacentOther} is {\em
true} a check is made to ensure that the vertex to be swapped is not
adjacent to a vertex in any other partition. Returns $0$ if
successfull, returns an error message and $-1$ if PartitionedDomain
has not been partitioned, $-2$ if {\em from} Subdomain does not exist,
$-3$ if {\em to Subdomain} does not exist, $-4$ if a vertex given by
{\em tag} does not exist, returns $-5$ if {\em notAdjacentOther} was
true and the vertex was adjacent to a vertex in another partition, and
returns $-6$ if no Element with a tag similar to {\em tag} exists
(this should not happen if element graph is built correctly). \\
The Element, Nodes, NodalLoads, SP\_Constraints and MP\_Constraints
that need to be moved between the two Subdomains, or between the
PartitionedDomain and Subdomains are also moved. NO ELEMENTAL LOADS 
are moved yet. \\


{\em virtual void swapBoundary(int from, int to, bool notAdjacentOther
= true); } \\ 
Method which when invoked will take all the boundary elements in
subdomain {\em from} that are connected to elements in subdomain {\em
to} and place them in subdomain {\em to}. If {\em adjacentVertexOther} is
{\em true} no Elements which are connected to elements in subdomains other 
than {\em to} and {\em from} are moved. Returns $0$ if successfull, returns 
an error message and $-1$ if PartitionedDomain has not been partitioned, $-2$ 
if {\em from} Subdomain does not exist, $-3$ if {\em to Subdomain}
does not exist. \\ The Elements, Nodes, NodalLoads, SP\_Constraints
and MP\_Constraints that need to be moved between the two Subdomains,
or between the PartitionedDomain and Subdomains are also moved. NO
ELEMENTAL LOADS are moved yet. It performs the operation by creating
an ID of vertices and then using code similar to that used in {\em
swapVertex()}; {\em swapVErtex()} is not called repeatedly as this was
found to be too slow. \\ 

{\em virtual int releaseVertex(int from, \\
\indent\indent\indent\indent\indent int vertexTag, \\
\indent\indent\indent\indent\indent Graph \&theWeightedPartitionGraph, \\
\indent\indent\indent\indent\indent double factorGreater, \\
\indent\indent\indent\indent\indent bool adjacentVertexOonly); } \\
Method which when invoked will take the element given by vertex
reference of the vertex whose tag is given by {\em vertexTag} from
subdomain {\em from} and place it in the subdomain to which it is most
attracted (to which it is most connected). If it is equally attracted
to two subdomains it is sent to the one with the lightest load (the
loads on the subdomains are given in the {\em
theWeightedPartitionGraph}. If the {\em mustReleaseToLighter} is {\em
true} a check is first made to see if the load on the intended
subdomain is lighter than the load in {\em from} and that the ratio in
load between from and the new domain is greater than {\em
factorGreater}; if this is the case the element is swapped, if not the
element is not swapped. An additional requirement that the vertex that
is to be swapped is not adjacent to any other partition can also be
set. \\ 
The method determines which partition the vertex is to be sent to and
then returns the result of invoking {\em swapVertex()} on itself, with
this partition used as the {\em to} argument in the arguments. \\

{\em virtual int releaseBoundary(int from, \\
\indent\indent\indent\indent\indent Graph \&theWeightedPartitionGraph, \\
\indent\indent\indent\indent\indent double factorGreater, \\
\indent\indent\indent\indent\indent bool adjacentVertexOonly); } \\
Method which when invoked will release all the elements on the
boundary of subdomain {\em from}. It performs the operation by
creating an ID of all the vertices on the boundary of the {\em from}
Subdomain. Then {\em releaseBoundary()} is invoked on all these vertices. \\




%\pagebreak
%\subsection{{\bf LoadBalancer}}
%% File: ~/domain/domain/loadBalancer/LoadBalancer.tex 
% What: "@(#) LoadBalancer.tex, revA"

\noindent {\bf Files}   \\
\indent \#include $<\tilde{ }$/domain/loadBalancer/LoadBalancer.h$>$  \\

\noindent {\bf Class Decleration}  \\
\indent class LoadBalancer \\

\noindent {\bf Class Hierarchy} \\
\indent {\bf LoadBalancer} \\


\noindent {\bf Description}  \\
\indent A LoadBalancer is an object used to balance a
PartitionedDomain. The LoadBalancer does this by invoking methods in
the DomainPartitioner object with which it is associated. \\

\noindent {\bf Class Interface}  \\
\indent\indent  // Constructor  \\
\indent\indent {\em LoadBalancer(); }\\ \\
\indent\indent // Destructor  \\
\indent\indent {\em virtual~ $\tilde{}$LoadBalancer();}  \\ \\
\indent\indent // Public Methods  \\
\indent\indent {\em virtual void setLinks(DomainPartitioner
\&thePartitioner);} \\ 
\indent\indent {\em virtual int balance(Graph \&theWeightedGraph) =0;} \\ \\
\indent\indent // Public Methods  \\
\indent\indent {\em DomainPartitioner *getDomainPartitioner(void); }\\


\noindent {\bf  Constructor  }\\
\indent {\em LoadBalancer(); }\\ 
Sets the pointer to the assocaited PartitionedDomain to be $0$. \\

\noindent {\bf Destructor } \\
\indent {\em virtual~ $\tilde{}$LoadBalancer();}  \\ 
Does nothing. Provided so the subclasses destructor will be called. \\

\noindent {\bf  Public Methods} \\
\indent {\em virtual void setLinks(DomainPartitioner
\&thePartitioner);} \\
Sets the pointer to the DomainPartitioner object associated with the
LoadBalancer to point to {\em thePartitioner}. \\

{\em virtual int balance(Graph \&theWeightedGraph) =0;} \\ 
Each subclass must provide an implementation of this method. This
method is invoked to balance the PartitionedDomain. The Graph {\em
theWeightedGraph} is a weighted graph in which the vertices represent
the Subdomains, the edges Subdomains sharing common boundary nodes and
the weight the cost of the previous Subdomain calculations (determined
by invoking {\em getCost()} on the Subdomains). \\


\noindent {\bf Protected Methods}  \\
\indent {\em DomainPartitioner *getDomainPartitioner(void); }\\
Returns a pointer to the DomainPartitioner. If no DomainPartitioner
has been set, a warning message is printed and $0$ returned. \\






%\pagebreak
%\subsubsection{ShedHeaviest}
%% File: ~/domain/domain/loadBalancer/ShedHeaviest.tex 
% What: "@(#) ShedHeaviest.tex, revA"

\noindent {\bf Files}   \\
\indent \#include $<\tilde{ }$/domain/loadBalancer/ShedHeaviest.h$>$  \\

\noindent {\bf Class Decleration}  \\
\indent class ShedHeaviest: public LoadBalancer \\

\noindent {\bf Class Hierarchy} \\
\indent  LoadBalancer \\
\indent\indent {\bf ShedHeaviest} \\


\noindent {\bf Description}  \\
\indent A ShedHeaviest is an object used to balance a
PartitionedDomain. It does this by shedding the boundary vertices on
the heaviest loaded partition (subdomain). \\

\noindent {\bf Class Interface}  \\
\indent\indent  // Constructors  \\
\indent\indent {\em ShedHeaviest(); }\\ 
\indent\indent {\em ShedHeaviest(double factorGreater, int
numReleases);} \\ \\
\indent\indent // Destructor  \\
\indent\indent {\em virtual~ $\tilde{}$ShedHeaviest();}  \\ \\
\indent\indent // Public Methods  \\
\indent\indent {\em virtual int balance(Graph \&theWeightedGraph) =0;} \\ \\


\noindent {\bf  Constructors  }\\
\indent {\em ShedHeaviest(); }\\ 
Sets {\em numRealeases} to $1$ and  {\em factorGreater} to
$1.0$. These are the paramemeters used in the {\em balance()}
method. \\ 

\indent {\em ShedHeaviest(double factorGreater, int numReleases);} \\
Sets the parameters used in the {\em balance()} method. \\

\noindent {\bf Destructor } \\
\indent {\em virtual~ $\tilde{}$ShedHeaviest();}  \\ 
Does nothing. \\

\noindent {\bf  Public Methods} \\
\indent {\em virtual int balance(Graph \&theWeightedGraph) =0;} \\ 
The heaviest loaded partition, {\em max}, is first determined by
iterating through the Graph {\em theWeightedGraph} looking at the
vertex weights. Then {\em releaseBoundary(max, theWieightedGraph,
true, factorGreater)} is invoked on the
DomainPartitioner {\em numRelease} times. Returns $0$ if successfull,
otherwise a negative number and a warning message are returned if
either no link has been set to the DomainPartitioner or {\em
releaseBoundary()} returns a negative number. \\





%\pagebreak
%\subsubsection{SwapHeavierToLighterNeighbours}
%% File: ~/domain/domain/loadBalancer/SwapHeavierToLighterNeighbours.tex 
% What: "@(#) SwapHeavierToLighterNeighbours.tex, revA"

\noindent {\bf Files}   \\
\indent \#include $<\tilde{ }$/domain/loadBalancer/SwapHeavierToLighterNeighbours.h$>$  \\

\noindent {\bf Class Declaration}  \\
\indent class SwapHeavierToLighterNeighbours: public LoadBalancer \\

\noindent {\bf Class Hierarchy} \\
\indent  LoadBalancer \\
\indent\indent {\bf SwapHeavierToLighterNeighbours} \\


\noindent {\bf Description}  \\
\indent A SwapHeavierToLighterNeighbours is an object used to balance a
PartitionedDomain. It does this by shedding the boundary vertices on
the heaviest loaded partition (subdomain). \\

\noindent {\bf Class Interface}  \\
\indent\indent  // Constructors  \\
\indent\indent {\em SwapHeavierToLighterNeighbours(); }\\ 
\indent\indent {\em SwapHeavierToLighterNeighbours(double
factorGreater, int numReleases);} \\ \\
\indent\indent // Destructor  \\
\indent\indent {\em virtual~ $\tilde{}$SwapHeavierToLighterNeighbours();}  \\ \\
\indent\indent // Public Methods  \\
\indent\indent {\em virtual int balance(Graph \&theWeightedGraph) =0;} \\ \\


\noindent {\bf  Constructors  }\\
\indent {\em SwapHeavierToLighterNeighbours(); }\\ 
Sets {\em numRealeases} to $1$ and {\em factorGreater} to $1.0$. These
are the parameters used in the {\em balance()} method. \\

\indent {\em SwapHeavierToLighterNeighbours(double factorGreater, int
numReleases);}\\ 
Sets the parameters used in the {\em balance()} method. \\

\noindent {\bf Destructor } \\
\indent {\em virtual~ $\tilde{}$SwapHeavierToLighterNeighbours();}  \\ 
Does nothing. \\

\noindent {\bf  Public Methods} \\
\indent {\em virtual int balance(Graph \&theWeightedGraph) =0;} \\ 
For {\em numRelease} times the Vertices of {\em theWeightedGraph} are
iterated through. For each Vertex, $i$, the weight of the Vertex is compared
to those of it's adjacent Vertices, $other$, (this is done by looping through
the adjacency ID of the Vertex), if the vertex weight is {\em
factorGreater} times greater than the other Vertices load then {\em
swapBoundary(i, other)} is invoked on the DomainPartitioner. Returns
$0$ if successful, otherwise a negative number and a warning message
are returned if either no link has been set to the DomainPartitioner
or {\em swapBoundary()} returns a negative number. \\





%\pagebreak
%\subsubsection{ReleaseHeavierToLighterNeighbours}
%% File: ~/domain/domain/loadBalancer/ReleaseHeavierToLighterNeighbours.tex 
% What: "@(#) ReleaseHeavierToLighterNeighbours.tex, revA"

\noindent {\bf Files}   \\
\indent \#include $<\tilde{
}$/domain/loadBalancer/ReleaseHeavierToLighterNeighbours.h$>$  \\ 

\noindent {\bf Clas Declaration}  \\
\indent clas ReleaseHeavierToLighterNeighbours: public LoadBalancer \\

\noindent {\bf Clas Hierarchy} \\
\indent  LoadBalancer \\
\indent\indent {\bf ReleaseHeavierToLighterNeighbours} \\


\noindent {\bf Description}  \\
\indent A ReleaseHeavierToLighterNeighbours is an object used to balance a
PartitionedDomain. It does this by shedding the boundary vertices on
the heaviest loaded partition (subdomain). \\

\noindent {\bf Clas Interface}  \\
\indent\indent  // Constructors  \\
\indent\indent {\em ReleaseHeavierToLighterNeighbours(); }\\ 
\indent\indent {\em ReleaseHeavierToLighterNeighbours(double
factorGreater, int numReleases);} \\ \\ 
\indent\indent // Destructor  \\
\indent\indent {\em virtual~ $\tilde{}$ReleaseHeavierToLighterNeighbours();}\\\\
\indent\indent // Public Methods  \\
\indent\indent {\em virtual int balance(Graph \&theWeightedGraph) =0;} \\ \\


\noindent {\bf  Constructors  }\\
\indent {\em ReleaseHeavierToLighterNeighbours(); }\\ 
Sets {\em numRealeases} to $1$ and {\em factorGreater} to $1.0$. These
are the paramemeters used in the {\em balance()} method. \\

\indent {\em ReleaseHeavierToLighterNeighbours(double factorGreater,
int numReleases);} \\ 
Sets the parameters used in the {\em balance()} method. \\

\noindent {\bf Destructor } \\
\indent {\em virtual~ $\tilde{}$ReleaseHeavierToLighterNeighbours();}  \\ 
Does nothing. \\

\noindent {\bf  Public Methods} \\
\indent {\em virtual int balance(Graph \&theWeightedGraph) =0;} \\ 
For {\em numRelease} times the Vertices of {\em theWeightedGraph} are
iterated through. For each Vertex, $i$, {\em releaseBoundary(i,
theWeightedGraph, true, factorGreater)} is invoked on
the DomainPartitioner. Returns $0$ if successful, otherwise a negative
number and a warning message are returned if either no link has been
set to the DomainPartitioner or {\em releaseBoundary()} returns a
negative number. \\ 









\pagebreak
\section{Analysis Classes}
To facilitate code re-use and to provide for a design which is both
flexible and extensible, object-oriented design principles can be
applied to the analysis algorithm. This is first done by identifying
the main tasks performed in a finite element analysis, abstracting
them into separate classes, and then specifying the interface for
these classes. It is important that the interfaces specified allow the
classes to work together to perform the analysis and allow new classes
to be introduced without the need to change existing classes. 
In this design an {\bf Analysis} object is an aggregation of objects
of the following types: 
\begin{enumerate}   

\item {\bf SolnAlgorithm}: The solution algorithm object is responsible for 
orchestrating the steps performed in the analysis.

\item {\bf AnalysisModel}: The AnalysisModel object is a container class for
storing and providing access to the following types of objects:
\begin{enumerate}
\item {\bf DOF\_Group}: The {\bf DOF\_Group} objects represent the
degrees-of-freedom at the {\bf Node}s or new degrees-of-freedom
introduced into the analysis to enforce the constraints.
\item {\bf FE\_Element}: The {\bf FE\_Element} objects represent the
{\bf Elements} in the {\bf Domain} or they are introduced to add stiffness
and/or load to the system of equations in order to enforce the
constraints. 
\end{enumerate}
The {\bf FE\_Element}s and {\bf DOF\_Group}s are important to the
design because: \begin{enumerate}
\item They remove from the {\bf Node} and {\bf Element} objects the
need to worry about the mapping between degrees-of-freedoms and
equation numbers. 
\item They also remove from the {\bf Node} and {\bf Element} class
interfaces methods for forming tangent and residual vectors, that are
used to form the system of equations. 
\item The subclasses of {\bf FE\_Element} and {\bf DOF\_Group} are
responsible for 
handling the constraints. This removes from the rest of the objects
the analysis aggregation the need to deal with the constraints. 
\end{enumerate}

\item {\bf Integrator}: The {\bf Integrator} object is responsible for
defining the contributions of the {\bf FE\_Element}s and {\bf
DOF\_Group}s to the system of equations and for updating the response
quantities at the {\bf DOF\_Group}s with the appropriate values given
the solution to the system of equations. 

\item {\bf ConstraintHandler}: The {\bf ConstraintHandler} object is
responsible for handling the constraints. It does this by creating
{\bf FE\_Element}s and {\bf DOF\_Group}s of the correct type.

\item {\bf DOF\_Numberer}: The {\bf DOF\_Numberer} object is
responsible for mapping equation numbers in the system of equations to
the degrees-of-freedom in the {\bf DOF\_Group}s. 

\end{enumerate}

\pagebreak

\subsection{{\bf Analysis}}
%File: ~/OOP/analysis/analysis/Analysis.tex
%What: "@(#) Analysis.tex, revA"

\noindent {\bf Files}   \\
\indent \#include $<\tilde{ }$/analysis/analysis/Analysis.h$>$  \\

\noindent {\bf Class Declaration}  \\
\indent class Analysis;  \\

\noindent {\bf Class Hierarchy} \\
\indent {\bf Analysis} \\

\noindent {\bf Description} \\ 
\indent The Analysis class is an abstract base class. Each Analysis
object will be associated with a single Domain, the Domain upon which
it will perform the analysis operations. The base Analysis class holds
a pointer to this Domain and will return this pointer to subclasses. \\

\noindent {\bf Class Interface} \\ 
\indent // Constructors  \\
\indent {\em Analysis(theDomain \&theDomain);}\\  \\
\indent // Destructor \\
\indent {\em virtual~ $\tilde{}$Analysis();}\\ \\
\indent // Pure Virtual Public Member Functions \\ 
\indent {\em virtual int analyze(void) = 0;} \\
\indent {\em virtual int domainChanged(void) = 0;} \\ \\
\indent // Protected Method \\
\indent {\em Domain *getDomainPtr(void);} \\


\noindent {\bf Constructors} 
\\ \indent {\em Analysis(theDomain \&theDomain);}\\ 
All analysis are associated with a single domain, this constructor
sets up the link between the analysis and the domain. \\

\noindent {\bf Destructor} \\
\indent {\em virtual~ $\tilde{}$Analysis();}\\ 
Does nothing. Provided so that the subclasses destructor will be
invoked. \\

\noindent {\bf Public Methods}\\
\indent {\em virtual int analyze(void) = 0;} \\
Invoked to perform the analysis on the domain, this is a pure virtual
function, i.e. all subclasses or their descendents must implement this
routine. Returns 0 if successful; a negative integer if not; the value
depends on the particular analysis class. \\

\indent {\em virtual int domainChanged(void) = 0;} \\
Invoked to inform the analysis that the finite element model has
changed, for example when new elements have been added. It is also a
virtual function. To return $0$ if successful, a negative number if
not.\\ 

\noindent {\bf Protected Methods}\\
\indent {\em Domain *getDomainPtr(void);} \\
Returns a pointer to the domain that was passed in the constructor. \\






\pagebreak \subsubsection{StaticAnalysis}
%File: ~/OOP/analysis/analysis/StaticAnalysis.tex
%What: "@(#) StaticAnalysis.tex, revA"

\noindent {\bf Files}   \\
\indent \#include $<\tilde{ }$/analysis/analysis/StaticAnalysis.h$>$  \\

\noindent {\bf Class Declaration}  \\
\indent class StaticAnalysis: public Analysis;  \\

\noindent {\bf Class Hierarchy} \\
\indent Analysis \\
\indent\indent {\bf StaticAnalysis} \\

\noindent {\bf Description} \\ 
\indent StaticAnalysis is a subclass of Analysis, it is used to
perform a static analysis on the Domain. The following are the
aggregates of such an analysis type: 
\begin{itemize}
\item {\bf AnalysisModel} - a container class holding the FE\_Element
and DOF\_Group objects created by the ConstraintHandler object. 
\item {\bf ConstraintHandler} - a class which creates the DOF\_Group
and FE\_Element objects, the type of objects created depending on how
the specified constraints in the domain are to be handled. 
\item {\bf DOF\_Numberer} - a class responsible for providing equation
numbers to the individual degrees of freedom in each DOF\_Group object.
\item {\bf LinearSOE} - a numeric class responsible for the creation
and subsequent solution of large systems of linear equations of the
form $Ax = b$, where $A$ is a matrix and $x$ and $b$ are vectors.
\item {\bf StaticIntegrator} - an algorithmic class which provides
methods which are invoked by the FE\_Element to determine their
current tangent and residual matrices; that is this is the class that
sets up the system of equations.  It also provides the {\em
update()} method which is invoked to set up the appropriate dof
response values once the solution algorithm has formed and solved the
system of equations.
\item {\bf EquiSolnAlgo} - an algorithmic class specifying the
sequence of operations to be performed in setting up and solving the
finite element equation which can be represented by the equation K(U)
U = P(U). 
\end{itemize}

\noindent {\bf Class Interface} \\
\indent // Constructors \\
\indent {\em StaticAnalysis(Domain \&theDomain, \\
\indent\indent\indent\indent ConstraintHandler \&theHandler, \\
\indent\indent\indent\indent DOF\_Numberer \&theNumberer, \\
\indent\indent\indent\indent AnalysisModel \&theModel,\\
\indent\indent\indent\indent EquiSolnAlgo \&theSolnAlgo,\\
\indent\indent\indent\indent LinearSOE \&theSOE, \\
\indent\indent\indent\indent StaticIntegrator \&theIntegrator \\
\indent\indent\indent\indent int numIncrements = 1);} \\ \\
\indent // Destructor \\
\indent {\em virtual~ $\tilde{}$StaticAnalysis();}\\  \\
\indent // Public Methods\\
\indent {\em int analyze(void);} \\
\indent {\em void clearAll(void);} \\
\indent {\em int domainChange(void);}\\  \\
\indent // Public Methods to vary the type of Analysis\\
\indent {\em int setNumIncrements(int numIncrements);}\\ 
\indent {\em int setAlgorithm(EquiSolnAlgo \&theAlgorithm);} \\
\indent {\em int setIntegrator(StaticIntegrator \&theIntegrator);}\\
\indent {\em int setLinearSOE(LinearSOE \&theSOE);} \\

\noindent {\bf Constructors} \\
\indent {\em StaticAnalysis(Domain \&theDomain, \\
\indent\indent\indent\indent\indent ConstraintHandler \&theHandler, \\
\indent\indent\indent\indent\indent DOF\_Numberer \&theNumberer, \\
\indent\indent\indent\indent\indent AnalysisModel \&theModel,\\
\indent\indent\indent\indent\indent EquiSolnAlgo \&theSolnAlgo,\\
\indent\indent\indent\indent\indent LinearSOE \&theSOE, \\
\indent\indent\indent\indent\indent StaticIntegrator \&theIntegrator, \\
\indent\indent\indent\indent\indent int numIncrements = 1);} \\ \\
The constructor is responsible for setting the links between the
objects in the aggregation. To do this it invokes {\em
setLinks(theDomain)} on {\em theModel}, {\em
setLinks(theDomain,theModel,theIntegrator)} on {\em theHandler}, 
{\em setLinks(theModel)} on {\em theNumberer}, {\em
setLinks(theModel, theSOE)} on {\em theIntegrator}, and  {\em
setLinks(theModel,theAnalysis, theIntegrator, theSOE)} on {\em
theSolnAlgo}.  The constructor also sets the number of analysis steps
that will be performed to be {\em numIncrements}.\\


\noindent {\bf Destructor} \\
\indent {\em $\tilde{ }$StaticAnalysis();}\\ 
Does nothing. {\em clearAll()} must be invoked if the destructor on
the objects in the aggregation need to be invoked. \\

\noindent {\bf Public Member Functions}\\
\indent {\em int analyze(void);} \\
Invoked to perform a static analysis on the FE\_Model. The analysis 
The StaticAnalysis object performs the following:
\begin{tabbing}
while \= \+ while \= while \= \kill
    for (int i=0; i$<$ numIncrements; i++) \{ \+ \\
    if (theDomain-$>$hasDomainChanged() == true) \+\\
	this-$>$domainChanged(); \- \\
    theIntegrator.newStep(); \\
    theSolnAlgo.solveCurrentStep(); \\
    theIntegrator.commit(); \- \\ \} \\
\end{tabbing}
\noindent The type of analysis performed, depends on the type of the
objects in the analysis aggregation. If any of the methods invoked
returns a negative number, an error message is printed, {\em
revertToLastCommit()} is invoked on the Domain, and a negative number
is immediately returned. Returns a $0$ if the algorithm is successful. \\

{\em void clearAll(void);} \\
Will invoke the destructor on all the objects in the aggregation. NOTE
this means they must have been constructed using {\em new()},
otherwise a segmentation fault can occur.\\

{\em void domainChange(void);}\\
This is a method invoked by the analysis during the analysis method if
the Domain has changed. The method invokes the following:
\begin{enumerate} 
\item It invokes {\em clearAll()} on {\em theModel} which causes the
AnalysisModel to clear out its list of FE\_Elements and DOF\_Groups,
and {\em clearAll()} on {\em theHandler}.
\item It then invokes {\em handle()} on {\em theHandler}. This causes
the constraint handler to recreate the appropriate FE\_Element and
DOF\_Groups to perform the analysis subject to the boundary conditions
in the modified domain.
\item It then invokes {\em number()} on {\em theNumberer}. This causes
the DOF numberer to assign equation numbers to the individual
dof's. Once the equation numbers have been set the numberer then
invokes {\em setID()} on all the FE\_Elements in the model. Finally
the numberer invokes {\em setNumEqn()} on the model.
\item It invokes {\em setSize(theModel.getDOFGraph())} on {\em
theSOE} which causes the system of equation to determine its size
based on the connectivity of the dofs in the analysis model. 
\item Finally {\em domainChanged()} is invoked on both {\em theIntegrator} and 
{\em theAlgorithm}. 
Returns $0$ if successful. At any stage above, if an error occurs the
method is stopped, a warning message is printed and a negative number
is returned. \\ 
\end{enumerate}

\indent {\em int setNumIncrements(int numIncrements);}\\ 
To set the number of incremental steps in the analysis to be {\em
numIncrements}. Returns $0$.\\

\indent {\em int setAlgorithm(EquiSolnAlgo \&newAlgorithm);}\\
To change the algorithm between analysis. It first invokes the
destructor on the old SolutionAlgorithm object associated with the
analysis. It then sets the SolutionAlgorithm 
associated with the analysis to be {\em newAlgorithm} and sets the
links for this object by invoking {\em setLinks()}. Checks then to
see if the domain has changed, if true it invokes {\em
domainChanged()}, otherwise it invokes {\em domainChanged()} on the
new SolutionAlgorithm. Returns $0$ if successful, a warning message
and a negative number if not.\\

\indent {\em int setIntegrator(StaticIntegrator \&newIntegrator);}\\
To change the integration scheme between analysis. It first invokes the
destructor on the old Integrator object associated with the
analysis. It then sets the SolutionAlgorithm 
associated with the analysis to be {\em newAlgorithm} and sets the
links for this object by invoking {\em setLinks()}. It also invokes
{\em setLinks()} on the ConstraintHandler and SolutionAlgorithm
objects. Checks then to see if the domain has changed, if true it
invokes {\em domainChanged()}, otherwise it invokes {\em
domainChanged()} on the new Integrator. Returns $0$ if
successful, a warning message and a negative number if not.\\

\indent {\em int setLinearSOE(LinearSOE \&newSOE);}\\
To change the linear system of equation object between analysis. It
first invokes the destructor on the old LinearSOE object associated
with the analysis. It then sets the SolutionAlgorithm 
associated with the analysis to be {\em newSOE}.
links for this object by invoking {\em setLinks()}. It then invokes
{\em setLinks()} on the ConstraintHandler and SolutionAlgorithm
objects. Checks then to see if the domain has changed, if true it
invokes {\em domainChanged()}, otherwise it invokes {\em
setSize()} on the new LinearSOE. Returns $0$ if successful, a warning
message and a negative number if not.\\ 




\pagebreak \subsubsection{{\bf TransientAnalysis}}
%File: ~/OOP/analysis/analysis/TransientAnalysis.tex
%What: "@(#) TransientAnalysis.tex, revA"

\noindent {\bf Files}   \\
\indent \#include $<\tilde{ }$/analysis/analysis/TransientAnalysis.h$>$  \\

\noindent {\bf Class Declaration}  \\
\indent class TransientAnalysis: public Analysis;  \\

\noindent {\bf Class Hierarchy} \\
\indent Analysis \\
\indent\indent {\bf TransientAnalysis} \\

\noindent {\bf Description} \\ 
\indent The TransientAnalysis class is an abstract class. Its purpose is
to define the interface common among all subclasses. A TransientAnalysis
object is responsible for performing a transient analysis on the domain. \\


\noindent {\bf Class Interface} \\
\indent // Constructor \\
\indent {\em TransientAnalysis(Domain \&theDomain, double tStart,
double tFinish);}\\  
\indent // Destructor \\
\indent {\em virtual~$\tilde{}$TransientAnalysis();}\\ \\
\indent // Public Methods\\
\indent {\em virtual void setTimeStart(double tStart);}\\
\indent {\em virtual void setTimeFinish(double tFinish);}\\ \\
\indent // Protected Data \\
\indent {\em double tStart} \\
\indent {\em double tFinish} \\

\noindent {\bf Constructor} \\
\indent {\em TransientAnalysis(Domain \&theDomain, double tStart,
double tFinish);}\\  
The Domain {\em theDomain} is passed to the Analysis classes
constructor. Sets the starting time and finishing time for the
transient analysis to {\em tStart} and {\em tFinish}. \\

\noindent {\bf Destructor} \\
\indent {\em virtual~$\tilde{}$TransientAnalysis();}\\ 
Does nothing. \\

\noindent {\bf Pure Virtual Public Member Functions}\\
\indent {\em virtual int analyze(void) =0;} \\
Invoked to perform a dynamic analysis on the model. The type of analysis
performed, depends on the type of the objects in the analysis
aggregation. Returns a $0$ if successful, otherwise a negative number
is returned; the value of which depends on the type of the analysis. \\

{\em virtual void setTimeStart(double tStart);}\\
To set the starting time of the TransientAnalysis to {\em tStart}. It
invokes {\em setCurrntTime(tStart)} on the associated domain object.\\

{\em virtual void setTimeFinish(double tFinish);}\\
To set the finishing time of the TransientAnalysis to {\em tFinish}. \\

\noindent {\bf Protected Member Data}\\
\indent {\em double tStart, tFinish} 


\pagebreak \subsubsection{DirectIntgrationAnalysis}
%File: ~/OOP/analysis/analysis/DirectIntegrationAnalysis.tex
%What: "@(#) DirectIntegrationAnalysis.tex, revA"

\noindent {\bf Files}   \\
\indent \#include $<\tilde{ }$/analysis/analysis/DirectIntegrationAnalysis.h$>$  \\

\noindent {\bf Class Declaration}  \\
\indent class DirectIntegrationAnalysis: public TransientAnalysis;  \\

\noindent {\bf Class Hierarchy} \\
\indent Analysis \\
\indent\indent TransientAnalysis \\
\indent\indent\indent {\bf DirectIntegrationAnalysis} \\

\noindent {\bf Description} \\ 
\indent DirectIntegrationAnalysis is a subclass of TransientAnalysis. It
is used to perform a transient analysis using an incremental approach
on the Domain. The following are the aggregates of such an analysis type: 
\begin{itemize}
\item {\bf AnalysisModel} - a container class holding the FE\_Element
and DOF\_Group objects created by the ConstraintHandler object. 
\item {\bf ConstraintHandler} - a class which creates the DOF\_Group
and FE\_Element objects, the type of objects created depending on how
the specified constraints in the domain are to be handled. 
\item {\bf DOF\_Numberer} - a class responsible for providing equation
numbers to the individual degrees of freedom in each DOF\_Group object.
\item {\bf LinearSOE} - a numeric class responsible for the creation
and subsequent solution of large systems of linear equations of the
form $Ax = b$, where $A$ is a matrix and $x$ and $b$ are vectors.
\item {\bf TransientIntegrator} - an algorithmic class which provides
methods which are invoked by the FE\_Element to determine their
current tangent and residual matrices; that is this is the class that
sets up the system of equations.  It also provides the {\em
update()} method which is invoked to set up the appropriate dof
response values once the solution algorithm has formed and solved the
system of equations.
\item {\bf EquiSolnAlgo} - an algorithmic class specifying the
sequence of operations to be performed in setting up and solving the
finite element equation which can be represented by the equation K(U)
U = P(U). 
\end{itemize}


\noindent {\bf Class Interface} \\
\indent // Constructor  \\
\indent {\em DirectIntegrationAnalysis(double tStart, double tFinal,
double $\delta t$, \\
\indent\indent\indent\indent\indent\indent Domain \&theDomain, \\
\indent\indent\indent\indent\indent\indent ConstraintHandler \&theHandler, \\
\indent\indent\indent\indent\indent\indent DOF\_Numberer \&theNumberer, \\
\indent\indent\indent\indent\indent\indent AnalysisModel \&theModel,\\
\indent\indent\indent\indent\indent\indent EquiSolnAlgo \&theSolnAlgo,\\
\indent\indent\indent\indent\indent\indent LinearSOE \&theSOE, \\
\indent\indent\indent\indent\indent\indent TransientIntegrator
\&theIntegrator);}  \\ \\
\indent // Destructor \\
\indent {\em virtual~ $\tilde{}$DirectIntegrationAnalysis();}\\  \\
\indent // Public Methods \\
\indent {\em int analyze(void);} \\
\indent {\em void clearAll(void);} \\
\indent {\em void domainChange(void);}\\ \\
\indent // Public Methods to vary the type of Analysis\\
\indent {\em void setDeltaT(double $\delta t$);} \\ 
\indent {\em int setAlgorithm(EquiSolnAlgo \&theAlgorithm);} \\
\indent {\em int setIntegrator(StaticIntegrator \&theIntegrator);}\\
\indent {\em int setLinearSOE(LinearSOE \&theSOE);} \\

\noindent {\bf Constructor} \\
\indent {\em DirectIntegrationAnalysis(double tStart, double tFinal,
double $\delta t$, \\
\indent\indent\indent\indent\indent\indent Domain \&theDomain, \\
\indent\indent\indent\indent\indent\indent ConstraintHandler \&theHandler, \\
\indent\indent\indent\indent\indent\indent DOF\_Numberer \&theNumberer, \\
\indent\indent\indent\indent\indent\indent AnalysisModel \&theModel,\\
\indent\indent\indent\indent\indent\indent EquiSolnAlgo \&theSolnAlgo,\\
\indent\indent\indent\indent\indent\indent LinearSOE \&theSOE, \\
\indent\indent\indent\indent\indent\indent TransientIntegrator
\&theIntegrator);} \\ 

{\em tStart}, {\em tFinish} and {\em thDomain} are passed to the
TransientAnalysis class constructor. The constructor is responsible
for setting up all links needed by the objects in the aggregation. It
invokes {\em setLinks(theDomain)} on {\em theModel}, {\em
setLinks(theDomain,theModel,theIntegrator)} on {\em theHandler}, 
{\em setLinks(theModel)} on {\em theNumberer},  {\em
setLinks(theModel, theSOE)} on {\em theIntegrator} and  {\em
setLinks(theModel,theAnalysis, theIntegrator, theSOE)} on {\em
theSolnAlgo}. \\

\noindent {\bf Destructor} \\
\indent {\em $\tilde{ }$DirectIntegrationAnalysis();}\\ 
Does nothing. {\em clearAll()} must be invoked if the destructor on
the objects in the aggregation need to be invoked. \\

\noindent {\bf Public Methods}\\
\indent {\em int analyze(void);} \\
Invoked to perform a transient analysis on the FE\_Model. The method
checks to see if the domain has changed before it performs the
analysis. The DirectIntegrationAnalysis object performs the following:
\begin{tabbing}
while \= \+ while \= while \= \kill
    double time = tStart; \\
    while (theDomain-$>$getCurrntTime() $<$ tFinish) \{ \+ \\
       if (theDomain-$>$hasDomainChanged() == true) \+\\
           this-$>$domainChanged(); \- \\

	theIntegrator-$>$newStep($\delta t$); \\
	theAlgorithm-$>$solveCurrentStep(); \\
        theIntegrator-$>$commit(); \- \\
    \}
\end{tabbing}
\noindent The type of analysis performed, depends on the type of the
objects in the analysis aggregation. If any of the methods invoked
returns a negative number, an error message is printed, {\em
revertToLastCommit()} is invoked on the Domain, and a negative number
is immediately returned. Returns a $0$ if the algorithm is successful. \\

{\em void clearAll(void);} \\
Will invoke the destructor on all the objects in the aggregation. NOTE
this means they must have been constructed using {\em new()},
otherwise a segmentation fault can occur.\\

{\em void domainChange(void);}\\
This is a method invoked by a domain which indicates to the analysis
that the domain has changed. The method invokes the following:
\begin{enumerate} 
\item It invokes {\em clearAll()} on {\em theModel} which causes the
AnalysisModel to clear out its list of FE\_Elements and DOF\_Groups,
and {\em clearAll()} on {\em theHandler}.
\item It then invokes {\em handle()} on {\em theHandler}. This causes
the constraint handler to recreate the appropriate FE\_Element and
DOF\_Groups to perform the analysis subject to the boundary conditions
in the modified domain.
\item It then invokes {\em number()} on {\em theNumberer}. This causes
the DOF numberer to assign equation numbers to the individual
dof's. Once the equation numbers have been set the numberer then
invokes {\em setID()} on all the FE\_Elements in the model. Finally
the numberer invokes {\em setNumEqn()} on the model.
\item It then invokes {\em domainChanged()} on {\em theIntegrator} and
{\em theAlgorithm} to inform these objects that changes have occurred
in the model.
\item It invokes {\em setSize(theModel.getDOFGraph())} on {\em
theSOE} which causes the system of equation to determine its size
based on the connectivity of the dofs in the analysis model. 
\item Finally it invokes {\em domainChanged()} on {\em theIntegrator} and 
{\em theAlgorithm}. 
Returns $0$ if successful. At any stage above, if an error occurs the
method is stopped, a warning message is printed and a negative number
is returned. \\ 
\end{enumerate}

{\em int setDeltaT(double $\delta t$);} \\
Sets the time increment used in the {\em analyze()} method to $\delta
t$. Returns $0$.\\

\indent {\em int setAlgorithm(EquiSolnAlgo \&newAlgorithm);}\\
To change the algorithm between analysis. It first invokes the
destructor on the old SolutionAlgorithm object associated with the
analysis. It then sets the SolutionAlgorithm 
associated with the analysis to be {\em newAlgorithm} and sets the
links for this object by invoking {\em setLinks()}. Checks then to
see if the domain has changed, if true it invokes {\em
domainChanged()}, otherwise it invokes {\em domainChanged()} on the
new SolutionAlgorithm. Returns $0$ if successful, a warning message
and a negative number if not.\\

\indent {\em int setIntegrator(TransientIntegrator \&newIntegrator);}\\
To change the integration scheme between analysis. It first invokes the
destructor on the old Integrator object associated with the
analysis. It then sets the SolutionAlgorithm 
associated with the analysis to be {\em newAlgorithm} and sets the
links for this object by invoking {\em setLinks()}. It also invokes
{\em setLinks()} on the ConstraintHandler and SolutionAlgorithm
objects. Checks then to see if the domain has changed, if true it
invokes {\em domainChanged()}, otherwise it invokes {\em
domainChanged()} on the new Integrator. Returns $0$ if
successful, a warning message and a negative number if not.\\

\indent {\em int setLinearSOE(LinearSOE \&newSOE);}\\
To change the linear system of equation object between analysis. It
first invokes the destructor on the old LinearSOE object associated
with the analysis. It then sets the SolutionAlgorithm 
associated with the analysis to be {\em newSOE}.
links for this object by invoking {\em setLinks()}. It then invokes
{\em setLinks()} on the ConstraintHandler and SolutionAlgorithm
objects. Checks then to see if the domain has changed, if true it
invokes {\em domainChanged()}, otherwise it invokes {\em
setSize()} on the new LinearSOE. Returns $0$ if successful, a warning
message and a negative number if not.\\ 


%\pagebreak \subsubsection{DomainDecompositionAnalysis}
%%File: ~/OOP/analysis/analysis/DomainDecompositionAnalysis.tex
%What: "@(#) DomainDecompositionAnalysis.tex, revA"

UNDER CONSTRUCTION

\noindent {\bf Files}   \\
\indent \#include $<\tilde{
}$/analysis/analysis/DomainDecompositionAnalysis.h$>$  \\ 

\noindent {\bf Class Declaration}  \\
\indent class DomainDecompositionAnalysis: public Analysis, public MovableObject;  \\

\noindent {\bf Class Hierarchy} \\
\indent Analysis \\
\indent MovableObject \\
\indent\indent {\bf DomainDecompositionAnalysis} \\

\noindent {\bf Description} \\ 
\indent DomainDecompositionAnalysis is a subclass of Analysis, 
it is used when performing an analysis using the domain decomposition
method to solve the equations. Its public member functions are all
virtual to allow for the generalization of the 
class. The following are the aggregates of such an analysis type:
\begin{itemize}
\item {\bf AnalysisModel} - a container class holding the FE\_Element
and DOF\_Group objects created by the ConstraintHandler object. 
\item {\bf ConstraintHandler} - a class which creates the DOF\_Group
and FE\_Element objects, the type of objects created depending on how
the specified constraints in the domain are to be handled. 
\item {\bf DOF\_Numberer} - a class responsible for providing equation
numbers to the individual degrees of freedom in each DOF\_Group object.
\item {\bf LinearSOE} - a numeric class responsible for the creation
and subsequent solution of large systems of linear equations of the
form $Ax = b$, where A is a matrix, and x and b are vectors.
\item {\bf IncrementalIntegrator} - an algorithmic class which provides
methods which are invoked by the FE\_Element to determine their
current tangent and residual matrices; that is this is the class that
sets up the system of equations.  It also provides the {\em
update()} method which is invoked to set up the appropriate dof
response values once the solution algorithm has formed and solved the
system of equations.
\item {\bf DomainDecompositionAlgo} - an algorithmic class specifying the
sequence of operations to be performed in determining the response
for the external dof and placing these in the system of equations.
\item {\bf DomainSolver} - an algorithmic class specifying the
sequence of operations to be performed in performing the numerical 
operations.
\end{itemize}


\noindent {\bf Class Interface} \\
\indent {\bf Constructors} \\
\indent {\em DomainDecompositionAnalysis(Subdomain \&theDomain, \\
\indent\indent\indent\indent\indent\indent\indent ConstraintHandler
\&theHandler, \\ 
\indent\indent\indent\indent\indent\indent\indent DOF\_Numberer \&theNumberer, \\
\indent\indent\indent\indent\indent\indent\indent AnalysisModel \&theModel,\\
\indent\indent\indent\indent\indent\indent\indent DomainDecompAlgo \&theSolnAlgo,\\
\indent\indent\indent\indent\indent\indent\indent IncrementalIntegrator \&theIntegrator,\\
\indent\indent\indent\indent\indent\indent\indent LinearSOE \&theSOE, \\
\indent\indent\indent\indent\indent\indent\indent DomainDecompositionSolver \&theSolver);} \\
\indent {\em DomainDecompositionAnalysis(Subdomain \&theDomain);} \\
\indent {\em DomainDecompositionAnalysis(int classTag, Subdomain
\&theDomain);}\\ \\
\indent // Destructor \\
\indent {\em virtual~ $\tilde{}$DomainDecompositionAnalysis();}\\ \\
\indent // Public Methods \\
\indent {\em virtual int analyze(void);} \\
\indent {\em virtual void domainChanged(void);} \\
\indent {\em int getNumExternalEqn(void);} \\
\indent {\em     virtual int  computeInternalResponse(void);}\\
\indent {\em     virtual int  formTangent(void);}\\
\indent {\em     virtual int  formResidual(void);}\\
\indent {\em int  formTangVectProduct(Vector \&u);}\\
\indent {\em virtual Matrix \&getTangent();}\\
\indent {\em virtual Vector \&getResidual();}\\
\indent {\em const Vector \&getTangVectProduct();}\\
\indent {\em int sendSelf(Channel \&theChannel, FEM\_ObjectBroker
\&theBroker);}\\ 
\indent {\em int recvSelf(Channel \&theChannel, FEM\_ObjectBroker
\&theBroker);}\\\\ 
\indent // Protected Member Functions  \\
\indent {\em Subdomain  *getSubdomainPtr(void) const;} \\
\indent {\em ConstraintHandler *getConstraintHandlerPtr(void) const;}\\
\indent {\em DOF\_Numberer *getDOF\_NumbererPtr(void) const;}\\
\indent {\em AnalysisModel  *getAnalysisModelPtr(void) const;}\\
\indent {\em DomainDecompAlgo  *getDomainDecompAlgoPtr(void) const;}\\
\indent {\em IncrementalIntegrator *getIncrementalIntegratorPtr(void) const;}\\
\indent {\em LinearSOE *getLinSOEPtr(void) const;}\\
\indent {\em DomainSolver *getDomainSolverPtr(void) const;} \\


\noindent {\bf Constructors} \\
\indent {\em DomainDecompositionAnalysis(Subdomain \&theDomain, \\
\indent\indent\indent\indent\indent\indent\indent ConstraintHandler
\&theHandler, \\ 
\indent\indent\indent\indent\indent\indent\indent DOF\_Numberer \&theNumberer, \\
\indent\indent\indent\indent\indent\indent\indent AnalysisModel \&theModel,\\
\indent\indent\indent\indent\indent\indent\indent DomainDecompAlgo \&theSolnAlgo,\\
\indent\indent\indent\indent\indent\indent\indent IncrementalIntegrator \&theIntegrator,\\
\indent\indent\indent\indent\indent\indent\indent LinearSOE \&theSOE, \\
\indent\indent\indent\indent\indent\indent\indent DomainDecompositionSolver \&theSolver);} \\
The constructor sets all the links required by the objects in the aggregation. To
do this it invokes {\em setLinks(theDomain)} on {\em theModel},
{\em setLinks(theSubdomain,theModel,theIntegrator)} on {\em theHandler}, 
{\em setLinks(theModel)} on {\em theNumberer}, it invokes {\em
setLinks(theModel, theSOE)} on {\em theIntegrator}, and it invokes {\em
setLinks(theModel,theIntegrator, theSOE,theSolver,theSubdomain)} on {\em
theSolnAlgo}. Finally it invokes {\em setAnalysis(*this)} on {\em theSubdomain}. \\

\indent {\em DomainDecompositionAnalysis(Subdomain \&theDomain);} \\
A constructor that is used when creating a DomainDecompositionObject which
is to receive itself afterwards. Sets the links to the Subdomain. It is
essential that this object {\em recvSelf()} before DomainDecompositionAnalysis
methods are invoked as their invocation will cause segmentation faults. 
Invokes {\em setAnalysis(this)} on the Subdomain. \\

{\em DomainDecompositionAnalysis(int classTag, Subdomain \&theDomain);}\\
Provided for subclasses to use. Invokes {\em setAnalysis(this)} on the
Subdomain. \\ 


\noindent {\bf Destructor} \\
\indent {\em virtual~ $\tilde{}$DomainDecompositionAnalysis();}\\ 

\noindent {\bf Public Methods}\\
\indent {\em virtual int analyze(void);} \\
Causes an error message to be output and returns $-1$. \\

{\em virtual void domainChanged(void);} \\
Method used to inform the object that the domain has changed. The
DomainDecompositionAnalysis object then performs the following: \\ {\em
\indent\indent theAnalysisModel-$>$clearAll(),
theConstraintHandler-$>$clearAll();\\ 
\indent\indent numExtEqn =
theConstraintHandler-$>$handle(theSubdomain-$>$getExternalNodes());\\ 
\indent\indent theDOFnumberer-$>$number(theExternalDOFsDOFGrps);  \\
\indent\indent theLinSysOfEqn-$>$setSize(theAnalysisModel-$>$getDOFGraph);  \\
\indent\indent theIntegrator-$>$domainChanged(); \\
\indent\indent theAlgorithm-$>$domainChanged(); \\ }
Finally {\em tangFormed} is marked as {\em false}. \\

{\em int getNumExternalEqn(void);} \\
A method to return the number of external degrees-of-freedom on the
Subdomain interface, this information is returned when {\em handle()}
is invoked on {\em theConstraintHandler}. \\

{\em     virtual int  computeInternalResponse(void);}\\
A method which invokes {\em solveCurrentStep()} on {\em theAlgorithm}.  \\

{\em     virtual int  formTangent(void);}\\
A method to form the condensed tangent matrix, given the current
number of internal dof. It first checks to see if the Subdomain has
changed, by  invoking {\em hasDomainChanged()} on the Subdomain; if it
has {\em invokeChangeOnAnalysis()} is invoked on the {\em
Subdomain}. It then checks to see if {\em counter} is equal to $-1$ or
not; a $-1$ indicating the tangent has already been formed in order
that the residual could be determined. If this is not the case {\em
formTangent()} is invoked on {\em theIntegrator}, {\em condenseA()} is
invoked on {\em theSolver} object, a flag is set to indicate that the
tangent has been formed, and the {\em counter} is incremented. Returns
a $0$ if successful, if either the {\em formTangent()} or {\em
condenseA()} method returns a negative number this number is
returned. \\  

{\em     virtual int  formResidual(void);}\\
A method to form the condensed residual vector, given the current
number of internal dof. A check to see if the Subdomain has changed is first made, 
this is done by invoking {\em hasDomainChanged()} on the Subdomain; if it has 
been modified {\em invokeChangeOnAnalysis()} 
is invoked on the {\em Subdomain}. If the tangent has not yet been formed
it invokes {\em formTangent()} on itself and sets the {\em counter} to $-1$.
To form the residual {\em formUnbalance()} is invoked on {\em theIntegrator} and
{\em condenseRHS(numInt)} is invoked on {\em theSolver}. Returns $0$ or the negative
number that was returned if either {\em formUnbalance()} or {\em condenseRHS()} failed.\\

{\em int  formTangVectProduct(Vector \&u);}\\
A method to form the product of the condensed tangent matrix times the
vector $u$. A check to see if the Subdomain has changed is first made, 
this is done by invoking {\em hasDomainChanged()} on the Subdomain; if it has 
been modified {\em invokeChangeOnAnalysis()} 
is invoked on the {\em Subdomain}. If the tangent has not yet been formed
it invokes {\em formTangent()} on itself and sets the {\em counter} to $-1$.
Finally the result of invoking {\em computeCondensedMatVect(numInt, u)} on {\em
theSolver} is returned. \\


{\em virtual Matrix \&getTangent();}\\
A method which returns the portion of A corresponding to internal
equation numbers. A check to see if the Subdomain has changed is first made, 
this is done by invoking {\em hasDomainChanged()} on the Subdomain; if it has 
been modified {\em invokeChangeOnAnalysis()} is invoked on the {\em Subdomain}. 
If the tangent has not yet been formed {\em formTangent()} is invoked. 
The method returns the result of invoking {\em getCondensedA()} on {\em theSolver()}. \\


{\em virtual Vector \&getResidual();}\\
A method which returns the portion of the $b$ corresponding
to the external equation numbers. A check to see if the Subdomain has changed is first made, 
this is done by invoking {\em hasDomainChanged()} on the Subdomain; if it has 
been modified {\em invokeChangeOnAnalysis()} is invoked on the {\em Subdomain}
and {\em formResidual()} is called.  The object returns the Vector obtained from 
invoking {\em getCondensedRHS()} on the solver. \\ 


{\em const Vector \&getTangVectProduct();}\\
Returns the result of invoking {\em getCondensedMatVect()} on the
solver. A check to see if the Subdomain has changed is first made, 
this is done by invoking {\em hasDomainChanged()} on the Subdomain; if it has 
been modified {\em invokeChangeOnAnalysis()} is invoked on the {\em Subdomain}. 
The object returns the Vector obtained from invoking {\em getCondensedMatVect()} 
on {\em theSolver}. \\


{\em int sendSelf(Channel \&theChannel, FEM\_ObjectBroker \&theBroker);}\\
Creates an ID and populates the ID with the class tags of the aggregates in
the aggregation. This ID is sent and then {\em sendSelf(theChannel,theBroker)} is
invoked on each of the aggregates. Returns 0.\\

{\em int recvSelf(Channel \&theChannel, FEM\_ObjectBroker \&theBroker);}\\
Creates an ID and receives data into it from {\em theChannel}. Based on the
class tags in the ID {\em theBroker} is then asked to return pointers to
new objects required in the aggregation.  {\em sendSelf(theChannel,theBroker)} is
invoked on each of these new aggregate objects. Finally {\em setLinks} is
invoked on each of these objects with the correct arguments and {\em setAnalysis(this)} 
is invoked on the {\em Subdomain}.  Returns 0.\\


\noindent {\bf Protected Methods}  \\
\indent {\em Subdomain  *getSubdomainPtr(void) const;} \\
A const member function which returns a pointer to {\em theSubdomain}. \\

{\em ConstraintHandler *getConstraintHandlerPtr(void) const;}\\
A const member function which returns a pointer to {\em theSubdomain}.\\

{\em DOF\_Numberer *getDOF\_NumbererPtr(void) const;}\\
A const member function which returns a pointer to {\em
theNumberer}.\\

{\em AnalysisModel  *getAnalysisModelPtr(void) const;}\\
A const member function which returns a pointer to {\em
theModel}.\\

{\em DomainDecompAlgo  *getDomainDecompAlgoPtr(void) const;}\\
A const member function which returns a pointer to {\em
theAlgorithm}.\\

{\em IncrementalIntegrator *getIncrementalIntegratorPtr(void) const;}\\
A const member function which returns a pointer to {\em
theIntegrator}.\\

{\em LinearSOE *getLinSOEPtr(void) const;}\\
A const member function which returns a pointer to {\em theSOE}.\\

\indent {\em DomainSolver *getDomainSolverPtr(void) const;} \\
A const member function which returns a pointer to {\em theSolver}.
associated with the DomainDecompositionAnalysis object. \\



\pagebreak \subsection{{\bf Integrator}}
%File: ~/OOP/analysis/integrator/Integrator.tex
%What: "@(#) Integrator.tex, revA"

\noindent {\bf Files}   \\
\indent \#include $<\tilde{ }$/analysis/integrator/Integrator.h$>$  \\

\noindent {\bf Class Declaration}  \\
\indent class Integrator: public MovableObject;  \\

\noindent {\bf Class Hierarchy} \\
\indent MovableObject \\
\indent\indent {\bf Integrator} \\

\noindent {\bf Description} \\ 
\indent The Integrator class is an abstract base class. Its purpose is
to define the interface common among all subclasses. An integrator
method is responsible for defining how the system of equations are set
up (this it does by specifying how the FE\_Element and DOF\_Group
objects of the analysis model construct the vectors and matrices asked
of them by the Analysis). It is also responsible for updating the
response quantities at the DOFs with the appropriate values; the values
are determined from the solution to the system of equations. \\

\noindent {\bf Class Interface} \\
\indent\indent // Constructor \\
\indent\indent {\em Integrator(int classTag);}\\ \\
\indent\indent // Destructor \\
\indent\indent {\em virtual $\tilde{ }$Integrator();}\\  \\
\indent\indent // Public Methods \\
\indent\indent {\em virtual int domainChanged(void); } \\
\indent\indent {\em  virtual int formEleTangent(FE\_Element *theEle) =0;} \\
\indent\indent {\em  virtual int formNodTangent(DOF\_Group *theDof) =0;} \\
\indent\indent {\em  virtual int formEleResidual(FE\_Element *theEle) =0;} \\
\indent\indent {\em  virtual int formNodUnbalance(DOF\_Group *theDof)
=0;} \\ \\
\indent\indent // Public Method added for Domain Decomposition \\
\indent\indent {\em virtual  int getLastResponse(Vector \&result, const
ID \&id) =0;} \\ \\
\indent\indent // Public Method added for Output \\
\indent\indent {\em virtual void Print(OPS_Stream \&s, int flag = 0) =0;}\\

\noindent {\bf Constructor} \\
\indent {\em Integrator(int classTag);}\\ 
{\em classTag} is passed to the MovableObject classes constructor. \\

\noindent {\bf Destructor} \\
\indent {\em virtual~$\tilde{}$Integrator();}\\ 
Does nothing. Provided so the subclass destructors will be invoked. \\

\noindent {\bf Public Methods}\\
\indent {\em virtual int domainChanged(void); } \\
Is called by the Analysis object. Refer to the Analysis classes to see
when it is invoked. To return $0$ if successful, a negative number if
not. This base class returns $0$.\\

\indent {\em  virtual int formEleTangent(FE\_Element *theEle) =0;} \\
Called upon to determine the FE\_Element {\em theEle}s matrix
contribution to the SystemOfEqn object. To return $0$ if successful,
a negative number otherwise. \\ 
 

\indent {\em  virtual int formNodTangent(DOF\_Group *theDof) =0;} \\
Called upon to determine the DOF\_Group {\em theDof}s matrix
contribution to the SystemOfEqn object. To return $0$ if successful,
a negative number otherwise. \\ 


\indent {\em  virtual int formEleResidual(FE\_Element *theEle) =0;} \\
Called upon to determine the FE\_Element {\em theEle}s vector
contribution to the SystemOfEqn object. To return $0$ if successful,
a negative number otherwise. \\ 


\indent {\em virtual int formNodUnbalance(DOF\_Group *theDof) =0;} \\
Called upon to determine the DOF\_Group {\em theDof}s vector
contribution to the SystemOfEqn object. To return $0$ if successful,
a negative number otherwise. \\ 

{\em virtual int getLastResponse(Vector \&result, const ID \&id) =0;} \\
Called upon to get the result quantities for the components specified
in the ID {\em id} and to place them in the Vector {\em result}. This
is provided for domain decomposition methods. To return $0$ if
successful, a negative number otherwise. \\ 

\indent {\em virtual void Print(OPS_Stream \&s, int flag = 0) =0;}\\
The Integrator is to send information to the stream based on the
integer {\em flag}. 



\pagebreak \subsubsection{{\bf IncrementalIntegrator}}
%File: ~/OOP/analysis/integrator/IncrementalIntegrator.tex
%What: "@(#) IncrementalIntegrator.tex, revA"

\noindent {\bf Files}   \\
\indent \#include $<\tilde{ }$/analysis/integrator/IncrementalIntegrator.h$>$  \\

\noindent {\bf Class Declaration}  \\
\indent class IncrementalIntegrator: public Integrator  \\

\noindent {\bf Class Hierarchy} \\
\indent MovableObject \\
\indent\indent Integrator \\
\indent\indent\indent {\bf IncrementalIntegrator} \\

\noindent {\bf Description} \\ 
\indent IncrementalIntegrator is an abstract class. A subclass of it
is used when performing a static or transient analysis using an
incremental displacement approach. Subclasses of
IncrementalIntegrators provide methods informing the FE\_Element and
DOF\_Group objects how to build the tangent and residual matrices and
vectors. They also provide the method for updating the response
quantities at the DOFs with appropriate values; these values being
some function of the solution to the linear system of equations. \\ 

\noindent {\bf Class Interface} \\
\indent\indent // Constructor \\
\indent\indent {\em IncrementalIntegrator(int classTag);}\\  \\
\indent\indent // Destructor \\
\indent\indent {\em virtual $\tilde{ }$IncrementalIntegrator();}\\ \\
\indent\indent // Public Methods \\
\indent\indent {\em void setLinks(AnalysisModel \&theMoedl, LinearSOE
\&theSOE);} \\  
\indent\indent {\em virtual int formTangent(void);} \\
\indent\indent {\em virtual int formUnbalance(void);} \\
\indent\indent {\em virtual int formEleTangent(FE\_Element *theEle) =0;} \\
\indent\indent {\em virtual int formEleResidual(FE\_Element *theEle) =0;} \\
\indent\indent {\em virtual int formNodTangent(DOF\_Group *theDof) =0;} \\
\indent\indent {\em virtual int formNodUnbalance(DOF\_Group *theDof) =0;} \\
\indent\indent {\em virtual int update(const Vector \&$\Delta U$) =0;}\\ 
\indent\indent {\em virtual int commit(void);} \\ \\
\indent\indent // Public Method added for Domain Decomposition \\
\indent\indent {\em virtual int getLastResponse(Vector \&result, const ID
\&id);} \\ \\
\indent\indent // Protected Methods  \\
\indent\indent {\em LinearSOE *getLinearSOEPtr(void) const;} \\
\indent\indent {\em AnalysisModel *getAnalysisModelPtr(void) const;} \\
\indent\indent {\em virtual int formNodalUnbalance(void);} \\
\indent\indent {\em virtual int formElementResidual(void);} \\

\noindent {\bf Constructor} \\
\indent {\em IncrementalIntegrator(int classTag);}\\ 
The integer {\em classTag} is passed to the Integrator classes
constructor. Pointers to the AnalysisModel and LinearSOE are set to
$0$.\\

\noindent {\bf Destructor} \\
\indent {\em virtual~ $\tilde{}$IncrementalIntegrator();}\\ 
Does nothing. \\

\noindent {\bf Public Methods}\\
\indent {\em void setLinks(AnalysisModel \&theModel, LinearSOE
\&theSOE);} \\  
Invoked by the Analysis object to set up the links the
IncrementalIntegrator objects needs to perform its operations.
Sets the pointers to the AnalysisModel and LinearSOE objects to point
to {\em theAnalaysisModel} and {\em theSOE}. \\

{\em virtual int formTangent(void);} \\
Invoked to form the structure tangent matrix. The method first loops
over all the FE\_Elements in the AnalysisModel telling them to form
their tangent and then it loops over the FE\_Elements again adding the
tangent to the LinearSOE objects A matrix. It performs the following:
\begin{tabbing}
while \= \+ while \= while \= \kill
FE\_EleIter \&theEles = theAnalysisModel.getFEs();\\
theSOE.zeroA();\\
while((elePtr = theEles1()) $\neq$ 0) \+ \\
if (theSOE.addA(elePtr-$>$getTangent(this),
elePtr-$>$getID(), $1.0$) $<$ 0) \+ \\
return $-1$; \-\- 
\end{tabbing}
\noindent Returns $0$ if successful, otherwise an error
message is printed an a $-1$ is returned if {\em setLinks()} has not
been called, or $-2$ if failure to add an FE\_Elements tangent to the
LinearSOE. The two loops are introduced to allow for efficient
parallel programming. THIS MAY CHANGE TO REDUCE MEMORY DEMANDS.\\  

{\em virtual int formUnbalance(void);} \\
Invoked to form the unbalance. The method fist zeros out the $B$
vector of the LinearSOE object and then invokes formElementResidual()
and formNodalUnbalance() on itself. \\
\begin{tabbing}
while \= \+ while \= while \= \kill
theSOE.zeroB();\\
this-$>$fromElementResidual(); \\
this-$>$formNodalUnbalance() \\
\end{tabbing}

\noindent If an error occurs in either of these two methods or if {\em
setLinks()} has not been called, an error message is printed and a
negative number is returned. Returns $0$ if successful. \\ 


\indent {\em virtual int formEleTangent(FE\_Element *theEle) =0;} \\
To inform the FE\_Element how to build its tangent matrix for addition
to the system of equations. The subclasses must provide the
implementation of this method. \\

{\em virtual int formEleResidual(FE\_Element *theEle) =0;} \\
To inform the FE\_Element how to build its residual vector for addition
to the system of equations. The subclasses must provide the
implementation of this method. \\

{\em virtual int formNodTangent(DOF\_Group *theDof) =0;} \\
To inform the DOF\_Group how to build its tangent matrix for addition
to the system of equations. The subclasses must provide the
implementation of this method. This is required in transient analysis
as th Node objects have mass. THIS MAY CHANGE.\\

{\em virtual int formNodUnbalance(DOF\_Group *theDof) =0;} \\
To inform the DOF\_Group how to build its residual vector for addition
to the system of equations. The subclasses must provide the
implementation of this method. \\

{\em virtual int update(const Vector \&$\Delta U$) =0;} \\
When invoked causes the integrator object to update the DOF\_Group
responses with the appropriate values based on the computed solution
to the system of equation object. The subclasses must provide an
implementation of this method. \\

{\em virtual int commit(void) =0;} \\
Invoked by the SolutionAlgorithm to inform the Integrator that current
state of domain is on solution path. Returns the result of invoking
{\em commitDomain()} on the AnalysisModel object associated with the
Integrator. \\


\noindent {\bf Public Method added for Domain Decomposition}\\
\indent {\em virtual int getLastResponse(Vector \&result, const ID \&id);} \\
Returns in {\em result} values for the last solution to the system of
equation object whose location in the solution vector is given by {\em
id}. For a location specified by a negative integer in {\em id} 0.0 will be
returned in {\em result}. Returns a $0$ if successful, a warning
message and a negative number is returned if an error occurs. $-1$ if
{\em setSize()} has not been called and a $-2$ if location in {\em id} is
greater than $order-1$ of $b$ vector.\\

\noindent {\bf Protected Methods}  \\
\indent {\em LinearSOE *getLinearSOEPtr(void) const;} \\
A const member function which returns a pointer to the LinearSOE
associated with the IncrementalIntegrator object, i.e. {\em theSOE}
passed in {\em setLinks()}. \\

{\em AnalysisModel *getAnalysisModelPtr(void) const;} \\
A const member function which returns a pointer to the AnalysisModel
associated with the IncrementalIntegrator object, i.e. {\em theModel}
passed in {\em setLinks()}. \\


{\em virtual int formNodalUnbalance(void);} \\
The method first
loops over all the DOF\_Group objects telling them to form their
unbalance and then adds this Vector to the $b$ vector of the LinearSOE
object, i.e. it performs the following: \\
\begin{tabbing}
while \= \+ while \= while \= \kill
DOF\_EleIter \&theDofs = theAnalysisModel.getDOFs();\\
theSOE.zeroB();\\
while((dofPtr = theDofs()) $\neq$ 0) \+ \\
theSOE.addB(dofPtr-$>$getUnbalance(theIntegrator),
dofPtr-$>$getID()) \\
\end{tabbing}
\noindent Returns $0$ if successful, otherwise a  negative number is
returned and a warning message is printed if an error occurred. Note,
no test is made to ensure {\em setLinks()} has been invoked.\\


{\em virtual int formElementResidual(void);} \\
Invoked to form residual vector (the C vector in theSOE). The method
iterates twice over the FE\_elements in the AnalysisModel, the first
time telling the FE\_Elements top form their residual and the second
time to add this residual to the LinearSOE objects $b$ vector, i.e. it
performs the following:
\begin{tabbing}
while \= \+ while \= while \= \kill
FE\_EleIter \&theEles = theAnalysisModel.getFEs();\\
while((elePtr = theEles()) $\neq$ 0) \{ \+ \\
theSOE.addA(elePtr-$>$getResidual(this), elePtr-$>$getID()) \+ \\ 
\end{tabbing}
\noindent Returns $0$ if successful, otherwise a warning message is
printed and a negative number is returned if an error occurs. Note, no
test is made to ensure {\em setLinks()} has been invoked.\\






\pagebreak \subsubsection{{\bf StaticIntegrator}}
%File: ~/OOP/analysis/integrator/StaticIntegrator.tex
%What: "@(#) StaticIntegrator.tex, revA"

\noindent {\bf Files}   \\
\indent \#include $<\tilde{ }$/analysis/integrator/StaticIntegrator.h$>$  \\

\noindent {\bf Class Declaration}  \\
\indent class StaticIntegrator: public Integrator  \\

\noindent {\bf Class Hierarchy} \\
\indent MovableObject \\
\indent\indent Integrator \\
\indent\indent\indent IncrementalIntegrator \\
\indent\indent\indent\indent {\bf StaticIntegrator} \\
\indent\indent\indent\indent\indent LoadControl \\
\indent\indent\indent\indent\indent ArcLength \\
\indent\indent\indent\indent\indent DisplacementControl \\

\noindent {\bf Description} \\ 
\indent StaticIntegrator is an abstract class. It is a subclass of
IncrementalIntegrator provided to implement the common methods among
integrator classes used in performing a static analysis on the
FE\_Model. The StaticIntegrator class provides an implementation of
the methods to form the FE\_Element and DOF\_Group contributions to
the tangent and residual. A pure virtual method {\em newStep()} is
also defined in the interface, this is the method first called at each
iteration in a static analysis, see the StaticAnalysis class. \\

In static nonlinear finite element problems we seek a solution
($\U$, $\lambda$) to the nonlinear vector function

\begin{equation}
\R(\U, \lambda) = \lambda \P - \F_R(\U) = \zero
\label{staticGenForm}
\end{equation}

The most widely used technique for solving the non-linear finite
element equation, equation~\ref{femGenForm}, is to use an incremental
scheme. In the incremental formulation, a solution to the equation is
sought at successive incremental steps.  

\begin{equation}
\R(\U_{n}, \lambda_n) = \lambda_n \P - \F_R(\U_{n})
\label{staticIncForm}
\end{equation}

The solution of this equation is typically obtained using an iterative
procedure, in which a sequence of approximations
($\U_{n}^{(i)}$, $\lambda_n^{(i)}$), $i=1,2, ..$ is obtained which
converges to the solution ($\U_n$, $\lambda_n)$. The most
frequently used iterative schemes, such as Newton-Raphson, modified
Newton, and quasi Newton schemes, are based on a Taylor expansion of
equation~\ref{staticIncForm} about ($\U_{n}$, $\lambda_n$):     

\begin{equation} 
\R(\U_{n},\lambda_n) = \lambda_n^{(i)} \P 
 - \f_{R}\left(\U_{n}^{(i)} \right) - \left[
\begin{array}{cc}
\K_n^{(i)} & -\P \\
\end{array} \right] 
\left\{
\begin{array}{c}
\U_{n} - \U_{n}^{(i)}  \\ 
\lambda_n - \lambda_n^{(i)} 
\end{array} \right\}
\label{staticFormTaylor}
\end{equation} 

\noindent which  a system of of $N$ equations with ($N+1$)
unknowns. Two solve this, an additional equation is required, the
constraint equation. The constraint equation used depends on the
static integration scheme, of which there are a number, for example
load control, arc length, and displacement control. \\

\pagebreak
\noindent {\bf Class Interface} \\
\indent\indent // Constructors \\
\indent\indent {\em StaticIntegrator(int classTag);}\\  \\
\indent\indent // Destructor \\
\indent\indent {\em virtual $\tilde{ }$StaticIntegrator();}\\  \\
\indent\indent // Public Methods \\
\indent\indent {\em virtual int formEleTangent(FE\_Element *theEle);} \\
\indent\indent {\em virtual int formEleResidual(FE\_Element *theEle);} \\
\indent\indent {\em virtual int formNodTangent(DOF\_Group *theDof);} \\
\indent\indent {\em virtual int formNodUnbalance(DOF\_Group *theDof);} \\
\indent\indent {\em virtual int newStep(void) =0;} \\

\noindent {\bf Constructors} \\
\indent {\em StaticIntegrator(int classTag);}\\ 
The integer {\em classTag} is passed to the
IncrementalIntegrator classes constructor. \\

\noindent {\bf Destructor} \\
\indent {\em virtual $\tilde{ }$StaticIntegrator();}\\ 
Does nothing. Provided so that the subclasses destructors will be called.\\

\noindent {\bf Public Methods}\\
\indent {\em virtual int formEleTangent(FE\_Element *theEle);} \\
To form the tangent matrix of the FE\_Element, {\em theEle}, is
instructed to zero this matrix and then add it's $K$ matrix to the
tangent, i.e. it performs the following: 
\begin{tabbing}
while \= \+ while \= while \= \kill
theEle-$>$zeroTang() \\
theEle-$>$addKtoTang() 
\end{tabbing}
\noindent The method returns $0$. \\

{\em virtual int formEleResidual(FE\_Element *theEle);} \\
To form the residual vector of the FE\_Element, {\em theEle}, is
instructed to zero the vector and then add it's $R$ vector to
the residual, i.e. it performs the following: 
\begin{tabbing}
while \= \+ while \= while \= \kill
theEle-$>$zeroResidual() \\
theEle-$>$addRtoResid() 
\end{tabbing}
\noindent The method returns $0$. \\

{\em virtual int formNodTangent(DOF\_Group *theDof);} \\
This should never be called in a static analysis. An error message is
printed if it is. Returns -1. \\


{\em virtual int formNodUnbalance(DOF\_Group *theDof);} \\
To form the unbalance vector of the DOF\_Group, {\em theDof}, is
instructed to zero the vector and then add it's $P$ vector to the
unbalance, i.e. it performs the following: 
\begin{tabbing}
while \= \+ while \= while \= \kill
theDof-$>$zeroUnbalance() \\
theDof-$>$addPtoUnbal() 
\end{tabbing}
\noindent The method returns $0$. \\




\pagebreak \subsubsection{ LoadControl}
%File: ~/OOP/analysis/integrator/LoadControl.tex
%What: "@(#) LoadControl.tex, revA"

\noindent {\bf Files}   \\
\indent \#include $<\tilde{ }$/analysis/integrator/LoadControl.h$>$  \\

\noindent {\bf Class Declaration}  \\
\indent class LoadControl: public StaticIntegrator  \\

\noindent {\bf Class Hierarchy} \\
\indent MovableObject \\
\indent\indent Integrator \\
\indent\indent\indent IncrementalIntegrator \\
\indent\indent\indent\indent StaticIntegrator \\
\indent\indent\indent\indent\indent {\bf LoadControl} \\

\noindent {\bf Description} \\ 
\indent LoadControl is a subclass of StaticIntegrator, it is
used to when performing a static analysis on the FE\_Model using the
load control method. In the load control method, the following
constraint equation is added to equation~\ref{staticFormTaylor} of the
StaticIntegrator class: 

\[ 
\lambda_n^{(i)} - \lambda_{n-1} = \delta \lambda_n
\]

\noindent where $\delta \lambda_n$ depends on $\delta \lambda_{n-1}$,
the load increment at the previous time step, $J_{n-1}$,
the number of iterations required to achieve convergence in the
previos load step, and $Jd$, the desired number of iteraions. $\delta
\lambda_n$ is bounded by $\delta \lambda_{min}$  and $\delta \lambda_{max}$. \\


\[ 
\delta \lambda_n = max \left( \delta \lambda_{min}, min \left(
\frac{Jd}{J_{n-1}} \delta \lambda_{n-1}, \delta \lambda_{max} \right) \right)
\]

Knowing $\lambda_n^{(i)}$ prior to each iteration, the $N+1$ unknowns
in equation~\ref{staticFormTaylor}, is reduced to $N$ unknowns and
results in the following equation:

\begin{equation} 
\R(\U_{n}) = \lambda_n^{(i)} \P 
 - \f_{R}\left(\U_{n}^{(i)} \right) - 
\K_n^{(i)} 
(\U_{n} - \U_{n}^{(i)})  
\label{staticFormLoadControl}
\end{equation} 

\noindent {\bf Class Interface} \\
\indent // Constructors \\
\indent {\em LoadControl(double $\delta \lambda_1$, int Jd, 
double $\delta \lambda_{min}$, double $\delta \lambda_{max}$);}\\ \\
\indent // Destructor \\
\indent {\em $\tilde{ }$LoadControl();}\\  \\
\indent // Public Methods \\
\indent {\em int newStep(void);} \\
\indent {\em int update(const Vector \&$\Delta U$);} \\
\indent {\em int setDeltaLambda(double $\delta
\lambda$;)} \\ \\
\indent // Public Methods for Output\\
\indent {\em int sendSelf(int commitTag, Channel \&theChannel);}\\ 
\indent {\em int recvSelf(int commitTag, Channel \&theChannel,
FEM\_ObjectBroker \&theBroker);}\\ 
\indent {\em int Print(OPS_Stream \&s, int flag = 0);}\\

\noindent {\bf Constructors} \\
\indent {\em LoadControl(double $\delta \lambda_1$, int Jd, 
double $\delta \lambda_{min}$, double $\delta \lambda_{max}$);}\\ \\
The integer INTEGRATOR\_TAGS\_LoadControl (defined in
$<$classTags.h$>$) is passed to the StaticIntegrator classes
constructor. $\delta \lambda_1$ is the load factor used in the first
step. The arguments $Jd$, $\delta \lambda_{min}$, and $\delta
\lambda_{max}$ are used in the determination of the increment in the
load factor at each step. \\



\noindent {\bf Destructor} \\
\indent {\em $\tilde{ }$LoadControl();}\\ 
Does nothing. \\

\noindent {\bf Public Methods}\\

{\em int newStep(void);} \\
The object obtains the current value of $\lambda$ from the AnalysisModel
object. It increments this by $\delta \lambda_n $.

\[ 
\delta \lambda_n = max \left( \delta \lambda_{min}, min \left(
\frac{Jd}{J_{n-1}} \delta \lambda_{n-1}, \delta \lambda_{max} \right) \right)
\]

\noindent It will then invoke
{\em applyLoadDomain(0.0, $\lambda$)} on the AnalysisModel
object. Returns $0$ if successful. A warning message is printed and a
$-1$ is returned if no AnalysisModel is associated with the object. \\

{\em int update(const Vector \&$\Delta U$);} \\
Invoked this causes the object to first increment the DOF\_Group
displacements with $\Delta U$, by invoking {\em incrDisp($\Delta U)$}
on the AnalysisModel, and then to update the domain, by invoking {\em
updateDomain()} on the AnalysisModel. Returns $0$ if successful, a
warning message and a $-1$ is returned if no AnalysisModel is
associated with the object. \\


\indent {\em int setDeltaLambda(double $\delta \lambda$;)} \\
Sets the value of the load increment in {\em newStep()} to be $\delta
\lambda$. Returns $0$.\\

{\em int sendSelf(int commitTag, Channel \&theChannel); } \\ 
Places in a vector if size 5 the value of $\delta \lambda_{n-1}$,
$Jd$, $J_{n-1}$, $\delta \lambda_{min}$ and $\delta \lambda_{max}$)
and then sends the Vector. Returns $0$ if successful, a warning
message is printed and a $-1$ is returned if {\em theChannel} fails to
send the Vector. \\ 

{\em int recvSelf(int commitTag, Channel \&theChannel, 
FEM\_ObjectBroker \&theBroker); } \\ 
Receives in a Vector of size 5 the data that was sent in {\em sendSelf()}.
Returns $0$ if successful, a warning message is printed, $\delta
\lambda$ is set to $0$, and a $-1$ is returned if {\em theChannel} 
fails to receive the Vector.\\

{\em int Print(OPS_Stream \&s, int flag = 0);}\\
The object sends to $s$ its type, the current value of $\lambda$, and
$\delta \lambda$. 

\pagebreak \subsubsection{ DisplacementControl}
%File: ~/OOP/analysis/integrator/DisplacementControl.tex
%What: "@(#) DisplacementControl.tex, revA"

\noindent {\bf Files}   \\
\indent \#include $<\tilde{ }$/analysis/integrator/DisplacementControl.h$>$  \\

UNDER CONSTRUCTION.\\

\noindent {\bf Class Declaration}  \\
\indent class DisplacementControl: public StaticIntegrator  \\

\noindent {\bf Class Hierarchy} \\
\indent MovableObject \\
\indent\indent Integrator \\
\indent\indent\indent IncrementalIntegrator \\
\indent\indent\indent\indent StaticIntegrator \\
\indent\indent\indent\indent\indent {\bf DisplacementControl} \\

\noindent {\bf Description} \\ 
\indent DisplacementControl is a subclass of StaticIntegrator, it is
used to when performing a static analysis on the FE\_Model using the
displacement control method. In the displacement control method the
displacement at a specified degree-of-freedom Uc is specified for each
iteration. The following constraint equation is added to
equation~\ref{staticFormTaylor} of the StaticIntegrator class: 

\[ 
Uc_n^{(i)} - Uc_{n-1} = \delta Uc_n
\]

\noindent where $\delta Uc_n$ depends on $\delta Uc_{n-1}$,
the displacement increment at the previous time step, $J_{n-1}$,
the number of iterations required to achieve convergence in the
previous load step, and $Jd$, the desired number of iteraions. $\delta
Uc_n$ is bounded by $\delta Uc_{min}$  and $\delta Uc_{max}$. \\


\[ 
\delta Ucn = max \left( \delta Uc{min}, min \left(
\frac{Jd}{J_{n-1}} \delta Uc{n-1}, \delta Uc{max} \right) \right)
\]

SOME THEORY.\\

\noindent {\bf Class Interface} \\
\indent // Constructors \\
\indent {\em DisplacementControl(int node, int dof, double $\delta Uc_1$, int Jd,  
double $\delta Uc_{min}$, double $\delta Uc_{max}$);}\\ \\
\indent // Destructor \\
\indent {\em $\tilde{ }$DisplacementControl();}\\  \\
\indent // Public Methods \\
\indent {\em int newStep(void);} \\
\indent {\em int update(const Vector \&$\Delta U$);} \\
\indent {\em int domainChanged(void);} \\ \\
\indent // Public Methods for Output\\
\indent {\em int sendSelf(int commitTag, Channel \&theChannel);}\\ 
\indent {\em int recvSelf(int commitTag, Channel \&theChannel,
FEM\_ObjectBroker \&theBroker);}\\ 
\indent {\em int Print(OPS_Stream \&s, int flag = 0);}\\

\noindent {\bf Constructors} \\
\indent {\em DisplacementControl(int node, int dof, double $\delta Uc_1$, int Jd,  
double $\delta Uc_{min}$, double $\delta Uc_{max}$);}\\ 
The integer INTEGRATOR\_TAGS\_DisplacementControl (defined in
$<$classTags.h$>$) is passed to the StaticIntegrator classes
constructor. $\delta Uc_1$ is the load factor used in the first
step. The arguments $Jd$, $\delta Uc_{min}$, and $\delta
Uc_{max}$ are used in the determination of the increment in the
load factor at each step. \\



\noindent {\bf Destructor} \\
\indent {\em $\tilde{ }$DisplacementControl();}\\ 
Does nothing. \\

\noindent {\bf Public Methods}\\

{\em int newStep(void);} \\
WHAT DO I DO?\\

{\em int update(const Vector \&$\Delta U$);} \\
WHAT DO I DO?\\

{\em int sendSelf(int commitTag, Channel \&theChannel); } \\ 
WHAT DO I DO?\\

{\em int recvSelf(int commitTag, Channel \&theChannel, 
FEM\_ObjectBroker \&theBroker); } \\ 
WHAT DO I DO?\\

{\em int Print(OPS_Stream \&s, int flag = 0);}\\
WHAT DO I DO?

\pagebreak \subsubsection{ ArcLength}
%File: ~/OOP/analysis/integrator/ArcLength.tex
%What: "@(#) ArcLength.tex, revA"

THE IMPLEMENTATION WILL HAVE TO CHANGE FOR DOMAIN-DECOMPOSITION
ANALYSIS .. AS DOES THE CONVERGENCE TEST STUFF .. THIS IS BECAUSE
USING DOT PRODUCTS OF VECTORS OBTAINED STRAIGHT FROM SYSTEM OF
EQUATION .. MAYBE MODIFY LinearSOE TO DO THE DOT PRODUCT .. WILL 
WORK IN DD IF ALL USE ONE SOE .. WHAT PetSC DOES, TALK WITH P. DEMMEL
ABOUT WHAT HE WILL PROVIDE. \\

\noindent {\bf Files}   \\
\indent \#include $<\tilde{ }$/analysis/integrator/ArcLength.h$>$  \\

\noindent {\bf Class Declaration}  \\
\indent class ArcLength: public StaticIntegrator  \\

\noindent {\bf Class Hierarchy} \\
\indent MovableObject \\
\indent\indent Integrator \\
\indent\indent\indent IncrementalIntegrator \\
\indent\indent\indent\indent StaticIntegrator \\
\indent\indent\indent\indent\indent {\bf ArcLength} \\

\noindent {\bf Description} \\ 
\indent ArcLength is a subclass of StaticIntegrator, it is
used to when performing a static analysis on the FE\_Model using an
arc length method. In the arc length method implemented by this class,
the following constraint equation is added to
equation~\ref{staticFormTaylor} of the StaticIntegrator class: 

\begin{equation}
\Delta \U_n^T \Delta \U_n  + \alpha^2 \Delta \lambda_n^2  = \Delta s^2
\end{equation}

where 

\[
\Delta \U_n = \sum_{j=1}^{i} \Delta \U_n^{(j)} = \Delta \U_n^{(i)} +
d\U^{(i)} 
\]

\[
\Delta \lambda_n = \sum_{j=1}^{i} \Delta \lambda_n^{(j)} = \Delta \lambda_n^{(i)} +
d\lambda^{(i)} 
\]

\noindent this equation cannot be added directly into
equation~\ref{staticFormTaylor} to produce a linear system of $N+1$
unknowns. To add this equation we make some assumptions ala Yang
(REF), which in so doing allows us to solve a system of $N$
unknowns using the method of ??(REF).  Rewriting
equation~\ref{staticFormTaylor} as  

\[
\K_n^{(i)} \Delta \U_n^{(i)} = \Delta \lambda_n^{(i)} \P +
\lambda_n^{(i)} \P - \F_R(\U_n^{(i)}) = \Delta \lambda_n^{(i)} \P + \R_n^{(i)}
\]

\noindent The idea of ?? is to separate this into two equations:

\def\Uh{\dot{\bf U}}
\def\Ub{\overline{\bf U}}

\[
\K_n^{(i)} \Delta \Uh_n^{(i)} = \P
\]

\[
\K_n^{(i)} \Delta \Ub_n^{(i)} = \R_n^{(i)}
\]

\noindent where now

\begin{equation}
 \Delta \U_n^{(i)} = \Delta \lambda_n^{(i)} \Delta \Uh_n^{(i)} + \Delta \Ub_n^{(i)}  
\label{splitForm}
\end{equation}

\noindent We now rewrite the constraint equation based on two conditions:

\begin{enumerate}
\item {\bf $i = 1$}: assuming $\R_n^{(1)} = \zero$, i.e. the system is
in equilibrium at the start of the iteration, the following is obtained

\[  \Delta \U_n^{(1)} = \Delta \lambda_n^{(1)} \Delta \Uh_n^{(1)} + \zero \]

\[ \Delta \lambda_n^{(1)} = \begin{array}{c} + \\ - \end{array}
\sqrt{\frac{\Delta s^2}{\Uh^T \Uh + \alpha^2}} \]

\noindent The question now is whether {\bf +} or {\bf -} should be
used. In this class, $d \lambda$ from the previous iteration $(n-1)$
is used, if it was positive {\bf +} is assumed, otherwise {\bf -}. This may
change. There are other ideas: ?(REF) number of negatives on diagonal
of decomposed matrix, ...

\item {\bf $i > 1$}

\[ \left( \Delta \U_n^{(i)} + d\U^{(i)} \right)^T \left( \Delta \U_n^{(i)} +
d\U^{(i)} \right) + \alpha^2 \left( \Delta \lambda_n^{(i)} + d\lambda^{(i)}
\right)^2 = \Delta s^2 \]

\[
\Delta {\U_n^{(i)}}^T\Delta \U_n^{(i)} + 2{d\U^{(i)}}^T\Delta \U_n^{(i)} + {d\U^{(i)}}^T d\U^{(i)}
+ \alpha^2 \Delta {\lambda_n^{(i)}}^2
+ 2 \alpha^2 d\lambda^{(i)} \Delta \lambda_n^{(i)} + \alpha^2 {d\lambda^{(i)}}^2
= \Delta s^2
\]

\noindent assuming the constraint equation was solved at $i-1$,
i.e. ${d\U^{(i)}}^T d\U^{(i)} + \alpha^2 {d\lambda^{(i)}}^2 = \Delta s^2$, this reduces to

\[
\Delta {\U_n^{(i)}}^T\Delta \U_n^{(i)} + 2{d\U^{(i)}}^T\Delta \U_n^{(i)} + 
\alpha^2 \Delta {\lambda_n^{(i)}}^2
+ 2 \alpha^2 d\lambda^{(i)} \Delta \lambda_n^{(i)} 
= 0
\]

\noindent substituting for $\Delta {\U_n^{(i)}} $ using
equation~\ref{splitForm} this can be expressed as:

\[
\Delta \lambda_n^{(i)^2} \left( \Delta \Uh_n^{(i)} \Delta \Uh_n^{(i)} +
\alpha^2 \right) +
2* \Delta \lambda_n^{(i)} \left( \Delta \Uh_n^{(i)} \Delta \Ub_n^{(i)}
+ d\U^{(i)} \Delta \Uh_n^{(i)} 
+ \alpha^2d \lambda^{(i)} \right)
\]
\[
+ \left (\Delta \Ub_n^{(i)} \Delta \Ub_n^{(i)} + d\U^{(i)} \Delta
\Ub_n^{(i)}
\right) =0 
\]

which is a quadratic in $\Delta \lambda_n^{(i)}$, which can be solved for two roots.
The root chosen is the one which will keep a positive angle between
the incremental displacement before and after this step.


\end{enumerate}
\noindent {\bf Class Interface} \\
\indent // Constructors \\
\indent {\em ArcLength(double arc, double $\alpha$ = 1.0);}\\ \\
\indent // Destructor \\
\indent {\em $\tilde{ }$ArcLength();}\\  \\
\indent // Public Methods \\
\indent {\em int newStep(void);} \\
\indent {\em int update(const Vector \&$\Delta U$);} \\
\indent {\em int domainChanged(void); }\\ \\
\indent // Public Methods for Output\\
\indent {\em int sendSelf(int commitTag, Channel \&theChannel);}\\ 
\indent {\em int recvSelf(int commitTag, Channel \&theChannel,
FEM\_ObjectBroker \&theBroker);}\\ 
\indent {\em int Print(OPS_Stream \&s, int flag = 0);}\\

\noindent {\bf Constructors} \\
\indent {\em ArcLength(double dS, double alpha = 1.0);}\\ \\
The integer INTEGRATOR\_TAGS\_ArcLength (defined in
$<$classTags.h$>$) is passed to the StaticIntegrator classes
constructor. The value of $\alpha$ is set to {\em alpha} and 
$\Delta s$ to {\em dS}. \\

\noindent {\bf Destructor} \\
\indent {\em $\tilde{ }$ArcLength();}\\ 
Invokes the destructor on the Vector objects created in {\em
domainChanged()}. \\

\noindent {\bf Public Methods}\\

{\em int newStep(void);} \\
{\em newStep()} performs the first iteration, that is it solves for 
$\lambda_n^{(1)}$ and $\Delta \U_n^{(1)}$ and updates the model with
$\Delta \U_n^{(1)}$ and increments the load factor by
$\lambda_n^{(1)}$. To do this it must set the rhs of the LinearSOE to
$\P$, invoke {\em formTangent()} on itself and solve the LinearSOE to
get $\Delta \Uh_n^{(1)}$. \\

{\em int update(const Vector \&$\Delta U$);} \\
Note the argument $\Delta U$ should be equal to $\Delta \Ub_n^{(i)}$.
The object then determines $\Delta \Uh_n^{(i)}$ by setting the rhs of
the linear system of equations to be $\P$ and then solving the
linearSOE. It then solves for
$\Delta \lambda_n^{(i)}$ and $\Delta \U_n^{(i)}$ and updates the model with
$\Delta \U_n^{(i)}$ and increments the load factor by $\Delta
\lambda_n^{(i)}$. Sets the vector $x$ in the LinearSOE object to be
equal to $\Delta \U_n^{(i)}$ before returning (this is for the
convergence test stuff. \\


\indent {\em int domainChanged(void); }\\ 
The object creates the Vector objects it needs. Vectors are created to
stor $\P$, $\Delta \Ub_n^{(i)}$, $\Delta \Uh_n^{(i)}$, $\Delta
\Ub_n^{(i)}$, $dU^{(i)}$. To form $\P$, the current load factor is
obtained from the model, it is incremented by $1.0$, {\em
formUnbalance()} is invoked on the object, and the $b$ vector is
obtained from the linearSOE. This is $\P$, the load factor on the
model is then decremented by $1.0$. \\

{\em int sendSelf(int commitTag, Channel \&theChannel); } \\ 
Places the values of $\Delta s$ and $\alpha$ in a
vector of size $2$ and invokes {\em sendVector()} on {\em theChannel}.
Returns $0$ if successful, a warning message is printed and
a $-1$ is returned if {\em theChannel} fails to send the Vector. \\

{\em int recvSelf(int commitTag, Channel \&theChannel, 
FEM\_ObjectBroker \&theBroker); } \\ 
Receives in a Vector of size 2 the values of $\Delta s$ and $\alpha$.
Returns $0$ if successful, a warning message is printed, $\delta
\lambda$ is set to $0$, and a $-1$ is returned if {\em theChannel}
fails to receive the Vector.\\

{\em int Print(OPS_Stream \&s, int flag = 0);}\\
The object sends to $s$ its type, the current value of $\lambda$, and
$\delta \lambda$. 

\pagebreak \subsubsection{ ArcLength1}
%File: ~/OOP/analysis/integrator/ArcLength1.tex
%What: "@(#) ArcLength1.tex, revA"

\noindent {\bf Files}   \\
\indent \#include $<\tilde{ }$/analysis/integrator/ArcLength1.h$>$  \\

\noindent {\bf Class Declaration}  \\
\indent class ArcLength1: public StaticIntegrator  \\

\noindent {\bf Class Hierarchy} \\
\indent MovableObject \\
\indent\indent Integrator \\
\indent\indent\indent IncrementalIntegrator \\
\indent\indent\indent\indent StaticIntegrator \\
\indent\indent\indent\indent\indent {\bf ArcLength1} \\

\noindent {\bf Description} \\ 
\indent ArcLength1 is a subclass of StaticIntegrator, it is
used to when performing a static analysis on the FE\_Model using a
simplified form of the arc length method. In the arc length method
implemented by this class, the following constraint equation is added to
equation~\ref{staticFormTaylor} of the StaticIntegrator class: 

\begin{equation}
\Delta \U_n^T \Delta \U_n  + \alpha^2 \Delta \lambda_n^2  = \Delta s^2
\end{equation}

where 

\[
\Delta \U_n = \sum_{j=1}^{i} \Delta \U_n^{(j)} = \Delta \U_n^{(i)} +
d\U^{(i)} 
\]

\[
\Delta \lambda_n = \sum_{j=1}^{i} \Delta \lambda_n^{(j)} = \Delta \lambda_n^{(i)} +
d\lambda^{(i)} 
\]

\noindent this equation cannot be added directly into
equation~\ref{staticFormTaylor} to produce a linear system of $N+1$
unknowns. To add this equation we make some assumptions ala Yang
(REF), which in so doing allows us to solve a system of $N$
unknowns using the method of ??(REF).  Rewriting
equation~\ref{staticFormTaylor} as  

\[
\K_n^{(i)} \Delta \U_n^{(i)} = \Delta \lambda_n^{(i)} \P +
\lambda_n^{(i)} \P - \F_R(\U_n^{(i)}) = \Delta \lambda_n^{(i)} \P + \R_n^{(i)}
\]

\noindent The idea of ?? is to separate this into two equations:

\def\Uh{\dot{\bf U}}
\def\Ub{\overline{\bf U}}

\[
\K_n^{(i)} \Delta \Uh_n^{(i)} = \P
\]

\[
\K_n^{(i)} \Delta \Ub_n^{(i)} = \R_n^{(i)}
\]

\noindent where now

\begin{equation}
 \Delta \U_n^{(i)} = \Delta \lambda_n^{(i)} \Delta \Uh_n^{(i)} +
\Delta \Ub_n^{(i)}  
\label{splitForm}
\end{equation}

\noindent We now rewrite the constraint equation based on two conditions:

\begin{enumerate}
\item {\bf $i = 1$}: assuming $\R_n^{(1)} = \zero$, i.e. the system is
in equilibrium at the start of the iteration, the following is obtained

\[  \Delta \U_n^{(1)} = \Delta \lambda_n^{(1)} \Delta \Uh_n^{(1)} + \zero \]

\[ \Delta \lambda_n^{(1)} = \begin{array}{c} + \\ - \end{array}
\sqrt{\frac{\Delta s^2}{\Uh^T \Uh + \alpha^2}} \]

\noindent The question now is whether {\bf +} or {\bf -} should be
used. In this class, $d \lambda$ from the previous iteration $(n-1)$
is used, if it was positive {\bf +} is assumed, otherwise {\bf -}. This may
change. There are other ideas: ?(REF) number of negatives on diagonal
of decomposed matrix, ...

\item {\bf $i > 1$}

\[ \left( \Delta \U_n^{(i)} + d\U^{(i)} \right)^T \left( \Delta \U_n^{(i)} +
d\U^{(i)} \right) + \alpha^2 \left( \Delta \lambda_n^{(i)} + d\lambda^{(i)}
\right)^2 = \Delta s^2 \]

\[
\Delta {\U_n^{(i)}}^T\Delta \U_n^{(i)} + 2{d\U^{(i)}}^T\Delta \U_n^{(i)} + {d\U^{(i)}}^T d\U^{(i)}
+ \alpha^2 \Delta {\lambda_n^{(i)}}^2
+ 2 \alpha^2 d\lambda^{(i)} \Delta \lambda_n^{(i)} + \alpha^2 {d\lambda^{(i)}}^2
= \Delta s^2
\]

\noindent assuming the constraint equation was solved at $i-1$,
i.e. ${d\U^{(i)}}^T d\U^{(i)} + \alpha^2 {d\lambda^{(i)}}^2 = \Delta s^2$, this reduces to

\[
\Delta {\U_n^{(i)}}^T\Delta \U_n^{(i)} + 2{d\U^{(i)}}^T\Delta \U_n^{(i)} + 
\alpha^2 \Delta {\lambda_n^{(i)}}^2
+ 2 \alpha^2 d\lambda^{(i)} \Delta \lambda_n^{(i)} 
= 0
\]

 \noindent For our ArcLength1 method we make the ADDITIONAL assumption that
$ 2{d\U^{(i)}}^T\Delta \U_n^{(i)} 
+ 2 \alpha^2 d\lambda^{(i)} \Delta \lambda_n^{(i)} $ $>>$
$ 
\Delta {\U_n^{(i)}}^T\Delta \U_n^{(i)} +
\alpha^2 \Delta {\lambda_n^{(i)}}^2
$
the constraint equation at step $i$ reduces to

\[
{d\U^{(i)}}^T\Delta \U_n^{(i)} 
+ \alpha^2 d\lambda^{(i)} \Delta \lambda_n^{(i)} = 0
\]

\noindent hence if the class was to solve an $N+1$ system of equations at
each step, the system would be:

\[ \left[
\begin{array}{cc}
\K_n^{(i)} & -\P \\
{d\U^{(i)}}^T & \alpha^2 d\lambda^{(i)} 
\end{array} \right] 
\left\{
\begin{array}{c}
\Delta \U_n^{(i)} \\
\Delta \lambda_n^{(i)}
\end{array} \right\} = \left\{
\begin{array}{c}
\lambda_n^{(i)} \P - \F_R(\U_n^{(i)}) \\
0
\end{array} \right\}
\]

\noindent instead of solving an $N+1$ system, equation~\ref{splitForm}
is used to give

\[
{d\U^{(i)}}^T \left(\Delta \lambda_n^{(i)} \Delta \Uh_n^{(i)} + \Delta
\Ub_n^{(i)}\right) 
+ \alpha^2 d\lambda^{(i)} \Delta \lambda_n^{(i)} = 0
\]

\noindent which knowing $\Uh_n^{(i)}$ and $\Ub_n^{(i)}$ can
be solved for $\Delta \lambda_n^{(i)}$ 

\[
\Delta \lambda_n^{(i)} = -\frac{{d\U^{(i)}}^T \Delta \Ub_n^{(i)}}{{d\U^{(i)}}^T \Delta
\Uh_n^{(i)} + \alpha^2 d\lambda^{(i)}}
\]

\end{enumerate}
\noindent {\bf Class Interface} \\
\indent // Constructors \\
\indent {\em ArcLength1(double arc, double $\alpha$ = 1.0);}\\ \\
\indent // Destructor \\
\indent {\em $\tilde{ }$ArcLength1();}\\  \\
\indent // Public Methods \\
\indent {\em int newStep(void);} \\
\indent {\em int update(const Vector \&$\Delta U$);} \\
\indent {\em int domainChanged(void); }\\ \\
\indent // Public Methods for Output\\
\indent {\em int sendSelf(int commitTag, Channel \&theChannel);}\\ 
\indent {\em int recvSelf(int commitTag, Channel \&theChannel,
FEM\_ObjectBroker \&theBroker);}\\ 
\indent {\em int Print(ostream \&s, int flag = 0);}\\

\noindent {\bf Constructors} \\
\indent {\em ArcLength1(double dS, double alpha = 1.0);}\\ \\
The integer INTEGRATOR\_TAGS\_ArcLength1 (defined in
$<$classTags.h$>$) is passed to the StaticIntegrator classes
constructor. The value of $\alpha$ is set to {\em alpha} and 
$\Delta s$ to {\em dS}. \\

\noindent {\bf Destructor} \\
\indent {\em $\tilde{ }$ArcLength1();}\\ 
Invokes the destructor on the Vector objects created in {\em
domainChanged()}. \\

\noindent {\bf Public Methods}\\

{\em int newStep(void);} \\
{\em newStep()} performs the first iteration, that is it solves for 
$\lambda_n^{(1)}$ and $\Delta \U_n^{(1)}$ and updates the model with
$\Delta \U_n^{(1)}$ and increments the load factor by
$\lambda_n^{(1)}$. To do this it must set the rhs of the LinearSOE to
$\P$, invoke {\em formTangent()} on itself and solve the LinearSOE to
get $\Delta \Uh_n^{(1)}$. \\

{\em int update(const Vector \&$\Delta U$);} \\
Note the argument $\Delta U$ should be equal to $\Delta \Ub_n^{(i)}$.
The object then determines $\Delta \Uh_n^{(i)}$ by setting the rhs of
the linear system of equations to be $\P$ and then solving the
linearSOE. It then solves for
$\Delta \lambda_n^{(i)}$ and $\Delta \U_n^{(i)}$ and updates the model with
$\Delta \U_n^{(i)}$ and increments the load factor by $\Delta
\lambda_n^{(i)}$. Sets the vector $x$ in the LinearSOE object to be
equal to $\Delta \U_n^{(i)}$ before returning (this is for the
convergence test stuff. \\


\indent {\em int domainChanged(void); }\\ 
The object creates the Vector objects it needs. Vectors are created to
stor $\P$, $\Delta \Ub_n^{(i)}$, $\Delta \Uh_n^{(i)}$, $\Delta
\Ub_n^{(i)}$, $dU^{(i)}$. To form $\P$, the current load factor is
obtained from the model, it is incremented by $1.0$, {\em
formUnbalance()} is invoked on the object, and the $b$ vector is
obtained from the linearSOE. This is $\P$, the load factor on the
model is then decremented by $1.0$. \\

{\em int sendSelf(int commitTag, Channel \&theChannel); } \\ 
Places the values of $\Delta s$ and $\alpha$ in a
vector of size $2$ and invokes {\em sendVector()} on {\em theChannel}.
Returns $0$ if successful, a warning message is printed and
a $-1$ is returned if {\em theChannel} fails to send the Vector. \\

{\em int recvSelf(int commitTag, Channel \&theChannel, 
FEM\_ObjectBroker \&theBroker); } \\ 
Receives in a Vector of size 2 the values of $\Delta s$ and $\alpha$.
Returns $0$ if successful, a warning message is printed, $\delta
\lambda$ is set to $0$, and a $-1$ is returned if {\em theChannel}
fails to receive the Vector.\\

{\em int Print(ostream \&s, int flag = 0);}\\
The object sends to $s$ its type, the current value of $\lambda$, and
$\delta \lambda$. 

\pagebreak \subsubsection{ MinUnbalDispNorm}
%File: ~/OOP/analysis/integrator/MinUnbalDispNorm.tex
%What: "@(#) MinUnbalDispNorm.tex, revA"

\noindent {\bf Files}   \\
\indent \#include $<\tilde{ }$/analysis/integrator/MinUnbalDispNorm.h$>$  \\

UNDER CONSTRUCTION.\\

\noindent {\bf Class Declaration}  \\
\indent class MinUnbalDispNorm: public StaticIntegrator  \\

\noindent {\bf Class Hierarchy} \\
\indent MovableObject \\
\indent\indent Integrator \\
\indent\indent\indent IncrementalIntegrator \\
\indent\indent\indent\indent StaticIntegrator \\
\indent\indent\indent\indent\indent {\bf MinUnbalDispNorm} \\

\noindent {\bf Description} \\ 
\indent MinUnbalDispNorm is a subclass of StaticIntegrator, it is
used to when performing a static analysis on the FE\_Model using the
minimum unbalanced displacement norm method. In this method WHAT

SOME THEORY.\\

\noindent {\bf Class Interface} \\
\indent // Constructors \\
\indent {\em MinUnbalDispNorm(double $\delta \lambda_1$, int Jd,
double $\delta \lambda_{min}$, double $\delta \lambda_{max}$);}\\ \\
\indent // Destructor \\
\indent {\em $\tilde{ }$MinUnbalDispNorm();}\\  \\
\indent // Public Methods \\
\indent {\em int newStep(void);} \\
\indent {\em int update(const Vector \&$\Delta U$);} \\
\indent {\em int domainChanged(void);} \\ \\
\indent // Public Methods for Output\\
\indent {\em int sendSelf(int commitTag, Channel \&theChannel);}\\ 
\indent {\em int recvSelf(int commitTag, Channel \&theChannel,
FEM\_ObjectBroker \&theBroker);}\\ 
\indent {\em int Print(ostream \&s, int flag = 0);}\\

\noindent {\bf Constructors} \\
\indent {\em MinUnbalDispNorm(int node, int dof, double $\delta Uc_1$, int Jd,  
double $\delta Uc_{min}$, double $\delta Uc_{max}$);}\\ 
The integer INTEGRATOR\_TAGS\_MinUnbalDispNorm (defined in
$<$classTags.h$>$) is passed to the StaticIntegrator classes
constructor. 


\noindent {\bf Destructor} \\
\indent {\em $\tilde{ }$MinUnbalDispNorm();}\\ 
Does nothing. \\

\noindent {\bf Public Methods}\\

{\em int newStep(void);} \\
WHAT DO I DO?\\

{\em int update(const Vector \&$\Delta U$);} \\
WHAT DO I DO?\\

{\em int sendSelf(int commitTag, Channel \&theChannel); } \\ 
WHAT DO I DO?\\

{\em int recvSelf(int commitTag, Channel \&theChannel, 
FEM\_ObjectBroker \&theBroker); } \\ 
WHAT DO I DO?\\

{\em int Print(ostream \&s, int flag = 0);}\\
WHAT DO I DO?

\pagebreak \subsubsection{{\bf TransientIntegrator}}
%File: ~/OOP/analysis/integrator/TransientIntegrator.tex
%What: "@(#) TransientIntegrator.tex, revA"

\noindent {\bf Files}   \\
\indent \#include $<\tilde{ }$/analysis/integrator/TransientIntegrator.h$>$  \\

\noindent {\bf Class Declaration}  \\
\indent class TransientIntegrator: public Integrator  \\

\noindent {\bf Class Hierarchy} \\
\indent MovableObject \\
\indent\indent Integrator \\
\indent\indent\indent IncrementalIntegrator \\
\indent\indent\indent\indent {\bf TransientIntegrator} \\
\indent\indent\indent\indent\indent Newmark \\
\indent\indent\indent\indent\indent HHT \\
\indent\indent\indent\indent\indent Wilson-$\Theta$ \\

\noindent {\bf Description} \\ 
\indent TransientIntegrator is an abstract subclass of IncrementalIntegrator.
A subclass of it is used when performing a nonlinear transient
analysis of the problem using a direct integration method. The
TransientIntegrator class redefines the {\em formTangent()} method of
the IncrementalIntegrator class and it defines a new method {\em
newStep()} which is invoked by the DirectIntegrationAnalysis class at
each new time step.  \\

In nonlinear transient finite element problems we seek a solution
($\U$, $\dot \U$, $\ddot \U$) to the nonlinear vector function

\begin{equation}
\R(\U,\Ud, \Udd) = \P(t) - \F_I(\Udd) - \F_R(\U, \Ud) = \zero
\label{femGenForm}
\end{equation}


The most widely used technique for solving the transient non-linear 
finite element equation, equation~\ref{femGenForm}, is to use an
incremental direct integration scheme. In the incremental formulation,
a solution to the equation is sought at successive time steps $\Delta
t$ apart.  

\begin{equation}
\R(\U_{n \Delta t},\Ud_{n \Delta t}, \Udd_{n \Delta t}) = \P(n \Delta t) -
\F_I(\Udd_{n \Delta t}) - \F_R(\U_{n \Delta t}, \Ud_{n \Delta t})
\label{fullTimeForm}
\end{equation}

For each time step, t, the integration schemes provide two operators,
$\I_1$ and $\I_2$, to relate the velocity and accelerations at the 
time step as a function of the displacement at the time step and the
response at previous time steps: 

\begin{equation} 
\dot \U_{t} = {\I}_1 (\U_t, \U_{t-\Delta t}, \dot \U_{t-\Delta t},
\ddot \U_{t - \Delta t}, \U_{t - 2\Delta t}, \dot \U_{t - 2 \Delta t}. ..., )
\label{I1}
\end{equation} 

\begin{equation} 
\ddot \U_{t} = {\I}_2 (\U_t, \U_{t-\Delta t}, \dot \U_{t-\Delta t},
\ddot \U_{t - \Delta t}, \U_{t - 2\Delta t}, \dot \U_{t - 2 \Delta t}. ..., )
\label{I2}
\end{equation} 

These allow us to rewrite equation~\ref{fullTimeForm}, in terms of a
single response quantity, typically the displacement:

\begin{equation}
\R(\U_t) = \P(t) - \F_I(\Udd_t) - \F_R(\U_t, \Ud_t)
\label{genForm}
\end{equation}

The solution of this equation is typically obtained using an iterative
procedure, i.e. making an initial prediction for $\U_{t}$,
denoted $\U_{t}^{(0)}$ a sequence of approximations $\U_{t}^{(i)}$,
$i=1,2, ..$ is obtained which converges (we hope) to the solution $\U_{t}$. The
most frequently used iterative schemes, such as Newton-Raphson,
modified Newton, and quasi Newton schemes, are based on a Taylor
expansion of equation~\ref{genForm} about $\U_{t}$:    

\begin{equation} 
\R(\U_{t}) = 
\R(\U_{t}^{(i)}) +
\left[ {\frac{\partial \R}{\partial \U_t} \vert}_{\U_{t}^{(i)}}\right]
\left( \U_{t} - \U_{t}^{(i)} \right) 
\end{equation}

$$
\R(\U_{t}) = 
 \P (t) 
 - \f_{I} \left( \ddot \U_{t}^{(i)} \right) -
\f_{R} \left( \dot \U_{t}^{(i)}, \U_{t}^{(i)} \right)
$$
\begin{equation} 
- \left[
   \M^{(i)} {\I}_2'
+  \C^{(i)} {\I}_1'
+ \K^{(i)}  \right]
 \left( \U_{t} - \U_{t}^{(i)} \right)
\label{femGenFormTaylor}
\end{equation} 

To start the iteration scheme, trial values for $\U_{t}$, $\dot
\U_{t} $ and $\ddot \U_{t} $ are required. These are
obtained by assuming $\U_{t}^{(0)} = \U_{t-\Delta t}$. The
$\dot \U_{t}^{(0)} $ and $\ddot \U_{t}^{(0)}$
can then be obtained from the operators for the integration scheme. \\

Subclasses of TransientIntegrators provide
methods informing the FE\_Element and DOF\_Group objects how to build
the tangent and residual matrices and vectors. They also provide the
method for updating the response quantities at the DOFs with
appropriate values; these values being some function of the solution
to the linear system of equations. \\ 


\noindent {\bf Class Interface } \\
\indent // Constructor \\
\indent {\em TransientIntegrator(int classTag);}\\  \\
\indent // Destructor  \\ 
\indent {\em virtual~ $\tilde{}$TransientIntegrator();}\\  \\
\indent // Public Methods \\
\indent {\em virtual int formTangent(void);} \\
\indent {\em virtual int formEleResidual(FE\_Element *theEle);} \\
\indent {\em virtual int formNodUnbalance(DOF\_Group *theDof);} \\
\indent {\em virtual void newStep(double deltaT) =0;} \\

\noindent {\bf Constructor} \\
\indent {\em TransientIntegrator(int classTag);}\\ 
The integer {\em classTag} is passed to the IncrementalIntegrator
classes constructor. \\

\noindent {\bf Destructor} \\
\indent {\em virtual~ $\tilde{}$TransientIntegrator();}\\ 
Does nothing. \\

\noindent {\bf Public Methods}\\
\indent {\em virtual int formTangent(void);} \\
Invoked to form the structure tangent matrix. The method is rewritten
for this class to include inertia effects from the nodes. The method
iterates over both the FE\_Elements and DOF\_Groups invoking methods
to form their contributions to the $A$ matrix of the LinearSOE and
then adding these contributions to the $A$ matrix. The method performs
the following:
\begin{tabbing}
while \= \+ while \= while \= \kill
theSysOfEqn.zeroA();\\
DOF\_EleIter \&theDofs = theAnalysisModel.getDOFs();\\
while((dofPtr = theDofs()) $\neq$ 0) \+ \\
dofPtr-$>$formTangent(theIntegrator); \\
theSOE.addA(dofPtr-$>$getTangent(this),
dofPtr-$>$getID()) \- \\
FE\_EleIter \&theEles = theAnalysisModel.getFEs();\\
while((elePtr = theEles()) $\neq$ 0) \+ \\
theSOE.addA(elePtr-$>$getTangent(this),
elePtr-$>$getID(), $1.0$) \\
\end{tabbing}
\noindent Returns $0$ if successful, otherwise a $-1$ if an error occurred while
trying to add the stiffness. The two loops are introduced for the
FE\_Elements, to allow for efficient parallel programming when the
FE\_Elements are associated with a ShadowSubdomain.\\ 

{\em virtual int formEleResidual(FE\_Element *theEle);} \\
Called upon by the FE\_Element {\em theEle} to determine it's
contribution to the rhs of the equation. The following are invoked
before $0$ is returned.
\begin{tabbing}
while \= \+ while \= while \= \kill
theEle-$>$zeroResidual() \\
theEle-$>$addRIncInertiaToResid() \\
\end{tabbing}

{\em virtual int formNodUnbalance(DOF\_Group *theDof);} \\
Called upon by the DOF\_Group {\em theDof} to determine it's
contribution to the rhs of the equation. The following are invoked
before $0$ is returned.
\begin{tabbing}
while \= \+ while \= while \= \kill
theDof-$>$zeroUnbalance() \\
theDof-$>$addPIncInertiaToUnbalance() \\
\end{tabbing}

{\em virtual int newStep(double deltaT) =0;} \\
Invoked to inform the integrator that the transient analysis is
proceeding to the next time step. To return $0$ if successful, a
negative number if not.\\






\pagebreak \subsubsection{Newmark}
%File: ~/OOP/analysis/integrator/Newmark.tex
%What: "@(#) Newmark.tex, revA"

\noindent {\bf Files}   \\
\indent \#include $<\tilde{ }$/analysis/integrator/Newmark.h$>$  \\

\noindent {\bf Class Declaration}  \\
\indent class Newmark: public TransientIntegrator  \\

\noindent {\bf Class Hierarchy} \\
\indent MovableObject \\
\indent\indent Integrator \\
\indent\indent\indent IncrementalIntegrator \\
\indent\indent\indent\indent TransientIntegrator \\
\indent\indent\indent\indent\indent {\bf Newmark} \\

\noindent {\bf Description} \\ 
\indent Newmark is a subclass of TransientIntegrator which implements
the Newmark method. In the Newmark method, to determine the
velocities, accelerations and displacements at time $t + \Delta t$,
the equilibrium equation (expressed for the TransientIntegrator) is
typically solved at time $t + \Delta t$ for $\U_{t+\Delta t}$,
i.e. solve: 

\[ \R (\U_{t + \Delta t}) = \P(t + \Delta t) - \F_I(\Udd_{t+\Delta t})
- \F_R(\Ud_{t + \Delta t},\U_{t + \Delta t}) \]

\noindent for $\U_{t+\Delta t}$. The following difference relations
are used to relate $\Ud_{t + \Delta t}$ and $\Udd_{t + \Delta t}$ to
$\U_{t + \Delta t}$ and the response quantities at time $t$:

\[
\dot \U_{t + \Delta t} = \frac{\gamma}{\beta \Delta t}
\left( \U_{t + \Delta t} - \U_t \right)
 + \left( 1 - \frac{\gamma}{\beta}\right) \dot \U_t + \Delta t \left(1
- \frac{\gamma}{2 \beta}\right) \ddot \U_t 
\]

\[
\ddot \U_{t + \Delta t} = \frac{1}{\beta {\Delta t}^2}
\left( \U_{t+\Delta t} - \U_t \right)
 - \frac{1}{\beta \Delta t} \dot \U_t + \left( 1 - \frac{1}{2
\beta} \right) \ddot \U_t 
\]

\noindent which  results in the following 

\[ \left[ \frac{1}{\beta \Delta t^2} \M + \frac{\gamma}{\beta \Delta t}
\C + \K \right] \Delta \U_{t + \Delta t}^{(i)} = \P(t + \Delta t) -
\F_I\left(\Udd_{t+\Delta  t}^{(i-1)}\right)
- \F_R\left(\Ud_{t + \Delta t}^{(i-1)},\U_{t + \Delta t}^{(i-1)}\right) \]

\noindent An alternative approach, which does not involve $\Delta t$
in the denumerator (useful for impulse problems), is to solve for the
accelerations at time $t + \Delta t$ 

\[ \R (\Udd_{t + \Delta t}) = \P(t + \Delta t) - \F_I(\Udd_{t+\Delta t})
- \F_R(\Ud_{t + \Delta t},\U_{t + \Delta t}) \]

\noindent where we use following functions to relate $\U_{t + \Delta
t}$ and $\Ud_{t + \Delta t}$ to $\Udd_{t + \Delta t}$ and the response
quantities at time $t$:

\[
\U_{t + \Delta t} = \U_t + \Delta t \Ud_t + \left[
\left(\frac{1}{2} - \beta\right)\Udd_t + \beta \Udd_{t + \Delta
t}\right] \Delta t^2
\]

\[
\Ud_{t + \Delta t} = \Ud_t + \left[ \left(1 - \gamma\right)\Udd_t +
\gamma\Udd_{t + \Delta t}\right] \Delta t
\]

\noindent which results in the following 

\[ \left[ \M + \gamma \Delta t \C + \beta \Delta t^2 \K \right] \Delta
\Udd_{t + \Delta t}^{(i)} = \P(t + \Delta t) - \F_I\left(\Udd_{t+\Delta 
t}^{(i-1)}\right)
- \F_R\left(\Ud_{t + \Delta t}^{(i-1)},\U_{t + \Delta
t}^{(i-1)}\right) \]


\pagebreak
\noindent {\bf Class Interface} \\
\indent // Constructors \\
\indent {\em Newmark(bool dispFlag = true);}\\ 
\indent {\em Newmark(double gamma, double beta, bool dispFlag = true);}\\ 
\indent {\em Newmark(double gamma, double beta, double alphaM, double
betaK, bool dispFlag = true);}\\ \\
\indent // Destructor \\
\indent {\em virtual~ $\tilde{}$Newmark();}\\ \\
\indent // Public Methods \\
\indent {\em int formEleTangent(FE\_Element *theEle);} \\
\indent {\em int formNodTangent(DOF\_Group *theDof);} \\
\indent {\em int domainChanged(void);}\\
\indent {\em int newStep(double deltaT);}\\
\indent {\em int update(const Vector \&$\Delta U$);} \\ \\
\indent // Public Methods for Output\\
\indent {\em int sendSelf(int commitTag, Channel \&theChannel);}\\ 
\indent {\em int recvSelf(int commitTag, Channel \&theChannel,
FEM\_ObjectBroker \&theBroker);}\\ 
\indent {\em int Print(OPS_Stream \&s, int flag = 0);}\\


\noindent {\bf Constructors} \\
\indent {\em Newmark(bool dispFlag = true);}\\ 
Sets $\gamma$ to $1/2$ and $\beta$ to $1/4$. Sets a flag indicating
whether the incremental solution is done in terms of displacement,
$\Delta \U$, if {\em dispFlag} is {\em true}, or  
acceleration, $\Delta \ddot \U$, if {\em dispFlag} is {\em false}. In
addition, a flag is set indicating that Rayleigh damping will not be used. \\


\indent {\em Newmark(double gamma, double beta, bool dispFlag = true);}\\ 
Sets $\gamma$ to {\em gamma} and $\beta$ to {\em beta}. Sets a flag
indicating whether the incremental solution is done in terms of
displacement or acceleration to {\em dispFlag} and a flag indicating
that Rayleigh damping will not be used. \\


\indent {\em Newmark(double gamma, double beta, double alphaM, double
betaK, bool dispFlag = true);}\\ 
This constructor is invoked if Rayleigh damping is to be used, 
i.e. $\D = \alpha_M M + \beta_K K$. 
Sets $\gamma$ to {\em gamma}, $\beta$ to {\em beta}, $\alpha_M$ to
{\em alphaM} and $\beta_K$ to {\em betaK}. Sets a flag indicating whether the
incremental solution is done in terms of displacement or acceleration
to {\em dispFlag} and a flag indicating that Rayleigh damping will 
be used. \\ 

\noindent {\bf Destructor} \\
\indent {\em virtual~ $\tilde{}$Newmark();}\\ 
Invokes the destructor on the Vector objects created. \\

\noindent {\bf Public Methods}\\
\indent {\em int formEleTangent(FE\_Element *theEle);} \\
This tangent for each FE\_Element is defined to be $\K_e = c1 \K + c2
\D + c3 \M$, where c1,c2 and c3 were determined in the last invocation
of the {\em newStep()} method.  The method returns $0$ after
performing the following operations:
\begin{tabbing}
while \= \+ while \= while \= \kill
if (RayleighDamping == false) \{ \+ \\
theEle-$>$zeroTang() \\
theEle-$>$addKtoTang(c1) \\
theEle-$>$addCtoTang(c2) \\
theEle-$>$addMtoTang(c3)  \- \\
\} else \{ \+ \\
theEle-$>$zeroTang() \\
theEle-$>$addKtoTang(c1 + c2 * $\beta_K$) \\
theEle-$>$addMtoTang(c3 + c2 * $\alpha_M$)  \- \\ 
\}
\end{tabbing}



{\em int formNodTangent(DOF\_Group *theDof);} \\
The method returns $0$ after performing the following operations:
\begin{tabbing}
while \= \+ while \= while \= \kill
theDof-$>$zeroUnbalance() \\
if (RayleighDamping == false)  \+ \\
theDof-$>$addMtoTang(c3)  \- \\
else \+ \\
theDof-$>$addMtoTang(c3 + c2 * $\alpha_M$)  \- \\ 
\end{tabbing}


{\em int domainChanged(void);}\\
If the size of the LinearSOE has changed, the object deletes any old Vectors
created and then creates $6$ new Vector objects of size equal to {\em
theLinearSOE-$>$getNumEqn()}. There is a Vector object created to store
the current displacement, velocity and accelerations at times $t$ and
$t + \Delta t$. The response quantities at time $t + \Delta t$ are
then set by iterating over the DOF\_Group objects in the model and
obtaining their committed values. 
Returns $0$ if successful, otherwise a warning message and a negative
number is returned: $-1$ if no memory was available for constructing
the Vectors. \\

{\em int newStep(double $\Delta t$);}\\
The following are performed when this method is invoked:
\begin{enumerate}
\item First sets the values of the three constants {\em c1}, {\em c2}
and {\em c3} depending on the flag indicating whether incremental
displacements or accelerations are being solved for at each iteration.
If {\em dispFlag} was {\em true}, {\em c1} is set to $1.0$, {\em c2} to $
\gamma / (\beta \Delta t)$ and {\em c3} to $1/ (\beta \Delta t^2)$. If
the flag is {\em false} {\em c1} is set to $\beta \Delta t^2$, {\em c2} to $
\gamma \Delta t$ and {\em c3} to $1.0$. 
\item Then the Vectors for response quantities at time $t$ are set
equal to those at time $t + \Delta t$.
\begin{tabbing}
while \= while \= while \= while \= \kill
\>\> $ \U_t = \U_{t + \Delta t}$ \\
\>\> $ \Ud_t = \Ud_{t + \Delta t} $ \\
\>\> $ \Udd_t = \Udd_{t + \Delta t} $ 
\end{tabbing}
\item Then the velocity and accelerations approximations at time $t +
\Delta t$ are set using the difference approximations if {\em
dispFlag} was {\em true}. (displacement and velocity if {\em false}).
\begin{tabbing}
while \= while \= while \= while \= \kill
\>\> if (displIncr == true) \{ \\
\>\>\> $ \dot \U_{t + \Delta t} = 
 \left( 1 - \frac{\gamma}{\beta}\right) \dot \U_t + \Delta t \left(1
- \frac{\gamma}{2 \beta}\right) \ddot \U_t $ \\
\>\>\> $ \ddot \U_{t + \Delta t} = 
 - \frac{1}{\beta \Delta t} \dot \U_t + \left( 1 - \frac{1}{2
\beta} \right) \ddot \U_t  $ \\
\>\> \} else \{ \\
\>\>\> $ \U_{t + \Delta t} = \U_t + \Delta t \Ud_t + \frac{\Delta
t^2}{2}\Udd_t$ \\
\>\>\> $ \Ud_{t + \Delta t} = \Ud_t +  \Delta t \Udd_t $ \\
\>\> \} 
\end{tabbing}
\item The response quantities at the DOF\_Group objects are updated
with the new approximations by invoking {\em setResponse()} on the
AnalysisModel with new quantities for time $t + \Delta t$.
\begin{tabbing}
while \= while \= while \= while \= \kill
\>\> theModel-$>$setResponse$(\U_{t + \Delta t}, \Ud_{t+\Delta t},
\Udd_{t+\Delta t})$ 
\end{tabbing}
\item current time is obtained from the AnalysisModel, incremented by
$\Delta t$, and {\em applyLoad(time, 1.0)} is invoked on the
AnalysisModel. 
\item Finally {\em updateDomain()} is invoked on the AnalysisModel.
\end{enumerate}
The method returns $0$ if successful, otherwise a negative number is
returned: $-1$ if $\gamma$ or $\beta$ are $0$, $-2$ if {\em dispFlag}
was true and $\Delta t$ is $0$, and $-3$ if {\em domainChanged()}
failed or has not been called. \\

{\em int update(const Vector \&$\Delta U$);} \\
Invoked this causes the object to increment the DOF\_Group
response quantities at time $t + \Delta t$. The displacement Vector is  
incremented by $ c1 * \Delta U$, the velocity Vector by $
c2 * \Delta U$, and the acceleration Vector by $c3 * \Delta U$.
The response at the DOF\_Group objects are then updated by invoking
{\em setResponse()} on the AnalysisModel with quantities at time $t +
\Delta t$. Finally {\em updateDomain()} is invoked on the 
AnalysisModel. 
\begin{tabbing}
while \= while \= while \= while \= \kill
\>\> if (displIncr == true) \{ \\
\>\>\> $ \U_{t + \Delta t} += \Delta \U$ \\
\>\>\> $ \dot \U_{t + \Delta t} += \frac{\gamma}{\beta \Delta t} \Delta \U $\\
\>\>\> $ \ddot \U_{t + \Delta t} += \frac{1}{\beta {\Delta t}^2} \Delta
\U $\\
\>\> \} else \{ \\
\>\>\> $ \Udd_{t + \Delta t} += \Delta \Udd$ \\
\>\>\> $ \U_{t + \Delta t} += \beta \Delta t^2 \Delta \Udd $\\
\>\>\> $ \Ud_{t + \Delta t} += \gamma \Delta t \Delta \Udd $\\
\>\> \} \\
\>\> theModel-$>$setResponse$(\U_{t + \Delta t}, \Ud_{t+\Delta t},
\Udd_{t+\Delta t})$ \\
\>\> theModel-$>$setUpdateDomain()
\end{tabbing}
Returns $0$ if successful. A warning message is printed and a negative number
returned if an error occurs: $-1$ if no associated AnalysisModel, $-2$
if the Vector objects have not been created, $-3$ if the Vector
objects and $\delta U$ are of different sizes. \\

{\em int sendSelf(int commitTag, Channel \&theChannel); } \\ 
Places in a Vector of size 6 the values of $\beta$, $\gamma$, {\em
dispFlag}, RayleighDampingFlag, $\alpha_M$ and $\beta_K$.  Then
invokes {\em sendVector()} on the Channel with this Vector. Returns
$0$ if successful, a warning message is printed and a $-1$ is 
returned if {\em theChannel} fails to send the Vector. \\ 

{\em int recvSelf(int commitTag, Channel \&theChannel, 
FEM\_ObjectBroker \&theBroker); } \\ 
Receives in a Vector of size 6 the values of $\beta$, $\gamma$, {\em
dispFlag}, RayleighDampingFlag, $\alpha_M$ and $\beta_K$. Returns $0$
if successful. A warning message is printed, $\gamma$ is set to 0.5,
$\beta$ to 0.25 and the Rayleigh damping flag set to {\em false}, and
a $-1$ is returned, if {\em theChannel} fails to receive the Vector.\\ 

{\em int Print(OPS_Stream \&s, int flag = 0);}\\
The object sends to $s$ its type, the current time, $\gamma$ and
$\beta$. If Rayleigh damping is specified, the constants $\alpha_M$ and
$\beta_K$ are also printed.








\pagebreak \subsubsection{HHT}
%File: ~/OOP/analysis/integrator/HHT.tex
%What: "@(#) HHT.tex, revA"

\noindent {\bf Files}   \\
\indent \#include $<\tilde{ }$/analysis/integrator/HHT.h$>$  \\

\noindent {\bf Class Declaration}  \\
\indent class HHT: public TransientIntegrator  \\

\noindent {\bf Class Hierarchy} \\
\indent MovableObject \\
\indent\indent Integrator \\
\indent\indent\indent IncrementalIntegrator \\
\indent\indent\indent\indent TransientIntegrator \\
\indent\indent\indent\indent\indent {\bf HHT} \\

\noindent {\bf Description} \\ 
\indent HHT is a subclass of TransientIntegrator which implements
the Hilber-Hughes-Taylor (HHT) method. In the HHT method, to determine the
velocities, accelerations and displacements at time $t + \Delta t$,
by solving the following equilibrium equation

\[ \R (\U_{t + \Delta t}) = \P(t + \Delta t) -
\F_I(\Udd_{t+\Delta t}) - \F_R(\Ud_{t + \alpha \Delta t},\U_{t +
\alpha \Delta t}) \] 

\noindent where

\[ \U_{t + \alpha} = \left( 1 - \alpha \right) \U_t + \alpha \U_{t +
\Delta t} \]

\[ \Ud_{t + \alpha} = \left( 1 - \alpha \right) \Ud_t + \alpha \Ud_{t +
\Delta t} \]

\noindent and the velocities and accelerations at time $t + \Delta t$
are determined using the Newmark relations. The HHT method results in
the following for determining the response at $t + \Delta t$

\[ \left[ \frac{1}{\beta \Delta t^2} \M + \frac{\alpha \gamma}{\beta
\Delta t} \C + \alpha \K \right] \Delta \U_{t + \Delta t}^{(i)} = \P(t
+ \Delta t) - \F_I\left(\Udd_{t+\Delta  t}^{(i-1)}\right)
- \F_R\left(\Ud_{t + \alpha \Delta t}^{(i-1)},\U_{t + \alpha \Delta
t}^{(i-1)}\right) \] 
\\

\noindent {\bf Class Interface} \\
\indent // Constructors \\
\indent {\em HHT();}\\ 
\indent {\em HHT(double alpha);}\\ 
\indent {\em HHT(double alpha, double alphaM, double betaK);}\\ \\
\indent // Destructor \\
\indent {\em virtual~ $\tilde{}$HHT();}\\ \\
\indent // Public Methods \\
\indent {\em int formEleTangent(FE\_Element *theEle);} \\
\indent {\em int formNodTangent(DOF\_Group *theDof);} \\
\indent {\em int domainChanged(void);}\\
\indent {\em int newStep(double deltaT);}\\
\indent {\em int update(const Vector \&$\Delta U$);} \\ 
\indent {\em int commit(void);}\\ \\
\indent // Public Methods for Output\\
\indent {\em int sendSelf(int commitTag, Channel \&theChannel);}\\ 
\indent {\em int recvSelf(int commitTag, Channel \&theChannel,
FEM\_ObjectBroker \&theBroker);}\\ 
\indent {\em int Print(ostream \&s, int flag = 0);}\\

\noindent {\bf Constructors} \\
\indent {\em HHT();}\\ 
The integer INTEGRATOR\_TAGS\_HHT is passed to the TransientIntegrator
constructor. $\alpha$, $\beta$ and $\gamma$ are set to 0.0. This
constructor should only be invoked by an FEM\_ObjectBroker. \\

\indent {\em HHT(double alpha);}\\ 
Sets $\alpha$ to {\em alpha}, $\gamma$ to $(1.5 - \alpha)$ and $\beta$
to $0.25*\alpha^2$. In addition, a flag is set indicating that Rayleigh
damping will not be used. \\

\indent {\em HHT(double alpha, double alphaM, double betaK);}\\ 
This constructor is invoked if Rayleigh damping is to be used, 
i.e. $\D = \alpha_M M + \beta_K K$. Sets $\alpha$ to {\em alpha},
$\gamma$ to $(1.5 - \alpha)$, $\beta$ to $0.25*\alpha^2$, $\alpha_M$ to
{\em alphaM} and $\beta_K$ to {\em betaK}. Sets a flag indicating
whether the incremental solution is done in terms of displacement or
acceleration to {\em dispFlag} and a flag indicating that Rayleigh
damping will be used. \\ 

Sets $\alpha$ to {\em alpha}, $\gamma$ to $(1.5 - \alpha)$ and $\beta$
to $0.25*\alpha^2$. In addition, a flag is set indicating that Rayleigh
damping will not be used. \\

\noindent {\bf Destructor} \\
\indent {\em virtual~ $\tilde{}$HHT();}\\ 
Invokes the destructor on the Vector objects created. \\

\noindent {\bf Public Methods}\\
\indent {\em int formEleTangent(FE\_Element *theEle);} \\
This tangent for each FE\_Element is defined to be $\K_e = c1\alpha \K
+ c2\alpha \D + c3 \M$, where c1,c2 and c3 were determined in the last invocation
of the {\em newStep()} method. Returns $0$ after performing the
following operations:  
\begin{tabbing}
while \= \+ while \= while \= \kill
if (RayleighDamping == false) \{ \+ \\
theEle-$>$zeroTang() \\
theEle-$>$addKtoTang(c1) \\
theEle-$>$addCtoTang(c2) \\
theEle-$>$addMtoTang(c3)  \- \\
\} else \{ \+ \\
theEle-$>$zeroTang() \\
theEle-$>$addKtoTang(c1 + c2 * $\beta_K$) \\
theEle-$>$addMtoTang(c3 + c2 * $\alpha_M$)  \- \\ 
\}
\end{tabbing}

{\em int formNodTangent(DOF\_Group *theDof);} \\
This performs the following:
\begin{tabbing}
while \= \+ while \= while \= \kill
if (RayleighDamping == false)  \+ \\
theDof-$>$addMtoTang(c3)  \- \\
else \+ \\
theDof-$>$addMtoTang(c3 + c2 * $\alpha_M$)  \- \\ 
\end{tabbing}


{\em int domainChanged(void);}\\
If the size of the LinearSOE has changed, the object deletes any old Vectors
created and then creates $8$ new Vector objects of size equal to {\em
theLinearSOE-$>$getNumEqn()}. There is a Vector object created to store
the current displacement, velocity and accelerations at times $t$ and
$t + \Delta t$, and the displacement and velocity at time $t + \alpha
\Delta t$. The response quantities at time $t + \Delta t$ are
then set by iterating over the DOF\_Group objects in the model and
obtaining their committed values. 
Returns $0$ if successful, otherwise a warning message and a negative
number is returned: $-1$ if no memory was available for constructing
the Vectors. \\


{\em int newStep(double $\Delta t$);}\\
The following are performed when this method is invoked:
\begin{enumerate}
\item First sets the values of the three constants {\em c1}, {\em c2}
and {\em c3}, {\em c1} is set to $1.0$, {\em c2} to $
\gamma / (\beta * \Delta t)$ and {\em c3} to $1/ (\beta * \Delta t^2)$.
\item Then the Vectors for response quantities at time $t$ are set
equal to those at time $t + \Delta t$.
\begin{tabbing}
while \= while \= while \= while \= \kill
\>\> $ \U_t = \U_{t + \Delta t}$ \\
\>\> $ \Ud_t = \Ud_{t + \Delta t} $ \\
\>\> $ \Udd_t = \Udd_{t + \Delta t} $ 
\end{tabbing}
\item Then the velocity and accelerations approximations at time $t +
\Delta t$ and the displacement and velocity at time $t + \alpha \Delta t$
are set using the difference approximations.
\begin{tabbing}
while \= while \= while \= while \= \kill
\>\> $ \U_{t + \alpha \Delta t} = \U_t$ \\
\>\> $ \dot \U_{t + \Delta t} = 
 \left( 1 - \frac{\gamma}{\beta}\right) \dot \U_t + \Delta t \left(1
- \frac{\gamma}{2 \beta}\right) \ddot \U_t $ \\
\>\> $ \Ud_{t + \alpha \Delta t} = (1 - \alpha) \Ud_t + \alpha \Ud_{t +
\Delta t}$ \\
\>\> $ \ddot \U_{t + \Delta t} = 
 - \frac{1}{\beta \Delta t} \dot \U_t + \left( 1 - \frac{1}{2
\beta} \right) \ddot \U_t  $ \\
\>\> theModel-$>$setResponse$(\U_{t + \alpha \Delta t}, \Ud_{t+\alpha
\Delta t}, \Udd_{t+\Delta t})$ 
\end{tabbing}
\item The response quantities at the DOF\_Group objects are updated
with the new approximations by invoking {\em setResponse()} on the
AnalysisModel with displacements and velocities at time $t + \alpha
\Delta t$ and the accelerations at time $t + \Delta t$.
\begin{tabbing}
while \= while \= while \= while \= \kill
\>\> theModel-$>$setResponse$(\U_{t + \alpha \Delta t}, \Ud_{t+\alpha
\Delta t}, \Udd_{t+\Delta t})$ 
\end{tabbing}
\item current time is obtained from the AnalysisModel, incremented by
$\Delta t$, and {\em applyLoad(time, 1.0)} is invoked on the
AnalysisModel. 
\item Finally {\em updateDomain()} is invoked on the AnalysisModel.
\end{enumerate}
The method returns $0$ if successful, otherwise a negative number is
returned: $-1$ if $\gamma$ or $\beta$ are $0$, $-2$ if {\em dispFlag}
was true and $\Delta t$ is $0$, and $-3$ if {\em domainChanged()}
failed or has not been called. \\



{\em int update(const Vector \&$\Delta U$);} \\
Invoked this first causes the object to increment the DOF\_Group
response quantities at time $t + \Delta t$. The displacement Vector is  
incremented by $ c1 * \Delta U$, the velocity Vector by $
c2 * \Delta U$, and the acceleration Vector by $c3 * \Delta U$. 
The displacement Vector at time $t + \alpha \Delta t$ is incremented
by $c1 \alpha \Delta U$ and the velocity Vector by $c2 \alpha \Delta U$.
The response quantities at the DOF\_Group objects are then updated
with the new approximations by invoking {\em setResponse()} on the
AnalysisModel with displacement and velocity at time $t + \alpha
\Delta t$ and the accelerations at time $t + \Delta t$. 
Finally {\em updateDomain()} is invoked on the AnalysisModel. 
\begin{tabbing}
while \= \+ while \= while \= \kill
\>\> $ \U_{t + \Delta t} += \Delta \U$ \\
\>\> $ \dot \U_{t + \Delta t} += \frac{\gamma}{\beta \Delta t} \Delta \U $\\
\>\> $ \ddot \U_{t + \Delta t} += \frac{1}{\beta {\Delta t}^2} \Delta \U $\\
\>\> $ \U_{t + \alpha \Delta t} += \alpha \Delta \U $ \\
\>\> $ \Ud_{t + \alpha \Delta t} += \frac{\alpha \gamma}{\beta \Delta t}
\Delta \U $\\ 
\>\> theModel-$>$setResponse$(\U_{t + \alpha \Delta t}, \Ud_{t+\alpha
\Delta t}, \Udd_{t+\Delta t})$ \\
\>\> theModel-$>$updateDomain()
\end{tabbing}
Returns
$0$ if successful. A warning message is printed and a negative number
returned if an error occurs: $-1$ if no associated AnalysisModel, $-2$
if the Vector objects have not been created, $-3$ if the Vector
objects and $\Delta U$ are of different sizes. \\


{\em int commit(void);}\\
First the response quantities at the DOF\_Group objects are updated
with the new approximations by invoking {\em setResponse()} on the
AnalysisModel with displacement, velocity and accelerations at time $t +
\Delta t$. Finally {\em updateDomain()} and {\em commitDomain()} are
invoked on the AnalysisModel. 
Returns $0$ if successful, a warning
message and a negative number if not: $-1$ if no AnalysisModel
associated with the object and $-2$ if {\em commitDomain()} failed. \\


{\em int sendSelf(int commitTag, Channel \&theChannel); } \\ 
Places in a Vector of size 6 the values of $\alpha$, $\beta$, $\gamma$, 
RayleighDampingFlag, $\alpha_M$ and $\beta_K$.  Then
invokes {\em sendVector()} on the Channel with this Vector. Returns
$0$ if successful, a warning message is printed and a $-1$ is
returned if {\em theChannel} fails to send the Vector. \\ 

{\em int recvSelf(int commitTag, Channel \&theChannel, 
FEM\_ObjectBroker \&theBroker); } \\ 
Receives in a Vector of size 6 the values of $\alpha$, $\beta$, $\gamma$, 
RayleighDampingFlag, $\alpha_M$ and $\beta_K$. Returns $0$
if successful. A warning message is printed, and a $-1$ is returned if
{\em theChannel} fails to receive the Vector.\\

{\em int Print(ostream \&s, int flag = 0);}\\
The object sends to $s$ its type, the current time, $\alpha$, $\gamma$ and
$\beta$. If Rayleigh damping is specified, the constants $\alpha_M$ and
$\beta_K$ are also printed.




\pagebreak \subsubsection{WilsonTheta}
%File: ~/OOP/analysis/integrator/WilsonTheta.tex
%What: "@(#) WilsonTheta.tex, revA"

\noindent {\bf Files}   \\
\indent \#include $<\tilde{ }$/analysis/integrator/WilsonTheta.h$>$  \\

\noindent {\bf Class Declaration}  \\
\indent class WilsonTheta: public TransientIntegrator  \\

\noindent {\bf Class Hierarchy} \\
\indent MovableObject \\
\indent\indent Integrator \\
\indent\indent\indent IncrementalIntegrator \\
\indent\indent\indent\indent TransientIntegrator \\
\indent\indent\indent\indent\indent {\bf WilsonTheta} \\

\noindent {\bf Description} \\ 
\indent WilsonTheta is a subclass of TransientIntegrator which implements
the Wilson$\Theta$ method. In the Wilson$\Theta$ method, to determine the 
velocities, accelerations and displacements at time $t + \theta \Delta
t$, $\theta \ge 1.37$, for $\U_{t+ \theta \Delta t}$ 

\[ \R (\U_{t + \theta \Delta t}) = \P(t + \theta \Delta t) -
\F_I(\Udd_{t+ \theta \Delta t}) 
- \F_R(\Ud_{t + \theta \Delta t},\U_{t + \theta \Delta t}) \]

\noindent where we use following functions to relate $\Ud_{t + \theta
\Delta t}$ and $\Udd_{t + \theta \Delta t}$ to $\U_{t + \theta \Delta
t}$ and the response quantities at time $t$:

\[
\dot \U_{t + \theta \Delta t} = \frac{3}{\theta \Delta t} \left(
\U_{t + \theta \Delta t} - \U_t \right)
 - 2 \dot \U_t + \frac{\theta \Delta t}{2} \ddot \U_t 
\]

\[
\ddot \U_{t + \theta \Delta t} = \frac{6}{\theta^2 \Delta t^2}
\left( \U_{t+\theta \Delta t} - \U_t \right)
 - \frac{6}{\theta \Delta t} \dot \U_t -2 \Udd_t
\]

\noindent which  results in the following for determining the responses at
$t + \theta \Delta t$ 

\[ \left[ \frac{6}{\theta^2 \Delta t^2} \M + \frac{3}{\theta \Delta t}
\C + \K \right] \Delta \U_{t + \theta \Delta t}^{(i)} = \P(t + \theta
\Delta t) - \F_I\left(\Udd_{t+\theta \Delta  t}^{(i-1)}\right) 
- \F_R\left(\Ud_{t + \theta \Delta t}^{(i-1)},\U_{t + \theta \Delta
t}^{(i-1)}\right) \]


\noindent The response quantities at time $t + \Delta t$ are then
determined using the following

\[
\Udd_{t + \Delta t} = \Udd_t + \frac{1}{\theta} \left( \Udd_{t +
\theta \Delta t} - \Udd_t \right)
\]

\[ \Ud_{t + \Delta t} = \Ud_t + \frac{\Delta t}{2}\left( \Udd_{t +
\Delta t} + \Udd_t \right) \]

\[ \U_{t + \Delta t} = \U_t + \Delta t\Ud_t + \frac{\Delta t^2}{6}\left(
\Udd_{t + \Delta t} + 2 \Udd_t \right) \]
\\

\pagebreak
\noindent {\bf Class Interface} \\
\indent // Constructors \\
\indent {\em WilsonTheta();}\\ 
\indent {\em WilsonTheta(double theta);}\\ 
\indent {\em WilsonTheta(double theta, double alphaM, double betaK);}\\ \\
\indent // Destructor \\
\indent {\em virtual~ $\tilde{}$WilsonTheta();}\\ \\
\indent // Public Methods \\
\indent {\em int formEleTangent(FE\_Element *theEle);} \\
\indent {\em int formNodTangent(DOF\_Group *theDof);} \\
\indent {\em int domainChanged(void);}\\
\indent {\em int newStep(double deltaT);}\\
\indent {\em int update(const Vector \&$\Delta U$);} \\ 
\indent {\em int commit(void);}\\ \\
\indent // Public Methods for Output\\
\indent {\em int sendSelf(int commitTag, Channel \&theChannel);}\\ 
\indent {\em int recvSelf(int commitTag, Channel \&theChannel,
FEM\_ObjectBroker \&theBroker);}\\ 
\indent {\em int Print(OPS_Stream \&s, int flag = 0);}\\

\noindent {\bf Constructors} \\
\indent {\em WilsonTheta();}\\ 
The integer INTEGRATOR\_TAGS\_WilsonTheta is passed to the TransientIntegrator
constructor. $\Theta$ is set to 0.0. This constructor should only be
invoked by an FEM\_ObjectBroker. \\



\indent {\em WilsonTheta(double theta);}\\ 
Sets $\Theta$ to {\em theta}, $\gamma$ to $(1.5 - \alpha)$ and $\beta$
to $0.25*\alpha^2$. addition, a flag is set indicating that Rayleigh
damping will not be used. \\ 

\indent {\em WilsonTheta(double theta);}\\ 
Sets $\Theta$ to {\em theta}, $\gamma$ to $(1.5 - \alpha)$ and $\beta$
to $0.25*\alpha^2$. In addition, a flag is set indicating that Rayleigh
damping will not be used. \\

\indent {\em WilsonTheta(double theta, double alphaM, double betaK);}\\ 
This constructor is invoked if Rayleigh damping is to be used, 
i.e. $\D = \alpha_M M + \beta_K K$. Sets $\Theta$ to {\em theta},
$\gamma$ to $(1.5 - \alpha)$, $\beta$ to $0.25*\alpha^2$, $\alpha_M$ to
{\em alphaM} and $\beta_K$ to {\em betaK}. Sets a flag indicating whether the
incremental solution is done in terms of displacement or acceleration
to {\em dispFlag} and a flag indicating that Rayleigh damping will 
be used. \\ 


\noindent {\bf Destructor} \\
\indent {\em virtual~ $\tilde{}$WilsonTheta();}\\ 
Invokes the destructor on the Vector objects created. \\

\noindent {\bf Public Methods}\\
\indent {\em int formEleTangent(FE\_Element *theEle);} \\
This tangent for each FE\_Element is defined to be $\K_e = c1 \K
+ c2  \D + c3 \M$, where c1,c2 and c3 were determined in the last invocation
of the {\em newStep()} method. Returns $0$ after performing the
following operations:  
\begin{tabbing}
while \= \+ while \= while \= \kill
if (RayleighDamping == false) \{ \+ \\
theEle-$>$zeroTang() \\
theEle-$>$addKtoTang(c1) \\
theEle-$>$addCtoTang(c2) \\
theEle-$>$addMtoTang(c3)  \- \\
\} else \{ \+ \\
theEle-$>$zeroTang() \\
theEle-$>$addKtoTang(c1 + c2 * $\beta_K$) \\
theEle-$>$addMtoTang(c3 + c2 * $\alpha_M$)  \- \\ 
\}
\end{tabbing}

{\em int formNodTangent(DOF\_Group *theDof);} \\
This performs the following:
\begin{tabbing}
while \= \+ while \= while \= \kill
theDof-$>$zeroUnbalance() \\
if (RayleighDamping == false)  \+ \\
theDof-$>$addMtoTang(c3)  \- \\
else \+ \\
theDof-$>$addMtoTang(c3 + c2 * $\alpha_M$)  \- \\ 
\end{tabbing}


{\em int domainChanged(void);}\\
If the size of the LinearSOE has changed, the object deletes any old Vectors
created and then creates $6$ new Vector objects of size equal to {\em
theLinearSOE-$>$getNumEqn()}. There is a Vector object created to store
the current displacement, velocity and accelerations at times $t$ and
$t + \Delta t$ (between {\em newStep()} and {\em commit()} the $t +
\Delta t$ quantities store $t + \Theta \Delta t$ quantities).
The response quantities at time $t + \Delta t$ are
then set by iterating over the DOF\_Group objects in the model and
obtaining their committed values. 
Returns $0$ if successful, otherwise a warning message and a negative
number is returned: $-1$ if no memory was available for constructing
the Vectors. \\


{\em int newStep(double $\Delta t$);}\\
The following are performed when this method is invoked:
\begin{enumerate}
\item First sets the values of the three constants {\em c1}, {\em c2}
and {\em c3}: {\em c1} is set to $1.0$, 
{\em c2} to $3 / (\Theta
\Delta t)$ and {\em c3} to $6 / (\Theta \Delta t)^2)$. 
\item Then the Vectors for response quantities at time $t$ are set
equal to those at time $t + \Delta t$.
\begin{tabbing}
while \= while \= while \= while \= \kill
\>\> $ \U_t = \U_{t + \Delta t}$ \\
\>\> $ \Ud_t = \Ud_{t + \Delta t} $ \\
\>\> $ \Udd_t = \Udd_{t + \Delta t} $ 
\end{tabbing}
\item Then the velocity and accelerations approximations
at time $t + \Theta \Delta t$ are set using the difference
approximations,
\begin{tabbing}
while \= while \= while \= while \= \kill
\>\> $ \U_{t + \theta \Delta t} = \U_t $ \\
\>\> $ \dot \U_{t + \theta \Delta t} = - 2 \dot \U_t + \frac{\theta
\Delta t}{2} \ddot \U_t  $\\
\>\> $ \ddot \U_{t + \theta \Delta t} = - \frac{6}{\theta \Delta t}
\dot \U_t -2 \Udd_t $ 
\end{tabbing}
\item The response quantities at the DOF\_Group objects are updated
with the new approximations by invoking {\em setResponse()} on the
AnalysisModel with quantities at time $t + \Theta \Delta t$.
\begin{tabbing}
while \= while \= while \= while \= \kill
\>\> theModel-$>$setResponse$(\U_{t + \theta \Delta t}, \Ud_{t+\theta
\Delta t}, \Udd_{t+ \theta \Delta t})$ 
\end{tabbing}
\item current time is obtained from the AnalysisModel, incremented by
$\Theta \Delta t$, and {\em applyLoad(time, 1.0)} is invoked on the
AnalysisModel. 
\item Finally {\em updateDomain()} is invoked on the AnalysisModel.
\end{enumerate}
The method returns $0$ if successful, otherwise a negative number is
returned: $-1$ if $\gamma$ or $\beta$ are $0$, $-2$ if {\em dispFlag}
was true and $\Delta t$ is $0$, and $-3$ if {\em domainChanged()}
failed or has not been called. \\



{\em int update(const Vector \&$\Delta U$);} \\
Invoked this first causes the object to increment the DOF\_Group
response quantities at time $t + \Theta \Delta t$. The displacement Vector is  
incremented by $ c1 * \Delta U$, the velocity Vector by $
c2 * \Delta U$, and the acceleration Vector by $c3 * \Delta U$. 
The response quantities at the DOF\_Group objects are then updated
with the new approximations by invoking {\em setResponse()} on the
AnalysisModel with displacements, velocities and accelerations at time
$t + \Theta \Delta t$.
Finally {\em updateDomain()} is invoked on the AnalysisModel. 
\begin{tabbing}
while \= while \= while \= while \= \kill
\>\> $ \U_{t + \theta \Delta t} += \Delta \U$ \\
\>\> $ \dot \U_{t + \theta \Delta t} += \frac{3}{\theta \Delta t}
\Delta \U  $\\
\>\> $ \ddot \U_{t + \theta \Delta t} += \frac{6}{\theta^2 \Delta
t^2} \Delta \U $ \\ 
\>\> theModel-$>$setResponse$(\U_{t + \alpha \theta t}, \Ud_{t+\theta
\Delta t}, \Udd_{t+ \theta \Delta t})$ \\
\>\> theModel-$>$updateDomain()
\end{tabbing}
Returns $0$ if successful. A warning message is printed and a negative number
returned if an error occurs: $-1$ if no associated AnalysisModel, $-2$
if the Vector objects have not been created, $-3$ if the Vector
objects and $\Delta U$ are of different sizes. \\



{\em int commit(void);}\\
First the quantities at time $t + \Delta t$ are determined using the
quantities at time $t$ and $t + \Theta \Delta t$.
Then the response quantities at the DOF\_Group objects are updated
with the new approximations by invoking {\em setResponse()} on the
AnalysisModel with displacement, velocity and accelerations at time $t +
\Delta t$. The time is obtained from the AnalysisModel and $(\Theta
-1) \Delta t$ is subtracted from the value. The time is set in the
Domain by invoking {\em setCurrentDomainTime(time)} on the
AnalysisModel. Finally {\em updateDomain()} and {\em commitDomain()}
are invoked on the AnalysisModel. 
\begin{tabbing}
while \= while \= while \= while \= \kill
\>\> $\Udd_{t + \Delta t} = \Udd_t + \frac{1}{\theta} \left( \Udd_{t +
\theta \Delta t} - \Udd_t \right)$ \\
\>\> $ \Ud_{t + \Delta t} = \Ud_t + \frac{\Delta t}{2}\left( \Udd_{t +
\Delta t} + \Udd_t \right) $ \\
\>\> $ \U_{t + \Delta t} = \U_t + \Delta t\Ud_t + \frac{\Delta t^2}{6}\left(
\Udd_{t + \Delta t} + 2 \Udd_t \right) $ \\
\>\> theModel-$>$setResponse$(\U_{t + \Delta t}, \Ud_{t+
\Delta t}, \Udd_{t+\Delta t})$ \\
\>\> time = theModel-$>$getDomainTime() \\
\>\> time -= $(\theta -1) * \Delta t$ \\
\>\> theModel-$>$setTime(time) \\
\>\> theModel-$>$commitDomain()
\end{tabbing}
Returns $0$ if successful, a warning
message and a negative number if not: $-1$ if no AnalysisModel
associated with the object and $-2$ if {\em commitDomain()} failed. \\


{\em int sendSelf(int commitTag, Channel \&theChannel); } \\ 
Places $\Theta$, Rayleigh damping flag, $\alpha_M$ and $\beta_K$ in a
vector if size 4 and invokes {\em sendVector} on the Channel with this
Vector. Returns $0$ if successful, a warning message is printed and a
$-1$ is returned if {\em theChannel} fails to send the Vector. \\ 

{\em int recvSelf(int commitTag, Channel \&theChannel, 
FEM\_ObjectBroker \&theBroker); } \\ 
Receives in a Vector of size 4 the value of $\Theta$, the Rayleigh
damping flag, $\alpha_M$ and $\beta_K$.. Returns $0$ if
successful, a warning message is printed,  $\Theta$ is set to $0$, the
Rayleigh damping flag to {\em false}, and a $-1$ is returned if {\em
theChannel} fails to receive the Vector.\\ 

{\em int Print(OPS_Stream \&s, int flag = 0);}\\
The object sends to $s$ its type, the current time, $\alpha$, $\gamma$ and
$\beta$. 



\pagebreak \subsection{{\bf SolutionAlgorithm}}
%File ~/OOP/analysis/algorithm/SolutionAlgorithm.tex
%What: "@(#) SolutionAlgorithm.tex, revA"

\noindent {\bf Files}   \\
\indent \#include $<\tilde{ }$/analysis/algorithm/SolutionAlgorithm.h$>$  \\

\noindent {\bf Class Declaration}  \\
\indent class SolutionAlgorithm: public MovableObject  \\

\noindent {\bf Class Hierarchy} \\
\indent MovableObject \\
\indent\indent {\bf SolutionAlgorithm} \\

\noindent {\bf Description} \\
\indent The SolutionAlgorithm class is an abstract base class. Its purpose
is to define the interface common among all its subclasses. A
SolutionAlgorithm object performs the steps in the analysis by specifying
the sequence of operations to be performed by members in the analysis
aggregation.\\


\noindent {\bf Class Interface} \\
\indent // Constructor \\
\indent {\em SolutionAlgorithm(int classTag);}\\  \\
\indent // Destructor \\
\indent {\em virtual~ $\tilde{}$SolutionAlgorithm();}\\  \\
\indent // Public Methods  \\
\indent {\em virtual int domainChanged(void); } \\
\indent {\em virtual  int  addRecorder(Recorder \&theRecorder);}\\
\indent {\em virtual int record(int track); } \\
\indent {\em virtual int playback(int track); } \\


\noindent {\bf Constructor} \\
\indent {\em SolutionAlgorithm(int classTag);}\\
The integer {\em classTag} is passed to the MovableObject classes
constructor. \\

\noindent {\bf Destructor} \\
\indent {\em virtual~ $\tilde{}$SolutionAlgorithm();}\\
Invokes the destructor on any recorder object added to the
SolutionAlgorithm and releases memory used to hold pointers to the
recorder objects. \\


\noindent {\bf Public Methods}  \\
\indent {\em virtual int domainChanged(void); } \\
Is called by the Analysis if the domain changes. It is called after
{\em domainChange()} has been called on the ConstraintHandler,
DOF\_Numberer and the Integrator and after {\em setSize()} has been
called on the SystemOfEqn object. For base class nothing is done and
$0$ is returned. The subclasses can provide their own implementation
of this method if anything needs to be done, e.g. memory allocation,
To return $0$ if successful, a negative number if not. \\


\indent {\em  virtual int  addRecorder(Recorder \&theRecorder);}\\
To add a recorder object {\em theRecorder} to the
SolutionAlgorithm. returns $0$ if successful, a warning message and a
$-1$ is returned if not enough memory is available. \\

\indent {\em virtual int record(int track); } \\
To invoke {\em record(track)} on any Recorder objects which have been added to the
SolutionAlgorithm. \\

\indent {\em virtual int playback(int track); } \\
To invoke {\em playback(track)} on any Recorder objects which have been added to the
SolutionAlgorithm. \\


\pagebreak \subsubsection{{\bf EquiSolnAlgo}}
%File ~/OOP/analysis/algorithm/EquiSolnAlgo.tex
%What: "@(#) EquiSolnAlgo.tex, revA"

\noindent {\bf Files}   \\
\indent \#include $<\tilde{
}$/analysis/algorithm/equiSolnAlgo/EquiSolnAlgo.h$>$  \\ 

\noindent {\bf Class Declaration}  \\
\indent class EquiSolnAlg: public SolutionAlgo;  \\

\noindent {\bf Class Hierarchy} \\
\indent MovableObject \\
\indent\indent SolutionAlgorithm \\
\indent\indent\indent {\bf EquiSolnAlgo} \\

\noindent {\bf Description} \\ 
\indent The EquiSolnAlgo class is an abstract base class. Its
purpose is to define the interface common among all subtypes. An
EquiSolnAlgo object defines the sequence of operations 
performed by the the Integrator and the LinearSOE objects in
solving the equilibrium equation $R(U) = 0$ given the current state of
the domain at each time step in a direct integration analysis or load
increment in a static analysis. \\

\noindent {\bf Class Interface} \\ 
\indent\indent // Constructor  \\ 
\indent\indent {\em EquiSolnAlgo(int classTag);}\\ \\
\indent\indent // Destructor  \\
\indent\indent {\em virtual~ $\tilde{}$EquiSolnAlgo();}\\ \\
\indent\indent // Public Methods  \\
\indent\indent {\em void setLinks(AnalysisModel \&theModel, \\
\indent\indent\indent\indent\indent IncrementalIntegrator \&theIntegrator, \\
\indent\indent\indent\indent\indent LinearSOE \&theSOE);} \\
\indent\indent {\em virtual int solveCurrentStep(void) = 0;} \\ \\
\indent\indent // Public Methods for Output \\
\indent\indent {\em virtual void Print(ostream \&s, int flag = 0)
=0;}\\ \\
\indent\indent // Public Methods for Pointers - not Protected for ConvergenceTest\\
\indent\indent {\em AnalysisModel *getAnalysisModelPtr(void) const;}
\\ 
\indent\indent {\em IncrementalIntegrator
*getIncrementalIntegratorPtr(void) const;} \\ 
\indent\indent {\em LinSysOfEqn *getLinearSOEptr(void) const;} \\ 

\noindent {\bf Constructor} \\ 
\indent {\em EquiSolnAlgo(int classTag);}\\ 
The integer {\em classTag} is passed to the SolutionAlgorithm classes
constructor. \\

\noindent {\bf Destructor} \\
\indent {\em virtual~ $\tilde{}$EquiSolnAlgo();}\\ 
Does nothing. \\


\noindent {\bf Public Methods}  \\
\indent {\em void setLinks(AnalysisModel \&theModel, \\
\indent\indent\indent\indent IncrementalIntegrator \&theIntegrator, \\
\indent\indent\indent\indent LinearSOE \&theSOE);} \\
Sets up the links needed by an object of this class, or a derived
class, to an AnalysisMode, IncrementalIntegrator and LinearSOE
object. Pointers to these objects can be obtained by subtypes using
the protected methods defined below. \\ 

\indent {\em virtual int solveCurrentStep(void) = 0;} \\
A method implemented by each subclass which specifies the steps taken
in order to get the system into an equilibrium state. It is a pure
virtual function, i.e. all subclasses or their descendents must
implement this routine. To return $0$ if algorithm succeeds, a negative
value otherwise. \\ 

\indent {\em virtual void Print(ostream \&s, int flag = 0) =0;}\\
The Integrator is to send information to the stream based on the
integer {\em flag}. \\

\indent {\em AnalysisModel *getAnalysisModelPtr(void) const;} \\
A const member function which returns a pointer to the AnalysisModel
associated with the EquiSolnAlgo object, {\em theModel} passed in {\em
setLinks()}. $0$ returned if one not yet associated. \\

{\em IncrementalIntegrator *getIncrementalIntegratorPtr(void) const;} \\
A const member function which returns a pointer to the StaticMethod
associated with the StaticSolnAlgo object, {\em theIntegrator} passed
in {\em setLinks()}. $0$ returned if one not yet associated. \\
 
{\em LinSysOfEqn *getLinearSOEptr(void) const;} \\
A const member function which returns a pointer to the LinearSOE
associated with the EquiSolnAlgo object, {\em theLinearSOE} passed in
{\em setLinks()}. $0$ returned if one not yet associated. \\







\pagebreak \subsubsection{Linear}
%File ~/OOP/analysis/algorithm/Linear.tex
%What: "@(#) Linear.tex, revA"

\noindent {\bf Files}   \\
\indent \#include $<\tilde{ }$/analysis/algorithm/equiSolnAlgo/Linear.h$>$  \\

\noindent {\bf Class Declaration}  \\
\indent class Linear: public EquiSolnAlg;  \\

\noindent {\bf Class Hierarchy} \\
\indent MovableObject \\
\indent\indent SolutionAlgorithm \\
\indent\indent\indent EquiSolnAlgo \\
\indent\indent\indent\indent {\bf Linear} \\

\noindent {\bf Description} \\ 
\indent The Linear class is an algorithmic class which uses the
linear solution algorithm to solve the equations. This is based on
a Taylor expansion of the linear system $\R(\U) = \zero$ about an
approximate solution $\U_{a}$.

\begin{equation} 
\R(\U) = 
\R(\U_{a}) +
\left[ {\frac{\partial \R}{\partial \U} \vert}_{\U_{a}}\right]
\left( \U - \U_{a} \right) 
\end{equation}
\noindent which can be expressed as:
\begin{equation} \
\K_{a} \Delta \U = \R(\U_{a})
\end{equation}
which is solved for $\Delta \U$ to give the approximation 
$\U = \U_{a} + \Delta \U$. \\

To start the iteration $\U_a = \U_{trial}$, i.e. the current trial
response quantities are chosen as approximate solution quantities. \\


\noindent {\bf Class Interface} \\ 
\indent\indent // Constructor \\ 
\indent\indent {\em Linear();}\\ \\
\indent\indent // Destructor \\
\indent\indent {\em ~ $\tilde{}$Linear();}\\  \\
\indent\indent // Public Methods  \\
\indent\indent {\em int solveCurrentStep(void);} \\ \\
\indent\indent // Public Methods  for Output \\
\indent\indent {\em int sendSelf(int commitTag, Channel \&theChannel);}\\
\indent\indent {\em int recvSelf(int commitTag, Channel \&theChannel, 
FEM\_ObjectBroker \&theBroker);}\\ 
\indent\indent {\em int Print(ostream \&s, int flag =0);} \\


\noindent {\bf Constructor} \\ 
\indent {\em Linear();}\\ 
The integer {\em EquiALGORITHM\_TAGS\_Linear} (defined in
$<$classTags.h$>$) is passed to the EquiSolnAlgo classes
constructor. \\

\noindent {\bf Destructor} \\
\indent {\em ~ $\tilde{}$Linear();}\\ 

\noindent {\bf Public Methods}  \\
\indent {\em int solveCurrentStep(void);} \\
This method performs the linear solution algorithm:
\begin{tabbing}
while \= \+ whilewhilewhilewhilewhilewhilewhilewhilewhile \= \kill
theIntegrator-$>$formTangent() \+ // form $\K_{a}$ \- \\
theIntegrator-$>$formUnbalance() // form $\R(\U_{a})$ \\
theSOE-$>$solveX() // solve for $\Delta \U$ \\
theIntegrator-$>$update(theSOE-$>$getX()) // set $\U = \U_{a} + \Delta \U$ \-  
\end{tabbing}

The method returns a 0 if successful, otherwise warning message is
printed and a negative number is returned; a $-1$ if error during {\em
formTangent()}, a $-2$ if error during {\em formUnbalance()}, a $-3$
if error during {\em solve()}, a $-4$ if error during {\em
update()}. If an error occurs in any of the above operations the
method stops at that routine, none of the subsequent operations are
invoked. A $-5$ is returned if any one of the links has not been
setup. \\


{\em int sendSelf(int commitTag, Channel \&theChannel);}\\
Does  nothing. Returns 0. \\


{\em int recvSelf(int commitTag, Channel \&theChannel, FEM\_ObjectBroker
\&theBroker);}\\ 
Does nothing. Returns 0. \\

{\em int Print(ostream \&s, int flag =0);} \\
Sends the string 'Linear Algorithm' to the stream.

\pagebreak \subsubsection{NewtonRaphson}
%File ~/OOP/analysis/algorithm/NewtonRaphson.tex
%What: "@(#) NewtonRaphson.tex, revA"

\noindent {\bf Files}   \\
\indent \#include $<\tilde{
}$/analysis/algorithm/equiSolnAlgo/NewtonRaphson.h$>$  \\ 

\noindent {\bf Class Declaration}  \\
\indent class NewtonRaphson: public EquiSolnAlg;  \\

\noindent {\bf Class Hierarchy} \\
\indent MovableObject \\
\indent\indent SolutionAlgorithm \\
\indent\indent\indent EquiSolnAlgo \\
\indent\indent\indent\indent {\bf NewtonRaphson} \\

\noindent {\bf Description} \\ 
\indent The NewtonRaphson class is an algorithmic class which obtains a
solution to a non-linear system using the Newton-Raphson iteration
scheme. The iteration scheme is based on a Taylor expansion of the
non-linear system of equations $\R(\U) = \zero$ about an approximate
solution $\U^{(i)}$:
\begin{equation} 
\R(\U) = 
\R(\U^{(i)}) +
\left[ {\frac{\partial \R}{\partial \U} \vert}_{\U^{(i)}}\right]
\left( \U - \U^{(i)} \right) 
\end{equation}

\noindent which can be expressed as:
\begin{equation} \
\K^{(i)}  \Delta \U^{(i)} = \R(\U^{(i)})
\end{equation}
which is solved for $\Delta \U^{(i)}$ to give approximation for
$\U^{(i+1)} = \U^{(i)} + \Delta \U^{(i)}$. To start the
iteration $\U^{(1)} = \U_{trial}$, i.e. the current trial
response quantities are chosen as initial response quantities. 
To stop the iteration, a test must be performed to see if convergence
has been achieved at each iteration. Each NewtonRaphson object is
associated with a ConvergenceTest object. It is this object which
determines if convergence has been achieved. \\

\noindent {\bf Class Interface} \\ 
\indent // Constructors \\ 
\indent {\em NewtonRaphson(ConvergenceTest \&theTest);}\\ 
\indent {\em NewtonRaphson();}\\ \\
\indent // Destructor \\
\indent {\em ~ $\tilde{}$NewtonRaphson();}\\  \\
\indent // Public Member Functions \\
\indent {\em int solveCurrentStep(void);} \\
\indent {\em void setTest(ConvergenceTest \&theTest);} \\ \\
\indent // Public Methods  for Output \\
\indent {\em int sendSelf(int commitTag, Channel \&theChannel);}\\ 
\indent {\em int recvSelf(int commitTag, Channel \&theChannel, 
FEM\_ObjectBroker \&theBroker);}\\ 
\indent {\em int Print(OPS_Stream \&s, int flag =0);} \\

\noindent {\bf Constructors} \\ 
\indent {\em NewtonRaphson(int theMaxNumIter, ConvergenceTest \&theTest);}\\ 
The constructor takes as an argument the ConvergenceTest object {\em
theTest}, the object which is used at the end of each iteration to
determine if convergence has been obtained. The
integer {\em EquiALGORITHM\_TAGS\_NewtonRaphson} (defined in
$<$classTags.h$>$) is passed to the EquiSolnAlgo classes
constructor. \\ 

\indent {\em NewtonRaphson();}\\
Provided for FEM\_ObjectBroker to instantiate a blank object with a
class tag of EquiALGORITHM\_TAGS\_NewtonRaphson is passed. {\em
recvSelf()} must be invoked on this object. \\

\noindent {\bf Destructor} \\
\indent {\em ~ $\tilde{}$NewtonRaphson();}\\ 
Does nothing. \\

\noindent {\bf Public Member Functions}  \\
\indent {\em int solveCurrentStep(void);} \\
When invoked the object first sets itself as the EquiSolnAlgo object
that the ConvergenceTest is testing and then it performs the
Newton-Raphson iteration algorithm: 
\begin{tabbing}
while \= \+ while \= \kill
theTest-$>$start() \\
theIntegrator-$>$formUnbalance(); \\
do \{ \+ \\
theIntegrator-$>$formTangent(); \\
theSOE-$>$solveX(); \\
theIntegrator-$>$update(theSOE-$>$getX()); \\
theIntegrator-$>$formUnbalance(); \- \\
\} while (theTest-$>$test() $==$ false)\- 
\end{tabbing}

\noindent The method returns a 0 if successful, otherwise a negative number is
returned; a $-1$ if error during {\em formTangent()}, a $-2$ if
error during {\em formUnbalance()}, a $-3$ if error during {\em
solve()}, and a $-4$ if error during {\em update()}.
If an error occurs in any of the above operations the method stops at
that routine, none of the subsequent operations are invoked. A $-5$ is
returned if any one of the links has not been setup. NOTE it is up to
ConvergenceTest to ensure an infinite loop situation is not encountered. \\

{\em void setTest(ConvergenceTest \&theTest);} \\
A method to set the ConvergenceTest object associated with the
Algorithm to be {\em theTest}. \\

{\em int sendSelf(int commitTag, Channel \&theChannel);}\\
Creates an ID object, puts the values of the {\em theTest} objects
class and database tags into this ID. It then invokes {\em
sendVector()} on the Channel object {\em theChannel} to send the data
to the remote object. It then invokes {\em sendSelf()} on {\em
theTest}. Returns $0$ if successful, the channel error if not. \\

{\em int recvSelf(int commitTag, Channel \&theChannel, FEM\_ObjectBroker
\&theBroker);}\\ 
Creates an ID object, invokes {\em recvVector()} on the Channel
object. Uses the data in the ID to create a ConvergenceTest object of
appropriate type and sets its dbTag. It then invokes {\em recvSelf()}
on this test object. \\

{\em int Print(OPS_Stream \&s, int flag =0);} \\
Sends the string 'NewtonRaphson' to the stream if {\em flag} equals $0$.

\pagebreak \subsubsection{ModifiedNewton}
%File ~/OOP/analysis/algorithm/ModifiedNewton.tex
%What: "@(#) ModifiedNewton.tex, revA"

\noindent {\bf Files}   \\
\indent \#include $<\tilde{
}$/analysis/algorithm/equiSolnAlgo/ModifiedNewton.h$>$  \\ 

\noindent {\bf Class Declaration}  \\
\indent class ModifiedNewton: public EquiSolnAlg;  \\

\noindent {\bf Class Hierarchy} \\
\indent MovableObject \\
\indent\indent SolutionAlgorithm \\
\indent\indent\indent EquiSolnAlgo \\
\indent\indent\indent\indent {\bf ModifiedNewton} \\

\noindent {\bf Description} \\ 
\indent The ModifiedNewton class is an algorithmic class which obtains a
solution to a non-linear system using the modified Newton-Raphson iteration
scheme. The Newton-Rapson iteration scheme is based on a Taylor expansion of the
non-linear system of equations $\R(\U) = \zero$ about an approximate
solution $\U{(i)}$:
\begin{equation} 
\R(\U) = 
\R(\U^{(i)}) +
\left[ {\frac{\partial \R}{\partial \U} \vert}_{\U^{(i)}}\right]
\left( \U - \U^{(i)} \right) 
\end{equation}

\noindent which can be expressed as:
\begin{equation} \
\K^{(i)}  \Delta \U{(i)} = \R(\U^{(i)})
\end{equation}
which is solved for $\Delta \U^{(i)}$ to give approximation for
$\U^{(i+1)} = \U^{(i)} + \Delta \U^{(i)}$. To start the
iteration $\U^{(1)} = \U_{trial}$, i.e. the current trial
response quantities are chosen as initial response quantities. \\

in the modified version the tangent is formed only once, i.e
\begin{equation} \
\K^{(1)}  \Delta \U^{(i)} = \R(\U^{(i)})
\end{equation}

To stop the iteration, a test must be performed to see if convergence
has been achieved at each iteration. Each NewtonRaphson object is
associated with a ConvergenceTest object. It is this object which
determines if convergence has been achieved. \\

\noindent {\bf Class Interface} \\ 
\indent // Constructors \\ 
\indent {\em ModifiedNewton(ConvergenceTest \&theTest);}\\ 
\indent {\em ModifiedNewton();}\\ \\
\indent // Destructor \\
\indent {\em ~ $\tilde{}$ModifiedNewton();}\\  \\
\indent // Public Member Functions \\
\indent {\em int solveCurrentStep(void);} \\
\indent {\em void setTest(ConvergenceTest \&theTest);} \\\\
\indent // Public Methods  for Output \\
\indent {\em int sendSelf(int commitTag, Channel \&theChannel);}\\ 
\indent {\em int recvSelf(int commitTag, Channel \&theChannel, 
FEM\_ObjectBroker \&theBroker);}\\ 
\indent {\em int Print(ostream \&s, int flag =0);} \\


\noindent {\bf Constructors} \\ 
\indent {\em ModifiedNewton(int theMaxNumIter, ConvergenceTest \&theTest);}\\ 
The constructor takes as an argument the ConvregenceTest object {\em
theTest}, the object which is used at the end of each iteration to
determine if convergence has been obtained. The
integer {\em EquiALGORITHM\_TAGS\_ModifiedNewton} (defined in
$<$classTags.h$>$) is passed to the EquiSolnAlgo classes
constructor. \\ 

\indent {\em ModifiedNewton();}\\
Provided for FEM\_ObjectBroker to instantiate a blank object with a
class tag of EquiALGORITHM\_TAGS\_ModofiedNewton. {\em
recvSelf()} must be invoked on this object. \\

\noindent {\bf Destructor} \\
\indent {\em ~ $\tilde{}$ModifiedNewton();}\\ 
Does nothing. \\

\noindent {\bf Public Member Functions}  \\
\indent {\em int solveCurrentStep(void);} \\
When invoked the object first sets itself as the EquiSolnAlgo object
that the ConvergenceTest is testing and then it performs the
modified Newton-Raphson iteration algorithm: 
\begin{tabbing}
while \= \+ while \= \kill
theTest-$>$start(); \\
theIntegrator-$>$formTangent(); \\
do \{ \+ \\
theIntegrator-$>$formUnbalance(); \\
theSOE-$>$solveX(); \\
theIntegrator-$>$update(theSOE-$>$getX()); \- \\
\} while (theTest-$>$test() $==$ false)\- \\
\end{tabbing}


\noindent The method returns a 0 if successful, otherwise a negative number is
returned; a $-1$ if error during {\em formTangent()}, a $-2$ if
error during {\em formUnbalance()}, a $-3$ if error during {\em
solve()}, and a $-4$ if error during {\em update()}.
If an error occurs in any of the above operations the method stops at
that routine, none of the subsequent operations are invoked. A $-5$ is
returned if any one of the links has not been setup.\\

{\em void setTest(ConvergenceTest \&theTest);} \\
A method to set the tolerance criteria of the algorithm to be equal to
the value {\em theTol}. \\

{\em int sendSelf(int commitTag, Channel \&theChannel);}\\
Creates an ID object, puts the values of the {\em theTest} objects
class and database tags into this ID.
It then invokes {\em sendVector()} on the Channel object {\em
theChannel} to send the data to the remote object. It then invokes
{\em sendSelf()} on {\em theTest}. Returns $0$ if successful, the
 channel error if not. \\

{\em int recvSelf(int commitTag, Channel \&theChannel, FEM\_ObjectBroker
\&theBroker);}\\ 
Creates an ID object, invokes {\em recvVector()} on the Channel
object. Uses the data in the ID to create a ConvergenceTest object of
appropriate type and sets its dbTag. It then invokes {\em recvSelf()}
on this test object.  \\

{\em int Print(ostream \&s, int flag =0);} \\
Sends the string 'ModifiedNewton' to the stream if {\em flag} equals $0$.


%\pagebreak \subsubsection{DomainDecompAlgo}
%%File ~/OOP/analysis/algorithm/DomainDecompAlgo.tex
%What: "@(#) DomainDecompAlgo.tex, revA"

\noindent {\bf Files}   \\
\indent \#include $<\tilde{
}$/analysis/algorithm/domainDecompAlgo/DomainDecompAlgo.h$>$  \\  

\noindent {\bf Class Declaration}  \\
\indent class DomainDEcompAlgo: public SolutionAlgorithm;  \\

\noindent {\bf Class Hierarchy} \\
\indent MovableObject \\
\indent\indent SolutionAlgorithm \\
\indent\indent\indent {\bf DomainDecompAlgo} \\

\noindent {\bf Description} \\ 
\indent The DomainDecompAlgo class is the SolutionAlgorithm subclass
used in a DomainDecompAnalysis. The class is responsible for updating
the DOF\_Group responses given the current solution to the interface
problem. \\ 

\noindent {\bf Class Interface} \\ 
\indent\indent // Constructor \\ 
\indent\indent {\em DomainDecompAlgo();}\\ \\
\indent\indent // Destructor \\
\indent\indent {\em $\tilde{ }$DomainDecompAlgo();}\\  \\
\indent\indent // Public Methods  \\
\indent\indent {\em int solveCurrentStep(void);} \\
\indent\indent {\em void setLinks(AnalysisModel \&theModel, \\
\indent\indent\indent\indent\indent IncrementalIntegrator  \&theIntegrator, \\
\indent\indent\indent\indent\indent LinearSOE   \&theSOE, \\
\indent\indent\indent\indent\indent DomainSolver   \&theSolver,	  \\
\indent\indent\indent\indent\indent Subdomain   \&theSubdomain);}\\
\indent\indent {\em int sendSelf(int commitTag, Channel \&theChannel);}\\
\indent\indent {\em int recvSelf(int commitTag, Channel \&theChannel, 
FEM\_ObjectBroker \&theBroker);} \\


\noindent {\bf Constructor} \\ 
\indent {\em DomainDecompAlgo();}\\ 

\noindent {\bf Destructor} \\
\indent {\em $\tilde{ }$DomainDecompAlgo();}\\ 
Does nothing. \\

\noindent {\bf Public Methods}  \\
\indent {\em int solveCurrentStep(void);} \\
This is a method which will set the external dof response in X of the
system of equation. The responses set will be those X in the higher
level SOE object corresponding to the subdomain. It performs the following:
\begin{tabbing}
while \= while \= \+ \kill
theSubdomain-$>$getLastExternalSysResponse(); \\
theSolver-$>$setComputedXext(extResponse); \\
theSolver-$>$solveXint(); \\
theIntegrator-$>$update(theSOE-$>$getX()); 
\end{tabbing}
\noindent Returns $0$. WILL CHANGE TO CHECK THAT THOSE METHODS INVOKED ALL
RETURN $0$ OTHERWISE PRINT WARNING AND RETURN NEGATIVE.

{\em void setLinks(AnalysisModel         \&theModel, \\
\indent\indent\indent IncrementalIntegrator  \&theIntegrator, \\
\indent\indent\indent LinearSOE              \&theSOE, \\
\indent\indent\indent DomainSolver           \&theSolver, \\
\indent\indent\indent Subdomain              \&theSubdomain);	 } \\
This is the function called by the DomainDecompAnalysis object to set
the links for the class. \\

{\em int sendSelf(int commitTag, Channel \&theChannel) \\
Returns $0$. \\

{\em int recvSelf(int commitTag, Channel \&theChannel, FEM\_ObjectBroker
\&theBroker);} \\
Returns $0$. \\







\pagebreak \subsection{AnalysisModel}
% File: ~/OOP/analysis/model/AnalysisModel.tex 
%What: "@(#) AnalysisModel.tex, revA"

MODIFY INTERFACE TO OFFER USER DEFINED STORAGE TYPE. \\

\noindent {\bf Files}   \\
\indent \#include $<$/analysis/model/AnalysisModel.h$>$  \\

\noindent {\bf Class Declaration}  \\
\indent class AnalysisModel : public MovableObject  \\

\noindent {\bf Class Hierarchy} \\
\indent MovableObject \\
\indent\indent {\bf SolutionAlgorithm} \\

\noindent {\bf Description}  \\
\indent AnalysisModel is a container class. This class is responsible
for holding and providing access to the FE\_Element and DOF\_Group
objects that the ConstraintHandler creates. It is also responsible
for updating the response quantities at the DOF\_Groups and for
triggering methods in the associated Domain. It provides operations
for the following: \begin{itemize} 
\item Population: methods so that the ConstraintHandler can add the
FE\_Element and DOF\_Group objects to the analysis model. 
\item Access: methods so that other classes in the analysis aggregation
can access the components of the AnalysisModel. 
\item Connectivity: methods such that the SysOfEqn can determine the
connectivity of the dof, which is needed for storage, sparsity, etc.
\item Update: methods for updating the individual DOFs with the
response quantities given by the AnalysisMethod.
\item Trigger: methods which trigger events in the domain.
\end{itemize} 
Each subclass of AnalysisModel must have its own subclasses
of FE\_ELEIter and DOF\_GrpIter. NOTE at the moment FE\_Element and
DOF\_Group objects are not TaggedObjects and for this reason
TaggedObjectStorage classes cannot be used for storage. This may
change to allow efficient storage classes to be used.\\

\noindent {\bf Class Interface}  \\
\indent // Constructors \\
\indent {\em AnalysisModel();}  \\
\indent {\em AnalysisModel(int classTag);}  \\ \\
\indent // Destructor  \\
\indent {\em virtual $\tilde{ }$AnalysisModel();}  \\\\
\indent // Public Methods - Population/Depopulation  \\
\indent {\em virtual bool addFE\_Element(FE\_Element *theElementPtr);}  \\
\indent {\em virtual bool addDOF\_Group(DOF\_Group *theDOFPtr);}  \\
\indent {\em virtual void clearAll(void);} \\ \\
\indent // Public Member Functions - Access  \\
\indent {\em virtual int getNumDOF\_Groups(void) const;}\\
\indent {\em virtual DOF\_Group *getDOF\_GroupPtr(int tag);}\\
\indent {\em virtual FE\_EleIter getFEs(void)=0;} \\
\indent {\em virtual DOF\_Group getDOFs(void)=0;} \\ \\
\indent // Public Member Functions - Connectivity  \\
\indent {\em virtual void setNumEqn(int numEqn);} \\
\indent {\em virtual int getNumEqn(void) const;} \\
\indent {\em virtual const Graph \&getDOFGraph(void);} \\
\indent {\em virtual const Graph \&getDOFGroupGraph(void);} \\ \\
\indent // Public Member Functions - Update  \\
\indent {\em virtual void setResponse(const const Vector \&disp, const
const Vector \&vel, const const Vector \&accel);}\\ 
\indent {\em virtual void setDisp(const const Vector \&disp);} \\
\indent {\em virtual void setVel(const const Vector \&vel);} \\
\indent {\em virtual void setAccel(const const Vector \&accel);} \\
\indent {\em virtual void incrDisp(const const Vector \&disp);} \\
\indent {\em virtual void incrVel(const const Vector \&vel);} \\
\indent {\em virtual void incrAccel(const const Vector \&accel);} \\\\
\indent // Public Member Functions - Trigger  \\
\indent {\em void setLinks(Domain \&theDomain)} \\
\indent {\em virtual void applyLoadDomain(double timeStep = 0.0,
double loadFactor = 1.0);}\\
\indent {\em virtual void commitDomain(void);} \\
\indent {\em virtual void updateDomain(void);}\\ \\
\indent {\em virtual void revertDomainToLastCommit(void);}\\ 
\indent {\em virtual double getCurrentDomainTime(void);} \\
\indent {\em virtual void   setCurrentDomainTime(double newTime);}\\
\indent {\em virtual double getCurrentDomainLoadFactor(void);}\\
\indent {\em virtual void   setCurrentDomainLoadFactor(double
newFactor);}\\ \\
\indent // Public Methods for Output\\
\indent {\em virtual int sendSelf(int commitTag, Channel \&theChannel);}\\ 
\indent {\em virtual int recvSelf(int commitTag, Channel \&theChannel,
FEM\_ObjectBroker \&theBroker);}\\  \\
\indent // Protected Member Functions  \\
\indent {\em Domain *getDomainPtr(void) const;} \\



\noindent {\bf Constructors}  \\
\indent {\em AnalysisModel();}  \\
Constructs an empty AnalysisModel. The constructor allocates
space for two arrays of 256 pointers to FE\_Elements and DOF\_Groups.
If not enough memory is available for these arrays, an error message
is printed and the program is terminated. Note these arrays grow
automatically if the problem needs it.\\

\indent {\em AnalysisModel(int classTag);}  \\
Provided for subclasses to be used. The storage of the FE\_Elements
and DOF\_Groups and iters to access them must be provided by the
subclass. \\

\noindent {\bf Destructor}  \\
\indent {\em virtual~ $\tilde{}$AnalysisModel();}  \\
Is responsible for returning to memory the arrays used for storing
pointers to the FE\_Element and DOF\_Groups which have been added to
the AnalysisModel. It is not responsible for deleting the individual
DOF\_Group and FE\_Element objects, that is the responsibility of the
ConstraintHandler. If the Graphs have been requested their destructor
is invoked.\\ 

\noindent {\bf Public Member Functions - Population/Depopulation}  \\
\indent {\em virtual bool addFE\_Element(FE\_Element *theElement);}  \\
Adds the FE\_Element pointed to by {\em theElement} to the domain and
invokes {\em setAnalysisModel(*this)} on the FE\_Element. If the
array for the FE\_Elements is large enough, it adds this pointer to
the array and increments the number of FE\_Elements in the array. If
the array is not large enough, a new one double in size is
constructed, all the old pointers are copied to this new array and the
new pointer is then added. If not enough room is available for this
array, an error message is printed and the program is
terminated. Returns {\em true}, otherwise {\em false} if {\em
theElement} is $0$ or derived class used which does not implement the
method. \\ 

{\em virtual bool addDOF\_Group(DOF\_Group *theGroup);}  \\
Adds the DOF\_Group pointed to by {\em theGroup} to the domain. If the
array for the DOF\_Groups is large enough, it adds this pointer to
the array and increments the number of DOF\_Groups in the array. If
the array is not large enough, a new one double in size is
constructed, all the old pointers are copied to this new array and the
new pointer is then added. If not enough room is available for this
array, an error message is printed and the program is
terminated. Returns {\em true}, otherwise {\em false} if {\em
theGroup} is $0$ or derived class used which does not implement the
method. \\ 



{\em virtual void clearAll(void);} \\
Clears from the model all FE\_Element and DOF\_Group objects that have
been added to the analysis model using the above two methods. It does
this by setting the components in the two arrays of pointers equal to
$0$ and setting the number of components to $0$. If the Graphs have
been created their destructor is invoked. Also sets {\em numEqn} to $0$. \\

\noindent {\bf Public Member Functions - Access}  \\
\indent {\em virtual int getNumDOF\_Groups(void) const;}\\
Returns the number of DOF\_Group objects that have been added to the
model.\\

\indent {\em virtual DOF\_Group *getDOF\_GroupPtr(int tag);}\\
Returns a pointer to the DOF\_Group object whose tag is given by {\em
tag}.  It first checks to see if the DOF\_Group object is at the
location in the array given by {\em tag}; if not it searches through
the array to find the DOF\_Group object. Returns a pointer to the
object if found, otherwise $0$ is returned.\\

\indent {\em virtual FE\_EleIter getFEs(void)=0;} \\
Returns an {\em FE\_EleIter} for the FE\_Elements of the model. \\

{\em virtual DOF\_Group getDOFs(void)=0;} \\
Returns a {\em DOF\_GrpIter} for the DOF\_Groups of the model. \\

\noindent {\bf Public Member Functions - Connectivity}  \\
\indent {\em virtual void setNumEqn(int numEqn);} \\
Sets the value of the number of equations in the model. Invoked by the
DOF\_Numberer when it is numbering the dofs. \\

\indent {\em virtual int getNumEqn(void) const;} \\
Returns the number of DOFs in the model which have been assigned
an equation number. Returns the value passed in {\em setNumEqn()},
if {\em setNumEqn()} was not invoked $0$ is returned.\\

{\em virtual const Graph \&getDOFGraph(void);} \\
Returns the DOF connectivity graph for the individual dofs in the
model. This graph is used by the system of equation object to
determine its size. If no graph has yet been constructed it creates
a new DOF\_Graph object using itself as the argument, otherwise it
returns a pointer to this graph. THIS WILL CHANGE WHEN I REMOVE
DOF\_Graph CLASS - will go through and construct the Graph.\\

{\em virtual const Graph \&getDOFGroupGraph(void);} \\
Returns the connectivity of the DOF\_Group objects in the model. 
This graph is used by the DOF\_Numberer to assign equation numbers to
the dofs. If no graph has yet been constructed it creates
a new DOF\_GroupGraph object using itself as the argument, otherwise it
returns a pointer to this graph. AGAIN WILL CHANGE.\\

\noindent {\bf Public Member Functions - Update}  \\
\indent {\em virtual void setResponse(const const Vector \&disp, const
const Vector \&vel, const const Vector \&accel);}\\ 
The model is responsible for invoking {\em setDisp(disp)}, {\em
setVel(vel)} and {\em setAccel(accel)} on each DOF\_Group in the
model. It does this by iterating over the DOF\_Group objects using the
iter. \\

{\em virtual void setDisp(const const Vector \&disp);} \\
The model is responsible for invoking {\em setDisp(disp)} on each
DOF\_Group in the model. It does this by getting an iter to the
DOF\_Group objects and iterating through them invoking {\em
setNodeDisp(disp)} on each DOF\_Group. \\

{\em virtual void setVel(const const Vector \&vel);} \\
The model is responsible for invoking {\em setVel(vel)} on each
DOF\_Group in the model. It does this by getting an iter to the
DOF\_Group objects and iterating through them invoking {\em
setNodeVel(vel)} on each DOF\_Group. \\


{\em virtual void setAccel(const const Vector \&accel);} \\
The model is responsible for invoking {\em setAccel(accel)} on each
DOF\_Group in the model. It does this by getting an iter to the
DOF\_Group objects and iterating through them invoking {\em
setNodeAccel(accel)} on each DOF\_Group. \\

{\em virtual void incrDisp(const const Vector \&disp);} \\
The model is responsible for invoking {\em incrNodeDisp(disp)} on each
DOF\_Group in the model. It does this by getting an iter to the
DOF\_Group objects and iterating through them invoking {\em
incrNodeDisp(disp)} on each DOF\_Group. \\


{\em virtual void incrVel(const const Vector \&vel);} \\
The model is responsible for invoking {\em incrNodeVel(vel)} on each
DOF\_Group in the model. It does this by getting an iter to the
DOF\_Group objects and iterating through them invoking {\em
incrNodeVel(vel)} on each DOF\_Group. \\


{\em virtual void incrAccel(const const Vector \&accel);} \\
The model is responsible for invoking {\em incrNodeAccel(accel)} on each
DOF\_Group in the model. It does this by getting an iter to the
DOF\_Group objects and iterating through them invoking {\em
incrNodeAccel(accel)} on each DOF\_Group. \\



\noindent {\bf Public Member Functions - Trigger}  \\
\indent {\em void setLinks(Domain \&theDomain)} \\
Method to set the link to the associated Domain. Invoked by
during the construction of the {\em Analysis} object.\\

{\em virtual void applyLoadDomain(double timeStep = 0.0, double
loadFactor = 1.0);}\\
Method which invokes {\em applyLoad(timeStep, loadFactor)} on the
domain. This method causes the domain ask the loads in the currently
set to apply themselves. If no Domain has been set nothing is done and an error
message is printed. \\ 


{\em virtual int commitDomain(void);} \\
Method which invokes {\em commit()} on the domain: this is an
operation which causes all nodes in the domain to take the current
values of response quantities and copy them into the accepted values. 
Returns $0$ if successful , a negative number if not: $-1$ if no
Domain has been set and $-2$ if {\em commit()} fails on the Domain.\\

{\em virtual int revertDomainToLastCommit(void);} \\
Method which invokes {\em revertToLastCommit()} on the domain: this is an
operation which causes all nodes in the domain to set the trial
response quantities equal to the last committed response quantities.
Returns $0$ if successful , a negative number if not: $-1$ if no
Domain has been set and $-2$ if {\em revertToLastCommit}() fails on
the Domain.\\ 

{\em virtual void updateDomain(void);}\\
Method which invokes {\em update()} on the domain. If no Domain has
been set nothing is done and an error message is printed. \\ 

\indent {\em virtual double getCurrentDomainTime(void);} \\
To get the current time in the Domain. If no Domain has been set a
warning message is printed and $0.0$ is returned, otherwise the result
of invoking {\em getCurrentTime()} on the Domain is returned. \\

\indent {\em virtual void   setCurrentDomainTime(double newTime);}\\
To set the current time in the Domain to be {\em newTime}. If no
Domain has been set a warning message is printed, otherwise 
{\em setCurrentTime(newTime)} is invoked on the Domain. \\

\indent {\em virtual double getCurrentDomainLoadFactor(void);}\\
To get the current load factor in the Domain. If no Domain has been set a
warning message is printed and $0.0$ is returned, otherwise the result
of invoking {\em getCurrentLoadFactor()} on the Domain is returned. \\

\indent {\em virtual void setCurrentDomainLoadFactor(double newFactor)};\\
To set the current load factor in the Domain to be {\em newFactor}. If no
Domain has been set a warning message is printed, otherwise 
{\em setCurrentLoadFactor(newFactor)} is invoked on the Domain. \\

\indent {\em virtual int sendSelf(int commitTag, Channel
\&theChannel);}\\ 
Returns $0$. Note the FE\_Elements and DOF\_Group objects are not sent
as they are not MovableObjects. AnalysisModel objects are only sent
when setting up a DomainDecompAnalysis on a remote process; only type
info and whatever subclasses might need need to be sent.\\

\indent\indent {\em virtual int recvSelf(int commitTag, Channel \&theChannel,
FEM\_ObjectBroker \&theBroker)};\\
Returns $0$.  \\

\noindent {\bf Protected Member Functions}  \\
\indent {\em Domain *getDomainPtr(void) const;} \\
Returns a pointer to the associated Domain, that is the Domain
set when {\em setLinks()} was last invoked. 




\pagebreak \subsection{FE\_Element}
% File: ~/OOP/analysis/model/FE_Element.tex 
%What: "@(#) FE_Element.tex, revA"

\noindent {\bf Files}   \\
\indent \#include $<\tilde{ }$/analysis/fe\_ele/FE\_Element.h$>$  \\

\noindent {\bf Class Declaration}  \\
\indent class FE\_Element;  \\

\noindent {\bf Description}  \\
\indent FE\_Element is a base class, subtypes of which are used to
enforce the constraints on the domain. An object of type FE\_Element
represents an element of the domain in the analysis. It enforces no
constraints other than single point homogeneous boundary conditions,
imposed on any of the elements nodes. It provides a similar interface
to that of an Element but modified to provide features useful to an
Analysis class.  The FE\_Element is responsible for: \begin{enumerate}
\item Holding information about the mapping between equation numbers
and the degrees-of-freedom at the 
element ends, this mapping is determined from the DOF\_Group objects
associated with the elements Node objects. \item Providing methods to
allow the integrator to combine the elements stiffness, mass and
damping matrices into the elements contribution to the structure
tangent matrix and the elements resisting force to the structure
unbalance. Obtaining the stiffness, damping and mass matrices from the
elements. \item Providing methods so other forces can be
determined. \end{enumerate} While the FE\_Element class is
associated with an element in the domain, subclasses do not have to
be. It is the subclasses that are used to implement the constraints
imposed on the nodal displacements in the domain. \\

\noindent {\bf Class Interface}  \\
\indent\indent // Constructors  \\
\indent\indent {\em FE\_Element(Element *theElementPtr);}  \\
\indent\indent {\em FE\_Element(int numDOFGroup, int numDOF);}  \\ \\
\indent\indent // Destructor  \\
\indent\indent {\em virtual~ $\tilde{}$FE\_Element();}  \\ \\
\indent\indent // Public Methods - Mapping  \\
\indent\indent {\em virtual const ID \&getDOFtags(void) const;} \\
\indent\indent {\em virtual const ID \&getID(void) const;} \\
\indent\indent {\em void setAnalysisModel(AnalysisModel \&theModel);} \\
\indent\indent {\em virtual void setID(void);} \\ \\
\indent\indent // Public Methods to form and obtain Tangent \& Residual \\
\indent\indent {\em virtual const Matrix \&getTangent(Integrator
*theIntegrator);} \\ 
\indent\indent {\em virtual const Vector \&getResidual(Integrator
*theIntegrator);} \\  \\
\indent\indent // Public Methods to allow Integrator to Build Tangent \\
\indent\indent {\em virtual void zeroTangent(void);} \\
\indent\indent {\em virtual void addKtToTang(double fact = 1.0); }\\
\indent\indent {\em virtual void addKsToTang(double fact = 1.0); }\\
\indent\indent {\em virtual void addCtoTang(double fact = 1.0); }\\
\indent\indent {\em virtual void addMtoTang(double fact = 1.0); }\\ \\
\indent\indent // Public Methods to allow Integrator to Build Residual \\
\indent\indent {\em virtual void zeroResidual(void);} \\
\indent\indent {\em virtual void addRtoResidual(double fact = 1.0);}\\
\indent\indent {\em virtual void addRIncInertiaToResidual(double fact = 1.0); }\\ 
\indent\indent {\em virtual void addK\_Force(const Vector \&disp,
double fact = 1.0); }\\ 
\indent\indent {\em virtual void addKtForce(const Vector \&disp,
double fact = 1.0); }\\ 
\indent\indent {\em virtual void addKsForce(const Vector \&disp,
double fact = 1.0); }\\ 
\indent\indent {\em virtual void addD\_Force(const Vector \&vel,
double fact = 1.0); }\\ 
\indent\indent {\em virtual void addM\_Force(const Vector \&accel,
double fact = 1.0); }\\  \\
\indent\indent // Public Methods to allow Element-by-Element strategies \\
\indent\indent {\em virtual const Vector \&getTangForce(const Vector
\&disp, double fact = 1.0);    }\\
\indent\indent {\em virtual const Vector \&getKtForce(const Vector
\&disp, double fact = 1.0);    }\\ 
\indent\indent {\em virtual const Vector \&getKsForce(const Vector
\&disp, double fact = 1.0);    }\\ 
\indent\indent {\em virtual const Vector \&getD\_Force(const Vector
\&vel, double fact = 1.0); }\\
\indent\indent {\em virtual const Vector \&getM\_Force(const Vector
\&accel, double fcat = 1.0);}\\ \\
\indent\indent // Public Methods added for Domain Decomposition \\
\indent\indent {\em Integrator *getLastIntegrator(void);} \\
\indent\indent {\em const Vector \&getLastResponse(void);} \\

\noindent {\bf Constructors}  \\
\indent {\em FE\_Element(Element *theElemen);}  \\
Constructs an empty FE\_Element with an associated element given by {\em
theElement}.  During construction it determines the number of unknown
dofs from the element. Constructs an ID for the mapping between dof's of
the element and equation numbers in the system of equation and an ID
to hold the tag of the DOF\_Group objects associated with each Node of
the element. If the result of invoking {\em
theElementPtr-$>$isSubdomain()} is {\em true} invokes {\em
setFE\_Element(this)} on the Subdomain; if false creates a Matrix 
for the tangent and a Vector for the residual to be stored. An error
message is printed and the program is terminated if no Domain object
is associated with the Element, a Node of the Element does not exist
in the Domain, each Node has not yet been associated with a DOF\_Group
object, or there is not enough memory for the Vectors and Matrices
required. \\  

{\em FE\_Element(int numDOFGroup, int numDOF);}  \\
Provided for subclasses. Constructs an empty FE\_Element with the
number of unknown dofs given by {\em numDOF} and the number of
associated DOF\_Group objects given by {\em numDOFGroup}, two empty IDs
are constructed to hold the mapping and the tags of the
DOF\_Groups. The subclass must fill in the ID for the tags of the
DOF\_Groups in order that {\em setID()} will work. No element is
associated with this FE\_Element. No space is allocated for the
tangent and residual matrix and vector, this is the responsibility of
the subclass. \\  

\noindent {\bf Destructor}  \\
\indent {\em virtual~ $\tilde{}$FE\_Element();}  \\
Deletes the IDs, Vectors and Matrices created by the constructor. \\

\noindent {\bf Public Methods - Mapping}  \\
\indent {\em virtual const ID \&getDOFtags(void) const;} \\
Returns a const ID containing the unique tag number of the
DOF\_Group objects associated with that FE\_Element. For this base class,
these are obtained from the DOF\_Groups associated with the Node objects
that are associated with the Element object passed in the constructor. This
ID is computed only once, during the creation of the object. \\

{\em virtual const ID \&getID(void) const;} \\
Returns a const ID containing the equation numbers associated with its
matrices and vectors. This ID will contain $0$'s unless the {\em
setID()} method has been called.\\

{\em void setAnalysisModel(AnalysisModel \&theModel);} \\
To set a link to the AnalysisModel in which the FE\_element
resides; this link is needed in {\em setID()}. Is invoked by the
AnalysisModel when the FE\_element is added to the AnalysisModel. \\


{\em virtual void setID(void);} \\
Causes the FE\_Element to determine the mapping between it's equation
numbers and the degrees-of-freedom. The $i-1$ component of the ID
contains the equation number that is associated with $i$'th
degree-of-freedom (a consequence of C indexing for IDs). The method is
to be invoked by the DOF\_Numberer after the DOF\_Groups have been assigned
their equation numbers. The base class uses the ID containing the tags of
the DOF\_Group objects to determine this by looping over the
DOF\_Group objects (identified in the DOF\_Group ID and obtained from
the AnalysisModel) and getting their mapping by invoking {\em
getID()}. Returns $0$ if successful, a warning message and a
negative number is returned if an error occurs:
$-1$ returned if no AnalysisModel link has been set, $-2$ if a
DOF\_Group object does not exist in the AnalysisModel and a $-3$ if
there are more dof's in the DOF\_Groups than dof's identified for the
FE\_Element. \\


\noindent {\bf Public Methods - Tangent and Residual}  \\
\indent {\em virtual const Matrix \&getTangent(Integrator
*theIntegrator);} \\ 
Causes the FE\_Element to determine it's contribution to the tangent
matrix and to return this matrix. If the Element is a
Subdomain it invokes {\em computeTangent()} and {\em getTang()} on the
Subdomain. Otherwise {\em formEleTangent(this)} is invoked on {\em
theIntegrator} and the new tangent matrix is returned.
Subclasses must provide their own implementation. If no Element is
passed in the constructor, a warning message is
printed and an error Matrix of size 1X1 is returned. \\


\indent {\em virtual const Vector \&getResidual(Integrator
*theIntegrator);} \\
Causes the FE\_Element to determine it's contribution to the residual
vector and to return this vector. If the Element is a Subdomain it invokes
{\em computeResidual()} and {\em getResistingForce()} on the Subdomain.
Otherwise {\em formEleResidual(this)} is invoked on {\em theIntegrator}
and the resuting residual vector is returned. 
Subclasses must provide their own implementation. If no Element is
passed in the constructor, a warning message and an error vector is
returned. \\


\indent {\em virtual void zeroTangent(void);} \\
Zeros the tangent matrix. If the Element is not a Subdomain invokes
{\em Zero()} on the tangent matrix. Subclasses must provide their own
implementation. Nothing is done and a warning message is printed if no
Element was passed in the constructor or the Element passed was a
Subdomain. \\ 


{\em virtual void addKtToTang(double fact = 1.0); }\\
Adds the product of {\em fact} times the element's tangent stiffness
matrix to the tangent. If no element is associated with the
FE\_Element nothing is added, if the element is not a Subdomain {\em
addMatrix(theEle-$>$getTangentStiff(),fact} is invoked on the tangent
matrix. Nothing is done and a warning message is printed if no Element
was passed in the constructor or the Element passed was a
Subdomain. \\  

{\em virtual void addKsToTang(double fact = 1.0); }\\
Adds the product of {\em fact} times the element's secant stiffness
matrix to the tangent. If no element is associated with the
FE\_Element nothing is added, if the element is not a Subdomain {\em
addMatrix(theEle-$>$getSecantStiff(),fact} is invoked on the tangent
matrix. Nothing is done and a warning message is printed if no Element
was passed in the constructor or the Element passed was a
Subdomain. \\  


{\em virtual void addCtoTang(double fact = 1.0); }\\
Adds the product of {\em fact} times the element's damping
matrix to the tangent. If no element is associated with the
FE\_Element nothing is added, if the element is not a Subdomain 
{\em addMatrix(theEle-$>$getDamp(),fact} is invoked on the tangent
matrix. Nothing is done and a warning message is printed if no Element was
passed in the constructor or the Element passed was a Subdomain. \\  


{\em virtual void addMtoTang(double fact = 1.0); }\\
Adds the product of {\em fact} times the element's mass
matrix to the tangent. If no element is associated with the
FE\_Element nothing is added, if the element is not a Subdomain 
{\em addMatrix(theEle-$>$getMass(),fact} is invoked on the tangent
matrix. Nothing is done and a warning message is printed if no Element was
passed in the constructor or the Element passed was a Subdomain. \\  

\indent {\em virtual void zeroResidual(void);} \\
Zeros the residual vector. If the Element is not a Subdomain invokes
{\em Zero()} on the residual vector. Subclasses must provide their own
implementation. Nothing is done and a warning message is printed if no
Element was passed in the constructor or the Element passed was a
Subdomain.\\ 


{\em virtual void addRtoResidual(double fact = 1.0); }\\
Adds to the residual vector the product of the elements residual load
vector and {\em fact}. If no element is associated with the
FE\_Element nothing is added, if the associated element is not a
Subdomain {\em addVector(myEle-$>$getResistingForce(),fact)} is
invoked on the residual. Nothing is done and a warning message is
printed if no Element was passed in the constructor or the Element
passed was a Subdomain. \\   

{\em virtual void addRIncInertiaToResidual(double fact = 1.0); }\\
Adds to the residual vector the product of the elements residual load
vector with inertia forces included and {\em fact}. If no element is
associated with the FE\_Element nothing is added, if the associated
element is not a Subdomain {\em
addVector(myEle-$>$getResistingForceIncInertia(),fact)} is 
invoked on the residual. Nothing is done and a warning message is
printed if no Element was passed in the constructor or the Element
passed was a Subdomain. \\   

{\em virtual void addKtForce(const Vector \&disp, double fact = 1.0);    }\\
Adds to the residual the product of elements current tangent stiffness matrix
and a Vector whose values are obtained by taking the product of {\em
fact} and those elements of the Vector {\em disp} associated with 
the FE\_Elements equation numbers. If no element is associated with the
FE\_Element or the Element is a Subdomain nothing is added and an
warning message is printed. An error message is also printed if invoking
{\em addMatrixVector()} on the residual vector returns $< 0$.\\

{\em virtual void addKsForce(const Vector \&disp, double fact = 1.0);    }\\
Adds to the residual the product of elements current tangent stiffness matrix
and a Vector whose values are obtained by taking the product of {\em
fact} and those elements of the Vector {\em disp} associated with 
the FE\_Elements equation numbers. If no element is associated with the
FE\_Element or the Element is a Subdomain nothing is added and an
warning message is printed. An error message is also printed if invoking
{\em addMatrixVector()} on the residual vector returns $< 0$.\\


{\em virtual void addD\_Force(const Vector \&vel, double fcat = 1.0); }\\
Adds to the residual the product of elements current damping matrix
and a Vector whose values are obtained by taking the product of {\em
fact} and those elements of the Vector {\em vel} associated with 
the FE\_Elements equation numbers. If no element is associated with the
FE\_Element or the Element is a Subdomain nothing is added and an
warning message is printed. An error message is also printed if invoking
{\em addMatrixVector()} on the residual vector returns $< 0$.\\


{\em virtual void addM\_Force(const Vector \&accel, double fact = 1.0); }\\
Adds to the residual the product of elements current mass matrix
and a Vector whose values are obtained by taking the product of {\em
fact} and those elements of the Vector {\em accel} associated with 
the FE\_Elements equation numbers. If no element is associated with the
FE\_Element or the Element is a Subdomain nothing is added and an
warning message is printed. An error message is also printed if invoking
{\em addMatrixVector()} on the residual vector returns $< 0$.\\

{\em virtual const Vector \&getTangForce(const Vector \&disp, double
fact = 1.0);    }\\
Returns the product of FE\_Elements current tangent matrix
and a Vector whose values are obtained by taking the product of {\em
fact} and those elements of the Vector {\em disp} associated with 
the FE\_Elements equation numbers. If the element associated with the
FE\_Element is a subdomain, the tangent is obtained by invoking {\em
getTang()} on the subdomain, otherwise the tangent is formed by
invoking {\em formEleTang(this)} on the integrator object last used in
a {\em getTangent()} or {\em getResidual()}.
If no element is associated with the
FE\_Element a zero vector is returned and an error message is
printed. An error message is also printed if invoking
{\em addMatrixVector()} on the force vector returns $< 0$. \\

{\em virtual const Vector \&getKtForce(const Vector \&disp, double
fact = 1.0);    }\\
Returns the product of elements current tangent stiffness matrix
and a Vector whose values are obtained by taking the product of {\em
fact} and those elements of the Vector {\em disp} associated with 
the FE\_Elements equation numbers. If no element is associated with the
FE\_Element or the associated element is a Subdomain an error vector
is returned and a warning message printed. \\

{\em virtual const Vector \&getKsForce(const Vector \&disp, double
fact = 1.0);    }\\
Returns the product of elements current secant stiffness matrix
and a Vector whose values are obtained by taking the product of {\em
fact} and those elements of the Vector {\em disp} associated with 
the FE\_Elements equation numbers. If no element is associated with the
FE\_Element or the associated element is a Subdomain an error vector
is returned and a warning message printed. \\


{\em virtual const Vector \&getD\_Force(const Vector \&vel, double
fact = 1.0); }\\
Returns the product of elements current damping matrix
and a Vector whose values are obtained by taking the product of {\em
fact} and those elements of the Vector {\em vel} associated with 
the FE\_Elements equation numbers. If no element is associated with the
FE\_Element or the associated element is a Subdomain a warning message
is printed and an error Vector is returned. \\

{\em virtual const Vector \&getM\_Force(const Vector \&accel, double
fcat = 1.0);}\\
Returns the product of elements current mass matrix
and a Vector whose values are obtained by taking the product of {\em
fact} and those elements of the Vector {\em accel} associated with 
the FE\_Elements equation numbers. If no element is associated with the
FE\_Element or the associated element is a Subdomain a warning message
is printed and an error Vector is returned. \\

{\em Integrator *getLastIntegrator(void);} \\
Method which returns the last integrator supplied in a {\em
formTangent()} or a {\em formResidual()}  invocation. \\

{\em const Vector \&getLastResponse(void);} \\
A method which invokes {\em getLastResponse()} on the Integrator
object that was last passed as an argument to any of the routines.
The FE\_Elements ID and the force Vector object is passed as arguments.
Returns the force Vector object if successful. If no element is
associated with the FE\_Element or no integrator has yet to be passed,
a warning message is printed and an error Vector is returned. \\




\pagebreak \subsubsection{PenaltySP\_FE}
% File: ~/OOP/analysis/fe_ele/penalty/PenaltySP_FE.tex 
%What: "@(#) PenaltySP_FE.tex, revA"

\noindent {\bf Files}   \\
\indent \#include $<\tilde{ }$/analysis/fe\_ele/penalty/PenaltySP\_FE.h$>$  \\

\noindent {\bf Class Declaration}  \\
\indent class PenaltySP\_FE: public FE\_Element ;  \\

\noindent {\bf Class Hierarchy}  \\
\indent FE\_Element \\
\indent\indent {\bf PenaltySP\_FE} \\ 

\noindent {\bf Description}  \\
\indent PenaltySP\_FE is a subclass of FE\_Element used to enforce a
single point constraint. It does this by adding $\alpha$ to the
tangent and $\alpha * (U\_s - U\_t)$ to the residual at the locations
corresponding to the constrained degree-of-freedom specified by the
SP\_Constraint, where $U_s$ is the specified value of the constraint
and $U_t$ the current trial displacement at the node corresponding to
the constraint.\\


\noindent {\bf Class Interface}  \\
\indent\indent // Constructor  \\
\indent\indent {\em PenaltySP\_FE(Domain \&theDomain, SP\_Constraint
\&theSP, double alpha = 1.0e8);} \\ \\
\indent\indent // Destructor  \\
\indent\indent {\em virtual~ $\tilde{}$PenaltySP\_FE();}  \\ \\
\indent\indent // Public Methods \\
\indent\indent {\em virtual void setID(void);} \\ 
\indent\indent {\em virtual const Matrix \&getTangent(Integrator
*theIntegrator);} \\  
\indent\indent {\em virtual const Vector \&getResidual(Integrator
*theIntegrator);} \\ 
\indent\indent {\em virtual const Vector \&getTangForce(const Vector
\&disp, double fact = 1.0);    }\\

\noindent {\bf Constructor}  \\
\indent {\em PenaltySP\_FE(Domain \&theDomain, SP\_Constraint \&theSP,
double alpha = 1.0e8);}\\
To construct a PenaltySP\_FE element to enforce the constraint
specified by the SP\_Constraint {\em theSP} using a value for
$\alpha$ of {\em alpha} (which, if none is specified, defaults to
$1.0e8$). The FE\_Element class constructor is called with 
the integers $1$ and $1$. A Matrix and a Vector object of order $1$
are created to return the tangent and residual contributions, with the
tangent entry being set at $\alpha$. A link to the Node in the  Domain
corresponding to the SP\_Constraint is also set. A warning message is
printed and program terminates if there is not enough memory or no
Node associated with the SP\_Constraint exists in the Domain.\\

\noindent {\bf Destructor}  \\
\indent {\em virtual~ $\tilde{}$PenaltySP\_FE();}  \\
Invokes the destructor on the Matrix and Vector objects created in the
constructor. \\

\noindent {\bf Public Methods}  \\
\indent {\em virtual void setID(void);} \\
Causes the PenaltySP\_FE to determine the mapping between it's equation
numbers and the degrees-of-freedom. From the Node object link, created
in the constructor, the DOF\_group corresponding to the Node
associated with the constraint is determined. From this {\em
DOF\_Group} object the mapping for the constrained degree of freedom
is determined and the ID in the base class is set. Returns $0$ if
successful. Prints a warning message and returns a negative number if
an error occurs: $-2$ if the
Node has no associated DOF\_Group, $-3$ if the constrained DOF
specified is invalid for this Node and $-4$ if the ID in the
DOF\_Group is too small for the Node. \\ 

\indent {\em virtual const Matrix \&getTangent(Integrator *theIntegrator);} \\
Returns the tangent Matrix created in the constructor. \\

\indent {\em virtual const Vector \&getResidual(Integrator *theIntegrator);} \\
Sets the FE\_Elements contribution to the residual to be
$\alpha * (U_s - U_t)$, where $U_s$ is the specified value of the
constraint and $U_t$ the current trial displacement at the node
corresponding to constrained degree-of-freedom. Prints a warning
message and sets this contribution to $0$ if the specified constrained
degree-of-freedom is invalid. Returns this residual Vector set.


{\em virtual const Vector \&getTangForce(const Vector \&disp, double
fact = 1.0);    }\\
Sets the FE\_Elements contribution to the residual to be
$\alpha * (U\_s - disp\_t)$, where $U\_s$ is the specified value of the
constraint and $disp\_t$ the value in {\em disp}
corresponding to constrained degree-of-freedom. Prints a warning
message and sets this contribution to $0$ if the mapping, determined in
{\em setID()}, for the the specified constrained degree-of-freedom lies 
outside {\em disp}. \\  












\pagebreak \subsubsection{PenaltyMP\_FE}
% File: ~/OOP/analysis/fe_ele/penalty/PenaltyMP_FE.tex 
%What: "@(#) PenaltyMP_FE.tex, revA"

\noindent {\bf Files}   \\
\indent \#include $<\tilde{ }$/analysis/fe\_ele/penalty/PenaltyMP\_FE.h$>$  \\

\noindent {\bf Class Declaration}  \\
\indent class PenaltyMP\_FE: public FE\_Element ;  \\

\noindent {\bf Class Hierarchy}  \\
\indent FE\_Element \\
\indent\indent {\bf PenaltyMP\_FE} \\ 

\noindent {\bf Description}  \\
\indent PenaltyMP\_FE is a subclass of FE\_Element used to enforce a
multi point constraint, of the form $\U_c = \C_{cr} \U_r$, where $\U_c$ are
the constrained degrees-of-freedom at the constrained node, $\U_r$ are
the retained degrees-of-freedom at the retained node and $\C_{cr}$ a
matrix defining the relationship between these degrees-of-freedom. 

To enforce the constraint a matrix $\alpha \C^T \C$ is added to the
tangent for the degrees-of-freedom $[\U_c$ $\U_r]$, where $\C = [-\I$ 
$\C_{cr}]$. Nothing is added to the residual. \\  

\noindent {\bf Class Interface}  \\
\indent\indent // Constructor  \\
\indent\indent {\em PenaltyMP\_FE(Domain \&theDomain, MP\_Constraint
\&theMP, double alpha);} \\ \\
\indent\indent // Destructor  \\
\indent\indent {\em virtual~ $\tilde{}$PenaltyMP\_FE();}  \\ \\
\indent\indent // Public Methods \\
\indent\indent {\em virtual void setID(void);} \\ 
\indent\indent {\em virtual const Matrix \&getTangent(Integrator
*theIntegrator);} \\  
\indent\indent {\em virtual const Vector \&getResidual(Integrator
*theIntegrator);} \\ 
\indent\indent {\em virtual const Vector \&getTangForce(const Vector
\&disp, double fact = 1.0);    }\\

\noindent {\bf Constructor}  \\
\indent {\em PenaltyMP\_FE(Domain \&theDomain, MP\_Constraint \&theMP,
double alpha);}\\
To construct a PenaltyMP\_FE element to enforce the constraint
specified by the MP\_Constraint {\em theMP} using a default value for
$\alpha$ of $alpha$. The FE\_Element class constructor is called with
the integers $2$ and the size of the {\em retainedID} plus the size of
the {\em constrainedID} at the MP\_Constraint {\em theMP}. A Matrix
and a Vector object are created for adding the contributions to the
tangent and the residual. The residual is zeroed. A Matrix is created
to store the $C$ Matrix. If the MP\_Constraint is not time varying,
the components of this Matrix are determined, then the contribution
to the tangent $\alpha C^TC$ is determined and finally the $C$ matrix
is destroyed. Links are set to the retained and constrained nodes.
A warning message is printed and the program is terminated if
either not enough memory is available for the Matrices and Vector or the
constrained and retained Nodes do not exist in the Domain.  \\


\noindent {\bf Destructor}  \\
\indent {\em virtual~ $\tilde{}$PenaltyMP\_FE();}  \\
Invokes delete on any Matrix or Vector objects created in the
constructor that have not yet been destroyed. \\

\noindent {\bf Public Methods}  \\
\indent {\em virtual void setID(void);} \\
Causes the PenaltyMP\_FE to determine the mapping between it's equation
numbers and the degrees-of-freedom. This information is obtained by
using the mapping information at the DOF\_Group objects associated with
the constrained and retained nodes to determine the mappings between
the degrees-of-freedom identified in the {\em constrainedID} and the
{\em retainedID} at the MP\_Constraint {\em theMP}. Returns $0$ if
successful. Prints a warning message and returns a negative number if
an error occurs: $-2$ if the
Node has no associated DOF\_Group, $-3$ if the constrained DOF
specified is invalid for this Node (sets corresponding ID component to
$-1$ so nothing is added to the tangent) and $-4$ if the ID in the
DOF\_Group is too small for the Node (again setting corresponding ID
component to $-1$). \\ 


\indent {\em virtual Matrix \&getTangent(Integrator *theIntegrator);} \\
If the MP\_Constraint is time-varying, from the MP\_Constraint
{\em theMP} it obtains the current $C_{cr}$ matrix; it then forms the
$C$ matrix and finally it sets the tangent matrix to be $\alpha
C^TC$. Returns the tangent matrix. \\

\indent {\em virtual const Vector \&getResidual(Integrator *theIntegrator);} \\
Returns the residual, a $\zero$ Vector. \\

{\em virtual const Vector \&getTangForce(const Vector \&disp, double
fact = 1.0);    }\\
CURRENTLY just returns the $0$ residual. THIS WILL NEED TO CHANGE FOR
ELE-BY-ELE SOLVERS. 




\pagebreak \subsubsection{LagrangeSP\_FE}
% File: ~/OOP/analysis/fe_ele/lagrange/LagrangeSP_FE.tex 
%What: "@(#) LagrangeSP_FE.tex, revA"

\noindent {\bf Files}   \\
\indent \#include $<\tilde{ }$/analysis/fe\_ele/lagrange/LagrangeSP\_FE.h$>$  \\

\noindent {\bf Class Declaration}  \\
\indent class LagrangeSP\_FE: public FE\_Element ;  \\

\noindent {\bf Class Hierarchy}  \\
\indent FE\_Element \\
\indent\indent {\bf LagrangeSP\_FE} \\ 

\noindent {\bf Description}  \\
\indent LagrangeSP\_FE is a subclass of FE\_Element used to enforce a
single point constraint. It does this by adding to the tangent and the
residual:
\[ \left[ \begin{array}{cc} 0 & \alpha \\ \alpha & 0 \end{array}
\right] ,
\left\{ \begin{array}{c} 0 \\ \alpha(u_s - u_t) \end{array} \right\} \]
\noindent at the locations
corresponding to the constrained degree-of-freedom specified by the
SP\_Constraint and the lagrange multiplier degree-of-freedom
introduced by the LagrangeConstraintHandler for this constraint, where
$U_s$ is the specified value of the constraint 
and $U_t$ the current trial displacement at the node corresponding to
the constraint.\\

\noindent {\bf Class Interface}  \\
\indent\indent // Constructor  \\
\indent\indent {\em LagrangeSP\_FE(Domain \&theDomain, SP\_Constraint
\&theSP, DOF\_Group \&theGroup, double alpha = 1.0);} \\ \\
\indent\indent // Destructor  \\
\indent\indent {\em virtual~ $\tilde{}$LagrangeSP\_FE();}  \\ \\
\indent\indent // Public Methods \\
\indent\indent {\em virtual void setID(void);} \\  
\indent\indent {\em virtual const Matrix \&getTangent(Integrator
*theIntegrator);} \\  
\indent\indent {\em virtual const Vector \&getResidual(Integrator
*theIntegrator);} \\  
\indent\indent {\em virtual const Vector \&getTangForce(const Vector
\&disp, double fact = 1.0);    }\\

\noindent {\bf Constructor}  \\
\indent {\em LagrangeSP\_FE(Domain \&theDomain, SP\_Constraint \&theSP,
DOF\_Group \&theGroup, double alpha = 1.0);}\\
To construct a LagrangeSP\_FE element to enforce the constraint
specified by the SP\_Constraint {\em theSP} using a value for
$\alpha$ of {\em alpha} (which, if none is specified, defaults to
$1.0$). The FE\_Element class constructor is called with 
the integers $2$ and $2$. A Matrix and a Vector object of order $2$
are created to return the tangent and residual contributions, with the
tangent entries (0,1) and (1,0) set at $\alpha$. A link to the Node in the Domain
corresponding to the SP\_Constraint is also set. A warning message is
printed and program terminates if there is not enough memory or no
Node associated with the SP\_Constraint exists in the Domain, or
DOF\_Group is associated with the Node.\\

\noindent {\bf Destructor}  \\
\indent {\em virtual~ $\tilde{}$LagrangeSP\_FE();}  \\
Invokes the destructor on the Matrix and Vector objects created in the
constructor. \\

\noindent {\bf Public Methods}  \\
\indent {\em virtual void setID(void);} \\
Causes the LagrangeSP\_FE to determine the mapping between it's equation
numbers and the degrees-of-freedom. From the Node object link, created
in the constructor, the DOF\_group corresponding to the Node
associated with the constraint is determined. From this {\em
DOF\_Group} object the mapping for the constrained degree of freedom
is determined and the myID(0) in the base class is set. The myID(1) is
determined from the Lagrange DOF\_Group {\em theGroup} passed in the
constructor. Returns $0$ if 
successful. Prints a warning message and returns a negative number if
an error occurs: $-2$ if the
Node has no associated DOF\_Group, $-3$ if the constrained DOF
specified is invalid for this Node and $-4$ if the ID in the
DOF\_Group is too small for the Node. \\ 

\indent {\em virtual const Matrix \&getTangent(Integrator *theIntegrator);} \\
Returns the tangent Matrix created in the constructor. \\

\indent {\em virtual const Vector \&getResidual(Integrator *theIntegrator);} \\
Sets the FE\_Elements contribution to the residual:
\[ \left\{ \begin{array}{c} 0 \\ \alpha(u_s - u_t) \end{array} \right\} \]
where $U_s$ is the specified value of the
constraint and $U_t$ the current trial displacement at the node
corresponding to constrained degree-of-freedom. Prints a warning
message and sets this contribution to $0$ if the specified constrained
degree-of-freedom is invalid. Returns this residual Vector.\\


{\em virtual const Vector \&getTangForce(const Vector \&disp, double
fact = 1.0);    }\\
Sets the FE\_Elements contribution to the residual:
\[ \left\{ \begin{array}{c} 0 \\ \alpha(u_s - u_t) \end{array} \right\} \]
where $U_s$ is the specified value of the
constraint and $U_t$ the current trial displacement in {\em disp}
corresponding to constrained degree-of-freedom. Prints a warning
message and sets this contribution to $0$ if the specified constrained
degree-of-freedom is invalid. \\ 












\pagebreak \subsubsection{LagrangeMP\_FE}
% File: ~/OOP/analysis/fe_ele/lagrange/LagrangeMP_FE.tex 
%What: "@(#) LagrangeMP_FE.tex, revA"

\noindent {\bf Files}   \\
\indent \#include $<\tilde{ }$/analysis/fe\_ele/lagrange/LagrangeMP\_FE.h$>$  \\

\noindent {\bf Class Declaration}  \\
\indent class LagrangeMP\_FE: public FE\_Element ;  \\

\noindent {\bf Class Hierarchy}  \\
\indent FE\_Element \\
\indent\indent {\bf LagrangeMP\_FE} \\ 

\noindent {\bf Description}  \\
\indent LagrangeMP\_FE is a subclass of FE\_Element used to enforce a
multi point constraint, of the form $\U_c = \C_{cr} \U_r$, where $\U_c$ are
the constrained degrees-of-freedom at the constrained node, $\U_r$ are
the retained degrees-of-freedom at the retained node and $\C_{cr}$ a
matrix defining the relationship between these degrees-of-freedom. 

To enforce the constraint the following are added to the tangent and
the residual:
\[ \left[ \begin{array}{cc} 0 & \alpha\C^t \\ \alpha\C & 0 \end{array}
\right] ,
\left\{ \begin{array}{c} 0 \\ 0 \end{array} \right\} \]
\noindent 
\noindent at the locations
corresponding to the constrained degree-of-freedoms specified by the
MP\_Constraint, i.e. $[\U_c$ $\U_r]$, and the lagrange multiplier
degrees-of-freedom introduced by the LagrangeConstraintHandler for
this constraint, $\C = [-\I$ $\C_{cr}]$. Nothing is added to the residual. \\  


\noindent {\bf Class Interface}  \\
\indent // Constructor  \\
\indent {\em LagrangeMP\_FE(Domain \&theDomain, MP\_Constraint
\&theMP, double alpha);} \\ \\
\indent // Destructor  \\
\indent {\em virtual~ $\tilde{}$LagrangeMP\_FE();}  \\ \\
\indent // Public Methods \\
\indent {\em virtual void setID(void);} \\ 
\indent {\em virtual const Matrix \&getTangent(Integrator
*theIntegrator);} \\  
\indent {\em virtual const Vector \&getResidual(Integrator
*theIntegrator);} \\  
\indent {\em virtual const Vector \&getTangForce(const Vector
\&disp, double fact = 1.0);    }\\

\noindent {\bf Constructor}  \\
\indent {\em LagrangeMP\_FE(Domain \&theDomain, MP\_Constraint \&theMP,
double alpha);}\\
To construct a LagrangeMP\_FE element to enforce the constraint
specified by the MP\_Constraint {\em theMP} using a default value for
$\alpha$ of $alpha$. The FE\_Element class constructor is called with
the integers $3$ and the two times the size of the {\em retainedID}
plus the size of the {\em constrainedID} at the MP\_Constraint {\em
theMP} plus . A Matrix and a Vector object are created for adding the
contributions to the tangent and the residual. The residual is
zeroed. If the
MP\_Constraint is not time varying, then the contribution to the
tangent is determined. Links are set to the retained and constrained
nodes. The DOF\_Group tag ID is set using the tag of the constrained
Nodes DOF\_Group, the tag of the retained Node Dof\_group and the tag
of the LagrangeDOF\_Group, {\em theGroup}. A warning message is printed and 
the program is terminated if either not enough memory is available for
the Matrices and Vector or the constrained and retained Nodes of their
DOF\_Groups do not exist. \\



\noindent {\bf Destructor}  \\
\indent {\em virtual~ $\tilde{}$LagrangeMP\_FE();}  \\
Invokes delete on any Matrix or Vector objects created in the
constructor that have not yet been destroyed. \\

\noindent {\bf Public Methods}  \\
\indent {\em virtual void setID(void);} \\
Causes the LagrangeMP\_FE to determine the mapping between it's equation
numbers and the degrees-of-freedom. This information is obtained by
using the mapping information at the DOF\_Group objects associated with
the constrained and retained nodes and the LagrangeDOF\_Group, {\em
theGroup}. Returns $0$ if
successful. Prints a warning message and returns a negative number if
an error occurs: $-2$ if the
Node has no associated DOF\_Group, $-3$ if the constrained DOF
specified is invalid for this Node (sets corresponding ID component to
$-1$ so nothing is added to the tangent) and $-4$ if the ID in the
DOF\_Group is too small for the Node (again setting corresponding ID
component to $-1$). \\ 

\indent {\em virtual const Matrix \&getTangent(Integrator *theIntegrator);} \\
If the MP\_Constraint is time-varying, from the MP\_Constraint
{\em theMP} it obtains the current $C_{cr}$ matrix; it then adds the
contribution to the tangent matrix. Returns this tangent Matrix.

\indent {\em virtual const Vector \&getResidual(Integrator *theIntegrator);} \\
Returns the residual, a $\zero$ Vector. \\

{\em virtual const Vector \&getTangForce(const Vector \&disp, double
fact = 1.0);    }\\
CURRENTLY just returns the $0$ residual. THIS WILL NEED TO CHANGE FOR
ELE-BY-ELE SOLVERS. 



\pagebreak \subsubsection{TransformationFE}
% File: ~/OOP/analysis/fe_ele/transformation/TransformationFE.tex 
%What: "@(#) TransformationFE.tex, revA"

\noindent {\bf Files}   \\
\indent \#include $<\tilde{ }$/analysis/fe\_ele/penalty/TransformationFE.h$>$  \\

UNDER CONSTRUCTION. \\

\noindent {\bf Class Declaration}  \\
\indent class TransformationFE: public FE\_Element ;  \\

\noindent {\bf Class Hierarchy}  \\
\indent FE\_Element \\
\indent\indent {\bf TransformationFE} \\ 

\noindent {\bf Description}  \\
\indent TransformationFE is a subclass of FE\_Element used to enforce a
multi point constraint, of the form $\U_c = \C_{cr} \U_r$, where $\U_c$ are
the constrained degrees-of-freedom at the constrained node, $\U_r$ are
the retained degrees-of-freedom at the retained node and $\C_{cr}$ a
matrix defining the relationship between these degrees-of-freedom. 

To enforce the constraint a matrix $\T^T \K \T$ is added to the
tangent and $\T^T \R$ is added to the residual where $\T$ is a block
diagonal matrix equal to WHAT?


\noindent {\bf Class Interface}  \\
\indent // Constructor  \\
\indent {\em TransformationFE(Domain \&theDomain,
TransformationConstraintHandler \&theHandler); \\ \\
\indent // Destructor  \\
\indent {\em virtual~ $\tilde{}$TransformationFE();}  \\ \\
\indent // Public Methods \\

\indent {\em    virtual const ID \&getDOFtags(void) const; }\\
\indent {\em   virtual const ID \&getID(void) const;}\\
\indent {\em    void setAnalysisModel(AnalysisModel \&theModel);}\\
\indent {\em virtual void setID(void);} \\ 
\indent {\em virtual const Matrix \&getTangent(Integrator
*theIntegrator);} \\  
\indent {\em virtual const Vector \&getResidual(Integrator
*theIntegrator);} \\ 
\indent {\em virtual const Vector \&getTangForce(const Vector
\&disp, double fact = 1.0);    }\\
\indent {\em    int addSP(SP\_Constraint \&theSP);}\\ \\
\indent // Protected Methods \\
\indent {\em transformResponse(const Vector \&modResponse, 
Vector \&unmodResponse);}\\

\noindent {\bf Constructor}  \\
\indent {\em TransformationFE(Domain \&theDomain,
TransformationConstraintHandler \&theHandler); \\


\noindent {\bf Destructor}  \\
\indent {\em virtual~ $\tilde{}$TransformationFE();}  \\


\noindent {\bf Public Methods}  \\
\indent {\em virtual void setID(void);} \\

\indent {\em virtual Matrix \&getTangent(Integrator *theIntegrator);} \\

\indent {\em virtual const Vector \&getResidual(Integrator *theIntegrator);} \\


{\em virtual const Vector \&getTangForce(const Vector \&disp, double
fact = 1.0);    }\\




\pagebreak \subsection{DOF\_Group}
% File: ~/OOP/analysis/model/DOF_Group.tex 
%What: "@(#) DOF_Group.tex, revA"

NEED A GETTANGFORCE() LIKE FE\_ELEMENT FOR ELE\_BY\_ELE SOLVERS. KEEP A
POINTER TO LAST INTEGRATOR. \\

\noindent {\bf Files}   \\
\indent \#include $<\tilde{ }$/analysis/dof\_grp/DOF\_Group.h$>$  \\

\noindent {\bf Class Decleration}  \\
\indent class DOF\_Group;  \\

\noindent {\bf Description}  \\
\indent DOF\_Group is a base class. An object of type DOF\_Group
represents an unconstrained node of the domain in the model. Each node
in the domain is associated with one DOF\_Group. DOF\_Groups are
called upon in the analysis to provide their contributions of
unbalanced load to the system of equations. Subclasses are used by the
constraint handler to to introduce new dofs into the analysis. 

The DOF\_Group is responsible for providing operations to set and
access the mapping betwwen equation numbers and DOFs, allowing the
Integrator to from the tangent (if nodal masses in transient problem)
and unbalanced load information, and for setting and obtaining the
nodal trial response quantities. \\


\noindent {\bf Class Interface}  \\
\indent // Constructors  \\
\indent {\em DOF\_Group(int tag, Node *theNode);}  \\
\indent {\em DOF\_Group(int tag, int numDOF);}  \\ \\
\indent // Destructor  \\
\indent {\em virtual~ $\tilde{}$DOF\_Group();}  \\\\
\indent // Public Methods - Mapping  \\
\indent {\em virtual void setID(int dof, int value);} \\
\indent {\em virtual void setID(const ID \&values);} \\
\indent {\em virtual const ID \&getID(void) const;} \\
\indent {\em virtual int getTag(void) const;} \\
\indent {\em virtual int getNumDOF(void) const;}\\
\indent {\em virtual int getNumFreeDOF(void) const;}\\
\indent {\em virtual int getNumConstrainedDOF(void) const;}\\ \\
\indent // Public Methods - Tangent \& Residual  \\ 
\indent {\em virtual const Matrix \&getTangent(Integrator *theIntegrator)} \\
\indent {\em virtual void zeroTangent(void);}\\
\indent {\em virtual void addMtoTang(double factt);}\\
\indent {\em virtual const Vector \&getUnbalance(Integrator
*theIntegrator); }\\ 
\indent {\em virtual void zeroUnbalance(void);}\\
\indent {\em virtual void  addPtoUnbalance(double fact = 1.0);} \\ \\
\indent {\em virtual void addMtoTang(const Vector \&$\ddot u$,
double fact);}\\ \\ 
\indent // Public Methods - Node Response  \\
\indent {\em virtual const Vector \&getCommittedDisp(void);} \\
\indent {\em virtual const Vector \&getCommittedVel(void);} \\
\indent {\em virtual const Vector \&getCommittedAccel(void);} \\
\indent {\em virtual int setNodeDisp(const Vector \&u);} \\
\indent {\em virtual int setNodeVel(const Vector \&$\dot u$);}\\
\indent {\em virtual int setNodeAccel(const Vector \&$\ddot u$);}\\
\indent {\em virtual int incrNodeDisp(const Vector \&u);} \\
\indent {\em virtual int incrNodeVel(const Vector \&$\dot u$);}\\
\indent {\em virtual int incrNodeAccel(const Vector \&$\ddot u$);}\\



\noindent {\bf Constructors}  \\
\indent {\em DOF\_Group(int tag, Node *theNode);}  \\
Constructs a  DOF\_Group with an associated node given by {\em
theNode} and a tag given by {\em tag}. During construction it
determines the number of unknown dofs from the node, {\em numDOF}, and
creates an Vector object to hold the unbalance information and an ID
object to hold mapping between degrees-of-freedom and equation
numbers. All values of the ID are set to $-2$ initially. It creates a
Vector to hold the unbalance loads and an ID to 
hold the mapping information, both of size equal to the number of
unknowns. If the size of the Vector or the ID is $0$, i.e. ran out of
memory, a warning message is orinted and {\em numDOF} set to $0$. \\

{\em DOF\_Group(int tag, int numDOF);}  \\
Provided for subclasses. Constructs a  DOF\_Group with the number of
unknown dofs given by {\em numDOF} and a tag given by {\em tag}. No
Node is associated with this DOF\_Group. Creates a Vector of size {\em
numDOF} to hold the unbalance information and an ID object to
hold mapping between degrees-of-freedom and equation numbers. All
values of the ID are set to $-2$ initially. If
the size of the Vector or the ID is $0$, i.e. ran out of memory, a
warning message is orinted and {\em numDOF} set to $0$. \\

\noindent {\bf Destructor}  \\
\indent {\em virtual~ $\tilde{}$DOF\_Group();}  \\
Invokes the destructor on the Vector created to hold the unbalance. \\

\noindent {\bf Public Methods - Equation Numbers}  \\
\indent {\em virtual void setID(int dof, int value);} \\
Operation to set the equation number of the {\em dof'th} DOF in the
DOF\_Group to {\em value}. In this base class, if {\em index} is a
valid location, $0 < index < numDOF-1$ the operator {\em (index) =
value} is invoked on the ID object created for the mapping, otherwise
an error message is printed. \\ 


{\em virtual void setID(const ID \&values);} \\
An operation to set the equation numbers of all the DOFs in the
DOF\_Group to be those given in the ID {\em values}. In this base
class, the operator {\em = values} is invoked on the ID object created
for the mapping. \\ 

{\em virtual const ID \&getID(void) const;} \\
A const member function which returns in an ID object the equation
numbers associated with the degrees-of-freedom in the DOF\_Group. The
size of the ID object is equal to the number of degrees-of-freedom
represented by the DOF\_Group. This base class returns the ID created
for the mapping.\\ 

{\em virtual int getTag(void) const;} \\
A const member function which returns the integer identifier {\em
tag} passed in the constructor. \\

{\em virtual int getNumDOF(void) const;}\\
Returns the total number of DOFs in the DOF\_Group. This base class
returns {\em numDOF}.\\

{\em virtual int getNumFreeDOF(void) const;}\\
Returns the total number of {\em free} DOFs in the DOF\_Group,
i.e. number of dof in the group who have not been assigned a negative
equation number. Determines this by looping through the ID containing
the mapping information. \\

{\em virtual int getNumConstrainedDOF(void) const;}\\
Returns the total number of {\em constrained} DOFs in the DOF\_Group,
i.e. number of dof in the group who have been assigned a negative
equation number. Determines this by looping through the ID containing
the mapping information. \\


\noindent {\bf Public Methods - Nodal Tangent contributions
for transient}  \\ 
\indent {\em virtual const Matrix \&getTangent(void)} \\
Returns the current tangent matrix for the DOF\_Group. If no tangent
matrix has been created, one of size(numDOF,numDOF) is constructed.
If not enough space is available for a new tangent matrix, an error
message is printed and an error Matrix is returned.. \\  

\indent {\em virtual void formTangent(Integrator *theIntegrator)} \\
To form the DOF\_Groups tangent. Invokes {\em formNodTangent(this)} on
The Integrator object {\em theIntegrator}. \\

\indent {\em virtual void zeroTangent(void);}\\
To zero the tangent. If a tangent matrix has been allocated,
will invoke {\em Zero()} on the tangent matrix.\\

\indent {\em virtual void addMtoTang(double fcat);}\\
To add {\em fact} times the nodal mass matrix to the tangent
matrix. In this base class, if a Matrix to store the tangent has not
yet been created, one of size (numDOF,numDOF) is now created; if
construction of this matrix fails an error is printed and an error
Matrix is returned. Invokes {\em addMatrix(theNode-$>$getMass,fact)}
on the Matrix, printing a warining message if this method returns a
$negative$ value. If no Node is associated with the DOF\_Group an
error message is printed and nothing is done. \\

\noindent {\bf Public Methods - Node Unbalance}  \\
\indent {\em virtual void zeroUnbalance(void);}\\
To zero the unbalance vector. Invokes {\em Zero()} on the vector
object used to store the unbalance information. \\

{\em virtual void formUnbalance(Integrator *theIntegrator); }\\
Causes the DOF\_Group to form its contribution to the residual. Invokes
{\em formNodUnbalance(this)} on {\em theIntegrator}.\\

{\em virtual const Vector \&getUnbalance() const; } \\
Returns the vector holding the unbalance. \\

{\em virtual void  addPtoUnbalance(double fact = 1.0);} \\
Adds the product of the unbalanced load at the node and {\em fact} to
the unbalance vector. A warning message is printed and nothing is done
if no node is associated with the DOF\_Group, otherwise {\em
addVector(theNode-$>$getUnbalancedLoad(),fact)} is invoked on the
unbalance vector and a warning message is printed if this method
returns a negative value. \\ 

{\em virtual void addMtoTang(const Vector \&$\ddot u$,
double fact);}\\ \\ 
Adds {\em fact} times the product of the associated nodes mass matrix
and the Vector $\ddot u$ to the unbalance vector. A warning message is
printed and nothing is done if no node is associated with the
DOF\_Group. \\

\noindent {\bf Public Methods - Node Response}  \\
\indent {\em virtual const Vector \&getCommittedDisp(void);} \\
To return the committed displacement at the node. Returns the result
of invoking {\em getDisp()} on the Node. If there is no associated
node object, an error message is printed and an error Vector is
returned. \\


\indent {\em virtual const Vector \&getCommittedVel(void);} \\
To return the committed velocity at the node. Returns the result of
invoking {\em getVel()} on the Node. If there is no associated node
object, an error message is printed and an error Vector is returned. \\

\indent {\em virtual const Vector \&getCommittedAccel(void);} \\
To return the committed velocity at the node. Returns result of
invoking  {\em getAccel()} on the Node. If there is no associated node
object, an error message is printed and an error Vector is returned. \\

\indent {\em virtual int setNodeDisp(const Vector \&u);} \\
This operation sets the value of the nodal trial displacement at the
associated node. The vector {\em u} is of size equal to the number of
equations in the AnalysisModel (this is not checked by the DOF\_Group).
The DOF\_Group object accesses the contents of the Vector {\em u} using
its assigned equation numbers. If a $-1$ exists for a DOF a $0$ value
is set as the corresponding nodal displacement for the node. Creates a Vector
to store the correct components of {\em u}, then invokes {\em setTrialDisp()}
on the node object; if no node object an error message is
printed. CURRENTLY THERE IS NO CHECK TO SEE IF myID(i) DOES NOT OVERFLOW
ADMISSABLE VALUES IN U - THIS NEEDS TO CHANGE \\


{\em virtual int setNodeVel(const Vector \&$u^{.}$);}\\
This operation sets the value of the nodal trial velocity at the
associated node. The vector {\em $u^{.}$} is of size equal to the number of
equations in the AnalysisModel (this is not checked by the DOF\_Group).
The DOF\_Group object accesses the contents of the Vector {\em $u^{.}$} using
its assigned equation numbers. If a $-1$ exists for a DOF a $0$ value
is set as the corresponding nodal velocity for the node.  Creates a Vector
to store the correct components of {\em u}, then invokes {\em setTrialVel()}
on the node object; if no node object an error message is printed. 
MUST CHANGE AS setNodeDisp \\

{\em virtual int setNodeAccel(const Vector \&$u^{..}$);}\\
This operation sets the value of the nodal trial acceleration at the
associated node. The vector {\em $u^{..}$} is of size equal to the number of
equations in the AnalysisModel (this is not checked by the DOF\_Group).
The DOF\_Group object accesses the contents of the Vector {\em $u^{..}$} using
its assigned equation numbers. If a $-1$ exists for a DOF a $0$ value
is set as the corresponding nodal acceleration for the node. The return
value is as outlined above for {\em setNodeAccel().} Creates a Vector
to store the correct components of {\em u}, then invokes {\em setTrialAccel()}
on the node object; if no node object an error message is printed. 
MUST CHANGE AS setNodeDisp \\



\indent {\em virtual int incrNodeDisp(const Vector \&u);} \\
\indent {\em virtual int incrNodeVel(const Vector \&$u^{.}$);}\\
\indent {\em virtual int incrNodeAccel(const Vector \&$u^{..}$);}\\
These methods are similar to those three just outlined, the only
difference being that the trial response quantities at the nodes are
incremented, not set, with the values now given; this is done by
invoking {\em incrTrialDisp()}, {\em incrTrialVel()}, and {\em 
incrTrialAccel()} respectivly on the nodes. MUST ALL CHANGE AS
setNodeDisp \\ 








\pagebreak \subsubsection{LagrangeDOF\_Group}
% File: ~/OOP/analysis/dof_grp/LagrangeDOF_Group.tex 
%What: "@(#) DOF_Group.tex, revA"

\noindent {\bf Files}   \\
\indent \#include $<\tilde{ }$/analysis/dof\_grp/LagrangeDOF\_Group.h$>$  \\

\noindent {\bf Class Declaration}  \\
\indent class LagrangeDOF\_Group: public DOF\_Group;  \\


\noindent {\bf Class Hierarchy}  \\
\indent DOF\_Group \\
\indent\indent {\bf LagrangeDOF\_Group} \\ 

\noindent {\bf Description}  \\
\indent LagrangeDOF\_Group is a subclass of DOF\_Group. It is used to
introduce the lagrange multipliers associated with single and
multi-point constraints into the system of equation.

\noindent {\bf Class Interface}  \\
\indent // Constructors  \\
\indent {\em LagrangeDOF\_Group(int tag, SP\_Constraint \&spPtr);}  \\
\indent {\em LagrangeDOF\_Group(int tag, MP\_Constraint \&mpPtr);}  \\ \\
\indent // Destructor  \\
\indent {\em virtual~ $\tilde{}$LagrangeDOF\_Group();}  \\\\
\indent // Public Methods - Tangent \& Residual  \\ 
\indent {\em virtual const Matrix \&getTangent(Integrator *theIntegrator)} \\
\indent {\em virtual const Vector \&getUnbalance(Integrator *theIntegrator); }\\
\indent // Public Methods - update trial response quantities  \\ 
\indent {\em virtual int setNodeDisp(const Vector \&u);} \\
\indent {\em virtual int setNodeVel(const Vector \&$\dot u$);}\\
\indent {\em virtual int setNodeAccel(const Vector \&$\ddot u$);}\\
\indent {\em virtual int incrNodeDisp(const Vector \&u);} \\
\indent {\em virtual int incrNodeVel(const Vector \&$\dot u$);}\\
\indent {\em virtual int incrNodeAccel(const Vector \&$\ddot u$);}\\

\noindent {\bf Constructors}  \\
\indent {\em LagrangeDOF\_Group(int tag, SP\_Constraint \&spPtr);}  \\
Invokes the DOF\_Group constructor with {\em tag) and the integer $1$. \\

\indent {\em LagrangeDOF\_Group(int tag, MP\_Constraint \&mpPtr);}  \\ \\
Invokes the DOF\_Group constructor with {\em tag) and an integer equal
to the size of the ID object returned from the MP\_Constraint {\em
mpPtr} when {\em getRetainedDOFs()} is invoked on the object. \\

\noindent {\bf Destructor}  \\
\indent {\em virtual~ $\tilde{}$LagrangeDOF\_Group();}  \\
Does nothing. \\

\noindent {\bf Public Methods} \\
\indent {\em virtual const Matrix \&getTangent(Integrator
*theIntegrator)} \\
Invokes {\em zero{}) on the tangent matrix and returns this matrix. 
The LagrangeMP\_FE and LagrangeSP\_FE objects add to the tangent. \\


\indent {\em virtual const Vector \&getUnbalance(Integrator
*theIntegrator); }\\
Invokes {\em zero{}) on the residual vector and returns this vector. 
The LagrangeMP\_FE and LagrangeSP\_FE objects add to the residual. \\

\indent {\em virtual int setNodeDisp(const Vector \&u);} \\
Does nothing. The lagrange multipliers are associated with no Nodes
in the Domain. \\


{\em virtual int setNodeVel(const Vector \&$u^{.}$);}\\
Does nothing. The lagrange multipliers are associated with no Nodes
in the Domain. \\

{\em virtual int setNodeAccel(const Vector \&$u^{..}$);}\\
Does nothing. The lagrange multipliers are associated with no Nodes
in the Domain. \\

\indent {\em virtual int incrNodeDisp(const Vector \&u);} \\
Does nothing. The lagrange multipliers are associated with no Nodes
in the Domain. \\

\indent {\em virtual int incrNodeVel(const Vector \&$u^{.}$);}\\
Does nothing. The lagrange multipliers are associated with no Nodes
in the Domain. \\

\indent {\em virtual int incrNodeAccel(const Vector \&$u^{..}$);}\\
Does nothing. The lagrange multipliers are associated with no Nodes
in the Domain. \\







\pagebreak \subsubsection{TransformationDOF\_Group}
% File: ~/OOP/analysis/dof_grp/TransformationDOF_Group.tex 
%What: "@(#) DOF_Group.tex, revA"

\noindent {\bf Files}   \\
\indent \#include $<\tilde{ }$/analysis/dof\_grp/TransformationDOF\_Group.h$>$  \\

UNDER CONSTRUCTION.\\

\noindent {\bf Class Declaration}  \\
\indent class TransformationDOF\_Group: public DOF\_Group;  \\


\noindent {\bf Class Hierarchy}  \\
\indent DOF\_Group \\
\indent\indent {\bf TransformationDOF\_Group} \\ 

\noindent {\bf Description}  \\
\indent TransformationDOF\_Group is a subclass of DOF\_Group. This
object stores the transformation matrix $\T$ used by the
TransformationFE objects when a node has been constrained with an MP
or SP\_Constraint.


\noindent {\bf Class Interface}  \\
\indent // Constructors  \\
\indent {\em TransformationDOF\_Group(int tag, Node *theNode,
	MP\_Constraint \&mpPtr);}  \\
\indent {\em TransformationDOF\_Group(int tag, Node *theNode);}  \\ \\
\indent // Destructor  \\
\indent {\em virtual~ $\tilde{}$TransformationDOF\_Group();}  \\\\
\indent // Public Methods - dealing with the ID and Transformation
matrix  \\ 
\indent {\em    int doneID(void);    }\\
\indent {\em    const ID \&getID(void) const; }\\
\indent {\em    virtual void setID(int dof, int value);    }\\
\indent {\em    Matrix *getT(void);}\\
\indent {\em    virtual int getNumDOF(void) const;    }\\
\indent {\em    virtual int getNumFreeDOF(void) const;}\\
\indent {\em    virtual int getNumConstrainedDOF(void) const;}\\\\
\indent // Public Methods - Tangent \& Residual  \\ 
\indent {\em virtual const Matrix \&getTangent(Integrator *theIntegrator)} \\
\indent {\em virtual const Vector \&getUnbalance(Integrator *theIntegrator); }\\
\indent // Public Methods - update trial response quantities  \\ 
\indent {\em virtual int setNodeDisp(const Vector \&u);} \\
\indent {\em virtual int setNodeVel(const Vector \&$\dot u$);}\\
\indent {\em virtual int setNodeAccel(const Vector \&$\ddot u$);}\\
\indent {\em virtual int incrNodeDisp(const Vector \&u);} \\
\indent {\em virtual int incrNodeVel(const Vector \&$\dot u$);}\\
\indent {\em virtual int incrNodeAccel(const Vector \&$\ddot u$);}\\




\pagebreak \subsection{{\bf ConstraintHandler}}
%File: ~/OOP/analysis/handler/ConstraintHandler.tex
%What: "@(#) ConstraintHandler.tex, revA"

\noindent {\bf Files}   \\
\indent \#include $<\tilde{ }$/analysis/handler/ConstraintHandler.h$>$  \\

\noindent {\bf Class Declaration}  \\
\indent class ConstraintHandler: public MovableObject  \\

\noindent {\bf Class Hierarchy} \\
\indent MovableObject \\
\indent\indent {\bf ConstraintHandler} \\

\noindent {\bf Description} \\ 
\indent The ConstraintHandler class is an abstract base class. Its purpose is
to define the interface common among all subclasses.  A constraint
handler is responsible for: \begin{enumerate} \item creating the
FE\_Element and DOF\_Group objects and adding them to the
AnalysisModel. \item setting the initial dof equation numbers to $-1$,
$-2$ or $-3$. A $-1$ indicates to the DOF\_Numberer object that no
equation number is to be allocated for this dof, a $-3$ that this dof
is to be among the last group of dof to be numbered. \item deleting
the DOF\_Group and FE\_Element objects that it created.\end{enumerate}


\noindent {\bf Class Interface} \\
\indent // Constructor \\
\indent {\em ConstraintHandler(int classTag);}\\  \\
\indent // Destructor \\
\indent {\em virtual~ $\tilde{}$ConstraintHandler();}\\  \\
\indent // Public Methods\\
\indent {\em virtual void setLinks(Domain \&theDomain, \\
\indent\indent\indent\indent AnalysisModel \&theModel, \\
\indent\indent\indent\indent Integrator \&theIntegrator);} \\
\indent {\em virtual int handle(const ID *nodeToBeNumberedLast
=0) =0;} \\ 
\indent {\em virtual void clearAll(void) =0;} \\ \\
\indent // Protected Methods  \\
\indent {\em Domain *getDomainPtr(void) const;} \\
\indent {\em AnalysisModel *getAnalysisModelPtr(void) const;} \\
\indent {\em Integrator *getIntegratorPtr(void) const;} \\


\noindent {\bf Constructor} \\
\indent {\em ConstraintHandler(int classTag);}\\ 
The integer {\em classTag} is passed to the MovableObject constructor. \\

\noindent {\bf Destructor} \\
\indent {\em virtual~ $\tilde{}$ConstraintHandler();}\\ 
Does nothing. \\

\noindent {\bf Public Methods }\\
\indent {\em virtual void setLinks(Domain \&theDomain, \\
\indent\indent\indent\indent\indent AnalysisModel \&theModel, \\
\indent\indent\indent\indent\indent Integrator \&theIntegrator);} \\
Invoked to set the links that the ConstraintHandler will need. These
include links to the Domain, {\em theDomain}, for which the
ConstraintHandler object will apply the constraints and the
AnalysisModel, {\em theModel}, to which the ConstraintHandler will add
the FE\_Element and DOF\_Group objects. \\

\indent {\em virtual int handle(const ID *nodesToBeNumberedLast =0) =0;} \\
Invoked to handle the constraints imposed on the domain by the
SP\_Constraints and MP\_Constraints. The ConstraintHandler object does
this by instantiating the appropriate FE\_ELement and DOF\_Group objects 
and adding them to the AnalysisModel. For all the dofs in each
DOF\_Group the ConstraintHandler sets initial equation numbers as
either $-1$, $-2$ or $-3$: A $-1$ indicates to the DOF\_Numberer object
that no equation number is to be allocated for this dof, a $-2$ that
an equation number is to be given for the dof, and a $-3$ that an
equation number is to be allocated and that this dof is to
be among the last group of dof to be numbered,i.e. all dof initially
assigned a $-3$ are to be given a higher equation number than those
given a $-2$. Those dof with a $-3$ should include all those dof
associated with the nodes whose tags are in {\em
nodesToBeNumberedLast}. Returns a positive number if successfully, a
negative integer if not; the positive number is to be set at the
number of dof assigned a value $-3$ (this will be the number of
external dof for a subdomain), the negative value of which depends on
the type of ConstraintHandler. For subdomains the constraint handler
is responsible for setting the FE\_Element by calling {\em
setFE\_elementPtr}. \\    

{\em virtual void clearAll(void) =0;} \\
Invoked to inform the ConstraintHandler object that the FE\_Elements
and DOF\_Groups it constructed are no longer part of the AnalysisModel. The
ConstraintHandler can delete these objects if necessary; or the
ConstraintHandler can store them and use them in subsequent calls to
{\em handle()}. \\ 

\noindent {\bf Protected Methods}  \\
\indent {\em Domain *getDomainPtr(void) const;} \\
A const member function to return the Domain object associated with
the ConstraintHandler, {\em theDomain}. \\

{\em AnalysisModel *getAnalysisModelPtr(void) const;} \\
A const member function to return the AnalysisModel object associated with
the ConstraintHandler, {\em theModel}. \\

{\em Integrator *getIntegratorPtr(void) const;} \\
A const member function to return the Integrator object associated with
the ConstraintHandler, {\em theIntegrator}. 







\pagebreak \subsubsection{PlainHandler}
%File: ~/OOP/analysis/handler/PlainHandler.tex
%What: "@(#) PlainHandler.tex, revA"

\noindent {\bf Files}   \\
\indent \#include $<\tilde{ }$/analysis/handler/PlainHandler.h$>$  \\

\noindent {\bf Class Declaration}  \\
\indent class PlainHandler: public ConstraintHandler  \\

\noindent {\bf Class Hierarchy} \\
\indent MovableObject \\
\indent\indent ConstraintHandler \\
\indent\indent\indent {\bf PlainHandler} \\

\noindent {\bf Description} \\ 
\indent The PlainHandler class is a class which only deals with
homogeneous single point constraints. To do this it creates regular
FE\_Element and DOF\_Group objects and enforces the constraints by
specifying that degrees-of-freedom which are constrained are not
assigned an equation number. Pointers to the DOF\_Group and
FE\_Element objects are kept in two arrays. \\

\noindent {\bf Class Interface} \\
\indent // Constructor \\
\indent {\em PlainHandler();}\\  \\
\indent // Destructor \\
\indent {\em $\tilde{ }$PlainHandler();}\\  \\
\indent // Public Methods\\
\indent {\em int handle(const ID *nodeToBeNumberedLast
=0);} \\ 
\indent {\em void clearAll(void);} \\ \\
\indent {\em int sendSelf(int commitTag, Channel \&theChannel); } \\
\indent {\em int recvSelf(int commitTag, Channel \&theChannel,
FEM\_ObjectBroker \&theBroker); } \\

\noindent {\bf Constructor} \\
\indent {\em PlainHandler();}\\ 
The integer {\em HANDLER\_TAG\_PlainHandler} (defined in
$<$classTags.h$>$) is passed to the PlainHandler constructor. \\

\noindent {\bf Destructor} \\
\indent {\em virtual~ $\tilde{}$PlainHandler();}\\ 
Invokes the destructor on all the FE\_Element and DOF\_Group objects
created in {\em handle}. Then invokes the destructor on the two
pointer arrays. \\

\noindent {\bf Public Methods }\\
\indent {\em int handle(const ID *nodeToBeNumberedLast =0);} \\
Determines the number of FE\_Elements and DOF\_Groups needed from the
Domain (a one to one mapping between Elements and FE\_Elements and
Nodes and DOF\_Groups) Creates two arrays of pointers to store the
FE\_elements and DOF\_Groups, returning a warning message and a $-2$
or $-3$ if not enough memory is available for these arrays. Then the
object will iterate through the Nodes of the Domain, creating a
DOF\_Group for each node and setting the initial id for each dof to
$-2$ if no SP\_Constraint exists for the dof, or $-1$ if a
SP\_Constraint exists or $-3$ if the node identifier is in {\em
nodesToBeNumberedLast}. The object then iterates through the Elements
of the Domain creating a FE\_Element for each Element, if the Element
is a Subdomain {\em setFE\_ElementPtr()} is invoked on the Subdomain
with the new FE\_Element as the argument. If not enough memory is
available for any DOF\_Group or FE\_element a warning message is
printed and a $-4$ or $-5$ is returned. If any MP\_Constraint objects
exist in the Domain a warning message is printed and $-6$ is
returned. If all is successful, the method returns the number of
degrees-of-freedom associated with the DOF\_Groups in {\em
nodesToBeNumberedLast}. \\ 

{\em void clearAll(void) =0;} \\
Currently this invokes delete on all the FE\_element and DOF\_Group
objects created in {\em handle()} and the arrays used to store
pointers to these objects. FOR ANALYSIS INVOLVING DYNAMIC LOAD
BALANCING, RE-MESHING AND CONTACT THIS MUST CHANGE. \\

{\em int sendSelf(int commitTag, Channel \&theChannel); } \\
Returns $0$. \\

{\em int recvSelf(int commitTag, Channel \&theChannel, FEM\_ObjectBroker
\&theBroker); } \\
Returns $0$. 

\pagebreak \subsubsection{PenaltyConstraintHandler}
%File: ~/OOP/analysis/handler/PenaltyConstraintHandler.tex
%What: "@(#) PenaltyConstraintHandler.tex, revA"

\noindent {\bf Files}   \\
\indent \#include $<\tilde{
}$/analysis/handler/PenaltyConstraintHandler.h$>$  \\ 

\noindent {\bf Class Declaration}  \\
\indent class PenaltyConstraintHandler: public ConstraintHandler  \\

\noindent {\bf Class Hierarchy} \\
\indent MovableObject \\
\indent\indent ConstraintHandler \\
\indent\indent\indent {\bf PenaltyConstraintHandler} \\

\noindent {\bf Description} \\ 
\indent The PenaltyConstraintHandler class is a class which deals with
both single and multi point constraints using the penalty method. 
This is done by, in addition to creating a DOF\_Group object
for each Node and an FE\_Element for each Element in the Domain,
creating either a PenaltySP\_FE or a PenaltyMP\_FE object for each
constraint in the Domain. It is these objects that enforce the
constraints by moifying the tangent matrix and residual vector. \\ 


\noindent {\bf Class Interface} \\
\indent // Constructor \\
\indent {\em PenaltyConstraintHandler(double alphaSP, double
alphaMP);}\\  \\ 
\indent // Destructor \\
\indent {\em virtual~ $\tilde{}$PenaltyConstraintHandler();}\\  \\
\indent // Public Methods\\
\indent {\em virtual int handle(const ID *nodeToBeNumberedLast
=0); } \\ 
\indent {\em virtual void clearAll(void);} \\ \\
\indent {\em int sendSelf(int commitTag, Channel \&theChannel); } \\
\indent {\em int recvSelf(int commitTag, Channel \&theChannel, FEM\_ObjectBroker
\&theBroker); } \\


\noindent {\bf Constructor} \\
\indent {\em PenaltyConstraintHandler(double alphaSp, double alphaMP);}\\ 
The integer {\em HANDLER\_TAG\_PenaltyConstraintHandler} (defined in
$<$classTags.h$>$) is passed to the PenaltyConstraintHandler
constructor. Stores {\em alphaSP} and {\em alphaMP} which are needed
to construct the PenaltySP\_FE and PenaltyMP\_FE objects in {\em
handle()}. \\

\noindent {\bf Destructor} \\
\indent {\em virtual~ $\tilde{}$PenaltyConstraintHandler();}\\ 
Currently invokes {\em clearAll()}, this will change when {\em
clearAll()} is rewritten. \\

\noindent {\bf Public Methods }\\
\indent {\em virtual int handle(const ID *nodeToBeNumberedLast =0) =0;} \\
Determines the number of FE\_Elements and DOF\_Groups needed from the
Domain (a one to one mapping between Elements and FE\_Elements,
SP\_Constraints and PenaltySP\_FEs, MP\_Constraints and PenaltyMP\_FEs and
Nodes and DOF\_Groups). Creates two arrays of pointers to store the
FE\_Elements and DOF\_Groups, returning a warning message and a $-2$
or $-3$ if not enough memory is available for these arrays. Then the
object will iterate through the Nodes of the Domain, creating a
DOF\_Group for each node and setting the initial id for each dof to
$-2$ or $-3$ if the node identifier is in {\em
nodesToBeNumberedLast}. The object then iterates through the Elements
of the Domain creating a FE\_Element for each Element, if the Element
is a Subdomain {\em setFE\_ElementPtr()} is invoked on the Subdomain
with the new FE\_Element as the argument. If not enough memory is
available for any DOF\_Group or FE\_element a warning message is
printed and a $-4$ or $-5$ is returned. 
The object then iterates through the SP\_Constraints
of the Domain creating a PenaltySP\_FE for each constraint, using the
Domain, the constraint and {\em alphaSP} as the arguments in the
constructor.
The object then iterates through the MP\_Constraints
of the Domain creating a PenaltyMP\_FE for each constraint, using the
Domain, the constraint and {\em alphaMP} as the arguments in the constructor.
Finally the method returns the
number of degrees-of-freedom associated with the DOF\_Groups in {\em
nodesToBeNumberedLast}. \\

{\em virtual void clearAll(void) =0;} \\
Currently this invokes delete on all the FE\_element and DOF\_Group
objects created in {\em handle()} and the arrays used to store
pointers to these objects. FOR ANALYSIS INVOLVING DYNAMIC LOAD
BALANCING, RE-MESHING AND CONTACT THIS MUST CHANGE. \\

{\em int sendSelf(int commitTag, Channel \&theChannel); } \\
Sends in a Vector of size 2 {\em alphaSP} and {\em alphaMP}. Returns
$0$ if successful. \\

{\em int recvSelf(int commitTag, Channel \&theChannel, FEM\_ObjectBroker
\&theBroker); } \\
Receives in a Vector of size 2 the values {\em alphaSP} and {\em
alphaMP}. Returns $0$ if successful. 

\pagebreak \subsubsection{LagrangeConstraintHandler}
%File: ~/OOP/analysis/handler/LagrangeConstraintHandler.tex
%What: "@(#) LagrangeConstraintHandler.tex, revA"

\noindent {\bf Files}   \\
\indent \#include $<\tilde{
}$/analysis/handler/LagrangeConstraintHandler.h$>$  \\ 

\noindent {\bf Class Declaration}  \\
\indent class LagrangeConstraintHandler: public ConstraintHandler  \\

\noindent {\bf Class Hierarchy} \\
\indent MovableObject \\
\indent\indent ConstraintHandler \\
\indent\indent\indent {\bf LagrangeConstraintHandler} \\

\noindent {\bf Description} \\ 
\indent The LagrangeConstraintHandler class is a class which deals with
both single and multi point constraints using the Lagrange
method. This is done by, in addition to creating a DOF\_Group object
for each Node and an FE\_Element for each Element in the Domain,
creating a LagrangeDOF\_Group object and either a LagrangeSP\_FE or a
LagrangeMP\_FE object for each constraint in the Domain. It is these
objects that enforce the constraints by modifying the tangent matrix
and residual vector. \\ 


\noindent {\bf Class Interface} \\
\indent // Constructor \\
\indent {\em LagrangeConstraintHandler(double alphaSP, double
alphaMP);}\\  \\ 
\indent // Destructor \\
\indent {\em virtual~ $\tilde{}$LagrangeConstraintHandler();}\\  \\
\indent // Public Methods\\
\indent {\em virtual int handle(const ID *nodeToBeNumberedLast
=0); } \\ 
\indent {\em virtual void clearAll(void);} \\ \\
\indent {\em int sendSelf(int commitTag, Channel \&theChannel); } \\
\indent {\em int recvSelf(int commitTag, Channel \&theChannel, FEM\_ObjectBroker
\&theBroker); } \\


\noindent {\bf Constructor} \\
\indent {\em LagrangeConstraintHandler(double alphaSp, double alphaMP);}\\ 
The integer {\em HANDLER\_TAG\_LagrangeConstraintHandler} (defined in
$<$classTags.h$>$) is passed to the LagrangeConstraintHandler
constructor. Stores {\em alphaSP} and {\em alphaMP} which are needed
to construct the LagrangeSP\_FE and LagrangeMP\_FE objects in {\em
handle()}. \\

\noindent {\bf Destructor} \\
\indent {\em virtual~ $\tilde{}$LagrangeConstraintHandler();}\\ 
Currently invokes {\em clearAll()}, this will change when {\em
clearAll()} is rewritten. \\

\noindent {\bf Public Methods }\\
\indent {\em virtual int handle(const ID *nodeToBeNumberedLast =0) =0;} \\
Determines the number of FE\_Elements and DOF\_Groups needed from the
Domain (a one to one mappinging between Elements and FE\_Elements,
SP\_Constraints and LagrangeSP\_FEs, MP\_Constraints and LagrangeMP\_FEs and
Nodes and DOF\_Groups). Creates two arrays of pointers to store the
FE\_Elements and DOF\_Groups, returning a warning message and a $-2$
or $-3$ if not enough memory is available for these arrays. Then the
object will iterate through the Nodes of the Domain, creating a
DOF\_Group for each node and setting the initial id for each dof to
$-2$ or $-3$ if the node identifier is in {\em
nodesToBeNumberedLast}. The object then iterates through the Elements
of the Domain creating a FE\_Element for each Element, if the Element
is a Subdomain {\em setFE\_ElementPtr()} is invoked on the Subdomain
with the new FE\_Element as the argument. If not enough memory is
available for any DOF\_Group or FE\_element a warning message is
printed and a $-4$ or $-5$ is returned. 
The object then iterates through the SP\_Constraints
of the Domain creating a LagrangeSP\_FE for each constraint, using the
Domain, the constraint and {\em alphaSP} as the arguments in the
constructor.
The object then iterates through the MP\_Constraints
of the Domain creating a LagrangeMP\_FE for each constraint, using the
Domain, the constraint and {\em alphaMP} as the arguments in the constructor.
Finally the method returns the
number of degrees-of-freedom associated with the DOF\_Groups in {\em
nodesToBeNumberedLast}. \\

{\em virtual void clearAll(void) =0;} \\
Currently this invokes delete on all the FE\_element and DOF\_Group
objects created in {\em handle()} and the arrays used to store
pointers to these objects. FOR ANALYSIS INVOLVING DYNAMIC LOAD
BALANCING, RE-MESHING AND CONTACT THIS MUST CHANGE. \\

{\em int sendSelf(int commitTag, Channel \&theChannel); } \\
Sends in a Vector of size 2 {\em alphaSP} and {\em alphaMP}. Returns
$0$ if successful. \\

{\em int recvSelf(int commitTag, Channel \&theChannel, FEM\_ObjectBroker
\&theBroker); } \\
Receives in a Vector of size 2 the values {\em alphaSP} and {\em
alphaMP}. Returns $0$ if successful. 

\pagebreak \subsubsection{TransformationConstraintHandler}
%File: ~/OOP/analysis/handler/TransformationConstraintHandler.tex
%What: "@(#) TransformationConstraintHandler.tex, revA"

\noindent {\bf Files}   \\
\indent \#include $<\tilde{
}$/analysis/handler/TransformationConstraintHandler.h$>$  \\ 

UNDER CONSTRUCTION.\\

\noindent {\bf Class Declaration}  \\
\indent class TransformationConstraintHandler: public ConstraintHandler  \\

\noindent {\bf Class Hierarchy} \\
\indent MovableObject \\
\indent\indent ConstraintHandler \\
\indent\indent\indent {\bf TransformationConstraintHandler} \\

\noindent {\bf Description} \\ 
\indent The TransformationConstraintHandler class is a class which deals with
both single and multi point constraints using the Transformation method. 
This is done by, in addition to creating a DOF\_Group object
for each Node and an FE\_Element for each Element in the Domain,
creating either a TransformationSP\_FE or a TransformationMP\_FE object for each
constraint in the Domain. It is these objects that enforce the
constraints by moifying the tangent matrix and residual vector. \\ 


\noindent {\bf Class Interface} \\
\indent // Constructor \\
\indent {\em TransformationConstraintHandler();}\\  \\ 
\indent // Destructor \\
\indent {\em virtual~ $\tilde{}$TransformationConstraintHandler();}\\  \\
\indent // Public Methods\\
\indent {\em virtual int handle(const ID *nodeToBeNumberedLast
=0); } \\ 
\indent {\em virtual void clearAll(void);} \\ 
\indent {\em int enforceSPs(void); }\\
\indent {\em int doneDOFids(void); }\\
\indent {\em int sendSelf(int commitTag, Channel \&theChannel); } \\
\indent {\em int recvSelf(int commitTag, Channel \&theChannel, FEM\_ObjectBroker
\&theBroker); } \\


\noindent {\bf Constructor} \\
\indent {\em TransformationConstraintHandler();}\\ 
The integer {\em HANDLER\_TAG\_TransformationConstraintHandler} (defined in
$<$classTags.h$>$) is passed to the TransformationConstraintHandler
constructor. 

\noindent {\bf Destructor} \\
\indent {\em virtual~ $\tilde{}$TransformationConstraintHandler();}\\ 
Currently invokes {\em clearAll()}, this will change when {\em
clearAll()} is rewritten. \\

\noindent {\bf Public Methods }\\
\indent {\em virtual int handle(const ID *nodeToBeNumberedLast =0) =0;} \\


\pagebreak \subsection{DOF\_Numberer}
%File: ~/OOP/analysis/numberer/DOF_Numberer.tex
%What: "@(#) DOF_Numberer.tex, revA"

\noindent {\bf Files}   \\
\indent \#include $<\tilde{ }$/analysis/numberer/DOF\_Numberer.h$>$  \\

\noindent {\bf Class Declaration}  \\
\indent class DOF\_Numberer: public MovableObject  \\

\noindent {\bf Class Hierarchy} \\
\indent MovableObject \\
\indent\indent {\bf DOF\_Numberer} \\

\noindent {\bf Description} \\ 
\indent The DOF\_Numberer class is a base class. Its purpose is
to define the interface common among all subclasses.  A DOF\_Numberer
object is responsible for assigning the equation numbers to the
individual dofs in each of the  DOF\_Groups in the AnalysisModel. The
base DOF\_Numberer uses a GraphNumberer object to first number the
DOF\_Groups, based on the ordering of the DOF\_Group objects, it
assigns the equation numbers to the individual
degrees-of-freedom. Subtypes may wish to implement the numbering in
a more efficient manner by using the FE\_Element and DOF\_Group objects
directly. \\


\noindent {\bf Class Interface} \\
\indent // Constructors \\
\indent {\em DOF\_Numberer(GraphNumberer \&theGraphNumberer);}\\ 
\indent {\em DOF\_Numberer(int classTag);}\\ 
\indent {\em DOF\_Numberer();}\\  \\
\indent // Destructor \\
\indent {\em virtual~ $\tilde{}$DOF\_Numberer();}\\  \\
\indent // Public Methods \\
\indent {\em void setLinks(AnalysisModel \&theModel);} \\ 
\indent {\em virtual int numberDOF(int lastDOF\_Group = -1);} \\
\indent {\em virtual int numberDOF(ID \&lastDOF\_Groups);} \\
\indent {\em virtual int sendSelf(int commitTag, Channel \&theChannel);}\\ 
\indent {\em virtual int recvSelf(int commitTag, Channel \&theChannel,
FEM\_ObjectBroker \&theBroker);}\\ \\
\indent // Protected Methods  \\
\indent {\em AnalysisModel *getAnalysisModelPtr(void) const;} \\
\indent {\em GraphNumberer *getGraphNumbererPtr(void) const;} \\


\noindent {\bf Constructors} \\
\indent {\em DOF\_Numberer(GraphNumberer \&theGraphNumberer);}\\ 
The integer {\em NUMBERER\_TAG\_DOF\_Numberer}
(defined in $<$classtags.h$>$) is passed to the MovableObject classes
constructor. Sets the GraphNumberer to be used in the numbering to {\em
theGraphNumberer()}. \\

\indent {\em DOF\_Numberer(int classTag);}\\ 
Provided for subclasses. The integer {\em classTag} is passed to the
MovableObject classes constructor. \\


\indent {\em DOF\_Numberer();}\\ 
Provided for FEM\_ObjectBroker. The integer {\em NUMBERER\_TAG\_DOF\_Numberer}
(defined in $<$classtags.h$>$) is passed to the MovableObject classes
constructor. Sets the GraphNumberer to be used in the numbering to be
$0$, the GraphNumberer object is created and linked in a {\em
recvSelf()} method invocation. \\


\noindent {\bf Destructor} \\
\indent {\em virtual~ $\tilde{}$DOF\_Numberer();}\\ 
Does nothing. \\

\noindent {\bf Public Methods }\\
\indent {\em void setLinks(AnalysisModel \&theModel);} \\
Invoked to set a link to the AnalysisModel from which the
DOF\_Numberer will number (provide the equation number in the
SystemOfEqn object) the degrees-of-freedoms in each DOF\_Group
objects.\\  

\indent {\em virtual int numberDOF(int lastDOF\_Group = -1);} \\
Invoked to assign the equation numbers to the dofs in the DOF\_Groups
and the FE\_Elements, ensuring that the dof's in the DOF\_Group whose
tag is given by {\em lastDOF\_Group} are numbered last in a $-2$ or
$-3$ group. The initial values of these equation numbers have been set
by the ConstraintHandler object to be $-1$, $-2$ or $-3$, all dofs
with a $-3$ are to be assigned higher equation numbers than those
assigned a $-2$. To set the {\em numEqn} in the AnalysisModel and to
return the number of equations {\em numEqn} if successful, a negative
number if not.  

This base class performs the ordering by getting an ID containing the
ordered DOF\_Group tags, obtained by invoking {\em
number(theModel-$>$getDOFGroupGraph(), lastDOF\_Group)} on the
GraphNumberer, {\em theGraphNumberer}, passed in the constructor. The
base class then makes two passes through the DOF\_Group objects in the
AnalysisModel by looping through this ID; in the first pass assigning the
equation numbers incrementally to any degree-of-freedom marked with a
$-2$ and in the second pass assigning the equation numbers
incrementally to any degree-of-freedom marked with a $-3$. It then
iterates through the FE\_Elements in the AnalsisModel invoking {\em
setID()} on each object. Finally {\em setNumEqn(numEqn)} is invoked on
the AnalysisModel. Return {\em numEqn} if successful, a warning
message and a negative number is returned if an error occurs; $-1$ is
returned if {\em setLinks()} has not yet been invoked, $-2$ if no
GraphNumberer was passed in the constructor, $-3$ if the
number of {\em DOF\_Groups} in AnalysisModel and size of ID returned
are not the same, and a $-4$ if there is no DOF\_Group corresponding
to one of the tags given in the ID.\\



\indent {\em virtual int numberDOF(ID \&lastDOF\_Groups);} \\
Invoked to assign the equation numbers to the dofs in the DOF\_Groups
and the FE\_Elements, ensuring that the dof's in the DOF\_Groups whose
tag is given in {\em lastDOF\_Groups} are numbered last in a $-2$ or
$-3$ group. The initial values of these equation numbers have been set
by the ConstraintHandler object to be $-1$, $-2$ or $-3$, all dofs
with a $-3$ are to be assigned higher equation numbers than those
assigned a $-2$. To set the {\em numEqn} in the AnalysisModel and to
return the number of equations {\em numEqn} if successful, a negative
number if not.  

This method in the base class is almost identical to the one just
described. The only difference is that the ID identifying the order of
the DOF\_Groups is obtained by invoking {\em
number(theModel-$>$getDOFGroupGraph(), lastDOF\_Groups)} on the
GraphNumberer. \\



\indent {\em virtual int sendSelf(Channel \&theChannel,
FEM\_ObjectBroker \&theBroker);}\\ 
The DOF\_Numberer sends the class identifier and database tag of the
GraphNumberer in a ID to the channel, if no GraphNumberer is
associated a $-1$ is sent as the class tag. The object then invokes
{\em sendSelf()} on the GraphNumberer. \\ 

\indent {\em virtual int recvSelf(int commitTag, Channel \&theChannel, FEM\_ObjectBroker \&theBroker);}\\
The DOF\_Numberer receives the class identifier and database tag of
the GraphNumberer in an ID from the channel, if no GraphNumberer is
associated a $-1$ is received. The DOF\_Numberer will then ask {\em
theBroker} for a GraphNumberer with that class identifier, it sets the
database tag for the GraphNumberer and it then invokes {\em
recvSelf()} on that GraphNumberer. \\  

\noindent {\bf Protected Methods}  \\
\indent {\em AnalysisModel *getAnalysisModelPtr(void) const;} \\
A const member function to return the AnalysisModel object associated with
the DOF\_Numberer, {\em theModel}. \\

\indent {\em GraphNumberer *getGraphNumbererPtr(void) const;} \\
A const member function to return the GraphNumberer object associated with
the DOF\_Numberer, {\em theGraphNumberer}. \\









\pagebreak \subsubsection{PlainNumberer}
%File: ~/OOP/analysis/numberer/PlainNumberer.tex
%What: "@(#) PlainNumberer.tex, revA"

\noindent {\bf Files}   \\
\indent \#include $<\tilde{ }$/analysis/numberer/PlainNumberer.h$>$  \\

\noindent {\bf Class Declaration}  \\
\indent class PlainNumberer: public DOF\_Numberer  \\

\noindent {\bf Class Hierarchy} \\
\indent MovableObject \\
\indent\indent DOF\_Numberer \\
\indent\indent\indent {\bf PlainNumberer} \\

\noindent {\bf Description} \\ 
\indent The PlainNumberer class is a DOF\_Numberer. An object of this
class assigns the equation numbers to thee DOF\_Groups based on the
order in which they are obtained from the DOF\_GroupIter object
obtained from the AnalyisModel. The class is useful for situations
where the SystemOfEqn or Solver objects will impose a renumbering on
the equations, which makes performing a complex numbering of the
degrees-of-freedom a waste of computational effort. \\


\noindent {\bf Class Interface} \\
\indent // Constructor \\
\indent {\em PlainNumberer();}\\ \\
\indent // Destructor \\
\indent {\em virtual~ $\tilde{}$PlainNumberer();}\\  \\
\indent // Public Methods \\
\indent {\em virtual int numberDOF(int lastDOF\_Group = -1);} \\
\indent {\em virtual int numberDOF(ID \&lastDOF\_Groups);} \\
\indent {\em virtual int sendSelf(int commitTag, Channel \&theChannel);}\\ 
\indent {\em virtual int recvSelf(int commitTag, Channel \&theChannel,
FEM\_ObjectBroker \&theBroker);}\\ \\


\noindent {\bf Constructor} \\
\indent {\em PlainNumberer();}\\ 
The integer {\em NUMBERER\_TAG\_PlainNumberer} is passed to the
DOF\_Numberer classes constructor. \\

\noindent {\bf Destructor} \\
\indent {\em $\tilde{ }$PlainNumberer();}\\ 
Does nothing. \\

\noindent {\bf Public Methods }\\
\indent {\em virtual int numberDOF(int lastDOF\_Group = -1);} \\
The PlainNumberer will twice obtain the DOF\_GroupIter from the
AnalysisModel. It iterates twice through the DOF\_Groups first
assigning the dofs with eqn numbers assigned -2 a number and 
then on the next pass the dofs assigned -3. The PlainNumberer then
invokes {\em setID()} on each FE\_Element in the
AnalysisModel. Finally it invokes {\em setEqnNum(numEqn)} on the
AnalyisModel. Returns a positive integer equal to the last equation
number set if successful, a negative number if not; the value of 
which depends on the type of the PlainNumberer. A PlainNumberer will
not use the {\em lastDOF\_Group} integer, if one is supplied a warning
message is printed. \\ 

\indent {\em virtual int numberDOF(ID \&lastDOF\_Groups);} \\
The method is identical to that outlined above. A PlainNumberer will
not use the {\em lastDOF\_Groups} ID, if this method is invoked a warning
message is printed. \\ 


\indent {\em virtual int sendSelf(int commitTag, Channel \&theChannel);}\\ 
Returns $0$. \\

\indent {\em virtual int recvSelf(int commitTag, Channel \&theChannel, FEM\_ObjectBroker \&theBroker);}\\
Returns $0$.









\pagebreak
\section{Modeling Classes}
Modeling classes are used to create the finite element model for a
given problem. The classes provided are subclasses of ModelBuilder. An
analyst will interact with a ModelBuilder object, to create the
Element, Node, Load and Constraint objects that define the 
model. 

\pagebreak
\subsection{{\bf ModelBuilder}}
%File: ~/OOP/modelbuilder/ModelBuilder.tex
%What: "@(#) ModelBuilder.tex, revA"

\noindent {\bf Files}   \\
\indent \#include $<\tilde{ }$/modelbuilder/ModelBuilder.h$>$  \\

\noindent {\bf Class Declaration}  \\
\indent class ModelBuilder;  \\

\noindent {\bf Class Hierarchy} \\
\indent {\bf ModelBuilder} \\

\noindent {\bf Description} \\ 
\indent The ModelBuilder class is an abstract base class. A subclass
of ModelBuilder is a class which creates the finite element
discretization of a structure: that is it discretizes the structure to
be modeled into Elements, Nodes, Constraints, etc. and adds these
components to the Domain.  \\

\noindent {\bf Class Interface} \\
\indent // Constructor \\ 
\indent {\em ModelBuilder(theDomain \&theDomain);}\\ \\
\indent // Destructor \\ 
\indent {\em virtual $\tilde{ }$ModelBuilder();}\\  \\
\indent // Public Methods \\ 
\indent {\em virtual buildFE\_Model(void) = 0;} \\ \\
\indent // Protected Methods \\
\indent {\em  Domain *getDomainPtr(void) const;} \\

\noindent {\bf Constructor} \\ 
\indent {\em ModelBuilder(theDomain \&theDomain);}\\ 
All models are associated with a single domain, this constructor
sets up the link between the model and the domain, setting its link
to the Domain object {\em theDomain}. \\

\noindent {\bf Destructor} \\
\indent {\em virtual~ $\tilde{}$ModelBuilder();}\\ 
Does nothing. \\

\noindent {\bf Public Methods} \\
\indent {\em virtual buildFE\_Model(void) = 0;} \\
The ModelBuilder will construct the Element, Node, Load and Constraint
objects and add them to the Domain object associated with the ModelBuilder. \\

\noindent {\bf Protected Methods} \\
\indent {\em  Domain *getDomainPtr(void) const;} \\
Returns a pointer to the Domain object passed in the constructor. This
method can be used in the subclasses to get a pointer the Domain object
to which to add the domain components. \\




\pagebreak
\subsubsection{PlaneFrame}
%File: ~/OOP/modelbuilder/PlaneFrame.tex
%What: "@(#) PlaneFrame.tex, revA"

\noindent {\bf Files}   \\
\indent \#include $<$/modelbuilder/PlaneFrame.h$>$  \\

\noindent {\bf Class Declaration}  \\
\indent class PlaneFrame: public ModelBuilder;  \\

\noindent {\bf Class Hierarchy} \\
\indent ModelBuilder \\
\indent\indent {\bf PlaneFrame} \\

\noindent {\bf Description} \\ 
\indent The PlaneFrame class is used to construct 2d plane frame
models from an input file of specified format. \\

\noindent {\bf Class Interface} \\ 
\indent // Constructor \\ 
\indent {\em PlaneFrame(theDomain \&theDomain);}\\ \\
\indent // Destructor \\
\indent {\em virtual~ $\tilde{}$PlaneFrame();}\\ \\
\indent // Public Methods \\
\indent {\em virtual buildFE\_Model(void) = 0;} \\


\noindent {\bf Constructor} \\ 
\indent {\em PlaneFrame(theDomain \&theDomain);}\\ 
The {\em theDomain} object is used by the ModelBuilder classes 
constructor. \\ 

\noindent {\bf Destructor} \\
\indent {\em virtual~ $\tilde{}$PlaneFrame();}\\ 
This does nothing. It is the responsibility of the Domain object to delete
all the domain components when the destructor is called on that object. \\

\noindent {\bf Public Methods} \\
\indent {\em virtual buildFE\_Model(void) = 0;} \\
The PlaneFrame will construct the Element, Node, Load and Constraint
objects and add them to the Domain object associated with the PlaneFrame. 
To do this the PlaneFrame object will prompt the user for the name of 
the input file; if the file cannot be opened an error message is
printed and the program terminates. A sample input file is given below: \\

\indent\indent 4 3 3 1 2\\
\indent\indent 1 3  0.0  0.0 \\
\indent\indent 2 3  0.0 10.0 \\
\indent\indent 3 3 20.0 10.0 \\
\indent\indent 4 3 20.0  0.0 \\
\indent\indent 2 1 10.0 200000.0 100.0 1 2 \\
\indent\indent 3 2 20.0 200000.0 100.0 2 3 \\
\indent\indent 2 3 10.0 200000.0 100.0 3 4 \\
\indent\indent 1 0 0.0 \\
\indent\indent 1 1 0.0 \\
\indent\indent 1 2 0.0 \\
\indent\indent 3 4 2 3 \\
\indent\indent 0 1 \\
\indent\indent 0 1 2 \\
\indent\indent 1.0 1.0 0.0 \\
\indent\indent 1.0 0.0 2.0 \\
\indent\indent 2 10.0 0.0 0.0 \\
\indent\indent 3 10.0 10.0 0.0 \\


\indent line 1 contains the number of nodes (4), elements (3), single
point constraints (3), multiple point constraints(1) and number of
nodal loads respectively (2). The next 4 lines contains the nodal
data; for each line a Node object is constructed using the tag and,
the number of dof, and the x and y coords specified. The next 3 lines
contains the elemental data; for each line a beam object is created,
the type of beam object depending on the first integer value of each
line (2 == beam2d02, 3 == beam2d03), each beam takes a tag, A,E,I and
the tags of the nodes at end1 and end2. 

The next 3 lines contains the single point
constraint data; for each line a SP\_Constraint object is created
(constrainedNode, constrainedDOF, value). 

The next 5 lines contain the data for the multipoint constraint, the
first line identifies the tag of the constrained node(3), the tag of
the retained node(4) and the number of dof in the relationship for
each node (2 for node 3 and 3 for node 4). The next line contains the
degrees-of-freedom in the constrained node $U_c$, (0 1), the next lines the
degrees-of-freedom in the retained node, $U_r$, (0 1 2) and the next
two lines the $C_{cr}$ matrix defining the relationship $U_c = C_{cr} U_r$.

whose tag is 5 which has two load cases. The next two lines contain
the NodalLoad information; for each line a NodalLoad is created which
acts on the node specified with the forces xForce, yForce and
moment. To returns 0 if successful, -1 otherwise. \\







%\pagebreak
%\subsection{{\bf PartitionedModelBuilder}}
%%File: ~/OOP/modelbuilder/PartitionedModelBuilder.tex
%What: "@(#) PartitionedModelBuilder.tex, revA"

\noindent {\bf Files}   \\
\indent \#include $<\tilde{ }$/modelbuilder/PartitionedModelBuilder.h$>$  \\

\noindent {\bf Class Declaration}  \\
\indent class PartitionedModelBuilder: public ModelBuilder, public MovableObject  \\

\noindent {\bf Class Hierarchy} \\
\indent  PartitionedModelBuilder \\
\indent  MovableObject \\
\indent\indent {\bf PartitionedModelBuilder} \\

\noindent {\bf Description} \\ 
\indent The PartitionedModelBuilder class is an abstract class. A subclass
of PartitionedModelBuilder is a class which creates a partitioned finite element
discretization of a structure: that is it discretizes the structure to
be modeled into Elements, Nodes, Constraints, etc. and adds these
components to the Subdomains of a PartitionedDomain.
PartitionedModelBuilders can be used for creating models for analysis
involving domain decomposition methods where there exist natural
partitions or where a model has previously been partitioned and this
information has been saved. \\

\noindent {\bf Class Interface} \\
\indent\indent // Constructors \\ 
\indent\indent {\em PartitionedModelBuilder(PartitionedDomain
\&theDomain, int classTag);}\\ 
\indent\indent {\em PartitionedModelBuilder(Subdomain \&aSubdomain, int
classTag);} \\ \\
\indent\indent {\em PartitionedModelBuilder(PartitionedDomain
\&theDomain, int classTag);}\\ \\
\indent\indent // Destructor \\ 
\indent\indent {\em virtual $\tilde{ }$PartitionedModelBuilder();}\\  \\
\indent\indent // Public Methods \\ 
\indent\indent {\em virtual buildFE\_Model(void);} \\
\indent\indent {\em virtual int buildInterface(int numSubdomains) =0;} \\ 
\indent\indent {\em virtual int buildSubdomain(int partition, int
numPartitions, Subdomain \&theSubdomain) =0;} \\ \\
\indent\indent // Protected Methods \\ 
\indent\indent {\em  PartitionedDomain *getPartitionedDomainPtr(void)
const;} \\

\noindent {\bf Constructors} \\ 
\indent {\em PartitionedModelBuilder(PartitionedDomain \&domain, int
classTag);}\\  
Typically, a PartitionedModelBuilder is associated with a PartitionedDomain,
this constructor sets up a link for the PartitionedModelBuilder and the domain,
setting its link to the Domain object {\em domain}. The Domain {\em
domain} is passed to the constructor for ModelBuilder, and the integer
{\em classTag} is passed to the MovableObject classes constructor. \\

\indent {\em PartitionedModelBuilder(Subomain \&domain, int classTag);}\\  
This is the constructor that is called when a PartitionedModelBuilder
is to be created by an FE\_ObjectBroker. The only method that can be
invoked on such an object is {\em buildSubdomain()}. \\

\noindent {\bf Destructor} \\
\indent {\em virtual~ $\tilde{}$PartitionedModelBuilder();}\\ 
Does nothing. \\


\noindent {\bf Public Methods} \\
\indent {\em virtual buildFE\_Model(void);} \\
The PartitionedModelBuilder will first check that the
PartitionedModelBuilder was constructed with a PartitioneDomain, if
not a warning message is printed and a $-1$ is returned. If o.k. the object 
then determines the number of Subdomains, {\em numSub} in the
PartitionedDomain. It then invokes {\em buildInterface(numSub)} on 
itself. Then for each Subdomain in the PartitionedDomain it invokes
{\em buildSubdomain(numSub, *this)}. If building the interface or any
of the subdomains fails, a warning message is printed and a negative
number is returned. Returns $0$ if successful. \\

\indent {\em virtual int buildInterface(int numSubdomains) =0;}\\ 
This method must be provided by the subclasses. It is used to add any
boundary nodes, nodal loads and constraints to the PartitionedDomain
object. The integer {\em numSubdomains} is passed to provide
information about the number of subdomains in the
PartitionedDomain. To return $0$ if successful, a negative number if not. \\


\indent {\em virtual int buildSubdomain(int partition, int
numPartitions, Subdomain \&theSubdomain) =0;} \\
This method must be provided by the subclasses. It is used to add
nodes, elements, loads and constraints to the subdomain {\em
theSubdomain}. The integers {\em partition} and {\em numPartitions}
are used to identify which partition is being built and the total
number of partitions. To return $0$ if successful, a negative number
if not. \\


\noindent {\bf Protected Methods} \\
\indent {\em  PartitionedDomain *getPartitionedDomainPtr(void) const;} \\
Returns a pointer to the PartitionedDomain object passed in the constructor.\\









\pagebreak
\section{Numerical Classes}
Numerical classes are used to pass numerical information between
objects and to handle the numerical operations in the solution
procedure. The classes provided include Matrix, Vector, ID,
SystemOfEquations, Solver and subclasses of these classes.

\pagebreak
\subsection{{\bf SystemOfEqn}}
%File: ~/OOP/system_of_eqn/SystemOfEqn.tex
%What: "@(#) SystemOfEqn.tex, revA"

\noindent {\bf Files}   \\
\indent \#include $<\tilde{ }$/system\_of\_eqn/SystemOfEqn.h$>$  \\

\noindent {\bf Class Declaration}  \\
\indent class SystemOfEqn:  public MovableObject \\

\noindent {\bf Class Hierarchy} \\
\indent MovableObject \\
\indent\indent {\bf SystemOfEqn} \\
\indent\indent\indent LinearSOE \\
\indent\indent\indent EigenSOE \\

\noindent {\bf Description}  \\
\indent SystemOfEqn is an abstract class. A SystemOfEqn object
is responsible for storing the system of equations it represents.
A Solver object, which is associated with the SystemOfEqn object, is
responsible for performing the numerical operations to solve for the
system of equations. \\ 

\noindent {\bf Class Interface} \\
\indent\indent {// Constructors}  \\ 
\indent\indent {\em SystemOfEqn(int classTag);}  \\ \\
\indent\indent {// Destructor}  \\ 
\indent\indent {\em virtual~ $\tilde{}$SystemOfEqn();}\\  \\
\indent\indent {// Public Methods}  \\ 
\indent\indent {\em virtual int solve(void) =0;} \\


\noindent {\bf Constructor}  \\
\indent {\em SystemOfEqn(int classTag);}  \\
The integer {\em classTag} is provided to the constructor for the
MovableObject.  \\

\noindent {\bf Destructor} \\
\indent {\em virtual~ $\tilde{}$SystemOfEqn();}\\ 
Does nothing. Declared to allow the subclass destructor to be called. \\

\noindent {\bf Public Method }  \\
\indent {\em virtual int solve(void) =0;} \\
Invoked to cause the system of equation object to solve itself. To
return $0$ if successful, negative number if not.






\pagebreak
\subsubsection{{\bf LinearSOE}}
%File: ~/OOP/system_of_eqn/linearSOE/LinearSOE.tex
%What: "@(#) LinearSOE.tex, revA"

\noindent {\bf Files}   \\
\indent \#include $<\tilde{ }$/system\_of\_eqn/linearSOE/LinearSOE.h$>$  \\

\noindent {\bf Class Declaration}  \\
\indent class LinearSOE: public SystmOfEqn \\

\noindent {\bf Class Hierarchy} \\
\indent MovableObject \\
\indent\indent SystemOfEqn \\
\indent\indent\indent {\bf LinearSOE} \\

\noindent {\bf Description}  \\
\indent LinearSOE is an abstract class. A LinearSOE object provides an
abstraction of a system of linear equations. A linear system of
equation of order $n$:

$$ 
\begin{array}{ccccccccc}
a_{0,0}x_0 & + & a_{0,1}x_1  & + & ... & + & a_{0,n-1}x_{n-1} & = & b_0  \\
a_{1,0}x_0 & + & a_{1,1}x_1  & + & ... & + & a_{1,n-1}x_{n-1} & = & b_1 \\
 ...  &  & ...  &  & & &  ... & & ... \\
a_{n-1,0}x_0 & + & a_{n-1,1}x_1  & + & ... & + & a_{n-1,n-1}x_{n-1} &
= & b_{n-1} \\ 
\end{array}
$$

\noindent can be expressed by the matrix equation $Ax=b$, where $A$ is a matrix
of order $n$ X $n$ and $b$ and $x$ are vectors or order $n$. A
LinearSOE object is responsible for storing these equations and for
providing methods at the interface to set up and obtain the
equations. Each LinearSOE object will be associated with a
LinearSOESolver object. It is the LinearSOESolver objects that is
responsible for solving the linear system of equations. \\


\noindent {\bf Class Interface} \\
\indent\indent {// Constructors}  \\ 
\indent\indent {\em LinearSOE(LinearSOESolver \&theSolver, int classTag);}  \\ \\
\indent\indent {// Destructor}  \\ 
\indent\indent {\em virtual $\tilde{ }$LinearSOE();}\\ \\
\indent\indent {// Public Methods}  \\ 
\indent\indent {\em virtual int solve(void);} \\
\indent\indent {\em virtual int setSize(const Graph \&theGraph) =0; } \\
\indent\indent {\em virtual int getNumEqn(void) =0; } \\

\indent\indent {\em virtual int addA(const Matrix \&theMatrix, const ID \& loc,
doublefact = 1.0) =0;} \\
\indent\indent {\em virtual int addB(const Vector \& theVector, const ID \& loc,
double fact = 1.0) =0;} \\
\indent\indent {\em virtual int setB(const Vector \& theVector, double
fact = 1.0) =0;} \\ 
\indent\indent {\em virtual void zeroA(void) =0;} \\
\indent\indent {\em virtual void zeroB(void) =0;} \\
\indent\indent {\em virtual const Vector \&getX(void) = 0;} \\
\indent\indent {\em virtual const Vector \&getB(void) = 0;} \\
\indent\indent {\em virtual double normRHS(void) =0;} \\
\indent\indent {\em virtual void setX(int loc, double value) =0;}\\\\
\indent\indent {// Protected Methods}  \\ 
\indent\indent {\em virtual int setSolver(LinearSOESolver
\&newSolver);} \\
\indent\indent {\em virtual LinearSOESolver *getSolver(void);}\\

\noindent {\bf Constructors}  \\
\indent {\em LinearSOE(LinearSOESolver \&theSolver, int classTag);}  \\
The integer {\em classTag} is passed to the constructor for the
SystemOfEqn. The constructor sets sets the pointer for the currently
associated LinearSOESolver object to point to {\em theSolver}. \\

\noindent {\bf Destructor} \\
\indent {\em virtual $\tilde{ }$LinearSOE();}\\ 
Does nothing. \\

\noindent {\bf Public Methods }  \\
\indent {\em virtual int solve(void);} \\
Causes the SystemOfEqn object to invoke {\em solve()} on the currently
associated LinearSOESolver object. Returns a $0$ if successful,
negative number if not; the actual value depending on the
LinearSOESolver. To solve a linear system of equations means to find
$x$ such that the equation $Ax=b$ is satisfied. \\

\indent {\em virtual int getNumEqn(void) =0; } \\
A method which returns the number of equations in the system, i.e. the number
of unknowns. \\  

\indent {\em virtual int setSize(const Graph \&G) =0; } \\
Invoked to allow the LinearSOE object to determine the size and sparsity of 
the matrix $A$ and vectors $x$ and $b$. This information can be
deduced from the number of vertices and the connectivity between the
vertices in the Graph object {\em G}. To return $0$ if
successful, a negative number if not. \\  

\indent {\em virtual int addA(const Matrix \&M, const ID \& loc,
doublefact = 1.0) =0;} \\
The LinearSOE object assembles {\em fact} times the Matrix {\em
M} into the matrix $A$. The Matrix is assembled into $A$ at the
locations given by the ID object {\em loc}, i.e. $a_{loc(i),loc(j)} +=
M(i,j)$. Numbering in $A$ starts at $(0,0)$, i.e. C style. If a
location specified is outside the range, i.e. $(-1,-1)$ the
corresponding entry in {\em M} is not added to $A$. To return $0$ if
successful, a negative number if not. \\ 

{\em virtual int addB(const Vector \& theVector, const ID \& loc,
double fact = 1.0) =0;} \\
The LinearSOE object assembles {\em fact} times the Vector {\em V} into
the vector $b$. The Vector is assembled into $b$ at the locations
given by the ID object {\em loc}, i.e. $b_{loc(i)} += V(i)$. If a
location specified is outside the range, e.g. $-1$, the corresponding
entry in {\em V} is not added to $b$. To return $0$ if successful, a
negative number if not.  \\ 

{\em virtual int setB(const Vector \& theVector,
double fact = 1.0) =0;} \\
The LinearSOE object sets the vector {\em b} to be {\em fact} times
the Vector {\em V}. To return $0$ if successful, a negative number if
not.  \\  

{\em virtual void zeroA(void) =0;} \\
To zero the matrix $A$, i.e. set all the components of $A$ to $0$. \\

{\em virtual void zeroB(void) =0;} \\
To zero the vector $b$, i.e. set all the components of $b$ to $0$. \\

{\em virtual const Vector \&getX(void) = 0;} \\
To return, as a Vector object, the vector $x$. A const reference is
returned, meaning the Vector that is returned cannot be modified, i.e.
no non-const method can be invoked on the Vector. \\

{\em virtual const Vector \&getB(void) = 0;} \\
To return as a Vector object the vector $b$. A const reference is
returned, meaning the Vector that is returned cannot be modified, i.e.
no non-const method can be invoked on the Vector. \\


{\em virtual double normRHS(void) =0;} \\
To return the 2-norm of the vector $x$. \\

{\em virtual void setX(int loc, double value) =0;}\\
The LinearSOE object is responsible for setting $x(loc) = value$. This
is needed in domain decomposition methods and could be useful in
iterative solution strategies when an initial approximation is known.\\


\indent {\em int setSolver(LinearSOESolver \&newSolver);}\\
This is invoked to set the currently associated LinearSOESolver object to 
be {\em newSolver}. Each subclass will provide it's own variation of {\em
setSolver()} method (needed so subclasses can verify type of Solver
object passed). the subclasses in their variation of the {\em setSolver()}
method (unless they wish to implement their own {\em solve()}
method) invoke this method. Returns $0$.  \\

\indent {\em     LinearSOESolver *getSolver(void);}\\
Returns a pointer to the associated LinearSOESolver object.







\pagebreak
\subsubsection{FullGenLinSOE}
%File: ~/OOP/system_of_eqn/linearSOE/FullGen/FullGenLinSOE.tex
%What: "@(#) FullGenLinSOE.tex, revA"

\noindent {\bf Files}   \\
\indent \#include $<\tilde{ }$/system\_of\_eqn/linearSOE/fullGEN/FullGenLinSOE.h$>$  \\

\noindent {\bf Class Declaration}  \\
\indent class FullGenLinSOE: public LinearSOE \\

\noindent {\bf Class Hierarchy} \\
\indent MovableObject \\
\indent\indent SystemOfEqn \\
\indent\indent\indent LinearSOE \\
\indent\indent\indent\indent {\bf FullGenLinSOE} \\

\noindent {\bf Description}  \\
\indent FullGenLinSOE is class which is used to store a full general
system. The $A$ matrix is stored in a 1d double array with $n*n$
elements, where $n$ is the size of the system. $A_{i,j}$ is stored at
location $(i + j*(n)$, where $i$ and $j$ range from $0$ to $n-1$,
i.e. C notation. For example when $n=3$: 

$$
\left[
\begin{array}{ccc}
a_{0,0} & a_{0,1}  & a_{0,2}  \\
a_{1,0} & a_{1,1} & a_{1,2}  \\
a_{2,0} & a_{2,1} & a_{2,2} \\
\end{array}
\right] 
$$

is stored in:

$$
\left[
\begin{array}{cccccccccccccccccccc}
a_{0,0} & a_{1,0}  & a_{2,0} & a_{0,1} & a_{1,1} & a_{2,1} &
a_{0,2} & a_{1,2} & a_{2,2}  \\
\end{array}
\right] 
$$

The $x$ and $b$ vectors are stored in 1d double arrays of length
$n$. To allow the solvers access to this data, the solvers that use
this class are all declared as friend classes. \\

\noindent {\bf Interface}  \\
\indent\indent {// Constructors} \\
\indent\indent {\em FullGenLinSOE(FullGenLinSolver \&theSolver);}  \\
\indent\indent {\em FullGenLinSOE(int N, FullGenLinSolver \&theSolver); }\\\\
\indent\indent {// Destructor} \\
\indent\indent {\em  $\tilde{}$FullGenLinSOE();}\\ \\
\indent\indent {// Public Methods }  \\
\indent\indent {\em  int setFullGenSolver(FullGenLinSolver \&newSolver);}\\
\indent\indent {\em int setSize(const Graph \&theGraph) =0; } \\
\indent\indent {\em int getNumEqn(void) =0; } \\
\indent\indent {\em int addA(const Matrix \&theMatrix, const ID \& loc,
doublefact = 1.0) =0;} \\
\indent\indent {\em int addB(const Vector \& theVector, const ID \& loc,
double fact = 1.0) =0;} \\
\indent\indent {\em int setB(const Vector \& theVector, 
double fact = 1.0) =0;} \\
\indent\indent {\em void zeroA(void) =0;} \\
\indent\indent {\em void zeroB(void) =0;} \\
\indent\indent {\em const Vector \&getX(void) = 0;} \\
\indent\indent {\em const Vector \&getB(void) = 0;} \\
\indent\indent {\em double normRHS(void) =0;} \\
\indent\indent {\em void setX(int loc, double value) =0;}\\
\indent\indent {\em int sendSelf(int commitTag, Channel \&theChannel);}\\ 
\indent\indent {\em int recvSelf(int commitTag, Channel \&theChannel,
FEM\_ObjectBroker \&theBroker);}\\ 



\noindent {\bf Constructors}  \\
\indent {\em FullGenLinSOE(FullGenLinSolver \&solver);}  \\
The {\em solver} and a unique class tag (defined in $<$classTags.h$>$)
are passed to the LinearSOE constructor. The system size is set
to $0$ and the matrix $A$ is marked as not having been factored. Invokes
{\em setLinearSOE(*this)} on the {\em solver}. No memory is
allocated for the 3 1d arrays. \\  


{\em FullGenLinSOE(int N, FullGenLinSolver \&solver); }\\
The {\em solver} and a unique class tag (defined in $<$classTags.h$>$)
are passed to the LinearSOE constructor. The system size is set
to $N$ and the matrix $A$ is marked as not having been
factored. Obtains memory from the heap for the 1d arrays storing the
data for $A$, $x$ and $b$ and stores the size of these arrays. If not
enough memory is available for these arrays a warning message is
printed and the system size is set to $0$. Invokes {\em
setLinearSOE(*this)} and {\em setSize()} on {\em solver},
printing a warning message if {\em setSize()} returns a negative
number. Also creates Vector objects for $x$ and $b$ using the {\em
(double *,int)} Vector constructor. \\

\noindent {\bf Destructor} \\
\indent {\em  $\tilde{}$FullGenLinSOE();}\\ 
Calls delete on any arrays created. \\

\noindent {\bf Public Methods} \\
\indent {\em  int setFullGEnSolver(FullGenLinSolver \&newSolver);}\\
Invokes {\em setLinearSOE(*this)} on {\em newSolver}.
If the system size is not equal to $0$, it also invokes {\em setSize()}
on {\em newSolver}, printing a warning and returning $-1$ if this
method returns a number less than $0$. Finally it returns the result
of invoking the LinearSOE classes {\em setSolver()} method. \\

\indent {\em int getNumEqn(void) =0; } \\
A method which returns the current size of the system. \\

\indent {\em  int setSize(const Graph \&theGraph); } \\ 
The size of the system is determined by invoking {\em getNumVertex()}
on {\em theGraph}. If the old space allocated for the 1d arrays is not
big enough, it the old space is returned to the heap and new space is
allocated from the heap. Prints a warning message, sets size to $0$
and returns a $-1$, if not enough memory is available on the heap for the 
1d arrays. If memory is available, the components of the arrays are
zeroed and $A$ is marked as being unfactored. If the system size has
increased, new Vector objects for $x$ and $b$ using the {\em (double
*,int)} Vector constructor are created. Finally, the result of
invoking {\em setSize()} on the associated Solver object is returned. \\


\indent {\em int addA(const Matrix \&M, const ID \& loc,
doublefact = 1.0) =0;} \\
First tests that {\em loc} and {\em M} are of compatible sizes; if not
a warning message is printed and a $-1$ is returned. The LinearSOE
object then assembles {\em fact} times the Matrix {\em 
M} into the matrix $A$. The Matrix is assembled into $A$ at the
locations given by the ID object {\em loc}, i.e. $a_{loc(i),loc(j)} +=
fact * M(i,j)$. If the location specified is outside the range,
i.e. $(-1,-1)$ the corresponding entry in {\em M} is not added to
$A$. If {\em fact} is equal to $0.0$ or $1.0$, more efficient steps
are performed. Returns $0$.  \\


{\em int addB(const Vector \& V, const ID \& loc,
double fact = 1.0) =0;} \\
First tests that {\em loc} and {\em V} are of compatible sizes; if not
a warning message is printed and a $-1$ is returned. The LinearSOE
object then assembles {\em fact} times the Vector {\em V} into
the vector $b$. The Vector is assembled into $b$ at the locations
given by the ID object {\em loc}, i.e. $b_{loc(i)} += fact * V(i)$. If a
location specified is outside the range, e.g. $-1$, the corresponding
entry in {\em V} is not added to $b$. If {\em fact} is equal to $0.0$,
$1.0$ or $-1.0$, more efficient steps are performed. Returns $0$. \\


{\em int setB(const Vector \& V, double fact = 1.0) =0;} \\
First tests that {\em V} and the size of the system are of compatible
sizes; if not a warning message is printed and a $-1$ is returned. The
LinearSOE object then sets the vector {\em b} to be {\em fact} times
the Vector {\em V}. If {\em fact} is equal to $0.0$, $1.0$ or $-1.0$,
more efficient steps are performed. Returns $0$. \\ 

{\em void zeroA(void) =0;} \\
Zeros the entries in the 1d array for $A$ and marks the system as not
having been factored. \\

{\em void zeroB(void) =0;} \\
Zeros the entries in the 1d array for $b$. \\

{\em const Vector \&getX(void) = 0;} \\
Returns the Vector object created for $x$. \\

{\em const Vector \&getB(void) = 0;} \\
Returns the Vector object created for $b$. \\

{\em double normRHS(void) =0;} \\
Returns the 2-norm of the vector $x$. \\

{\em void setX(int loc, double value) =0;}\\
If {\em loc} is within the range of $x$, sets $x(loc) = value$. \\

\indent {\em int sendSelf(int commitTag, Channel \&theChannel);}\\ 
Returns $0$. The object does not send any data or connectivity
information as this is not needed in the finite element design. \\

\indent {\em int recvSelf(int commitTag, Channel \&theChannel, FEM\_ObjectBroker
\&theBroker);}\\ 
Returns $0$. The object does not receive any data or connectivity
information as this is not needed in the finite element design.



\pagebreak
\subsubsection{BandGenLinSOE}
%File: ~/OOP/system_of_eqn/linearSOE/bandGEN/BandGenLinSOE.tex
%What: "@(#) BandGenLinSOE.tex, revA"

\noindent {\bf Files}   \\
\indent \#include $<\tilde{ }$/system\_of\_eqn/linearSOE/bandGEN/BandGenLinSOE.h$>$  \\

\noindent {\bf Class Declaration}  \\
\indent class BandGenLinSOE: public LinearSOE \\

\noindent {\bf Class Hierarchy} \\
\indent MovableObject \\
\indent\indent SystemOfEqn \\
\indent\indent\indent LinearSOE \\
\indent\indent\indent\indent {\bf BandGenLinSOE} \\

\noindent {\bf Description}  \\
\indent BandGenLinSOE is class which is used to store a banded
unsymmetric system with kl subdiagonals and ku superdiagonals. The $A$
matrix is stored in a 1d double array with (kl+ku+1)*n elements, where
n is the size of the system. $A_{i,j}$ is stored at location
$(ku+1+i-j) + j*(ku+1+kl)$, where $i$ and $j$ range from $0$ to$n-1$, i.e. C
notation. For example when $n=5$, $kl = 2$ and $ku=1$: 

$$
\left[
\begin{array}{ccccc}
a_{0,0} & a_{0,1}  & 0 & 0 & 0 \\
a_{1,0} & a_{1,1} & a_{1,2} & 0 & 0 \\
a_{2,0} & a_{2,1} & a_{2,2} & a_{2,3} & 0  \\
0 & a_{3,1} & a_{3,2} & a_{3,3} & a_{3,4} \\
0 & 0 & a_{4,2} & a_{4,3} & a_{4,4} \\
\end{array}
\right] 
$$

is stored in:

$$
\left[
\begin{array}{ccccccccccccccccccccc}
* & a_{0,0} & a_{1,0}  & a_{2,0} & a_{0,1} & a_{1,1} & a_{2,1} &
a_{3,1} & a_{1,2} & a_{2,2} & a_{3,2} & a_{4,2} & a_{2,3} & a_{3,3} &
a_{4,3} & * & a_{3,4} & a_{4,4} & * & * \\
\end{array}
\right] 
$$

The $x$ and $b$ vectors are stored in 1d double arrays of length
$N$. To allow the solvers access to this data, the solvers that use
this class are all declared as friend classes. \\ 


\noindent {\bf Interface}  \\
\indent\indent {// Constructors} \\
\indent\indent {\em BandGenLinSOE(BandGenLinSolver \&theSolver);}  \\
\indent\indent {\em BandGenLinSOE(int N, int numSuperDiagonals, int numSubDiagonal,
		  BandGenLinSolver \&theSolver);        }\\ \\
\indent\indent {// Destructor} \\
\indent\indent {\em  $\tilde{}$BandGenLinSOE();}\\ \\
\indent\indent {// Public Methods }  \\
\indent\indent {\em  int setBandGenSolver(BandGenLinSolver \&newSolver);}\\
\indent\indent {\em int setSize(const Graph \&theGraph) =0; } \\
\indent\indent {\em int getNumEqn(void) =0; } \\
\indent\indent {\em int addA(const Matrix \&theMatrix, const ID \& loc,
doublefact = 1.0) =0;} \\
\indent\indent {\em int addB(const Vector \& theVector, const ID \& loc,
double fact = 1.0) =0;} \\
\indent\indent {\em int setB(const Vector \& theVector, 
double fact = 1.0) =0;} \\
\indent\indent {\em void zeroA(void) =0;} \\
\indent\indent {\em void zeroB(void) =0;} \\
\indent\indent {\em const Vector \&getX(void) = 0;} \\
\indent\indent {\em const Vector \&getB(void) = 0;} \\
\indent\indent {\em double normRHS(void) =0;} \\
\indent\indent {\em void setX(int loc, double value) =0;}\\
\indent\indent {\em int sendSelf(int commitTag, Channel \&theChannel);}\\ 
\indent\indent {\em int recvSelf(int commitTag, Channel \&theChannel,
FEM\_ObjectBroker \&theBroker);}\\ 


\noindent {\bf Constructors}  \\
\indent {\em BandGenLinSOE(BandGenLinSolver \&theSolver);}  \\
The {\em solver} and a unique class tag (defined in $<$classTags.h$>$)
are passed to the LinearSOE constructor. The system size is set
to $0$ and the matrix $A$ is marked as not having been factored. Invokes
{\em setLinearSOE(*this)} on the {\em theSolver}. No memory is
allocated for the 3 1d arrays. \\  


{\em BandGenLinSOE(int N, int numSuperDiagonals, int numSubDiagonal,
		  BandGenLinSolver \&theSolver);        }\\
The {\em solver} and a unique class tag (defined in $<$classTags.h$>$)
are passed to the LinearSOE constructor. 
Sets the size of the system to $N$, the number of superdiagonals to 
{\em numSuperDiagonals} and number of subdiagonals to {\em
numSubDiagonals}. Allocates memory for the arrays; if not enough
memory is available a warning message is printed and the system size
is set to $0$. Sets the solver to be called when solving the
equations to {\em theSolver}. Invokes {\em setLinearSOE(*this)} and
{\em setSize()} on the {\em theSolver}. Also creates Vector objects
for $x$ and $b$ using the {\em (double *,int)} Vector constructor. \\


\noindent {\bf Destructor} \\
\indent {\em $\tilde{ }$BandGenLinSOE();}\\ 
Calls delete on any arrays created. \\

\noindent {\bf Public Methods}  \\
\indent {\em  int setBandGenSolver(BandGenLinSolver \&newSolver);}\\
Invokes {\em setLinearSOE(*this)} on {\em newSolver}.
If the system size is not equal to $0$, it also invokes {\em setSize()}
on {\em newSolver}, printing a warning and returning the returned value if this
method returns a number less than $0$. Finally it returns the result
of invoking the LinearSOE classes {\em setSolver()} method. \\

\indent {\em int getNumEqn(void) =0; } \\
A method which returns the current size of the system. \\

\indent {\em  int setSize(const Graph \&G); } \\ 
The size of the system is determined by looking at the adjacency ID of
each Vertex in the Graph object {\em G}. This is done by first setting
{\em kl} and {\em ku} equal to $0$ and then checking for each Vertex
in {\em G}, whether any of the vertex tags in the Vertices adjacency
ID results in {\em kl} or {\em ku} being increased. Knowing {\em kl},
{\em ku} and the size of the system (the number of Vertices in {\em G}),
a check to see if the previously allocated 1d arrays for $A$, $x$ and
$b$ are large enough. If the memory portions allocated for the 1d
arrays are not big enough, the old space is returned to the heap and
new space is allocated from the heap. Prints a warning message if
not enough memory is available on the heap for the 1d arrays and
returns a $-1$. If memory is available, the components of the arrays
are zeroed and $A$ is marked as being unfactored. If the system size
has increased, new Vector objects for $x$ and $b$ using the {\em (double
*,int)} Vector constructor are created. Finally, the result of
invoking {\em setSize()} on the associated Solver object is returned. \\


\indent {\em int addA(const Matrix \&M, const ID \& loc,
doublefact = 1.0) =0;} \\
First tests that {\em loc} and {\em M} are of compatible sizes; if not
a warning message is printed and a $-1$ is returned. The LinearSOE
object then assembles {\em fact} times the Matrix {\em 
M} into the matrix $A$. The Matrix is assembled into $A$ at the
locations given by the ID object {\em loc}, i.e. $a_{loc(i),loc(j)} +=
fact * M(i,j)$. If the location specified is outside the range,
i.e. $(-1,-1)$ the corresponding entry in {\em M} is not added to
$A$. If {\em fact} is equal to $0.0$ or $1.0$, more efficient steps
are performed. Returns $0$.  \\


{\em int addB(const Vector \& V, const ID \& loc,
double fact = 1.0) =0;} \\
First tests that {\em loc} and {\em V} are of compatible sizes; if not
a warning message is printed and a $-1$ is returned. The LinearSOE
object then assembles {\em fact} times the Vector {\em V} into
the vector $b$. The Vector is assembled into $b$ at the locations
given by the ID object {\em loc}, i.e. $b_{loc(i)} += fact * V(i)$. If a
location specified is outside the range, e.g. $-1$, the corresponding
entry in {\em V} is not added to $b$. If {\em fact} is equal to $0.0$,
$1.0$ or $-1.0$, more efficient steps are performed. Returns $0$. \\

{\em int setB(const Vector \& V, double fact = 1.0) =0;} \\
First tests that {\em V} and the size of the system are of compatible
sizes; if not a warning message is printed and a $-1$ is returned. The
LinearSOE object then sets the vector {\em b} to be {\em fact} times
the Vector {\em V}. If {\em fact} is equal to $0.0$, $1.0$ or $-1.0$,
more efficient steps are performed. Returns $0$. \\ 

{\em void zeroA(void) =0;} \\
Zeros the entries in the 1d array for $A$ and marks the system as not
having been factored. \\

{\em void zeroB(void) =0;} \\
Zeros the entries in the 1d array for $b$. \\

{\em const Vector \&getX(void) = 0;} \\
Returns the Vector object created for $x$. \\

{\em const Vector \&getB(void) = 0;} \\
Returns the Vector object created for $b$. \\

{\em double normRHS(void) =0;} \\
Returns the 2-norm of the vector $x$. \\

{\em void setX(int loc, double value) =0;}\\
If {\em loc} is within the range of $x$, sets $x(loc) = value$. \\

\indent {\em int sendSelf(int commitTag, Channel \&theChannel);}\\ 
Returns $0$. The object does not send any data or connectivity
information as this is not needed in the finite element design. \\

\indent {\em int recvSelf(int commitTag, Channel \&theChannel, FEM\_ObjectBroker
\&theBroker);}\\ 
Returns $0$. The object does not receive any data or connectivity
information as this is not needed in the finite element design.













\pagebreak
\subsubsection{BandSPDLinSOE}
%File: ~/OOP/system_of_eqn/linearSOE/bandSPD/BandSPDLinSOE.tex
%What: "@(#) BandSPDLinSOE.tex, revA"

\noindent {\bf Files}   \\
\indent \#include $<\tilde{ }$/system\_of\_eqn/linearSOE/bandSPD/BandSPDLinSOE.h$>$  \\

\noindent {\bf Class Declaration}  \\
\indent class BandSPDLinSOE: public LinearSOE \\

\noindent {\bf Class Hierarchy} \\
\indent MovableObject \\
\indent\indent SystemOfEqn \\
\indent\indent\indent LinearSOE \\
\indent\indent\indent\indent {\bf BandSPDLinSOE} \\

\noindent {\bf Description}  \\
\indent BandSPDLinSOE is class which is used to store a banded symmetric system
with ku superdiagonals. The $A$ matrix is stored in a
1d double array with (ku+1)*n elements, where n is the size of the
system. $A_{i,j}$ is stored at location $(ku+1+i-j) +
j*(ku+1)$, where $i$ and $j$ range from $0$ to$n-1$, i.e. C
notation. For example when $n=5$, $ku = 2$: 

$$
\left[
\begin{array}{ccccc}
a_{0,0} & a_{0,1}  & a_{0,1} & 0 & 0 \\
a_{1,0} & a_{1,1} & a_{1,2} & a_{1,3} & 0 \\
a_{2,0} & a_{2,1} & a_{2,2} & a_{2,3} & a_{2,4}  \\
0 & a_{3,1} & a_{3,2} & a_{3,3} & a_{3,4} \\
0 & 0 & a_{4,2} & a_{4,3} & a_{4,4} \\
\end{array}
\right] 
$$

is stored in:

$$
\left[
\begin{array}{cccccccccccccccccccc}
* & * & a_{0,0} & * & a_{0,1}  & a_{1,1} & a_{0,2} & a_{1,2} & a_{2,2} &
a_{1,3} & a_{2,3} & a_{3,3} & a_{2,4} & a_{3,4} & a_{4,4}\\
\end{array}
\right] 
$$

The $X$ and $B$ vectors are stored in 1d double arrays of length $N$. \\

\noindent {\bf Interface}  \\
\indent\indent {// Constructors} \\
\indent\indent {\em BandSPDLinSOE(BandSPDLinSolver \&theSolver);}  \\
\indent\indent {\em BandSPDLinSOE(int N, int ku, BandSPDLinSolver
\&theSolver);}\\ \\
\indent\indent {// Destructor} \\
\indent\indent {\em  $\tilde{}$BandSPDLinSOE();}\\ \\
\indent\indent {// Public Methods }  \\
\indent\indent {\em  int setBandSPDSolver(BandSPDLinSolver \&newSolver);}\\
\indent\indent {\em int setSize(const Graph \&theGraph) =0; } \\
\indent\indent {\em int getNumEqn(void) =0; } \\
\indent\indent {\em int addA(const Matrix \&theMatrix, const ID \& loc,
doublefact = 1.0) =0;} \\
\indent\indent {\em int addB(const Vector \& theVector, const ID \& loc,
double fact = 1.0) =0;} \\
\indent\indent {\em int setB(const Vector \& theVector, 
double fact = 1.0) =0;} \\
\indent\indent {\em void zeroA(void) =0;} \\
\indent\indent {\em void zeroB(void) =0;} \\
\indent\indent {\em const Vector \&getX(void) = 0;} \\
\indent\indent {\em const Vector \&getB(void) = 0;} \\
\indent\indent {\em double normRHS(void) =0;} \\
\indent\indent {\em void setX(int loc, double value) =0;}\\
\indent\indent {\em int sendSelf(int commitTag, Channel \&theChannel);}\\ 
\indent\indent {\em int recvSelf(int commitTag, Channel \&theChannel,
FEM\_ObjectBroker \&theBroker);}\\ 


\noindent {\bf Constructors}  \\
\indent {\em BandSPDLinSOE(BandSPDLinSolver \&solver);}  \\
The {\em solver} and a unique class tag (defined in $<$classTags.h$>$)
are passed to the LinearSOE constructor. The system size is set
to $0$ and the matrix $A$ is marked as not having been factored. Invokes
{\em setLinearSOE(*this)} on the {\em solver}. No memory is
allocated for the 3 1d arrays. \\  

{\em BandSPDLinSOE(int N, int ku, BandSPDLinSolver \&theSolver);}\\
The {\em solver} and a unique class tag (defined in $<$classTags.h$>$)
are passed to the LinearSOE constructor. 
Sets the size of the system to $N$, the number of superdiagonals to 
{\em ku}. Allocates memory for the arrays; if not enough
memory is available a warning message is printed and the system size
is set to $0$. Sets the solver to be called when solving the
equations to {\em theSolver}. Invokes {\em setLinearSOE(*this)} and
{\em setSize()} on the {\em theSolver}, printing a warning message if
{\em setSize()} returns a negative number. Also creates Vector objects
for $x$ and $b$ using the {\em (double *,int)} Vector constructor. \\


\noindent {\bf Destructor} \\
\indent {\em  $\tilde{}$BandSPDLinSOE();}\\ 
Calls delete on any arrays created. \\

\noindent {\bf Public Methods }  \\
\indent {\em  int setBandSPDSolver(BandSPDLinSolver \&newSolver);}\\
Invokes {\em setLinearSOE(*this)} on {\em newSolver}.
If the system size is not equal to $0$, it also invokes {\em setSize()}
on {\em newSolver}, printing a warning and returning the returned value if this
method returns a number less than $0$. Finally it returns the result
of invoking the LinearSOE classes {\em setSolver()} method. \\

\indent {\em int getNumEqn(void) =0; } \\
A method which returns the current size of the system. \\

\indent {\em  int setSize(const Graph \&G); } \\ 
The size of the system is determined by looking at the adjacency ID of
each Vertex in the Graph object {\em G}. This is done by first setting
{\em ku} equal to $0$ and then checking for each Vertex
in {\em G}, whether any of the vertex tags in the Vertices adjacency
ID results in {\em ku} being increased. Knowing {\em ku} and the size
of the system (the number of Vertices in {\em G}, a check to see if
the previously allocated 1d arrays for $A$, $x$ and $b$ are large
enough. If the memory portions allocated for the 1d arrays are not big
enough, the old space is returned to the heap and new space is
allocated from the heap. Prints a warning message if not enough
memory is available on the heap for the 1d arrays and returns a
$-1$. If memory is available, the components of the arrays are zeroed
and $A$ is marked as being unfactored. If the system size has
increased, new Vector objects for $x$ and $b$ using the {\em (double
*,int)} Vector constructor are created. Finally, the result of
invoking {\em setSize()} on the associated Solver object is
returned. \\ 


\indent {\em int addA(const Matrix \&M, const ID \& loc,
doublefact = 1.0) =0;} \\
First tests that {\em loc} and {\em M} are of compatible sizes; if not
a warning message is printed and a $-1$ is returned. The LinearSOE
object then assembles {\em fact} times the Matrix {\em 
M} into the matrix $A$. The Matrix is assembled into $A$ at the
locations given by the ID object {\em loc}, i.e. $a_{loc(i),loc(j)} +=
fact * M(i,j)$. If the location specified is outside the range,
i.e. $(-1,-1)$ the corresponding entry in {\em M} is not added to
$A$. If {\em fact} is equal to $0.0$ or $1.0$, more efficient steps
are performed. Returns $0$.  \\


{\em int addB(const Vector \& V, const ID \& loc,
double fact = 1.0) =0;} \\
First tests that {\em loc} and {\em V} are of compatible sizes; if not
a warning message is printed and a $-1$ is returned. The LinearSOE
object then assembles {\em fact} times the Vector {\em V} into
the vector $b$. The Vector is assembled into $b$ at the locations
given by the ID object {\em loc}, i.e. $b_{loc(i)} += fact * V(i)$. If a
location specified is outside the range, e.g. $-1$, the corresponding
entry in {\em V} is not added to $b$. If {\em fact} is equal to $0.0$,
$1.0$ or $-1.0$, more efficient steps are performed. Returns $0$. \\


{\em int setB(const Vector \& V, double fact = 1.0) =0;} \\
First tests that {\em V} and the size of the system are of compatible
sizes; if not a warning message is printed and a $-1$ is returned. The
LinearSOE object then sets the vector {\em b} to be {\em fact} times
the Vector {\em V}. If {\em fact} is equal to $0.0$, $1.0$ or $-1.0$,
more efficient steps are performed. Returns $0$. \\ 

{\em void zeroA(void) =0;} \\
Zeros the entries in the 1d array for $A$ and marks the system as not
having been factored. \\

{\em void zeroB(void) =0;} \\
Zeros the entries in the 1d array for $b$. \\

{\em const Vector \&getX(void) = 0;} \\
Returns the Vector object created for $x$. \\

{\em const Vector \&getB(void) = 0;} \\
Returns the Vector object created for $b$. \\

{\em double normRHS(void) =0;} \\
Returns the 2-norm of the vector $x$. \\

{\em void setX(int loc, double value) =0;}\\
If {\em loc} is within the range of $x$, sets $x(loc) = value$. \\

\indent {\em int sendSelf(int commitTag, Channel \&theChannel);}\\ 
Returns $0$. The object does not send any data or connectivity
information as this is not needed in the finite element design. \\

\indent {\em int recvSelf(int commitTag, Channel \&theChannel, FEM\_ObjectBroker
\&theBroker);}\\ 
Returns $0$. The object does not receive any data or connectivity
information as this is not needed in the finite element design.


\pagebreak
\subsubsection{ProfileSPDLinSOE}
%File: ~/OOP/system_of_eqn/linearSOE/profileSPD/ProfileSPDLinSOE.tex
%What: "@(#) ProfileSPDLinSOE.tex, revA"

\noindent {\bf ProfileSPDLinSOE} \\

\noindent {\bf Files}   \\
\indent \#include $<\tilde{ }$ProfileSPDLinSOE.h$>$  \\

\noindent {\bf Class Decleration}  \\
\indent class ProfileSPDLinSOE: public LinearSOE \\

\noindent {\bf Class Hierarchy} \\
\indent MovableObject \\
\indent\indent SystemOfEqn \\
\indent\indent\indent LinearSOE \\
\indent\indent\indent\indent {\bf ProfileSPDLinSOE} \\

\noindent {\bf Description}  \\
\indent ProfileSPDLinSOE is class which is used to store a symmetric
system of equations using a profile storage scheme. The upper
triangular part of $A$ is stored in a 1d double array with the diagonals of
$A$ located at positions given by an integer array $iLoc$. 
For example when $n=5$ and $A$ as shown below:

$$
\left[
\begin{array}{ccccc}
a_{0,0} & a_{0,1}  & 0 & 0 & a_{0,4} \\
a_{1,0} & a_{1,1} & a_{1,2} & a_{1,3} & 0 \\
a_{2,0} & a_{2,1} & a_{2,2} & a_{2,3} & a_{2,4}  \\
0 & a_{3,1} & a_{3,2} & a_{3,3} & a_{3,4} \\
0 & 0 & a_{4,2} & a_{4,3} & a_{4,4} \\
\end{array}
\right] 
$$

this is stored using:


$$ A =
\left[
\begin{array}{cccccccccccccccccccc}
a_{0,0} & a_{0,1}  & a_{1,1} & a_{1,2} & a_{2,2} & a_{1,3} &
a_{2,3} & a_{3,3} & a_{0,4} & 0 & a_{2,4} & a_{3,4} & a_{4,4}\\
\end{array}
\right] 
$$

and 

$$ iLoc =
\left[
\begin{array}{cccccccccccccccccccc}
1 & 3 & 5 & 8 & 13 \\
\end{array}
\right] 
$$
Note $iLoc$ stores the diagonal locations using Fortran indexing. This
is to facilitate calls to Fortran libraries, e.g. Digital's DXML.
The $x$ and $b$ vectors are stored in 1d double arrays of length $N$. \\


\noindent {\bf Interface}  \\
\indent\indent {// Constructors} \\
\indent\indent {\em ProfileSPDLinSOE(Solver \&theSolver);}  \\
\indent\indent {\em ProfileSPDLinSOE(int N, int *iLoc,
		  ProfileSPDLinSolver \&theSolver);}\\ \\
\indent\indent {// Destructor} \\
\indent\indent {\em  $\tilde{}$ProfileSPDLinSOE();}\\ \\
\indent\indent {// Public Methods }  \\
\indent\indent {\em  int setProfileSPDSolver(ProfileSPDLinSolver \&newSolver);}\\
\indent\indent {\em int setSize(const Graph \&theGraph) =0; } \\
\indent\indent {\em int getNumEqn(void) =0; } \\
\indent\indent {\em int addA(const Matrix \&theMatrix, const ID \& loc,
doublefact = 1.0) =0;} \\
\indent\indent {\em int addB(const Vector \& theVector, const ID \& loc,
double fact = 1.0) =0;} \\
\indent\indent {\em int setB(const Vector \& theVector, 
double fact = 1.0) =0;} \\
\indent\indent {\em void zeroA(void) =0;} \\
\indent\indent {\em void zeroB(void) =0;} \\
\indent\indent {\em const Vector \&getX(void) = 0;} \\
\indent\indent {\em const Vector \&getB(void) = 0;} \\
\indent\indent {\em double normRHS(void) =0;} \\
\indent\indent {\em void setX(int loc, double value) =0;}\\
\indent\indent {\em int sendSelf(int commitTag, Channel \&theChannel);}\\ 
\indent\indent {\em int recvSelf(int commitTag, Channel \&theChannel,
FEM\_ObjectBroker \&theBroker);}\\ 

\noindent {\bf Constructors}  \\
\indent {\em ProfileSPDLinSOE(Solver \&theSolver);}  \\
The {\em solver} and a unique class tag (defined in $<$classTags.h$>$)
are passed to the LinearSOE constructor. The system size is set
to $0$ and the matrix $A$ is marked as not having been factored. Invokes
{\em setLinearSOE(*this)} on the {\em solver}. No memory is
allocated for the 1d arrays. \\  


{\em ProfileSPDLinSOE(int N, int *newIloc,
		  ProfileSPDLinSolver \&theSolver); }\\
The {\em solver} and a unique class tag (defined in $<$classTags.h$>$)
are passed to the LinearSOE constructor. The system size is set
to $N$ and the matrix $A$ is marked as not having been
factored or condensed. Obtains memory from the heap for the 1d arrays storing the
data for $A$, $x$, $b$ and $iLoc$ and stores the size of these arrays. If not
enough memory is available for these arrays a warning message is
printed and the system size is set to $0$. The size of $A$ is given
by $newIloc(N-1)$, if this is not a valid address in {\em newIloc} a
segmentation fault or erronious results will result. The contents of
$iLoc$ are set equal to those of {\em newIloc}. Invokes {\em
setLinearSOE(*this)} and {\em setSize()} on {\em solver},
printing a warning message if {\em setSize()} returns a negative
number. Also creates Vector objects for $x$ and $b$ using the {\em
(double *,int)} Vector constructor. \\

\noindent {\bf Destructor} \\
\indent {\em virtual~ $\tilde{}$ProfileSPDLinSOE();}\\ 
Calls delete on any arrays created. \\

\noindent {\bf Public Methods }  \\
\indent {\em  int setProfileSPDSolver(ProfileSPDLinSolver \&newSolver);}\\
Invokes {\em setLinearSOE(*this)} on {\em newSolver}.
If the system size is not equal to $0$, it also invokes {\em setSize()}
on {\em newSolver}, printing a warning and returning the returned value if this
method returns a number less than $0$. Finally it returns the result
of invoking the LinearSOE classes {\em setSolver()} method. \\

\indent {\em int getNumEqn(void) =0; } \\
A method which returns the current size of the system. \\

\indent {\em  int setSize(const Graph \&G); } \\ 
The size of the system is determined by looking at the adjacency ID of
each Vertex in the Graph object {\em G}. This is done by first
determining the column height for each Vertex $i$ in {\em G}, done by
setting $iLoc(i)$ equal to $0$ and then checking for each Vertex
in {\em G}, whether any of the vertex tags in the Vertices adjacency
ID results in $iLoc(i)$ being increased. Knowing the col height of
each column, the values of {\em iLoc} can be determined. Knowing {\em
iLoc} and the size of the system (the number of Vertices in {\em G}, 
a check to see if the previously allocated 1d arrays for $A$, $x$ and
$b$ are large enough. If the memory portions allocated for the 1d
arrays are not big enough, the old space is returned to the heap and
new space is allocated from the heap. Printins a warning message if
not enough memory is available on the heap for the 1d arrays and
returns a $-1$. If memory is available, the components of the arrays
are zeroed and $A$ is marked as being unfactored. If the system size
has increased, new Vector objects for $x$ and $b$ using the {\em
(double *,int)} Vector constructor are created. Finally, the result of 
invoking {\em setSize()} on the associated Solver object is
returned. \\ 


\indent {\em int addA(const Matrix \&M, const ID \& loc,
doublefact = 1.0) =0;} \\
First tests that {\em loc} and {\em M} are of compatable sizes; if not
a warning message is printed and a $-1$ is returned. The LinearSOE
object then assembles {\em fact} times the Matrix {\em 
M} into the matrix $A$. The Matrix is assembled into $A$ at the
locations given by the ID object {\em loc}, i.e. $a_{loc(i),loc(j)} +=
fact * M(i,j)$. If the location specified is outside the range,
i.e. $(-1,-1)$ the corrseponding entry in {\em M} is not added to
$A$. If {\em fact} is equal to $0.0$ or $1.0$, more efficient steps
are performed. Returns $0$.  \\


{\em int addB(const Vector \& V, const ID \& loc,
double fact = 1.0) =0;} \\
First tests that {\em loc} and {\em V} are of compatable sizes; if not
a warning message is printed and a $-1$ is returned. The LinearSOE
object then assembles {\em fact} times the Vector {\em V} into
the vector $b$. The Vector is assembled into $b$ at the locations
given by the ID object {\em loc}, i.e. $b_{loc(i)} += fact * V(i)$. If a
location specified is outside the range, e.g. $-1$, the corresponding
entry in {\em V} is not added to $b$. If {\em fact} is equal to $0.0$,
$1.0$ or $-1.0$, more efficient steps are performed. Returns $0$. \\


{\em int setB(const Vector \& V, double fact = 1.0) =0;} \\
First tests that {\em V} and the size of the system are of compatable
sizes; if not a warning message is printed and a $-1$ is returned. The
LinearSOE object then sets the vector {\em b} to be {\em fact} times
the Vector {\em V}. If {\em fact} is equal to $0.0$, $1.0$ or $-1.0$,
more efficient steps are performed. Returns $0$. \\ 

{\em void zeroA(void) =0;} \\
Zeros the entries in the 1d array for $A$ and marks the system as not
having been factored. \\

{\em void zeroB(void) =0;} \\
Zeros the entries in the 1d array for $b$. \\

{\em const Vector \&getX(void) = 0;} \\
Returns the Vector object created for $x$. \\

{\em const Vector \&getB(void) = 0;} \\
Returns the Vector object created for $b$. \\

{\em double normRHS(void) =0;} \\
Returns the 2-norm of the vector $x$. \\

{\em void setX(int loc, double value) =0;}\\
If {\em loc} is within the range of $x$, sets $x(loc) = value$. \\

\indent {\em int sendSelf(int commitTag, Channel \&theChannel);}\\ 
Returns $0$. The object does not send any data or connectivity
information as this is not needed in the finite element design. \\

\indent {\em int recvSelf(int commitTag, Channel \&theChannel, FEM\_ObjectBroker
\&theBroker);}\\ 
Returns $0$. The object does not receive any data or connectivity
information as this is not needed in the finite element design.






\pagebreak
\subsubsection{SparseGenColLinSOE}
%File: ~/OOP/system_of_eqn/linearSOE/sparseGen/SparseGenColLinSOE.tex
%What: "@(#) SparseGenColLinSOE.tex, revA"

\noindent {\bf Files}   \\
\indent \#include $<\tilde{ }$/system\_of\_eqn/linearSOE/sparseGen/SparseGenColLinSOE.h$>$  \\

\noindent {\bf Class Declaration}  \\
\indent class SparseGenColLinSOE: public LinearSOE \\

\noindent {\bf Class Hierarchy} \\
\indent MovableObject \\
\indent\indent SystemOfEqn \\
\indent\indent\indent LinearSOE \\
\indent\indent\indent\indent {\bf SparseGenColLinSOE} \\

\noindent {\bf Description}  \\
\indent SparseGenColLinSOE is class which is used to store the matrix
equation $Ax=b$ of order $size$ using a sparse column-compacted storage
scheme for $A$. The $A$ matrix is stored in a 1d double array with
$nnz$ elements, where $nnz$ is the number of non-zeroes in the matrix
$A$. Two additional 1d integer arrays $rowA$ and $colStartA$ are used to
store information about the location of the coefficients, with $colStartA(i)$
storing the location in the 1d double array of the start of column $i$
and $rowA(j)$ identifying the row in $A$ to which the
$j'th$ component in the 1d double array. $colStartA$ is of
dimension $size+1$ and $rowA$ of dimension $nnz$. For example

$$
\left[
\begin{array}{ccccc}
a_{0,0} & 0 & a_{0,2}  & a_{0,3} & 0  \\
a_{1,0} & a_{1,1} & 0 & 0 & 0  \\
0 & a_{2,1} & a_{2,2} & 0 & 0 \\
0 & 0 & 0 & a_{3,3} & a_{3,4} \\
a_{4,0} & a_{4,1} & 0 & 0 & a_{4,4}
\end{array}
\right] 
$$

is stored in:

$$
\left[
\begin{array}{cccccccccccccc}
a_{0,0} & a_{1,0}  & a_{4,0} & a_{1,1} & a_{2,1} & a_{4,1} &
a_{0,2} & a_{2,2} & a_{0,3} & a_{3,3} & a_{3,4} & a_{4,4}  \\
\end{array}
\right] 
$$

with

$$
colStartA =
\left[
\begin{array}{cccccccccccccc}
0 & 3 & 6 & 8 & 10 & 12
\end{array}
\right] 
$$

and

$$
rowA =
\left[
\begin{array}{cccccccccccccc}
0 & 1 & 4 & 1 & 2 & 4 & 0 & 2 & 0 & 3 & 3 & 4 
\end{array}
\right] 
$$
The $x$ and $b$ vectors are stored in 1d double arrays of length $n$. \\

\noindent {\bf Interface}  \\
\indent\indent {// Constructors} \\
\indent\indent {\em SparseGenColLinSOE(SparseGenColLinSolver \&theSolver);}  \\
\indent\indent {\em SparseGenColLinSOE(int N, int NNZ, int *colStartA,
int *rowA, SparseGenColLinSolver \&theSolver); }\\\\
\indent\indent {// Destructor} \\
\indent\indent {\em  $\tilde{}$SparseGenColLinSOE();}\\ \\
\indent\indent {// Public Methods }  \\
\indent\indent {\em  int setSparseGenSolver(SparseGenColLinSolver \&newSolver);}\\
\indent\indent {\em int setSize(const Graph \&theGraph) =0; } \\
\indent\indent {\em int getNumEqn(void) =0; } \\
\indent\indent {\em int addA(const Matrix \&theMatrix, const ID \& loc,
doublefact = 1.0) =0;} \\
\indent\indent {\em int addB(const Vector \& theVector, const ID \& loc,
double fact = 1.0) =0;} \\
\indent\indent {\em int setB(const Vector \& theVector, 
double fact = 1.0) =0;} \\
\indent\indent {\em void zeroA(void) =0;} \\
\indent\indent {\em void zeroB(void) =0;} \\
\indent\indent {\em const Vector \&getX(void) = 0;} \\
\indent\indent {\em const Vector \&getB(void) = 0;} \\
\indent\indent {\em double normRHS(void) =0;} \\
\indent\indent {\em void setX(int loc, double value) =0;}\\
\indent\indent {\em int sendSelf(int commitTag, Channel \&theChannel);}\\ 
\indent\indent {\em int recvSelf(int commitTag, Channel \&theChannel,
FEM\_ObjectBroker \&theBroker);}\\ 


\noindent {\bf Constructors}  \\
\indent {\em SparseGenColLinSOE(SparseGenColLinSolver \&solver);}  \\
The {\em solver} and a unique class tag (defined in $<$classTags.h$>$)
are passed to the LinearSOE constructor. The system size is set
to $0$ and the matrix $A$ is marked as not having been factored. Invokes
{\em setLinearSOE(*this)} on the {\em solver}. No memory is
allocated for the 3 1d arrays. \\  


{\em SparseGenColLinSOE(int N, int NNZ, int *colStartA,
int *rowA, SparseGenColLinSolver \&theSolver); }\\
The {\em solver} and a unique class tag (defined in $<$classTags.h$>$)
are passed to the LinearSOE constructor. The system size is set
to $N$, the number of non-zeros is set to $NNZ$ and the matrix $A$ is
marked as not having been factored. Obtains memory from the heap for
the 1d arrays storing the data for $A$, $x$ and $b$ and stores the
size of these arrays. If not enough memory is available for these
arrays a warning message is printed and the system size is set to
$0$. Invokes {\em setLinearSOE(*this)} and {\em setSize()} on {\em solver},
printing a warning message if {\em setSize()} returns a negative
number. Also creates Vector objects for $x$ and $b$ using the {\em
(double *,int)} Vector constructor. It is up to the user to ensure
that {\em colStartA} and {\em rowA} are of the correct size and
contain the correct data. \\

\noindent {\bf Destructor} \\
\indent {\em  $\tilde{}$SparseGenColLinSOE();}\\ 
Calls delete on any arrays created. \\

\noindent {\bf Public Methods} \\
\indent {\em  int setSolver(SparseGenColLinSolver \&newSolver);}\\
Invokes {\em setLinearSOE(*this)} on {\em newSolver}.
If the system size is not equal to $0$, it also invokes {\em setSize()}
on {\em newSolver}, printing a warning and returning $-1$ if this
method returns a number less than $0$. Finally it returns the result
of invoking the LinearSOE classes {\em setSolver()} method. \\

\indent {\em int getNumEqn(void) =0; } \\
A method which returns the current size of the system. \\

\indent {\em  int setSize(const Graph \&theGraph); } \\ 
The size of the system is determined from the Graph object {\em
theGraph}, which must contain {\em size} vertices labelled $0$ through
$size-1$, the adjacency list for each vertex containing the associated
vertices in a column of the matrix $A$. The size is determined by
invoking {\em getNumVertex()} on {\em theGraph} and the number of
non-zeros is determined by looking at the size of the adjacenecy list
of each vertex in the graph, allowing space for the diagonal term. If
the old space allocated for the 1d arrays is not big enough, it the
old space is returned to the heap and new space is allocated from the
heap. Prints a warning message, sets size to $0$ and returns a $-1$,
if not enough memory is available on the heap for the 1d arrays. If
memory is available, the components of the arrays are 
zeroed and $A$ is marked as being unfactored. If the system size has
increased, new Vector objects for $x$ and $b$ using the {\em (double
*,int)} Vector constructor are created. The $colStartA$ and $rowA$ are
then determined by looping through the vertices, setting $colStartA(i)
= colStartA(i-1) + 1 + $the size of Vertices $i$ adjacency list and 
placing the contents of $i$ and the adjacency list into $rowA$ in
ascending order. Finally, the result of invoking {\em setSize()} on
the associated Solver object is returned. \\ 


\indent {\em int addA(const Matrix \&M, const ID \& loc,
doublefact = 1.0) =0;} \\
First tests that {\em loc} and {\em M} are of compatible sizes; if not
a warning message is printed and a $-1$ is returned. The LinearSOE
object then assembles {\em fact} times the Matrix {\em 
M} into the matrix $A$. The Matrix is assembled into $A$ at the
locations given by the ID object {\em loc}, i.e. $a_{loc(i),loc(j)} +=
fact * M(i,j)$. If the location specified is outside the range,
i.e. $(-1,-1)$ the corrseponding entry in {\em M} is not added to
$A$. If {\em fact} is equal to $0.0$ or $1.0$, more efficient steps
are performed. Returns $0$.  \\


{\em int addB(const Vector \& V, const ID \& loc,
double fact = 1.0) =0;} \\
First tests that {\em loc} and {\em V} are of compatible sizes; if not
a warning message is printed and a $-1$ is returned. The LinearSOE
object then assembles {\em fact} times the Vector {\em V} into
the vector $b$. The Vector is assembled into $b$ at the locations
given by the ID object {\em loc}, i.e. $b_{loc(i)} += fact * V(i)$. If a
location specified is outside the range, e.g. $-1$, the corresponding
entry in {\em V} is not added to $b$. If {\em fact} is equal to $0.0$,
$1.0$ or $-1.0$, more efficient steps are performed. Returns $0$. \\


{\em int setB(const Vector \& V, double fact = 1.0) =0;} \\
First tests that {\em V} and the size of the system are of compatible
sizes; if not a warning message is printed and a $-1$ is returned. The
LinearSOE object then sets the vector {\em b} to be {\em fact} times
the Vector {\em V}. If {\em fact} is equal to $0.0$, $1.0$ or $-1.0$,
more efficient steps are performed. Returns $0$. \\ 

{\em void zeroA(void) =0;} \\
Zeros the entries in the 1d array for $A$ and marks the system as not
having been factored. \\

{\em void zeroB(void) =0;} \\
Zeros the entries in the 1d array for $b$. \\

{\em const Vector \&getX(void) = 0;} \\
Returns the Vector object created for $x$. \\

{\em const Vector \&getB(void) = 0;} \\
Returns the Vector object created for $b$. \\

{\em double normRHS(void) =0;} \\
Returns the 2-norm of the vector $x$. \\

{\em void setX(int loc, double value) =0;}\\
If {\em loc} is within the range of $x$, sets $x(loc) = value$. \\


\indent {\em int sendSelf(int commitTag, Channel \&theChannel);}\\ 
Returns $0$. The object does not send any data or connectivity
information as this is not needed in the finite element design. \\

\indent {\em int recvSelf(int commitTag, Channel \&theChannel, FEM\_ObjectBroker
\&theBroker);}\\ 
Returns $0$. The object does not receive any data or connectivity
information as this is not needed in the finite element design.



\pagebreak
\subsubsection{SparseGenRowLinSOE}
%%File: ~/OOP/system_of_eqn/linearSOE/sparseGen/SparseGenColLinSOE.tex
%What: "@(#) SparseGenColLinSOE.tex, revA"

\noindent {\bf Files}   \\
\indent \#include $<\tilde{ }$/system\_of\_eqn/linearSOE/sparseGen/SparseGenColLinSOE.h$>$  \\

\noindent {\bf Class Declaration}  \\
\indent class SparseGenColLinSOE: public LinearSOE \\

\noindent {\bf Class Hierarchy} \\
\indent MovableObject \\
\indent\indent SystemOfEqn \\
\indent\indent\indent LinearSOE \\
\indent\indent\indent\indent {\bf SparseGenColLinSOE} \\

\noindent {\bf Description}  \\
\indent SparseGenColLinSOE is class which is used to store the matrix
equation $Ax=b$ of order $size$ using a sparse column-compacted storage
scheme for $A$. The $A$ matrix is stored in a 1d double array with
$nnz$ elements, where $nnz$ is the number of non-zeroes in the matrix
$A$. Two additional 1d integer arrays $rowA$ and $colStartA$ are used to
store information about the location of the coefficients, with $colStartA(i)$
storing the location in the 1d double array of the start of column $i$
and $rowA(j)$ identifying the row in $A$ to which the
$j'th$ component in the 1d double array. $colStartA$ is of
dimension $size+1$ and $rowA$ of dimension $nnz$. For example

$$
\left[
\begin{array}{ccccc}
a_{0,0} & 0 & a_{0,2}  & a_{0,3} & 0  \\
a_{1,0} & a_{1,1} & 0 & 0 & 0  \\
0 & a_{2,1} & a_{2,2} & 0 & 0 \\
0 & 0 & 0 & a_{3,3} & a_{3,4} \\
a_{4,0} & a_{4,1} & 0 & 0 & a_{4,4}
\end{array}
\right] 
$$

is stored in:

$$
\left[
\begin{array}{cccccccccccccc}
a_{0,0} & a_{1,0}  & a_{4,0} & a_{1,1} & a_{2,1} & a_{4,1} &
a_{0,2} & a_{2,2} & a_{0,3} & a_{3,3} & a_{3,4} & a_{4,4}  \\
\end{array}
\right] 
$$

with

$$
colStartA =
\left[
\begin{array}{cccccccccccccc}
0 & 3 & 6 & 8 & 10 & 12
\end{array}
\right] 
$$

and

$$
rowA =
\left[
\begin{array}{cccccccccccccc}
0 & 1 & 4 & 1 & 2 & 4 & 0 & 2 & 0 & 3 & 3 & 4 
\end{array}
\right] 
$$
The $x$ and $b$ vectors are stored in 1d double arrays of length $n$. \\

\noindent {\bf Interface}  \\
\indent\indent {// Constructors} \\
\indent\indent {\em SparseGenColLinSOE(SparseGenColLinSolver \&theSolver);}  \\
\indent\indent {\em SparseGenColLinSOE(int N, int NNZ, int *colStartA,
int *rowA, SparseGenColLinSolver \&theSolver); }\\\\
\indent\indent {// Destructor} \\
\indent\indent {\em  $\tilde{}$SparseGenColLinSOE();}\\ \\
\indent\indent {// Public Methods }  \\
\indent\indent {\em  int setSparseGenSolver(SparseGenColLinSolver \&newSolver);}\\
\indent\indent {\em int setSize(const Graph \&theGraph) =0; } \\
\indent\indent {\em int getNumEqn(void) =0; } \\
\indent\indent {\em int addA(const Matrix \&theMatrix, const ID \& loc,
doublefact = 1.0) =0;} \\
\indent\indent {\em int addB(const Vector \& theVector, const ID \& loc,
double fact = 1.0) =0;} \\
\indent\indent {\em int setB(const Vector \& theVector, 
double fact = 1.0) =0;} \\
\indent\indent {\em void zeroA(void) =0;} \\
\indent\indent {\em void zeroB(void) =0;} \\
\indent\indent {\em const Vector \&getX(void) = 0;} \\
\indent\indent {\em const Vector \&getB(void) = 0;} \\
\indent\indent {\em double normRHS(void) =0;} \\
\indent\indent {\em void setX(int loc, double value) =0;}\\
\indent\indent {\em int sendSelf(int commitTag, Channel \&theChannel);}\\ 
\indent\indent {\em int recvSelf(int commitTag, Channel \&theChannel,
FEM\_ObjectBroker \&theBroker);}\\ 


\noindent {\bf Constructors}  \\
\indent {\em SparseGenColLinSOE(SparseGenColLinSolver \&solver);}  \\
The {\em solver} and a unique class tag (defined in $<$classTags.h$>$)
are passed to the LinearSOE constructor. The system size is set
to $0$ and the matrix $A$ is marked as not having been factored. Invokes
{\em setLinearSOE(*this)} on the {\em solver}. No memory is
allocated for the 3 1d arrays. \\  


{\em SparseGenColLinSOE(int N, int NNZ, int *colStartA,
int *rowA, SparseGenColLinSolver \&theSolver); }\\
The {\em solver} and a unique class tag (defined in $<$classTags.h$>$)
are passed to the LinearSOE constructor. The system size is set
to $N$, the number of non-zeros is set to $NNZ$ and the matrix $A$ is
marked as not having been factored. Obtains memory from the heap for
the 1d arrays storing the data for $A$, $x$ and $b$ and stores the
size of these arrays. If not enough memory is available for these
arrays a warning message is printed and the system size is set to
$0$. Invokes {\em setLinearSOE(*this)} and {\em setSize()} on {\em solver},
printing a warning message if {\em setSize()} returns a negative
number. Also creates Vector objects for $x$ and $b$ using the {\em
(double *,int)} Vector constructor. It is up to the user to ensure
that {\em colStartA} and {\em rowA} are of the correct size and
contain the correct data. \\

\noindent {\bf Destructor} \\
\indent {\em  $\tilde{}$SparseGenColLinSOE();}\\ 
Calls delete on any arrays created. \\

\noindent {\bf Public Methods} \\
\indent {\em  int setSolver(SparseGenColLinSolver \&newSolver);}\\
Invokes {\em setLinearSOE(*this)} on {\em newSolver}.
If the system size is not equal to $0$, it also invokes {\em setSize()}
on {\em newSolver}, printing a warning and returning $-1$ if this
method returns a number less than $0$. Finally it returns the result
of invoking the LinearSOE classes {\em setSolver()} method. \\

\indent {\em int getNumEqn(void) =0; } \\
A method which returns the current size of the system. \\

\indent {\em  int setSize(const Graph \&theGraph); } \\ 
The size of the system is determined from the Graph object {\em
theGraph}, which must contain {\em size} vertices labelled $0$ through
$size-1$, the adjacency list for each vertex containing the associated
vertices in a column of the matrix $A$. The size is determined by
invoking {\em getNumVertex()} on {\em theGraph} and the number of
non-zeros is determined by looking at the size of the adjacenecy list
of each vertex in the graph, allowing space for the diagonal term. If
the old space allocated for the 1d arrays is not big enough, it the
old space is returned to the heap and new space is allocated from the
heap. Prints a warning message, sets size to $0$ and returns a $-1$,
if not enough memory is available on the heap for the 1d arrays. If
memory is available, the components of the arrays are 
zeroed and $A$ is marked as being unfactored. If the system size has
increased, new Vector objects for $x$ and $b$ using the {\em (double
*,int)} Vector constructor are created. The $colStartA$ and $rowA$ are
then determined by looping through the vertices, setting $colStartA(i)
= colStartA(i-1) + 1 + $the size of Vertices $i$ adjacency list and 
placing the contents of $i$ and the adjacency list into $rowA$ in
ascending order. Finally, the result of invoking {\em setSize()} on
the associated Solver object is returned. \\ 


\indent {\em int addA(const Matrix \&M, const ID \& loc,
doublefact = 1.0) =0;} \\
First tests that {\em loc} and {\em M} are of compatible sizes; if not
a warning message is printed and a $-1$ is returned. The LinearSOE
object then assembles {\em fact} times the Matrix {\em 
M} into the matrix $A$. The Matrix is assembled into $A$ at the
locations given by the ID object {\em loc}, i.e. $a_{loc(i),loc(j)} +=
fact * M(i,j)$. If the location specified is outside the range,
i.e. $(-1,-1)$ the corrseponding entry in {\em M} is not added to
$A$. If {\em fact} is equal to $0.0$ or $1.0$, more efficient steps
are performed. Returns $0$.  \\


{\em int addB(const Vector \& V, const ID \& loc,
double fact = 1.0) =0;} \\
First tests that {\em loc} and {\em V} are of compatible sizes; if not
a warning message is printed and a $-1$ is returned. The LinearSOE
object then assembles {\em fact} times the Vector {\em V} into
the vector $b$. The Vector is assembled into $b$ at the locations
given by the ID object {\em loc}, i.e. $b_{loc(i)} += fact * V(i)$. If a
location specified is outside the range, e.g. $-1$, the corresponding
entry in {\em V} is not added to $b$. If {\em fact} is equal to $0.0$,
$1.0$ or $-1.0$, more efficient steps are performed. Returns $0$. \\


{\em int setB(const Vector \& V, double fact = 1.0) =0;} \\
First tests that {\em V} and the size of the system are of compatible
sizes; if not a warning message is printed and a $-1$ is returned. The
LinearSOE object then sets the vector {\em b} to be {\em fact} times
the Vector {\em V}. If {\em fact} is equal to $0.0$, $1.0$ or $-1.0$,
more efficient steps are performed. Returns $0$. \\ 

{\em void zeroA(void) =0;} \\
Zeros the entries in the 1d array for $A$ and marks the system as not
having been factored. \\

{\em void zeroB(void) =0;} \\
Zeros the entries in the 1d array for $b$. \\

{\em const Vector \&getX(void) = 0;} \\
Returns the Vector object created for $x$. \\

{\em const Vector \&getB(void) = 0;} \\
Returns the Vector object created for $b$. \\

{\em double normRHS(void) =0;} \\
Returns the 2-norm of the vector $x$. \\

{\em void setX(int loc, double value) =0;}\\
If {\em loc} is within the range of $x$, sets $x(loc) = value$. \\


\indent {\em int sendSelf(int commitTag, Channel \&theChannel);}\\ 
Returns $0$. The object does not send any data or connectivity
information as this is not needed in the finite element design. \\

\indent {\em int recvSelf(int commitTag, Channel \&theChannel, FEM\_ObjectBroker
\&theBroker);}\\ 
Returns $0$. The object does not receive any data or connectivity
information as this is not needed in the finite element design.



\pagebreak
\subsubsection{UmfpackGenLinSOE}
UNDER CONSTRUCTION. \\

\pagebreak
\subsubsection{SymSparseLinSOE}
UNDER CONSTRUCTION. \\

\pagebreak
\subsection{{\bf Solver}}
%File: ~/OOP/system_of_eqn/Solver.tex
%What: "@(#) Solver.tex, revA"

\noindent {\bf Files}   \\
\indent \#include $<\tilde{ }$/system\_of\_eqn/Solver.h$>$  \\

\noindent {\bf Class Declaration}  \\
\indent class Solver:  public MovableObject \\

\noindent {\bf Class Hierarchy} \\
\indent MovableObject \\
\indent\indent {\bf Solver} \\

\noindent {\bf Description}  \\
\indent Solver is an abstract class. A Solver object is responsible for performing
the numerical operations on its associated SystemOfEqn object. \\

\noindent {\bf Class Interface} \\
\indent\indent {// Constructors}  \\ 
\indent\indent {\em Solver(int classTag);}  \\ \\
\indent\indent {// Destructor}  \\ 
\indent\indent {\em virtual~ $\tilde{}$Solver();}\\  \\
\indent\indent {// Public Methods}  \\ 
\indent\indent {\em virtual int solve(void) =0;} \\


\noindent {\bf Constructor}  \\
\indent {\em Solver(int classTag);}  \\
The integer {\em classTag} is passed to the MovableObject classes
constructor. \\ 

\noindent {\bf Destructor} \\
\indent {\em virtual~ $\tilde{}$Solver();}\\ 
Does nothing. Provided so the subclasses destructor will be called. \\

\noindent {\bf Public Methods }  \\
\indent {\em virtual int solve(void) =0;} \\
Causes the solver to solve the system of equations. Returns $0$ if
successful , negative number if not; the actual value depending on
the type of Solver.\\






\pagebreak
\subsubsection{{\bf LinearSOESolver}}
%File: ~/OOP/system_of_eqn/linearSOE/LinearSOESolver.tex
%What: "@(#) LinearSOESolver.tex, revA"

\noindent {\bf Files}   \\
\indent \#include $<\tilde{ }$/system\_of\_eqn/linearSOE/LinearSOESolver.h$>$  \\

\noindent {\bf Class Declaration}  \\
\indent class LinearSOESolver: public Solver  \\

\noindent {\bf Class Hierarchy} \\
\indent MovableObject \\
\indent\indent  Solver \\
\indent\indent\indent {\bf LinearSOESolver} \\

\noindent {\bf Description}  \\
\indent LinearSOESolver is an abstract class. A LinearSOESolver object is
responsible for solving the LinearSOE object that it is associated
with. That is, to find $x$ such that the matrix equation $Ax=b$ is
satisfied. \\

\noindent {\bf Interface}  \\
\indent\indent {// Constructor} \\
\indent\indent {\em LinearSOESolver(int classTag);}  \\ \\
\indent\indent {// Destructor} \\
\indent\indent {\em virtual~ $\tilde{}$LinearSOESolver();}\\  \\
\indent\indent {// Public Methods }  \\
\indent\indent {\em virtual int solve(void) =0;} \\
\indent\indent {\em virtual int setSize(void) =0;} \\

\noindent {\bf Constructor}  \\
\indent {\em LinearSOESolver(int classTag);}  \\
The integer {\em classTag} is passed to the Solver. \\

\noindent {\bf Destructor} \\
\indent {\em virtual~ $\tilde{}$LinearSOESolver();}\\ 
Does nothing. Provided so the subclasses destructor will be called. \\

\noindent {\bf Public Methods }  \\
\indent {\em virtual int solve(void) =0;} \\
Causes the LinearSOESolver to solve the system of equations $Ax=b$ for $x$.
Returns $0$ if successful , negative number if not; the actual value depending on
the type of LinearSOESolver. The result of the solve are to be stored
in the $x$ vector of the LinearSOE by the object.\\

\indent {\em virtual int setSize(void) =0;} \\
This is invoked by the {\em LinearSOE} object when {\em setSize()} has
been invoked on it. Solvers may sometimes need to store additional
data that needs to be updated if the size of the system of equation
changes. \\






\pagebreak
\subsubsection{{\bf FullGenLinSolver}}
%File: ~/OOP/system_of_eqn/linearSOE/FullGEN/FullGenLinSolver.tex
%What: "@(#) FullGenLinSolver.tex, revA"

\noindent {\bf Files}   \\
\indent \#include $<\tilde{ }$/system\_of\_eqn/linearSOE/fullGEN/FullGenLinSolver.h$>$  \\

\noindent {\bf Class Declaration}  \\
\indent class FullGenLinSolver: public LinearSOESolver  \\

\noindent {\bf Class Hierarchy} \\
\indent MovableObject \\
\indent\indent  Solver \\
\indent\indent\indent LinearSOESolver \\
\indent\indent\indent\indent {\bf FullGenLinSolver} \\

\noindent {\bf Description}  \\
\indent FullGenLinSolver is an abstract class.  The FullGEnLinSolver
class provides access for each subclass to the FullGenLinSOE object
through the pointer {\em theSOE}, which is a protected pointer. \\

\noindent {\bf Interface}  \\
\indent\indent // Constructor \\
\indent\indent {\em FullGenLinSolver(int classTag);}  \\ \\
\indent\indent // Destructor \\
\indent\indent {\em virtual~ $\tilde{}$FullGenLinSolver();}\\  \\
\indent\indent // Public Methods \\
\indent\indent {\em virtual int setLinearSOE(FullGenLinSOE \&theSOE);} \\

\noindent {\bf Constructor}  \\
\indent {\em FullGenLinSolver(int classTag);}  \\
The integer {\em classTag} is passed to the LinearSOESolver classes
constructor. \\ 

\noindent {\bf Destructor} \\
\indent {\em virtual~ $\tilde{}$FullGenLinSolver();}\\ 
Does nothing, provided so the subclasses destructor will be called. \\

\noindent {\bf Public Methods }  \\
\indent {\em virtual int setLinearSOE(FullGenLinSOE \&theSOE);} \\
Sets the link to the FullGEnLinSOE object {\em theSOE}. This is the
object on which the solver will perform the numerical computations. \\







\pagebreak
\subsubsection{FullGenLinLapackSolver}
%File: ~/OOP/system_of_eqn/linearSOE/FullGEN/FullGenLinLapackSolver.tex
%What: "@(#) FullGenLinLapackSolver.tex, revA"

\noindent {\bf Files}   \\
\indent \#include $<\tilde{ }$/system\_of\_eqn/linearSOE/fullGEN/FullGenLinLapackSolver.h$>$  \\

\noindent {\bf Class Declaration}  \\
\indent class FullGenLinLapackSolver: public FullGenLinSolver  \\

\noindent {\bf Class Hierarchy} \\
\indent  MovableObject \\
\indent\indent  Solver \\
\indent\indent\indent LinearSOESolver \\
\indent\indent\indent\indent  FullGenLinSolver \\
\indent\indent\indent\indent\indent {\bf FullGenLinLapackSolver} \\

\noindent {\bf Description}  \\
\indent A FullGenLinLapackSolver object can be constructed to solve
a FullGenLinSOE object. It obtains the solution by making calls on the
the LAPACK library. The class is defined to be a friend of the 
FullGenLinSOE class (see $<$FullGenLinSOE.h$>$). \\

\noindent {\bf Interface}  \\
\indent\indent // Constructor \\
\indent\indent {\em FullGenLinLapackSolver();}  \\ \\
\indent\indent // Destructor \\
\indent\indent {\em $\tilde{ }$FullGenLinLapackSolver();}\\  \\
\indent\indent // Public Methods \\
\indent\indent {\em int solve(void);} \\
\indent\indent {\em int setSize(void);} \\
\indent\indent {\em  int sendSelf(int commitTag, Channel \&theChannel);} \\ 
\indent\indent {\em  int recvSelf(int commitTag, Channel \&theChannel, FEM\_ObjectBroker
\&theBroker);} \\ 

\noindent {\bf Constructor}  \\
\indent {\em FullGenLinLapackSolver();}  \\
A unique class tag (defined in $<$classTags.h$>$) is passed to the
FullGenLinSolver constructor. Sets the size of {\em iPiv} to $0$, {\em
iPiv} being an integer array needed by the LAPACK routines. \\

\noindent {\bf Destructor} \\
\indent {\em  $\tilde{ }$FullGenLinLapackSolver();}\\ 
Invokes delete on {\em iPiv} to free the memory it was allocated. \\


\noindent {\bf Public Methods }  \\
\indent {\em int solve(void);} \\
First copies $B$ into $X$ and then solves the FullGenLinSOE system 
it is associated with (pointer kept by parent class) by calling the LAPACK 
routines {\em dgesv()}, if the system is marked as not having been factored,
or {\em dgetrs()}, if system is marked as having been factored. If the
solution is successfully obtained, i.e. the LAPACK routines return $0$
in the INFO argument, it marks the system has having been
factored and returns $0$, otherwise it prints a warning message and
returns INFO. The solve process changes $A$ and $X$. \\   

\indent {\em int setSize(void);} \\
Is used to construct a 1d integer array, {\em iPiv} that is needed by
the LAPACK solvers. It checks to see if current size of {\em iPiv} is
large enough, if not it deletes the cold and creates a larger
array. Returns $0$ if sucessfull, prints a warning message and returns
a $-1$ if not enough memory is available for this new array. \\


\indent {\em  int sendSelf(int commitTag, Channel \&theChannel);} \\ 
Does nothing but return $0$. \\

\indent {\em  int recvSelf(int commitTag, Channel \&theChannel, FEM\_ObjectBroker
\&theBroker);} \\ 
Does nothing but return $0$. \\






\pagebreak
\subsubsection{{\bf BandGenLinSolver}}
%File: ~/OOP/system_of_eqn/linearSOE/bandGEN/BandGenLinSolver.tex
%What: "@(#) BandGenLinSolver.tex, revA"

\noindent {\bf Files}   \\
\indent \#include $<\tilde{ }$/system\_of\_eqn/linearSOE/bandGEN/BandGenLinSolver.h$>$  \\

\noindent {\bf Class Declaration}  \\
\indent class BandGenLinSolver: public LinearSOESolver  \\

\noindent {\bf Class Hierarchy} \\
\indent MovableObject \\
\indent\indent  Solver \\
\indent\indent\indent LinearSOESolver \\
\indent\indent\indent\indent {\bf BandGenLinSolver} \\

\noindent {\bf Description}  \\
\indent BandGenLinSolver is an abstract class.  The BandGenLinSolver
class provides access for each subclass to the BandGenLinSOE object
through the pointer {\em theSOE}, which is a protected pointer. \\


\noindent {\bf Interface}  \\
\indent\indent // Constructor \\
\indent\indent {\em BandGenLinSolver(int classTag);}  \\ \\
\indent\indent // Destructor \\
\indent\indent {\em virtual~ $\tilde{}$BandGenLinSolver();}\\  \\
\indent\indent // Public Methods \\
\indent\indent {\em virtual int setLinearSOE(BandGenLinSOE \&theSOE);} \\

\noindent {\bf Constructor}  \\
\indent {\em BandGenLinSolver(int classTag);}  \\
The integer {\em classTag} is passed to the LinearSOESolver classes
constructor. \\ 

\noindent {\bf Destructor} \\
\indent {\em virtual~ $\tilde{}$BandGenLinSolver();}\\ 
Does nothing, provided so the subclasses destructor will be called. \\

\noindent {\bf Public Methods }  \\
\indent {\em virtual int setLinearSOE(BandGenLinSOE \&theSOE);} \\
The method sets up the link between the BandGenLinSOE object and the
BandGenLinSolver, that it is sets the pointer the subclasses use.  \\






\pagebreak
\subsubsection{BandGenLinLapackSolver}
%File: ~/OOP/system_of_eqn/linearSOE/bandGEN/BandGenLinLapackSolver.tex
%What: "@(#) BandGenLinLapackSolver.tex, revA"

\noindent {\bf Files}   \\
\indent \#include $<\tilde{ }$/system\_of\_eqn/linearSOE/bandGEN/BandGenLinLapackSolver.h$>$  \\

\noindent {\bf Class Declaration}  \\
\indent class BandGenLinLapackSolver: public BandGenLinSolver  \\

\noindent {\bf Class Hierarchy} \\
\indent MovableObject \\
\indent\indent  Solver \\
\indent\indent\indent LinearSOESolver \\
\indent\indent\indent\indent BandGenLinSolver \\
\indent\indent\indent\indent\indent {\bf BandGenLinLapackSolver} \\

\noindent {\bf Description}  \\
\indent A BandGenLinLapackSolver object can be constructed to solve
a BandGenLinSOE object. It obtains the solution by making calls on the
the LAPACK library. The class is defined to be a friend of the 
BandGenLinSOE class (see $<$BandGenLinSOE.h$>$). \\


\noindent {\bf Interface}  \\
\indent\indent // Constructor \\
\indent\indent {\em BandGenLinLapackSolver();}  \\ \\
\indent\indent // Destructor \\
\indent\indent {\em $\tilde{ }$BandGenLinLapackSolver();}\\  \\
\indent\indent // Public Methods \\
\indent\indent {\em int solve(void);} \\
\indent\indent {\em int setSize(void);} \\
\indent\indent {\em  int sendSelf(int commitTag, Channel \&theChannel);} \\ 
\indent\indent {\em  int recvSelf(int commitTag, Channel \&theChannel, FEM\_ObjectBroker
\&theBroker);} \\ 


\noindent {\bf Constructor}  \\
\indent {\em BandGenLinLapackSolver();}  \\
A unique class tag (defined in $<$classTags.h$>$) is passed to the
BandGenLinSolver constructor. Sets the size of {\em iPiv} to $0$, {\em
iPiv} being an integer array needed by the LAPACK routines. \\

\noindent {\bf Destructor} \\
\indent {\em  $\tilde{ }$BandGenLinLapackSolver();}\\ 
Invokes delete on {\em iPiv} to free the memory allocated to store the
array. \\ 

\noindent {\bf Public Methods }  \\
\indent {\em virtual int solve(void);} \\
The solver first copies the B vector into X and then solves the
BandGenLinSOE system by calling the LAPACK routines {\em
dgbsv()}, if the system is marked as not having been factored, and {\em
dgbtrs()} if system is marked as having been factored. If the
solution is successfully obtained, i.e. the LAPACK routines return $0$
in the INFO argument, it marks the system has having been
factored and returns $0$, otherwise it prints a warning message and
returns INFO. The solve process changes $A$ and $X$. \\   

\indent {\em int setSize(void);} \\
Is used to construct a 1d integer array, {\em iPiv} that is needed by
the LAPACK solvers. It checks to see if current size of {\em iPiv} is
large enough, if not it deletes the cold and creates a larger
array. Returns $0$ if sucessfull, prints a warning message and returns
a $-1$ if not enough memory is available for this new array. \\


\indent {\em  int sendSelf(int commitTag, Channel \&theChannel);} \\ 
Does nothing but return $0$. \\

\indent {\em  int recvSelf(int commitTag, Channel \&theChannel, FEM\_ObjectBroker
\&theBroker);} \\ 
Does nothing but return $0$. \\


\pagebreak
\subsubsection{{\bf BandSPDLinSolver}}
%File: ~/OOP/system_of_eqn/linearSOE/bandSPD/BandSPDLinSolver.tex
%What: "@(#) BandSPDLinSolver.tex, revA"

\noindent {\bf Files}   \\
\indent \#include $<\tilde{ }$/system\_of\_eqn/linearSOE/bandSPD/BandSPDLinSolver.h$>$  \\

\noindent {\bf Class Declaration}  \\
\indent class BandSPDLinSolver: public LinearSOESolver  \\

\noindent {\bf Class Hierarchy} \\
\indent MovableObject \\
\indent\indent  Solver \\
\indent\indent\indent LinearSOESolver \\
\indent\indent\indent\indent {\bf BandSPDLinSolver} \\

\noindent {\bf Description}  \\
\indent BandSPDLinSolver is an abstract class. The BandSPDLinSolver
class provides access for each subclass to the BandSPDLinSOE object
through the pointer {\em theSOE}, which is a protected pointer. \\

\noindent {\bf Interface}  \\
\indent\indent // Constructor \\
\indent\indent {\em BandSPDLinSolver(int classTag);}  \\ \\
\indent\indent // Destructor \\
\indent\indent {\em virtual~ $\tilde{}$BandSPDLinSolver();}\\  \\
\indent\indent // Public Methods \\
\indent\indent {\em virtual int setLinearSOE(BandSPDLinSOE \&theSOE);} \\

\noindent {\bf Constructor}  \\
\indent {\em BandSPDLinSolver(int classTag);}  \\
The integer {\em classTag} is passed to the LinearSOESolver classes
constructor. \\ 

\noindent {\bf Destructor} \\
\indent {\em virtual~ $\tilde{}$BandSPDLinSolver();}\\ 
Does nothing. Provided so the subclasses destructor will be called. \\

\noindent {\bf Public Methods}  \\
\indent {\em virtual int setLinearSOE(BandSPDLinSOE \&theSOE);} \\
The method sets up the link between the BandSPDLinSOE object and the
BandSPDLinSolver, that it is sets the pointer the subclasses use.  \\







\pagebreak
\subsubsection{BandSPDLinLapackSolver}
%File: ~/OOP/system_of_eqn/linearSOE/bandSPD/BandSPDLinLapackSolver.tex
%What: "@(#) BandSPDLinLapackSolver.tex, revA"

\noindent {\bf Files}   \\
\indent \#include $<\tilde{ }$/system\_of\_eqn/linearSOE/bandSPD/BandSPDLinLapackSolver.h$>$  \\

\noindent {\bf Class Declaration}  \\
\indent class BandSPDLinLapackSolver: public BandSPDLinSolver  \\

\noindent {\bf Class Hierarchy} \\
\indent MovableObject \\
\indent\indent  Solver \\
\indent\indent\indent LinearSOESolver \\
\indent\indent\indent\indent BandSPDLinSolver \\
\indent\indent\indent\indent\indent {\bf BandSPDLinLapackSolver} \\

\noindent {\bf Description}  \\
\indent A BandSPDLinLapackSolver object can be constructed to solve
a BandSPDLinSOE object. It obtains the solution by making calls on the
the LAPACK library. The class is defined to be a friend of the 
BandSPDLinSOE class (see $<$BandSPDLinSOE.h$>$). \\


\noindent {\bf Interface}  \\
\indent\indent // Constructor \\
\indent\indent {\em BandSPDLinLapackSolver();}  \\ \\
\indent\indent // Destructor \\
\indent\indent {\em $\tilde{ }$BandSPDLinLapackSolver();}\\  \\
\indent\indent // Public Methods \\
\indent\indent {\em int solve(void);} \\
\indent\indent {\em int setSize(void);} \\
\indent\indent {\em int sendSelf(int commitTag, Channel \&theChannel);}\\ 
\indent\indent {\em int recvSelf(int commitTag, Channel \&theChannel,
FEM\_ObjectBroker \&theBroker);}\\ 


\noindent {\bf Constructor}  \\
\indent {\em BandSPDLinLapackSolver();}  \\
A unique class tag (defined in $<$classTags.h$>$) is passed to the
BandSPDLinSolver constructor. \\


\noindent {\bf Destructor} \\
\indent {\em  $\tilde{ }$BandSPDLinLapackSolver();}\\ 
Does nothing. \\

\noindent {\bf Public Member Functions }  \\
\indent {\em virtual int solve(void);} \\
The solver first copies the B vector into X and then solves the
BandSPDLinSOE system by calling the LAPACK routines {\em 
dpbsv()}, if the system is marked as not having been factored,
and {\em dpbtrs()} if system is marked as having been factored. 
If the solution is successfully obtained, i.e. the LAPACK routines
return $0$ in the INFO argument, it marks the system has having been 
factored and returns $0$, otherwise it prints a warning message and
returns INFO. The solve process changes $A$ and $X$. \\   


\indent {\em int setSize(void);} \\
Does nothing but return $0$. \\

\indent {\em  int sendSelf(int commitTag, Channel \&theChannel);} \\ 
Does nothing but return $0$. \\

\indent {\em  int recvSelf(int commitTag, Channel \&theChannel, FEM\_ObjectBroker
\&theBroker);} \\ 
Does nothing but return $0$. \\





\pagebreak
\subsubsection{{\bf ProfileSPDLinSolver}}
%File: ~/OOP/system_of_eqn/linearSOE/ProfileSPD/ProfileSPDLinSolver.tex
%What: "@(#) ProfileSPDLinSolver.tex, revA"

\noindent {\bf ProfileSPDLinSolver} \\

\noindent {\bf Files}   \\
\indent \#include $<\tilde{ }$ProfileSPDLinSolver.h$>$  \\

\noindent {\bf Class Decleration}  \\
\indent class ProfileSPDLinSolver: public LinearSOESolver  \\

\noindent {\bf Class Hierarchy} \\
\indent  Solver \\
\indent\indent LinearSOESolver \\
\indent\indent\indent {\bf ProfileSPDLinSolver} \\

\noindent {\bf Description}  \\
\indent ProfileSPDLinSolver is an abstract class.  The ProfileSPDLinSolver
class provides access for each subclass to the ProfileSPDLinSOE object
through the pointer {\em theSOE}, which is a protected pointer. \\

\noindent {\bf Interface}  \\
\indent\indent // Constructor \\
\indent\indent {\em ProfileSPDLinSolver(int classTag);}  \\ \\
\indent\indent // Destructor \\
\indent\indent {\em virtual~ $\tilde{}$ProfileSPDLinSolver();}\\  \\
\indent\indent // Public Methods \\
\indent\indent {\em virtual int setLinearSOE(ProfileSPDLinSOE \&theSOE);} \\

\noindent {\bf Constructor}  \\
\indent {\em ProfileSPDLinSolver(int classTag);}  \\
The integer {\em classTag} is passed to the LinearSOESolver classes
constructor. \\ 

\noindent {\bf Destructor} \\
\indent {\em virtual~ $\tilde{}$ProfileSPDLinSolver();}\\ 
Does nothing, provided so the subclasses destructor will be called. \\

\noindent {\bf Public Methods }  \\
\indent {\em virtual int setLinearSOE(ProfileSPDLinSOE \&theSOE);} \\
The method sets up the link between the ProfileSPDLinSOE object and the
ProfileSPDLinSolver, that it is sets the pointer the subclasses use.  \\

\pagebreak
\subsubsection{ProfileSPDLinDirectSolver}
%File: ~/OOP/system_of_eqn/linearSOE/profileSPD/ProfileSPDLinDirectSolver.tex
%What: "@(#) ProfileSPDLinDirectSolver.tex, revA"

UNDER CONSTRUCTION

\noindent {\bf Files}   \\
\indent \#include $<\tilde{
}$/system\_of\_eqn/linearSOE/profileSPD/ProfileSPDLinDirectSolver.h$>$
\\ 

\noindent {\bf Class Declaration}  \\
\indent class ProfileSPDLinDirectSolver: public LinearSOESolver  \\

\noindent {\bf Class Hierarchy} \\
\indent MovableObject \\
\indent\indent  Solver \\
\indent\indent\indent LinearSOESolver \\
\indent\indent\indent\indent ProfileSPDLinSolver \\
\indent\indent\indent\indent\indent {\bf ProfileSPDLinDirectSolver} \\

\noindent {\bf Description}  \\
\indent A ProfileSPDLinDirectSolver object can be constructed to solve
a ProfileSPDLinSOE object. It does this by direct means, using the
$LDL^t$ variation of the cholesky factorization. The matrx $A$ is
factored one column at a time using a left-looking approach. No BLAS
or LAPACK routines are called for the factorization or subsequent
substitution. \\ 

\noindent {\bf Interface}  \\
\indent\indent Constructor \\
\indent\indent {\em ProfileSPDLinDirectSolver();}  \\ \\
\indent\indent Destructor \\
\indent\indent {\em  $\tilde{ }$ProfileSPDLinDirectySolver();}\\  \\
\indent\indent Public Methods \\
\indent\indent {\em int solve(void);} \\
\indent\indent {\em int setSize(void);} \\
\indent\indent {\em int sendSelf(int commitTag, Channel \&theChannel);}\\ 
\indent\indent {\em int recvSelf(int commitTag, Channel \&theChannel,
FEM\_ObjectBroker \&theBroker);}\\ 


\noindent {\bf Constructor}  \\
\indent {\em ProfileSPDLinDirectSolver();}  \\
A unique class tag (defined in $<$classTags.h$>$) is passed to the
ProfileSPDLinSolver constructor. \\


\noindent {\bf Destructor} \\
\indent {\em $\tilde{ }$ProfileSPDLinDirectSolver();}\\ 
Does nothing. \\

\noindent {\bf Public Member Functions }  \\
\indent {\em virtual int solve(void);} \\
The solver first copies the B vector into X.
FILL IN
The solve process changes $A$ and $X$. \\   


\indent {\em int setSize(void);} \\
Does nothing but return $0$. \\

\indent {\em  int sendSelf(int commitTag, Channel \&theChannel);} \\ 
Does nothing but return $0$. \\

\indent {\em  int recvSelf(int commitTag, Channel \&theChannel, FEM\_ObjectBroker
\&theBroker);} \\ 
Does nothing but return $0$. \\









\pagebreak
\subsubsection{ProfileSPDLinDirectBlockSolver}
%File: ~/OOP/system_of_eqn/linearSOE/profileSPD/ProfileSPDLinDirectBlockSolver.tex
%What: "@(#) ProfileSPDLinDirectBlockSolver.tex, revA"

UNDER CONSTRUCTION 

\noindent {\bf Files}   \\
\indent \#include $<\tilde{
}$/system\_of\_eqn/linearSOE/profileSPD/ProfileSPDLinDirectBlockSolver.h$>$
\\ 

\noindent {\bf Class Decleration}  \\
\indent class ProfileSPDLinDirectBlockSolver: public LinearSOESolver  \\

\noindent {\bf Class Hierarchy} \\
\indent MovableObject \\
\indent\indent  Solver \\
\indent\indent\indent LinearSOESolver \\
\indent\indent\indent\indent ProfileSPDLinSolver \\
\indent\indent\indent\indent\indent {\bf ProfileSPDLinDirectBlockSolver} \\

\noindent {\bf Description}  \\
\indent A ProfileSPDLinDirectBlockSolver object can be constructed to
solve a ProfileSPDLinSOE object. It does this by direct means, using
the $LDL^t$ variation of the cholesky factorization. The matrx $A$ is
factored one block row at a time using a right-looking approach. No BLAS
or LAPACK routines are called for the factorization or subsequent substitution. \\

\noindent {\bf Interface}  \\
\indent\indent Constructor \\
\indent\indent {\em ProfileSPDLinDirectBlockSolver();}  \\ \\
\indent\indent Destructor \\
\indent\indent {\em $\tilde{ }$ProfileSPDLinDirectySolver();}\\  \\
\indent\indent Public Methods \\
\indent\indent {\em int solve(void);} \\
\indent\indent {\em int setSize(void);} \\
\indent\indent {\em int sendSelf(int commitTag, Channel \&theChannel);}\\ 
\indent\indent {\em int recvSelf(int commitTag, Channel \&theChannel,
FEM\_ObjectBroker \&theBroker);}\\ 


\noindent {\bf Constructor}  \\
\indent {\em ProfileSPDLinDierctBlockSolver();}  \\
A unique class tag (defined in $<$classTags.h$>$) is passed to the
ProfileSPDLinSolver constructor. \\


\noindent {\bf Destructor} \\
\indent {\em $\tilde{ }$ProfileSPDLinDierctBlockSolver();}\\ 
Does nothing. \\

\noindent {\bf Public Member Functions }  \\
\indent {\em int solve(void);} \\
The solver first copies the B vector into X.
FILL IN
The solve process changes $A$ and $X$. \\   


\indent {\em int setSize(void);} \\
Does nothing but return $0$. \\

\indent {\em  int sendSelf(int commitTag, Channel \&theChannel);} \\ 
Does nothing but return $0$. \\

\indent {\em  int recvSelf(int commitTag, Channel \&theChannel, FEM\_ObjectBroker
\&theBroker);} \\ 
Does nothing but return $0$. \\









\pagebreak
\subsubsection{ProfileSPDLinDirectThreadSolver}
%File: ~/OOP/system_of_eqn/linearSOE/profileSPD/ProfileSPDLinDirectThreadSolver.tex
%What: "@(#) ProfileSPDLinDirectThreadSolver.tex, revA"

UNDER CONSTRUCTION

\noindent {\bf Files}   \\
\indent \#include $<\tilde{
}$/system\_of\_eqn/linearSOE/profileSPD/ProfileSPDLinDirectThreadSolver.h$>$
\\ 

\noindent {\bf Class Decleration}  \\
\indent class ProfileSPDLinDirectThreadSolver: public LinearSOESolver  \\

\noindent {\bf Class Hierarchy} \\
\indent MovableObject \\
\indent\indent  Solver \\
\indent\indent\indent LinearSOESolver \\
\indent\indent\indent\indent ProfileSPDLinSolver \\
\indent\indent\indent\indent\indent {\bf ProfileSPDLinDirectThreadSolver} \\

\noindent {\bf Description}  \\
\indent A ProfileSPDLinDirectThreadSolver object can be constructed to
solve a ProfileSPDLinSOE object. It does this in parallel using
threads by direct means, using the $LDL^t$ variation of the cholesky
factorization. The matrx $A$ is factored one row block at a time using
a left-looking approach. Within a row block the factorization is
performed by $NP$ threads. No BLAS or LAPACK routines are called 
for the factorization or subsequent substitution. \\

\noindent {\bf Interface}  \\
\indent\indent Constructor \\
\indent\indent {\em ProfileSPDLinDirectThreadSolver(int numThreads);}  \\ \\
\indent\indent Destructor \\
\indent\indent {\em $\tilde{ }$ProfileSPDLinDirectThreadySolver();}\\  \\
\indent\indent Public Methods \\
\indent\indent {\em int solve(void);} \\
\indent\indent {\em  int setSize(void);} \\
\indent\indent {\em  int sendSelf(Channel \&theChannel, FEM\_ObjectBroker
\&theBroker);} \\ 
\indent\indent {\em  int recvSelf(Channel \&theChannel, FEM\_ObjectBroker
\&theBroker);} \\ 


\noindent {\bf Constructor}  \\
\indent {\em ProfileSPDLinDirectThreadSolver(int numThreads);}  \\
A unique class tag (defined in $<$classTags.h$>$) is passed to the
ProfileSPDLinSolver constructor. \\


\noindent {\bf Destructor} \\
\indent {\em $\tilde{ }$ProfileSPDLinDirectThreadSolver();}\\ 
Does nothing. \\

\noindent {\bf Public Member Functions }  \\
\indent {\em  int solve(void);} \\
The solver first copies the B vector into X.
FILL IN
The solve process changes $A$ and $X$. \\   

\indent {\em  int setSize(void);} \\
Does nothing but return $0$. \\

\indent {\em  int sendSelf(Channel \&theChannel, FEM\_ObjectBroker
\&theBroker);} \\ 
Does nothing but return $0$. \\

\indent {\em  int recvSelf(Channel \&theChannel, FEM\_ObjectBroker
\&theBroker);} \\ 
Does nothing but return $0$. \\









\pagebreak
\subsubsection{ProfileSPDLinDirectSkypackSolver}
%File: ~/OOP/system_of_eqn/linearSOE/bandSPD/ProfileSPDLinDirectSkypackSolver.tex
%What: "@(#) ProfileSPDLinDirectSkypackSolver.tex, revA"

\noindent {\bf Files}   \\
\indent \#include $<\tilde{ }$/system\_of\_eqn/linearSOE/profileSPD/ProfileSPDLinDirectSkypackSolver.h$>$  \\

\noindent {\bf Class Decleration}  \\
\indent class ProfileSPDLinDirectSkypackSolver: public LinearSOESolver  \\

\noindent {\bf Class Hierarchy} \\
\indent MovableObject \\
\indent\indent  Solver \\
\indent\indent\indent LinearSOESolver \\
\indent\indent\indent\indent ProfileSPDLinSolver \\
\indent\indent\indent\indent\indent {\bf ProfileSPDLinDirectSkypackSolver} \\

\noindent {\bf Description}  \\
\indent A ProfileSPDLinDirectSkypackSolver object can be constructed
to solve a ProfileSPDLinSOE object. It does this by direct means using
the routines supplied in the SKYPACK library, a library which uses the
BLAS levels 1,2 and 3 for the factorization and substitution. 

The routines in SKYPACK require a number of work areas: {\em int
block[3]} and {\em double invD[size]}. In addition, to allow the use
of the BLAS 2 and 3, work areas {\em double rw[mRows x mCols]}, {\em
double tw[mRows x mRows]} and {\em int index[max(mCols,mRows)]} are 
created. \\


\noindent {\bf Interface}  \\
\indent\indent Constructors \\
\indent\indent {\em ProfileSPDLinDirectSkypackSolver();}  \\ 
\indent\indent {\em ProfileSPDLinDirectSkypackSolver(int mCols, mRows);}  \\ \\
\indent\indent Destructor \\
\indent\indent {\em  $\tilde{}$ProfileSPDLinDirectSkypackSolver();}\\  \\
\indent\indent Public Methods \\
\indent\indent {\em int solve(void);} \\
\indent\indent {\em  int setSize(void);} \\
\indent\indent {\em int sendSelf(int commitTag, Channel \&theChannel);}\\ 
\indent\indent {\em int recvSelf(int commitTag, Channel \&theChannel,
FEM\_ObjectBroker \&theBroker);}\\ 


\noindent {\bf Constructors}  \\
\indent {\em ProfileSPDLinDirectSkypackSolver();}  \\
A unique class tag (defined in $<$classTags.h$>$) is passed to the
ProfileSPDLinSolver constructor. Sets {\em mCols} and {\em mRows}
equal to $0$ and does not try and allocate any memory for the work arrays.\\

\indent {\em ProfileSPDLinDirectSkypackSolver(int mCols, int mRows);}  \\
A unique class tag (defined in $<$classTags.h$>$) is passed to the
ProfileSPDLinSolver constructor. Sets {\em mCols} and {\em mRows} and
allocates space in memory for the work arrays {\em rw}, {\em tw} and
{\em index}. If not enough memory is available in memory, {\em mCols}
and {\em mRows} is set equal to $0$ and an error message is printed. \\

\noindent {\bf Destructor} \\
\indent {\em $\tilde{}$ProfileSPDLinDirectSkypackSolver();}\\ 
Invokes delete on any work areas that have been constructed. \\

\noindent {\bf Public Member Functions }  \\
\indent {\em  int solve(void);} \\
The solver first copies the B vector into X and then solves the
BandSPDLinSOE system. If the matrix has not been factored, the matrix
is first factored using the SKYPACK routine {\em skysf2()}, if {\em
mCols} and {\em mRows} equal $0$, or {\em skypf2()}. {\em skysf2()} is
a routine which uses the BLAS level 1 routines, {\em skypf2()} is a
routine which uses BLAS levels 2 and 3. If {\em skypf2()}
has been called, {\em invD} is set up. Once the matrix has been
factored, {\em skyss()} is called. If the solution is sucessfully
obtained, i.e. the {\em skyss()} routine returns $0$ in the INFO
argument, $0$ is returned, otherwise it prints a warning message and
returns INFO. The solve process changes $A$ and $X$. \\   

\indent {\em  int setSize(void);} \\
Is responsible for setting the {\em block} information required by the
SKYPACK routines (block[0]=1; block[1]=size, block[2]=1) and for
creating space for the {\em invD} work array. Returns $0$ if
successfull, otherwise a warning message is printed and a $-1$ is
returned. \\

\indent {\em  int sendSelf(int commitTag, Channel \&theChannel);} \\ 
Does nothing but return $0$. \\

\indent {\em  int recvSelf(int commitTag, Channel \&theChannel, FEM\_ObjectBroker
\&theBroker);} \\ 
Does nothing but return $0$. \\









\pagebreak
\subsubsection{{\bf SparseGenColLinSolver}}
%File: ~/OOP/system_of_eqn/linearSOE/sparseGEN/SparseGenColLinSolver.tex
%What: "@(#) SparseGenColLinSolver.tex, revA"

\noindent {\bf Files}   \\
\indent \#include $<\tilde{ }$/system\_of\_eqn/linearSOE/SparseGen/SparseGenColLinSolver.h$>$  \\

\noindent {\bf Class Declaration}  \\
\indent class SparseGenColLinSolver: public LinearSOESolver  \\

\noindent {\bf Class Hierarchy} \\
\indent MovableObject \\
\indent\indent  Solver \\
\indent\indent\indent LinearSOESolver \\
\indent\indent\indent\indent {\bf SparseGenColLinSolver} \\

\noindent {\bf Description}  \\
\indent SparseGenColLinSolver is an abstract class.  The SparseGenColLinSolver
class provides access for each subclass to the SparseGenColLinSOE object
through the pointer {\em theSOE}, which is a protected pointer. \\

\noindent {\bf Interface}  \\
\indent\indent // Constructor \\
\indent\indent {\em SparseGenColLinSolver(int classTag);}  \\ \\
\indent\indent // Destructor \\
\indent\indent {\em virtual~ $\tilde{}$SparseGenColLinSolver();}\\  \\
\indent\indent // Public Methods \\
\indent\indent {\em virtual int setLinearSOE(SparseGenColLinSOE \&theSOE);} \\

\noindent {\bf Constructor}  \\
\indent {\em SparseGenColLinSolver(int classTag);}  \\
The integer {\em classTag} is passed to the LinearSOESolver classes
constructor. \\ 

\noindent {\bf Destructor} \\
\indent {\em virtual~ $\tilde{}$SparseGenColLinSolver();}\\ 
Does nothing, provided so the subclasses destructor will be called. \\

\noindent {\bf Public Methods }  \\
\indent {\em virtual int setLinearSOE(SparseGenColLinSOE \&theSOE);} \\
Sets the link to the SparseGenColLinSOE object {\em theSOE}. This is the
object on which the solver will perform the numerical computations. \\






\pagebreak
\subsubsection{SuperLU}
%File: ~/OOP/system_of_eqn/linearSOE/SparseGenCol/SuperLU.tex
%What: "@(#) SuperLU.tex, revA"

\noindent {\bf Files}   \\
\indent \#include $<\tilde{
}$/system\_of\_eqn/linearSOE/fullGEN/SuperLU.h$>$  \\ 

\noindent {\bf Class Declaration}  \\
\indent class SuperLU: public SparseGenColLinSolver  \\

\noindent {\bf Class Hierarchy} \\
\indent  MovableObject \\
\indent\indent  Solver \\
\indent\indent\indent LinearSOESolver \\
\indent\indent\indent\indent  SparseGenColLinSolver \\
\indent\indent\indent\indent\indent {\bf SuperLU} \\

\noindent {\bf Description}  \\
\indent A SuperLU object can be constructed to solve
a SparseGenColLinSOE object. It obtains the solution by making calls on the
the SuperLU library developed at UC Berkeley by Prof. James Demmel, 
Xiaoye S. Li and John R. Gilbert.
The SuperLU library contains a set of subroutines to solve a sparse
linear system  $AX=B$. It uses Gaussian elimination with partial
pivoting (GEPP). The columns of A may be preordered before
factorization; the preordering for sparsity is completely separate
from the factorization and a number of ordering schemes are provided. \\

\noindent {\bf Interface}  \\
\indent // Constructor \\
\indent {\em SuperLU();}  \\ \\
\indent // Destructor \\
\indent {\em $\tilde{ }$SuperLU();}\\  \\
\indent // Public Methods \\
\indent {\em int solve(void);} \\
\indent {\em int setSize(void);} \\
\indent {\em int sendSelf(int commitTag, Channel \&theChannel);}\\ 
\indent {\em int recvSelf(int commitTag, Channel \&theChannel,
FEM\_ObjectBroker \&theBroker);}\\ 

\noindent {\bf Constructor}  \\
\indent {\em SuperLU(int permSpec =0, double thresh = 0.0, int panelSize =6,
int relax = 6);}  \\
A unique class tag (defined in $<$classTags.h$>$) is passed to the
SparseGenColLinSolver constructor. Saves the values for the arguments
{\em permSpec}, {\em panelSize}, {\em relax} and {\em thresh} that
will be used when calling the SuperLU routines in {\em solve()} and
{\em setSize()}.

{\em permSpec} defines the ordering routine used in defining the
column permutations {\em permC}: $0$ uses the original ordering
supplied, $1$ defines a min-degree ordering based on $A^TA$ and $2$ a
min-degree ordering based on $A^T + A$. {\em relax} defines the min
number of columns in a subtree for the subtree to be considered a
single supernode. {\em thresh} defines the pivoting threshold: at
step j of the Gaussian elimination if (abs$(A_{jj}) \ge$ {\em thresh}
(max$ i \ge j$ abs($A_{ij}$)). A value for {\em thresh} of $0.0$
definines no pivoting, a value of $1.0$ classical partial pivoting.
{\em panelSize} defines the number of consecutive columns used as a
panel in the elimination. For more information on these values see the
SuperLU manual. \\


\noindent {\bf Destructor} \\
\indent {\em  $\tilde{ }$SuperLU();}\\ 
Invokes delete on {\em permR}, {\em permC} and {\em etree} arrays. \\


\noindent {\bf Public Methods }  \\
\indent {\em int solve(void);} \\
First copies $B$ into $X$ and then solves the FullGenLinSOE system 
it is associated with (pointer kept by parent class) by calling the SeuperLU
routine {\em dgstrf()}, if the system is marked as not having been factored,
or {\em dgstrs()}, if system is marked as having been factored. If the
solution is successfully obtained, i.e. the SuperLU routines return $0$
in the INFO argument, it marks the system has having been
factored and returns $0$, otherwise it prints a warning message and
returns INFO. The solve process changes $A$ and $X$ and sets the char
{\em rafact} to {\em Y}. \\   

\indent {\em int setSize(void);} \\
Obtains the size of the system from it's associaed SparseGenColLinSOE
object. With this information it creates space for the integer arrays
{\em permR}, {\em permC} and {\em etree}. It then creates the
a SuperMatrix for A by calling the SuperLU routine {\em
dCreate\_CompCol\_Matrix()}, sets the column permutation {\em permR}
by calling the SuperLU routine {\em get\_perm\_c(permSpec, A, permC)},
applies this permutation and determines the elimination tree {\em
etree} by calling the SuperLU routine {\em sp\_preorder()}. It then
creates a SuperMatrix for X by calling the SuperLU routine 
{\em dCreate\_Dense\_Matrix()}.
Returns $0$ if successful, prints a warning message and returns
a $-1$ if not enough memory is available for the arrays. \\


\indent {\em  int sendSelf(int commitTag, Channel \&theChannel);} \\ 
Does nothing but return $0$. \\

\indent {\em  int recvSelf(int commitTag, Channel \&theChannel, FEM\_ObjectBroker
\&theBroker);} \\ 
Does nothing but return $0$. \\










\pagebreak
\subsubsection{ThreadedSuperLU}
%\input{../system_of_eqn/linearSOE/sparseGEN/ThreadedSuperLU}

\pagebreak
\subsubsection{{\bf SparseGenRowLinSolver}}
%%File: ~/OOP/system_of_eqn/linearSOE/sparseGEN/SparseGenColLinSolver.tex
%What: "@(#) SparseGenColLinSolver.tex, revA"

\noindent {\bf Files}   \\
\indent \#include $<\tilde{ }$/system\_of\_eqn/linearSOE/SparseGen/SparseGenColLinSolver.h$>$  \\

\noindent {\bf Class Declaration}  \\
\indent class SparseGenColLinSolver: public LinearSOESolver  \\

\noindent {\bf Class Hierarchy} \\
\indent MovableObject \\
\indent\indent  Solver \\
\indent\indent\indent LinearSOESolver \\
\indent\indent\indent\indent {\bf SparseGenColLinSolver} \\

\noindent {\bf Description}  \\
\indent SparseGenColLinSolver is an abstract class.  The SparseGenColLinSolver
class provides access for each subclass to the SparseGenColLinSOE object
through the pointer {\em theSOE}, which is a protected pointer. \\

\noindent {\bf Interface}  \\
\indent\indent // Constructor \\
\indent\indent {\em SparseGenColLinSolver(int classTag);}  \\ \\
\indent\indent // Destructor \\
\indent\indent {\em virtual~ $\tilde{}$SparseGenColLinSolver();}\\  \\
\indent\indent // Public Methods \\
\indent\indent {\em virtual int setLinearSOE(SparseGenColLinSOE \&theSOE);} \\

\noindent {\bf Constructor}  \\
\indent {\em SparseGenColLinSolver(int classTag);}  \\
The integer {\em classTag} is passed to the LinearSOESolver classes
constructor. \\ 

\noindent {\bf Destructor} \\
\indent {\em virtual~ $\tilde{}$SparseGenColLinSolver();}\\ 
Does nothing, provided so the subclasses destructor will be called. \\

\noindent {\bf Public Methods }  \\
\indent {\em virtual int setLinearSOE(SparseGenColLinSOE \&theSOE);} \\
Sets the link to the SparseGenColLinSOE object {\em theSOE}. This is the
object on which the solver will perform the numerical computations. \\






\pagebreak
\subsubsection{UmfpackGenLinSolver}
UNDER CONSTRUCTION. \\

\pagebreak
\subsubsection{SymSparseLinSolver}
UNDER CONSTRUCTION. \\


%\pagebreak
%\subsubsection{{\bf DomainSolver}}
%%File: ~/OOP/system_of_eqn/linearSOE/DomainSolver.tex
%What: "@(#) DomainSolver.tex, revA"

\noindent {\bf Files}   \\
\indent \#include $<\tilde{ }$/system\_of\_eqn/linearSOE/DomainSolver.h$>$  \\

\noindent {\bf Class Declaration}  \\
\indent class DomainSolver: public LinearSOESolver  \\

\noindent {\bf Class Hierarchy} \\
\indent  MovableObject \\
\indent\indent  Solver \\
\indent\indent\indent  LinearSOESolver \\
\indent\indent\indent\indent {\bf DomainSolver} \\

\noindent {\bf Description}  \\
\indent DomainSolver is an abstract class. DomainSolver objects
are responsible for performing the numerical operations required for
the domain decomposition methods.  \\

\noindent {\bf Interface}  \\
\indent\indent // Constructor  \\
\indent\indent {\em DomainSolver(int classsTag);}  \\ \\
\indent\indent // Destructor  \\
\indent\indent {\em virtual~ $\tilde{}$DomainSolver();}\\ \\
\indent\indent // Public Methods \\
\indent\indent {\em virtual int condenseA(int numInt) =0;} \\
\indent\indent {\em virtual int condenseRHS(int numInt, Vector *u = 0) =0;} \\
\indent\indent {\em virtual int computeCondensedMatVect(int numInt, Vector \&u) =0;} \\
\indent\indent {\em virtual Matrix \&getCondensedA(void) =0;} \\
\indent\indent {\em virtual Vector \&getCondensedRHS(void) =0;} \\
\indent\indent {\em virtual Vector \&getCondensedMatVect(void) =0;} \\
\indent\indent {\em virtual int setComputedXext(const Vector \&u) =0;} \\
\indent\indent {\em virtual int solveXint(void) =0;} \\






\noindent {\bf Constructor}  \\
\indent {\em DomainSolver(int classsTag);}  \\
{\em classTag} is needed by the LinearSOESolver objects constructor. \\

\noindent {\bf Destructor} \\
\indent {\em virtual~ $\tilde{}$DomainSolver();}\\ 

\noindent {\bf Public Methods}  \\
\indent {\em virtual int condenseA(int numInt) =0;} \\
Causes the condenser to form $A_{ee}^* = A_{ee} -A_{ei} A_{ii}^{-1} A_{ie}$, where
$A_{ii}$ is the first {\em numInt} rows of the $A$ matrix.  The
original $A$ is changed as a result. $A_{ee}^*$ is to be stored in $A_{ee}$. \\

{\em virtual int condenseRHS(int numInt) =0;} \\
Causes the condenser to form $B_e^* = B_e - A_{ei} A_{ii}^{-1} B_i$, where $A_{ii}$ 
is the first {\em numInt} rows of $A$. The original $B$ is changed as a result. 
$B_e^*$ is to be stored in $B_e$. \\

{\em virtual int computeCondensedMatVect(Vector \&u, int numInt) =0;} \\
Causes the condenser to form $A_{ee} u$. \\

{\em virtual Matrix \&getCondensedA(void) =0;} \\
Returns the contents of $A_{ee}$ as a matrix. \\

{\em virtual Vector \&getCondensedRHS(void) =0;} \\
Returns the contents of $B_e$ as a Vector. \\

{\em virtual Vector \&getCondensedMatVect(void) =0;} \\
Returns the contents of the last call to {\em
computeCondensedMatVect()}. \\

{\em virtual int setComputedXext(const Vector \&u) =0;} \\
Sets the computed value of the unknowns in $X_e$ corresponding to the
external equations to {\em u}. The number of external equations is
given by the size of vector $u$.\\

{\em  virtual int solveXint(void) =0;} \\
To compute the internal equation numbers $X_i$ given the value set
for the external equations in the last call to {\em setComputedXext()}. \\



%\pagebreak
%\subsubsection{ProfileSPDLinSubstrSolver}
%%File: ~/OOP/system_of_eqn/linearSOE/profileSPD/ProfileSPDLinSubstrSolver.tex
%What: "@(#) ProfileSPDLinSubstrSolver.tex, revA"

UNDER CONSTRUCTION

\noindent {\bf Files}   \\
\indent \#include $<\tilde{
}$/system\_of\_eqn/linearSOE/profileSPD/ProfileSPDLinSubstrSolver.h$>$ \\ 

\noindent {\bf Class Declaration}  \\
\indent class ProfileSPDLinSubstrSolver: public DomainSolver, public
ProfileSPDLinSubstrSolver  \\ 

\noindent {\bf Class Hierarchy} \\
\indent  MovableObject \\
\indent\indent  Solver \\
\indent\indent\indent  LinearSOESolver \\
\indent\indent\indent\indent  DomainSolver \\
\indent\indent\indent\indent  ProfileSPDLinDirectSolver \\
\indent\indent\indent\indent\indent {\bf ProfileSPDLinSubstrSolver} \\

\noindent {\bf Description}  \\
\indent A ProfileSPDLinSubstrSolver object will perform the numerical
substructuring operations on a ProfileSPDLinSOE object. EXPAND. \\

\noindent {\bf Interface}  \\
\indent\indent // Constructor  \\
\indent\indent {\em ProfileSPDLinSubstrSolver(double tol = 1.0e-12);}  \\ \\
\indent\indent // Destructor  \\
\indent\indent {\em $\tilde{ }$ProfileSPDLinSubstrSolver();}\\ \\
\indent\indent // Public Methods \\
\indent\indent {\em int condenseA(int numInt);} \\
\indent\indent {\em int condenseRHS(int numInt, Vector *u = 0);} \\
\indent\indent {\em int computeCondensedMatVect(int numInt, Vector \&u);} \\
\indent\indent {\em Matrix \&getCondensedA(void);} \\
\indent\indent {\em Vector \&getCondensedRHS(void);} \\
\indent\indent {\em Vector \&getCondensedMatVect(void);} \\
\indent\indent {\em int setComputedXext(const Vector \&u);} \\
\indent\indent {\em  int solveXint(void);} \\






\noindent {\bf Constructor}  \\
\indent {\em ProfileSPDLinSubstrSolver(double tol = 1.0e-12);}  \\

\noindent {\bf Destructor} \\
\indent {\em $\tilde{ }$ProfileSPDLinSubstrSolver();}\\ 

\noindent {\bf Public Methods}  \\
\indent {\em int condenseA(int numInt);} \\
Causes the condenser to form $A_{ee}^* = A_{ee} -A_{ei} A_{ii}^{-1} A_{ie}$, where
$A_{ii}$ is the first {\em numInt} rows of the $A$ matrix.  The
original $A$ is changed as a result. $A_{ee}^*$ is to be stored in $A_{ee}$. \\

{\em int condenseRHS(int numInt);} \\
Causes the condenser to form $B_e^* = B_e - A_{ei} A_{ii}^{-1} B_i$, where $A_{ii}$ 
is the first {\em numInt} rows of $A$. The original $B$ is changed as a result. 
$B_e^*$ is to be stored in $B_e$. \\

{\em int computeCondensedMatVect(Vector \&u, int numInt);} \\
Causes the condenser to form $A_{ee} u$. \\

{\em Matrix \&getCondensedA(void);} \\
Returns the contents of $A_{ee}$ as a matrix. \\

{\em Vector \&getCondensedRHS(void);} \\
Returns the contents of $B_e$ as a Vector. \\

{\em Vector \&getCondensedMatVect(void);} \\
Returns the contents of the last call to {\em
computeCondensedMatVect()}. \\

{\em int setComputedXext(const Vector \&u);} \\
Sets the computed value of the unknowns in $X_e$ corresponding to the
external equations to {\em u}. The number of external equations is
given by the size of vector $u$.\\

{\em  int solveXint(void);} \\
To compute the internal equation numbers $X_i$ given the value set
for the external equations in the last call to {\em setComputedXext()}. \\







\pagebreak
\section{Material Classes}

Material classes are used to provide information to the Elements about
the material. There is one main class MaterialModel. The Element
objects query the MaterialModel objects to obtain the current value of
stress and the tangent defining the stress-strain relationship for the
current value of strain at the point in the domain represented by the
MaterialModel object.  

\pagebreak 
\subsection{{\bf MaterialModel}}
%File: ~/OOP/material/Material.tex
%What: "@(#) Material.tex, revA"

PRESENTLY LITTLE IN THE INTERFACE .. THIS MAY CHANGE IF MAKE GENERAL
i.e 1D, 2D and 3d PROBLEMS RETURN MATRICES AND VECTORS .. IF CHANGE,
INTERFACE FOR UniaxialMaterial MAY THEN CHANGE. \\ 

\noindent {\bf Files}   \\
\indent \#include $<\tilde{ }$/material/Material.h$>$  \\

\noindent {\bf Class Declaration}  \\
\indent class Material: public TaggedObject, public MovableObject \\

\noindent {\bf Class Hierarchy} \\
\indent TaggedObject \\
\indent MovableObject \\
\indent\indent {\bf Material} \\

\noindent {\bf Description}  \\
\indent Material is an abstract class. The Material class
provides the interface that all Material writers must provide
when introducing new Material subclasses. A Material object
is responsible for keeping track of stress, strain and the
constitution for a particular point in the domain. \\ 

\noindent {\bf Class Interface} \\
\indent // Constructor \\
\indent {\em Material (int tag, int classTag);}  \\ \\
\indent // Destructor \\
\indent {\em virtual~ $\tilde{}$Material ();}\\ \\

\noindent {\bf Constructor}  \\
\indent {\em Material (int tag, int classTag);}  \\
To construct a Material whose unique integer among Materials in the
domain is given by {\em tag}, and whose class identifier is given
by {\em classTag}. These integers are passed to the TaggedObject and
MovableObject class constructors. \\

\noindent {\bf Destructor} \\
\indent {\em virtual~ $\tilde{}$Material ();}\\ 






\pagebreak \subsubsection{{\bf UniaxialMaterial}}
%File: ~/OOP/material/UniaxialMaterial.tex
%What: "@(#) UniaxialMaterial.tex, revA"

INTERFACE MAY CHANGE IF MAKE MATERIAL MORE GENERAL. \\

\noindent {\bf Files}   \\
\indent \#include $<\tilde{ }$/material/UniaxialMaterial.h$>$  \\

\noindent {\bf Class Declaration}  \\
\indent class UniaxialMaterial: public Material \\

\noindent {\bf Class Hierarchy} \\
\indent TaggedObject \\
\indent MovableObject \\
\indent\indent Material \\
\indent\indent\indent {\bf UniaxialMaterial} \\

\noindent {\bf Description}  \\
\indent UniaxialMaterial is an abstract class. The
UniaxialMaterial class provides the interface that all
UniaxialMaterial writers must provide when introducing new
UniaxialMaterial subclasses. A UniaxialMaterial object 
is responsible for keeping track of stress, strain and the
constitution for a particular point in the domain. \\ 

\noindent {\bf Class Interface} \\
\indent // Constructor \\
\indent {\em UniaxialMaterial (int tag, int classTag);}  \\ \\
\indent // Destructor \\
\indent {\em virtual $\tilde{ }$UniaxialMaterial ();}\\ \\
\indent // Public Methods \\
\indent {\em virtual int setTrialStrain (double strain) = 0; } \\
\indent {\em virtual double getStress (void) = 0; } \\
\indent {\em virtual double getTangent (void) = 0; } \\
\indent {\em virtual int commitState (void) = 0; } \\
\indent {\em virtual int revertToLastCommit (void) = 0; } \\
\indent {\em virtual int revertToStart (void) = 0; } \\
\indent {\em virtual UniaxialMaterial *getCopy (void) = 0; } \\


\noindent {\bf Constructor}  \\
\indent {\em UniaxialMaterial (int tag, int classTag);}  \\
To construct a UniaxialMaterial whose unique integer among
UniaxialMaterials in the domain is given by {\em tag}, and whose class
identifier is given by {\em classTag}. These integers are passed to
the Material class constructor. \\

\noindent {\bf Destructor} \\
\indent {\em virtual $\tilde{ }$UniaxialMaterial ();} \\
Does nothing. \\ 

\noindent {\bf Public Methods} \\
\indent {\em virtual int setTrialStrain (double strain) = 0; }  \\
Sets the value of the trial strain, that value used by {\em
getStress()} and {\em getTangent()}, to be {\em strain}. To
return $0$ if successful, a negative number if not. \\

\indent {\em virtual double getStress (void) = 0; } \\
To return the current value of stress for the trial strain. \\

\indent {\em virtual double getTangent (void) = 0; } \\
To return the current value of the tangent for the trial strain. \\

\indent {\em virtual int commitState (void) = 0; } \\
To accept the current value of the trial strain as being on the
solution path. To return $0$ if successful, a negative number if not. \\

\indent {\em virtual int revertToLastCommit (void) = 0; } \\
To cause the material to revert to the state at the last commit. To
return $0$ if successful, a negative number if not. \\

\indent {\em virtual int revertToStart (void) = 0; } \\
Invoked to cause the material to revert to its original state in its
undeformed configuration. To return $0$ if successful, a negative
number if not. \\

\indent {\em virtual UniaxialMaterial *getCopy (void) = 0; } \\
To return an exact copy of the material. \\


\pagebreak \subsubsection{ElasticMaterial}
%File: ~/OOP/material/ElasticMaterial.tex
%What: "@(#) ElasticMaterial.tex, revA"

\noindent {\bf Files}   \\
\indent \#include $<\tilde{ }$/material/ElasticMaterial.h$>$  \\

\noindent {\bf Class Declaration}  \\
\indent class ElasticMaterial: public MaterialModel \\

\noindent {\bf Class Hierarchy} \\
\indent TaggedObject \\
\indent MovableObject \\
\indent\indent MaterialModel \\
\indent\indent\indent UniaxialMaterial \\
\indent\indent\indent\indent {\bf ElasticMaterial} \\

\noindent {\bf Description}  \\
\indent ElasticMaterial provides the abstraction of an elastic
uniaxial material, i.e. the stress-strain relationship is given by the
linear equation $\sigma = E * \epsilon$. \\

\noindent {\bf Class Interface} \\
\indent // Constructor \\
\indent {\em ElasticMaterial(int tag, double E);}  \\ \\
\indent // Destructor \\
\indent {\em $\tilde{ }$ElasticMaterial();}\\ \\
\indent // Public Methods \\
\indent {\em int setTrialStrain(double strain); } \\
\indent {\em double getStress(void); } \\
\indent {\em double getTangent(void); } \\
\indent {\em int commitState(void); } \\
\indent {\em int revertToLastCommit(void); } \\
\indent {\em int revertToStart(void); } \\
\indent {\em UniaxialMaterial *getCopy(void); } \\ \\
\indent // Public Methods for Output\\
\indent {\em int sendSelf(int commitTag, Channel \&theChannel); }\\
\indent {\em int recvSelf(int commitTag, Channel \&theChannel, 
FEM\_ObjectBroker \&theBroker); }\\
\indent {\em void Print(OPS_Stream \&s, int flag =0);} \\

\noindent {\bf Constructor}  \\
\indent {\em ElasticMaterial(int tag, int classTag);}  \\
To construct an ElasticMaterial with an integer identifier {\em
tag}, an elastic tangent modulus of $E$ and a current strain
$\epsilon$ of $0.0$. The integers {\em tag} and
MAT\_TAG\_ElasticMaterial, defined in $<$classTags.h$>$, are passed 
to the UniaxialMaterial classes constructor. \\

\noindent {\bf Destructor} \\
\indent {\em virtual~ $\tilde{}$ElasticMaterial();}\\ 
Does nothing. \\

\noindent {\bf Public Methods} \\
\indent {\em int setTrialStrain(double strain); }  \\
Sets the value of the trial strain, $\epsilon$ to be {\em
strain}. Returns $0$.\\

\indent {\em double getStress(void); } \\
Returns the product of $E * \epsilon$, where $\epsilon$ is the current
trial strain. \\

\indent {\em double getTangent(void); } \\
Returns the value of $E$ passed in the constructor. \\

\indent {\em int commitState(void); } \\
Returns $0$. \\

\indent {\em int revertToLastCommit(void); } \\
Returns $0$. \\

\indent {\em int revertToStart(void); } \\
Returns $0$. \\

\indent {\em UniaxialMaterial *getCopy(void); } \\
returns a pointer to a new ElasticMaterial object, constructed
using the same values of {\em tag} and $E$. It is up to the caller to
ensure that the destructor is invoked. \\

\indent {\em    int sendSelf(int commitTag, Channel \&theChannel); }\\
Creates a Vector of size $2$ into which it places  {\em tag} and
{\em E}. Invokes {\em sendVector()} on  {\em theChannel} using the
ElasticMaterialObjects {\em dbTag}, the integer {\em commitTag} and
the Vector as arguments. Returns $0$ if successful, a warning message
and a negative number are returned if the Channel object fails to send
the Vector. \\

\indent {\em    int recvSelf(int commitTag, Channel \&theChannel, 
		 FEM\_ObjectBroker \&theBroker); }\\
Creates a Vector of size $2$. Invokes {\em recvVector()} on  {\em
theChannel} using the ElasticMaterialObjects {\em dbTag}, the integer
{\em commitTag} and the Vector as arguments. Using the data in the
Vector to set it's {\em tag} and $E$. Returns $0$ if successful, a
warning message is printed, {\em tag} and $E$ are set to $0.0$, and a
negative number is returned if the Channel object fails to receive
the Vector. \\

\indent {\em    void Print(OPS_Stream \&s, int flag =0);} \\
Prints to the stream {\em s} the objects {\em tag} and $E$ values. \\


\pagebreak \subsubsection{ElasticPPMaterial}
%File: ~/OOP/material/ElasticPPModel.tex
%What: "@(#) ElasticPPModel.tex, revA"

UNDER CONSTRUCTION. \\

\noindent {\bf Files}   \\
\indent \#include $<\tilde{ }$/material/ElasticPPModel.h$>$  \\

\noindent {\bf Class Declaration}  \\
\indent class ElasticPPMaterial: public MaterialModel \\

\noindent {\bf Class Hierarchy} \\
\indent TaggedObject \\
\indent MovableObject \\
\indent\indent MaterialModel \\
\indent\indent\indent UniaxialMaterial \\
\indent\indent\indent\indent {\bf ElasticPPMaterial} \\

\noindent {\bf Description}  \\
\indent ElasticPPMaterial provides the abstraction of an elastic
perfectly plastic uniaxial material, i.e. the stress-strain
relationship is given by the linear equation $\sigma = E * \epsilon$
while the material is elastic and $| \sigma = E * \epsilon_p |$  while
the material is undergoing plastic deformation. SOME MORE THEORY. \\

\noindent {\bf Class Interface} \\
\indent // Constructor \\
\indent {\em ElasticPPMaterial(int tag, double E, double ep);}  \\ \\
\indent // Destructor \\
\indent {\em $\tilde{ }$ElasticPPMaterial();}\\ \\
\indent // Public Methods \\
\indent {\em int setTrialStrain(double strain); } \\
\indent {\em double getStress(void); } \\
\indent {\em double getTangent(void); } \\
\indent {\em int commitState(void); } \\
\indent {\em int revertToLastCommit(void); } \\
\indent {\em int revertToStart(void); } \\
\indent {\em UniaxialMaterial *getCopy(void); } \\ \\
\indent // Public Methods for Output\\
\indent {\em    int sendSelf(int commitTag, Channel \&theChannel); }\\
\indent {\em    int recvSelf(int commitTag, Channel \&theChannel, 
		 FEM\_ObjectBroker \&theBroker); }\\
\indent {\em    void Print(ostream \&s, int flag =0);} \\



\pagebreak \subsubsection{ParallelMaterial}
%File: ~/OOP/material/ParallelModel.tex
%What: "@(#) ParallelModel.tex, revA"

UNDER CONSTRUCTION. POSSIBLE NAME CHANGE IF MATERIAL GENERAL.\\

\noindent {\bf Files}   \\
\indent \#include $<\tilde{ }$/material/ParallelModel.h$>$  \\

\noindent {\bf Class Declaration}  \\
\indent class ParallelModel: public MaterialModel \\

\noindent {\bf Class Hierarchy} \\
\indent TaggedObject \\
\indent MovableObject \\
\indent\indent MaterialModel \\
\indent\indent\indent UniaxialMaterial \\
\indent\indent\indent\indent {\bf ParallelModel} \\

\noindent {\bf Description}  \\
\indent A ParallelModel object is an aggregation of
UniaxialMaterial objects all considered acting in parallel. SOME
THEORY. \\ 

\noindent {\bf Class Interface} \\
\indent // Constructor \\
\indent {\em ParallelModel(int tag, int numModel,
UniaxialMaterial **theModels);}  \\ \\
\indent // Destructor \\
\indent {\em $\tilde{ }$ParallelModel();}\\ \\
\indent // Public Methods \\
\indent {\em int setTrialStrain(double strain); } \\
\indent {\em double getStress(void); } \\
\indent {\em double getTangent(void); } \\
\indent {\em int commitState(void); } \\
\indent {\em int revertToLastCommit(void); } \\
\indent {\em int revertToStart(void); } \\
\indent {\em UniaxialMaterial *getCopy(void); } \\ \\
\indent // Public Methods for Output\\
\indent {\em    int sendSelf(int commitTag, Channel \&theChannel); }\\
\indent {\em    int recvSelf(int commitTag, Channel \&theChannel, 
		 FEM\_ObjectBroker \&theBroker); }\\
\indent {\em    void Print(ostream \&s, int flag =0);} \\




\pagebreak \subsubsection{Concrete01}
%File: ~/OOP/material/Concrete01.tex
%What: "@(#) Concrete01.tex, revA"

UNDER CONSTRUCTION.\\

\noindent {\bf Files}   \\
\indent \#include $<\tilde{ }$/material/Concrete01.h$>$  \\

\noindent {\bf Class Declaration}  \\
\indent class Concrete01: public MaterialModel \\

\noindent {\bf Class Hierarchy} \\
\indent TaggedObject \\
\indent MovableObject \\
\indent\indent MaterialModel \\
\indent\indent\indent UniaxialMaterial \\
\indent\indent\indent\indent {\bf Concrete01} \\

\noindent {\bf Description}  \\
\indent 
Provides a uniaxial Kent-Scott-Park concrete model with linear
unloading/reloading according to the work of Karsan-Jirsa and no
strength in tension. The model contains a compressive strength of fpc,
a strain at the compressive strength of epsc0, a crushing strength of
fpcu, and a strain at the crushing strength of epscu. Compressive
concrete parameters should be input as negative numeric values for
this model to behave properly. Specification of minimum and maximum
failure strains through the -min and -max switches is optional. The
argument matTag is used to uniquely identify the material object among
material objects in the BasicBuilder object. \\

\noindent {\bf Class Interface} \\
\indent // Constructor \\
\indent {\em Concrete01(int tag, double fpc, double eco, double fpcu,
double ecu,double epsmin=NEG\_INF\_STRAIN, double
epsmax=POS\_INF\_STRAIN);} \\ \\
\indent // Destructor \\
\indent {\em $\tilde{ }$Concrete01();}\\ \\
\indent // Public Methods \\
\indent {\em int setTrialStrain(double strain); } \\
\indent {\em double getStress(void); } \\
\indent {\em double getTangent(void); } \\
\indent {\em int commitState(void); } \\
\indent {\em int revertToLastCommit(void); } \\
\indent {\em int revertToStart(void); } \\
\indent {\em UniaxialMaterial *getCopy(void); } \\ \\
\indent // Public Methods for Output\\
\indent {\em    int sendSelf(int commitTag, Channel \&theChannel); }\\
\indent {\em    int recvSelf(int commitTag, Channel \&theChannel, 
		 FEM\_ObjectBroker \&theBroker); }\\
\indent {\em    void Print(ostream \&s, int flag =0);} \\




\pagebreak \subsubsection{Steel01}
%File: ~/OOP/material/Steel01.tex
%What: "@(#) Steel01.tex, revA"

UNDER CONSTRUCTION. \\

\noindent {\bf Files}   \\
\indent \#include $<\tilde{ }$/material/Steel01.h$>$  \\

\noindent {\bf Class Declaration}  \\
\indent class Steel01 : public MaterialModel \\

\noindent {\bf Class Hierarchy} \\
\indent TaggedObject \\
\indent MovableObject \\
\indent\indent MaterialModel \\
\indent\indent\indent UniaxialMaterial \\
\indent\indent\indent\indent {\bf Steel01} \\

\noindent {\bf Description}  \\
\indent Steel01 provides the abstraction of a bilinear steel model
with isotropic hardening. The model contains a yield strength of fy,
an initial elastic tangent of E0, and a hardening ratio of b. The
optional parameters a1, a2, a3, and a4 control the amount of isotropic
hardening (default values are provided). Specification of minimum and
maximum failure strains through the -min and -max switches is optional
and must appear after the specification of the hardening parameters,
if present. The argument matTag is used to uniquely identify the
material object among material objects in the BasicBuilder object. 


\noindent {\bf Class Interface} \\
\indent // Constructor \\
\indent {\em Steel01 (int tag, double fy, double E0, double b,
                      double a1, double a2, double a3, double a4,
                      double epsmin, double epsmax);}  \\ \\
\indent // Destructor \\
\indent {\em $\tilde{ }$Steel01();}\\ \\
\indent // Public Methods \\
\indent {\em int setTrialStrain(double strain); } \\
\indent {\em double getStress(void); } \\
\indent {\em double getTangent(void); } \\
\indent {\em int commitState(void); } \\
\indent {\em int revertToLastCommit(void); } \\
\indent {\em int revertToStart(void); } \\
\indent {\em ElasticMaterial *getCopy(void); } \\ \\
\indent // Public Methods for Output\\
\indent {\em    int sendSelf(int commitTag, Channel \&theChannel); }\\
\indent {\em    int recvSelf(int commitTag, Channel \&theChannel, 
		 FEM\_ObjectBroker \&theBroker); }\\
\indent {\em    void Print(ostream \&s, int flag =0);} \\
\indent // Private Methods \\
\indent {\em void determineTrialState (double dStrain); } \\
\indent {\em void detectLoadReversal (double dStrain); } \\
\indent {\em void setHistoryVariables (); } \\

\noindent {\bf Constructor}  \\
\indent {\em Steel01 (int tag, double fy, double E0, double b,
                      double a1, double a2, double a3, double a4,
                      double epsmin, double epsmax);}  \\

\noindent {\bf Destructor} \\
\indent {\em virtual~ $\tilde{}$ElasticMaterial();}\\ 
Does nothing. \\

\noindent {\bf Public Methods} \\
\indent {\em int setTrialStrain(double strain); }  \\





\pagebreak \subsubsection{{\bf NDMaterial}}
%File: ~/OOP/material/nD/NDMaterial.tex
%What: "@(#) NDMaterial.tex, revA"

INTERFACE MAY CHANGE IF MAKE MATERIAL MORE GENERAL. \\

\noindent {\bf Files}   \\
\indent \#include $<\tilde{ }$/material/nD/NDMaterial.h$>$  \\

\noindent {\bf Class Declaration}  \\
\indent class NDMaterial : public Material \\

\noindent {\bf Class Hierarchy} \\
\indent TaggedObject \\
\indent MovableObject \\
\indent\indent Material \\
\indent\indent\indent {\bf NDMaterial} \\

\noindent {\bf Description}  \\
\indent NDMaterial is an abstract class. The
NDMaterial class provides the interface that all
NDMaterial writers must provide when introducing new
NDMaterial subclasses. An NDMaterial object 
is responsible for keeping track of stress, strain and the
constitution for a particular point in the domain. \\ 

\noindent {\bf Class Interface} \\
\indent // Constructor \\
\indent {\em NDMaterial (int tag, int classTag);}  \\ \\
\indent // Destructor \\
\indent {\em virtual $\tilde{ }$NDMaterial ();}\\ \\
\indent // Public Methods \\
\indent {\em virtual int setTrialStrain (const Vector \&strain) = 0; } \\
\indent {\em virtual const Vector \&getStress (void) = 0; } \\
\indent {\em virtual const Matrix \&getTangent (void) = 0; } \\
\indent {\em virtual int commitState (void) = 0; } \\
\indent {\em virtual int revertToLastCommit (void) = 0; } \\
\indent {\em virtual int revertToStart (void) = 0; } \\
\indent {\em virtual NDMaterial *getCopy (void) = 0; } \\


\noindent {\bf Constructor}  \\
\indent {\em NDMaterial (int tag, int classTag);}  \\
To construct a NDMaterial whose unique integer among
NDMaterials in the domain is given by {\em tag}, and whose class
identifier is given by {\em classTag}. These integers are passed to
the Material class constructor. \\

\noindent {\bf Destructor} \\
\indent {\em virtual $\tilde{ }$NDMaterial ();} \\
Does nothing. \\ 

\noindent {\bf Public Methods} \\
\indent {\em virtual int setTrialStrain (const Vector \&strain) = 0; }  \\
Sets the value of the trial strain vector, that value used by {\em
getStress()} and {\em getTangent()}, to be {\em strain}. To return $0$
if successful and a negative number if not. \\

\indent {\em virtual const Vector \&getStress (void) = 0; } \\
To return the material stress vector at the current trial strain. \\

\indent {\em virtual const Matrix \&getTangent (void) = 0; } \\
To return the material tangent stiffness matrix at the current trial
strain. \\

\indent {\em virtual int commitState (void) = 0; } \\
To accept the current value of the trial strain vector as being on the
solution path. To return $0$ if successful, a negative number if not. \\

\indent {\em virtual int revertToLastCommit (void) = 0; } \\
To cause the material to revert to its last committed state. To
return $0$ if successful, a negative number if not. \\

\indent {\em virtual int revertToStart (void) = 0; } \\
Invoked to cause the material to revert to its original state in its
undeformed configuration. To return $0$ if successful, a negative
number if not. \\

\indent {\em virtual NDMaterial *getCopy (void) = 0; } \\
Returns a pointer to a new NDMaterial,
which is an exact copy of this instance. It is up to the caller to
ensure that the destructor is invoked. \\




\pagebreak \subsubsection{{\bf SectionForceDeformation}}
%File: ~/OOP/material/section/SectionForceDeformation.tex
%What: "@(#) SectionForceDeformation.tex, revA"

\noindent {\bf Files}   \\
\indent \#include $<\tilde{ }$/material/section/SectionForceDeformation.h$>$  \\

\noindent {\bf Class Declaration}  \\
\indent class SectionForceDeformation : public Material \\

\noindent {\bf Class Hierarchy} \\
\indent TaggedObject \\
\indent MovableObject \\
\indent\indent Material \\
\indent\indent\indent {\bf SectionForceDeformation} \\

\noindent {\bf Description}  \\
\indent SectionForceDeformation provides the interface which all
SectionForceDeformation models must implement. \\

\noindent {\bf Class Interface} \\
\indent // Constructor \\
\indent {\em SectionForceDeformation (int tag, int classTag);} \\
\indent // Destructor \\
\indent {\em virtual $\tilde{ }$SectionForceDeformation ();}\\ \\
\indent // Public Methods \\
\indent {\em virtual int setTrialSectionDeformation (const Vector \&def) = 0; } \\
\indent {\em virtual const Vector \&getSectionDeformation (void) = 0; } \\
\indent {\em virtual const Vector \&getStressResultant (void) = 0; } \\
\indent {\em virtual const Vector \&getPrevStressResultant (void) = 0; } \\
\indent {\em virtual const Matrix \&getSectionTangent (void) = 0; } \\
\indent {\em virtual const Matrix \&getPrevSectionTangent (void) = 0; } \\
\indent {\em virtual const Matrix \&getSectionFlexibility (void); } \\
\indent {\em virtual const Matrix \&getPrevSectionFlexibility (void); } \\
\indent {\em virtual int commitState (void) = 0; } \\
\indent {\em virtual int revertToLastCommit (void) = 0; } \\
\indent {\em virtual int revertToStart (void) = 0; } \\
\indent {\em virtual SectionForceDeformation *getCopy (void) = 0; } \\
\indent {\em virtual const ID \&getType (void) = 0; } \\
\indent {\em virtual int getOrder (void) = 0; } \\ \\
\indent // Public Methods for Output\\
\indent {\em virtual int sendSelf (int commitTag, Channel
\&theChannel) = 0; }\\
\indent {\em virtual int recvSelf (int commitTag, Channel \&theChannel, 
FEM\_ObjectBroker \&theBroker) = 0; }\\
\indent {\em virtual void Print (ostream \&s, int flag = 0) = 0;} \\

\noindent {\bf Constructor}  \\
\indent {\em SectionForceDeformation (int tag, int classTag);}  \\
To construct a SectionForceDeformation object whose unique integer tag among
SectionForceDeformation objects in the domain is given by {\em tag}, and whose class
identifier is given by {\em classTag}. These integers are passed to
the Material class constructor. \\

\noindent {\bf Destructor} \\
\indent {\em virtual $\tilde{ }$SectionForceDeformation ();}\\ 
Does nothing. \\

\noindent {\bf Public Methods} \\
\indent {\em virtual int setTrialSectionDeformation (const Vector \&def) = 0; }  \\
To set the value of the trial section deformation vector, $\esec$ to be {\em
def}. To return $0$ if successful, a negative number if not.\\

\indent {\em virtual const Vector \&getSectionDeformation (void) = 0; } \\
To return the trial section deformation vector, $\esec$. \\

\indent {\em virtual const Vector \&getStressResultant (void) = 0; } \\
To return the section resisting forces, $\ssec$, at the current trial state. \\

\indent {\em virtual const Vector \&getPrevStressResultant (void) = 0; } \\
To return the section resisting forces, $\ssec$, from the previous trial state. \\

\indent {\em virtual const Matrix \&getSectionTangent (void) = 0; } \\
To return the section tangent stiffness matrix, $\ksec$, at the current trial state. \\

\indent {\em virtual const Matrix \&getPrevSectionTangent (void) = 0; } \\
To return the section tangent stiffness matrix, $\ksec$, from the previous trialstate. \\

\indent {\em virtual const Matrix \&getSectionFlexibility (void); } \\
Obtains the section tangent stiffness matrix, $\ksec$, and returns its
inverse, the section flexibility matrix, $\fsec$, via an
explicit matrix inversion.  NOTE: The explicit matrix
inversion provides default behavior and may be overridden in
subclasses to suit specific SectionForceDeformation implementations. \\

\indent {\em virtual const Matrix \&getPrevSectionFlexibility (void); } \\
Returns the section flexibility matrix, $\fsec$, from the previous trial
state. NOTE: This function provides default behavior and may be
overridden in subclasses to suit specific SectionForceDeformation
implementations. \\

\indent {\em virtual int commitState (void) = 0; } \\
To commit the section state.  Returns $0$ if successful and a negative
number if not. \\

\indent {\em virtual int revertToLastCommit (void) = 0; } \\
To revert the section to its last committed state.  Returns $0$ if
successful and a negative number if not. \\

\indent {\em virtual int revertToStart (void) = 0; } \\
To revert the section to its initial state. Returns $0$ if successful
and a negative number if not. \\

\indent {\em virtual SectionForceDeformation *getCopy (void) = 0; } \\
To return a pointer to a new SectionForceDeformation object, which is a copy 
of this instance. It is up to the caller to ensure that the destructor is invoked. \\

\indent {\em virtual const ID \&getType (void) = 0; } \\
To return the section ID code that indicates the ordering and type of
response quantities returned by the section. Lets the calling object
(e.g. an Element) know how to interpret the quantites returned by this
SectionForceDeformation model. \\

\indent {\em virtual int getOrder (void) = 0; } \\
To return the number of response quantities provided by the section. \\

\indent {\em    int sendSelf (int commitTag, Channel \&theChannel); }\\
FILL IN. \\

\indent {\em    int recvSelf (int commitTag, Channel \&theChannel, 
		 FEM\_ObjectBroker \&theBroker); }\\
FILL IN. \\

\indent {\em    void Print(ostream \&s, int flag = 0) = 0;} \\
To print section information to the stream {\em s} based on the value of {\em flag}. \\


\pagebreak \subsubsection{GenericSection1D}
%File: ~/OOP/material/section/GenericSection1D.tex
%What: "@(#) GenericSection1D.tex, revA"

\noindent {\bf Files}   \\
\indent \#include $<\tilde{ }$/material/section/GenericSection1D.h$>$  \\

\noindent {\bf Class Declaration}  \\
\indent class GenericSection1D : public SectionForceDeformation \\

\noindent {\bf Class Hierarchy} \\
\indent TaggedObject \\
\indent MovableObject \\
\indent\indent Material \\
\indent\indent\indent SectionForceDeformation \\
\indent\indent\indent\indent {\bf GenericSection1D} \\

\noindent {\bf Description}  \\
\indent GenericSection1D provides a wrapper around a UniaxialMaterial
so that any UniaxialMaterial may be used to model section response. 
The design of this class follows the Object Adapter pattern in 
{\em Design Patterns} by Gamma et al.\\

\noindent {\bf Class Interface} \\
\indent // Constructor \\
\indent {\em GenericSection1D (int tag, UniaxialMaterial \&m,
int code);} \\
\indent // Destructor \\
\indent {\em $\tilde{ }$GenericSection1D ();}\\ \\
\indent // Public Methods \\
\indent {\em int setTrialSectionDeformation (const Vector \&def); } \\
\indent {\em const Vector \&getSectionDeformation (void); } \\
\indent {\em const Vector \&getStressResultant (void); } \\
\indent {\em const Vector \&getPrevStressResultant (void); } \\
\indent {\em const Matrix \&getSectionTangent (void); } \\
\indent {\em const Matrix \&getPrevSectionTangent (void); } \\
\indent {\em const Matrix \&getSectionFlexibility (void); } \\
\indent {\em const Matrix \&getPrevSectionFlexibility (void); } \\
\indent {\em int commitState (void); } \\
\indent {\em int revertToLastCommit (void); } \\
\indent {\em int revertToStart (void); } \\
\indent {\em SectionForceDeformation *getCopy (void); } \\
\indent {\em const ID \&getType (void); } \\
\indent {\em int getOrder (void); } \\ \\
\indent // Public Methods for Output\\
\indent {\em int sendSelf (int commitTag, Channel
\&theChannel); }\\
\indent {\em int recvSelf (int commitTag, Channel \&theChannel, 
FEM\_ObjectBroker \&theBroker); }\\
\indent {\em void Print (ostream \&s, int flag = 0);} \\

\noindent {\bf Constructor}  \\
\indent {\em GenericSection1D (int tag, UniaxialMaterial \&m,
int code);}  \\
Constructs a GenericSection1D whose unique integer tag among
SectionForceDeformation objects in the domain is given by {\em tag}. Obtains
a copy of the UniaxialMaterial {\em m} via a call to {\em getCopy()}.
The section code is set to be {\em code}. \\

\noindent {\bf Destructor} \\
\indent {\em $\tilde{ }$GenericSection1D ();}\\ 
Invokes the destructor of the UniaxialMaterial. \\

\noindent {\bf Public Methods} \\
\indent {\em int setTrialSectionDeformation (const Vector \&def); }  \\
Sets the trial section deformation vector, $\esec$, to be {\em def}, then invokes
{\em setTrialStrain()} on the UniaxialMaterial. \\

\indent {\em const Vector \&getSectionDeformation (void); } \\
Returns the trial section deformation vector, $\esec$. \\

\indent {\em const Vector \&getStressResultant (void); } \\
Sets the section resisting force, $\ssec$, to be the result of invoking 
{\em getStress()} on the UniaxialMaterial, then returns $\ssec$. \\

\indent {\em const Vector \&getPrevStressResultant (void); } \\
Returns the section resisting force, $\ssec$, from the previous trial state. \\

\indent {\em const Matrix \&getSectionTangent (void); } \\
Sets the section tangent stiffness matrix, $\ksec$, to be the result of
invoking {\em getTangent()} on the UniaxialMaterial, then returns $\ksec$. \\

\indent {\em const Matrix \&getPrevSectionTangent (void); } \\
Returns the section tangent stiffness matrix, $\ksec$ from the previous trial state. \\

\indent {\em const Matrix \&getSectionFlexibility (void); } \\
Sets the section flexibility matrix, $\fsec$, to be the inverse of the result
of invoking {\em getTangent()} on the UniaxialMaterial, then returns $\fsec$.
This function overrides the base class implementation. \\

\indent {\em const Matrix \&getPrevSectionFlexibility (void); } \\
Returns the section flexibility matrix, $\fsec$, from the previous trial state. \\
This function overrides the base class implementation. \\

\indent {\em int commitState (void); } \\
Invokes {\em commitState()} on the UniaxialMaterial and returns the
result of that invocation. \\

\indent {\em int revertToLastCommit (void); } \\
Invokes {\em revertToLastCommit()} on the UniaxialMaterial and returns the
result of that invocation. \\

\indent {\em int revertToStart (void); } \\
Invokes {\em revertToStart()} on the UniaxialMaterial and returns the
result of that invocation. \\

\indent {\em SectionForceDeformation *getCopy (void); } \\
Returns a pointer to a new instance of GenericSection1D, using the
same tag, UniaxialMaterial reference, and code. It is up to the caller to
ensure that the destructor is invoked. \\

\indent {\em const ID \&getType (void); } \\
Returns the section ID code that indicates the type of response quantity 
returned by this instance of GenericSection1D. \\

\indent {\em int getOrder (void); } \\
Returns 1. \\

\indent {\em int sendSelf (int commitTag, Channel \&theChannel); }\\
FILL IN. \\

\indent {\em int recvSelf (int commitTag, Channel \&theChannel, 
		 FEM\_ObjectBroker \&theBroker); }\\
FILL IN. \\

\indent {\em void Print (ostream \&s, int flag = 0);} \\
Prints to the stream {\em s} the object's {\em tag}, then invokes
{\em Print()} on the UniaxialMaterial using the same values of {\em s}
and {\em flag}. \\


\pagebreak \subsubsection{GenericSectionND}
%File: ~/OOP/material/section/GenericSectionND.tex
%What: "@(#) GenericSectionND.tex, revA"

\noindent {\bf Files}   \\
\indent \#include $<\tilde{ }$/material/section/GenericSectionND.h$>$  \\

\noindent {\bf Class Declaration}  \\
\indent class GenericSectionND : public SectionForceDeformation \\

\noindent {\bf Class Hierarchy} \\
\indent TaggedObject \\
\indent MovableObject \\
\indent\indent Material \\
\indent\indent\indent SectionForceDeformation \\
\indent\indent\indent\indent {\bf GenericSectionND} \\

\noindent {\bf Description}  \\
\indent GenericSectionND provides a wrapper around a NDMaterial
so that any NDMaterial may be used to model section response. 
The design of this class follows the Object Adapter pattern in 
{\em Design Patterns} by Gamma et al.\\

\noindent {\bf Class Interface} \\
\indent // Constructor \\
\indent {\em GenericSectionND (int tag, NDMaterial \&m,
int code);} \\
\indent // Destructor \\
\indent {\em $\tilde{ }$GenericSectionND ();}\\ \\
\indent // Public Methods \\
\indent {\em int setTrialSectionDeformation (const Vector \&def); } \\
\indent {\em const Vector \&getSectionDeformation (void); } \\
\indent {\em const Vector \&getStressResultant (void); } \\
\indent {\em const Vector \&getPrevStressResultant (void); } \\
\indent {\em const Matrix \&getSectionTangent (void); } \\
\indent {\em const Matrix \&getPrevSectionTangent (void); } \\
\indent {\em int commitState (void); } \\
\indent {\em int revertToLastCommit (void); } \\
\indent {\em int revertToStart (void); } \\
\indent {\em SectionForceDeformation *getCopy (void); } \\
\indent {\em const ID \&getType (void); } \\
\indent {\em int getOrder (void); } \\ \\
\indent // Public Methods for Output\\
\indent {\em int sendSelf (int commitTag, Channel
\&theChannel); }\\
\indent {\em int recvSelf (int commitTag, Channel \&theChannel, 
FEM\_ObjectBroker \&theBroker); }\\
\indent {\em void Print (OPS_Stream \&s, int flag = 0);} \\

\noindent {\bf Constructor}  \\
\indent {\em GenericSectionND (int tag, NDMaterial \&m,
const ID \&code);}  \\
Constructs a GenericSectionND whose unique integer tag among
SectionForceDeformation objects in the domain is given by {\em tag}. Obtains
a copy of the NDMaterial {\em m} via a call to {\em getCopy()}.
The section code is set to be {\em code}. \\

\noindent {\bf Destructor} \\
\indent {\em $\tilde{ }$GenericSectionND ();}\\ 
Invokes the destructor of the NDMaterial. \\

\noindent {\bf Public Methods} \\
\indent {\em int setTrialSectionDeformation (const Vector \&def); }  \\
Sets the trial section deformation vector, $\esec$, to be {\em def}, then invokes
{\em setTrialStrain()} on the NDMaterial. \\

\indent {\em const Vector \&getSectionDeformation (void); } \\
Returns the trial section deformation vector, $\esec$. \\

\indent {\em const Vector \&getStressResultant (void); } \\
Sets the section resisting force, $\ssec$, to be the result of invoking 
{\em getStress()} on the NDMaterial, then returns $\ssec$. \\

\indent {\em const Vector \&getPrevStressResultant (void); } \\
Returns the section resisting force, $\ssec$, from the previous trial state. \\

\indent {\em const Matrix \&getSectionTangent (void); } \\
Sets the section tangent stiffness matrix, $\ksec$, to be the result of
invoking {\em getTangent()} on the NDMaterial, then returns $\ksec$. \\

\indent {\em const Matrix \&getPrevSectionTangent (void); } \\
Returns the section tangent stiffness matrix, $\ksec$ from the previous trial state. \\

\indent {\em int commitState (void); } \\
Invokes {\em commitState()} on the NDMaterial and returns the
result of that invocation. \\

\indent {\em int revertToLastCommit (void); } \\
Invokes {\em revertToLastCommit()} on the NDMaterial and returns the
result of that invocation. \\

\indent {\em int revertToStart (void); } \\
Invokes {\em revertToStart()} on the NDMaterial and returns the
result of that invocation. \\

\indent {\em SectionForceDeformation *getCopy (void); } \\
Returns a pointer to a new instance of GenericSectionND, using the
same tag, NDMaterial reference, and code. It is up to the caller to
ensure that the destructor is invoked. \\

\indent {\em const ID \&getType (void); } \\
Returns the section ID code that indicates the type of response quantities
returned by this instance of GenericSectionND. \\

\indent {\em int getOrder (void); } \\
Returns the result of invoking {\em getOrder()} on the NDMaterial. \\

\indent {\em int sendSelf (int commitTag, Channel \&theChannel); }\\
FILL IN. \\

\indent {\em int recvSelf (int commitTag, Channel \&theChannel, 
		 FEM\_ObjectBroker \&theBroker); }\\
FILL IN. \\

\indent {\em void Print (OPS_Stream \&s, int flag = 0);} \\
Prints to the stream {\em s} the object's {\em tag}, then invokes
{\em Print()} on the NDMaterial using the same values of {\em s}
and {\em flag}. \\


\pagebreak \subsubsection{ElasticSection2D}
%File: ~/OOP/material/section/ElasticSection2D.tex
%What: "@(#) ElasticSection2D.tex, revA"

\noindent {\bf Files}   \\
\indent \#include $<\tilde{ }$/material/section/ElasticSection2D.h$>$  \\

\noindent {\bf Class Declaration}  \\
\indent class ElasticSection2D : public SectionForceDeformation \\

\noindent {\bf Class Hierarchy} \\
\indent TaggedObject \\
\indent MovableObject \\
\indent\indent Material \\
\indent\indent\indent SectionForceDeformation \\
\indent\indent\indent\indent {\bf ElasticSection2D} \\

\noindent {\bf Description}  \\
\indent ElasticSection2D provides the implementation of a
section which exhibits uncoupled elastic behavior in axial, moment,
and shear response. \\

\noindent {\bf Class Interface} \\
\indent // Constructor \\
\indent {\em ElasticSection2D (int tag, double E, double A, double I,
double G, double alpha = 5/6.0);} \\ 
\indent {\em ElasticSection2D ();} \\ \\
\indent // Destructor \\
\indent {\em $\tilde{ }$ElasticSection2D ();} \\ \\
\indent // Public Methods \\
\indent {\em int setTrialSectionDeformation (const Vector \&def); } \\
\indent {\em const Vector \&getSectionDeformation (void); } \\
\indent {\em const Vector \&getStressResultant (void); } \\
\indent {\em const Vector \&getPrevStressResultant (void); } \\
\indent {\em const Matrix \&getSectionTangent (void); } \\
\indent {\em const Matrix \&getPrevSectionTangent (void); } \\
\indent {\em const Matrix \&getSectionFlexibility (void); } \\
\indent {\em const Matrix \&getPrevSectionFlexibility (void); } \\
\indent {\em int commitState (void); } \\
\indent {\em int revertToLastCommit (void); } \\
\indent {\em int revertToStart (void); } \\
\indent {\em SectionForceDeformation *getCopy (void); } \\
\indent {\em const ID \&getType (void); } \\
\indent {\em int getOrder (void); } \\ \\
\indent // Public Methods for Output\\
\indent {\em int sendSelf (int commitTag, Channel \&theChannel); }\\
\indent {\em int recvSelf (int commitTag, Channel \&theChannel, 
FEM\_ObjectBroker \&theBroker); }\\
\indent {\em void Print (ostream \&s, int flag = 0);} \\

\noindent {\bf Constructor}  \\
\indent {\em ElasticSection2D (int tag, double E, double I, double A,
double G, double alpha);}  \\
To construct an ElasticSection2D with an integer identifier {\em
tag}, an elastic modulus of $E$, a second moment of area $I$,
a section area of $A$, an elastic shear modulus of $G$, and a shape factor
of $alpha$. The integers {\em tag} and MAT\_TAG\_ElasticSection2D, defined in
$<$classTags.h$>$, are passed to the SectionForceDeformation
class constructor. \\

\indent {\em ElasticSection2D ();} \\
Constructs a blank ElasticSection2D object. \\

\noindent {\bf Destructor} \\
\indent {\em $\tilde{}$ElasticSection2D ();} \\ 
Does nothing. \\

\noindent {\bf Public Methods} \\
\indent {\em int setTrialSectionDeformation (const Vector \&def); }  \\
Sets the value of the section deformation vector, $\esec$ to be {\em
def}. The section deformation vector, $\esec$, is defined by

\begin{equation}
\esec := \left[
   \begin{array}{c}
       \epsilon_a \\
       \kappa_z   \\
       \gamma_y
   \end{array} 
 \right]
\end{equation}

where $\epsilon_a$ is the axial strain, $\kappa_z$ is the curvature about
the local z-axis, and $\gamma_y$ is the shear strain along the local y-axis.
Returns $0$. \\

\indent {\em const Vector \&getSectionDeformation (void); } \\
Returns the current value of the section deformation vector, $\esec$. \\

\indent {\em const Vector \&getStressResultant (void); } \\
Returns the section stress resultants, $\ssec$, the product of the 
section stiffness matrix, $\ksec$, and the section deformation 
vector, $\esec$,

\begin{equation}
\ssec = \ksec \esec = \left[
   \begin{array}{c}
       P \\
       M_z   \\
       V_y
   \end{array} 
 \right]
\end{equation}

where $P$ is the axial force, $M_z$ is the bending moment about the
local z-axis, and $V_y$ is the shear force along the local y-axis. \\

\indent {\em const Vector \&getPrevStressResultant (void); } \\
Returns the value of $\ssec$ calculated at the previous state determination. \\

\indent {\em const Matrix \&getSectionTangent (void); } \\
Returns the section stiffness matrix, $\ksec$, where 

\begin{equation}
\ksec = \left[
   \begin{array}{ccc}
       EA &  0 &  0 \\
        0 & EI &  0 \\
        0 &  0 & \alpha GA
   \end{array} 
 \right]
\end{equation} \\

\indent {\em const Matrix \&getPrevSectionTangent (void); } \\
Returns the section stiffness matrix, $\ksec$. \\

\indent {\em const Matrix \&getSectionFlexibility (void); } \\
Overrides the base class implementation and returns the section
flexibility matrix, $\fsec$, where

\begin{equation}
\fsec = \left[
   \begin{array}{ccc}
       \frac{1}{EA} &            0 &                  0  \\
                  0 & \frac{1}{EI} &                  0  \\
                  0 &            0 & \frac{1}{\alpha GA}
   \end{array} 
 \right]
\end{equation} \\

\indent {\em const Matrix \&getPrevSectionFlexibility (void); } \\
Overrides the base class implementation and returns the section
flexibility matrix, $\fsec$. \\

\indent {\em int commitState (void); } \\
Returns $0$. \\

\indent {\em int revertToLastCommit (void); } \\
Returns $0$. \\

\indent {\em int revertToStart (void); } \\
Returns $0$. \\

\indent {\em SectionForceDeformation *getCopy (void); } \\
Returns a pointer to a new ElasticSection2D object, constructed
using the same values of {\em tag}, $E$, $A$, $I$, $G$, and $\alpha.  
It is up to the caller to ensure that the destructor is invoked. \\

\indent {\em const ID \&getType (void); } \\
Returns the section ID code that indicates the ordering of
section response quantities. For this section, axial response is the
first quantity, bending about the local z-axis is the second, and
shear along the local y-axis is third. \\

\begin{equation}
code := \left[
   \begin{array}{c}
       2 \\
       1 \\
       3 
   \end{array} 
 \right]
\end{equation} \\

\indent {\em int getOrder (void); } \\
Returns 3. \\

\indent {\em    int sendSelf(int commitTag, Channel \&theChannel); }\\
Creates a Vector of size $6$ into which it places  {\em tag},
$E$, $A$, $I$, $G$, and $\alpha$. Invokes {\em sendVector()} on  {\em theChannel} using the
ElasticSection2D objects {\em dbTag}, the integer {\em commitTag} and
the Vector as arguments. Returns $0$ if successful, a warning message
and a negative number are returned if the Channel object fails to send
the Vector. \\

\indent {\em    int recvSelf(int commitTag, Channel \&theChannel, 
		 FEM\_ObjectBroker \&theBroker); }\\
Creates a Vector of size $6$. Invokes {\em recvVector()} on  {\em
theChannel} using the ElasticSection2D objects {\em dbTag}, the integer
{\em commitTag} and the Vector as arguments. Using the data in the
Vector to set its {\em tag}, $E$, $A$, $I$, $G$, and $\alpha$. Returns $0$ if successful, a
warning message is printed and a negative number is returned if the Channel 
object fails to receive the Vector. \\

\indent {\em    void Print(ostream \&s, int flag =0);} \\
Prints to the stream {\em s} the objects {\em tag}, $E$, $A$, $I$, $G$, and
$\alpha$ values. \\





\pagebreak \subsubsection{ElasticSection3D}
%File: ~/OOP/material/section/ElasticSection3D.tex
%What: "@(#) ElasticSection3D.tex, revA"

\noindent {\bf Files}   \\
\indent \#include $<\tilde{ }$/material/section/ElasticSection3D.h$>$  \\

\noindent {\bf Class Declaration}  \\
\indent class ElasticSection3D : public SectionForceDeformation \\

\noindent {\bf Class Hierarchy} \\
\indent TaggedObject \\
\indent MovableObject \\
\indent\indent Material \\
\indent\indent\indent SectionForceDeformation \\
\indent\indent\indent\indent {\bf ElasticSection3D} \\

\noindent {\bf Description}  \\
\indent ElasticSection3D provides the implementation of a
section which exhibits uncoupled elastic behavior in axial, moment,
shear, and torsion response. \\

\noindent {\bf Class Interface} \\
\indent // Constructor \\
\indent {\em ElasticSection3D (int tag, double E, double Iz,
double Iy, double J, double A, double G, double alpha = 5/6.0);} \\ 
\indent {\em ElasticSection3D ();} \\ \\
\indent // Destructor \\
\indent {\em $\tilde{ }$ElasticSection3D ();} \\ \\
\indent // Public Methods \\
\indent {\em int setTrialSectionDeformation (const Vector \&def); } \\
\indent {\em const Vector \&getSectionDeformation (void); } \\
\indent {\em const Vector \&getStressResultant (void); } \\
\indent {\em const Vector \&getPrevStressResultant (void); } \\
\indent {\em const Matrix \&getSectionTangent (void); } \\
\indent {\em const Matrix \&getPrevSectionTangent (void); } \\
\indent {\em const Matrix \&getSectionFlexibility (void); } \\
\indent {\em const Matrix \&getPrevSectionFlexibility (void); } \\
\indent {\em int commitState (void); } \\
\indent {\em int revertToLastCommit (void); } \\
\indent {\em int revertToStart (void); } \\
\indent {\em SectionForceDeformation *getCopy (void); } \\
\indent {\em const ID \&getType (void); } \\
\indent {\em int getOrder (void); } \\ \\
\indent // Public Methods for Output\\
\indent {\em int sendSelf (int commitTag, Channel \&theChannel); }\\
\indent {\em int recvSelf (int commitTag, Channel \&theChannel, 
FEM\_ObjectBroker \&theBroker); }\\
\indent {\em void Print (OPS_Stream \&s, int flag = 0);} \\

\noindent {\bf Constructor}  \\
\indent {\em ElasticSection3D (int tag, double E, double Iz,
double Iy, double J, double A, double G, double alpha = 5/6.0);} \\ 
To construct an ElasticSection3D with an integer identifier {\em
tag}, an elastic modulus of $E$, a second moment of area about the local
z-axis, $I_z$, a second moment of area about the local y-axis, $I_y$, a polar
moment of intertia of $J$, a section area of $A$, an elastic shear modulus of $G$, 
and a shape factor of $\alpha$. The integers {\em tag} and MAT\_TAG\_ElasticSection3D,
defined in $<$classTags.h$>$, are passed to the SectionForceDeformation
class constructor. \\

\indent {\em ElasticSection3D ();} \\
Constructs a blank ElasticSection3D object. \\

\noindent {\bf Destructor} \\
\indent {\em $\tilde{}$ElasticSection3D ();} \\ 
Does nothing. \\

\noindent {\bf Public Methods} \\
\indent {\em int setTrialSectionDeformation (const Vector \&def); }  \\
Sets the value of the section deformation vector, $\esec$ to be {\em
def}. The section deformation vector, $\esec$, is defined by

\begin{equation}
\esec := \left[
   \begin{array}{c}
       \epsilon_a \\
       \kappa_z   \\
       \kappa_y   \\
       \gamma_y   \\
       \gamma_z   \\
       \phi
   \end{array} 
 \right]
\end{equation}

where $\epsilon_a$ is the axial strain, $\kappa_z$ is the curvature about
the local z-axis, $\kappa_z$ is the curvature about the local z-axis, 
$\gamma_y$ is the shear strain along the local y-axis, $\gamma_z$
is the shear strain along the local z-axis, and $\phi$ is the angle of
twist. Returns $0$. \\

\indent {\em const Vector \&getSectionDeformation (void); } \\
Returns the current value of the section deformation vector, $\esec$. \\

\indent {\em const Vector \&getStressResultant (void); } \\
Returns the section stress resultants, $\ssec$, the product of the 
section stiffness matrix, $\ksec$, and the section deformation 
vector, $\esec$,

\begin{equation}
\ssec = \ksec \esec = \left[
   \begin{array}{c}
       P     \\
       M_z   \\
       M_z   \\
       V_y   \\
       V_y   \\
       T
   \end{array} 
 \right]
\end{equation}

where $P$ is the axial force, $M_z$ is the bending moment about the
local z-axis, $M_y$ is the bending moment about the local y-axis, 
$V_y$ is the shear force along the local y-axis, $V_z$ is the shear force
along the local z-axis, and $T$ is the torque. \\

\indent {\em const Vector \&getPrevStressResultant (void); } \\
Returns the value of $\ssec$ calculated at the previous state determination. \\

\indent {\em const Matrix \&getSectionTangent (void); } \\
Returns the section stiffness matrix, $\ksec$, where 

\begin{equation}
\ksec = \left[
   \begin{array}{cccccc}
       EA &  0 &  0 & 0 & 0 & 0  \\
        0 & EI_z & 0 & 0 & 0 & 0 \\
        0 & 0 & EI_y & 0 & 0 & 0 \\
        0 & 0 & 0 & \alpha GA & 0 & 0 \\
        0 & 0 & 0 & 0 & \alpha GA & 0 \\
        0 & 0 & 0 & 0 & 0 & GJ
   \end{array} 
 \right]
\end{equation} \\

\indent {\em const Matrix \&getPrevSectionTangent (void); } \\
Returns the section stiffness matrix, $\ksec$. \\

\indent {\em const Matrix \&getSectionFlexibility (void); } \\
Overrides the base class implementation and returns the section
flexibility matrix, $\fsec$, where

\begin{equation}
\fsec = \left[
   \begin{array}{cccccc}
       \frac{1}{EA} & 0 & 0 & 0 & 0 & 0 \\
       0 & \frac{1}{EI_z} & 0 & 0 & 0 & 0 \\
       0 & 0 & \frac{1}{EI_y} & 0 & 0 & 0 \\
       0 & 0 & 0 & \frac{1}{\alpha GA} & 0 & 0 \\ 
       0 & 0 & 0 & 0 & \frac{1}{\alpha GA} & 0 \\
       0 & 0 & 0 & 0 & 0 & \frac{1}{GJ}
   \end{array} 
 \right]
\end{equation} \\

\indent {\em const Matrix \&getPrevSectionFlexibility (void); } \\
Overrides the base class implementation and returns the section
flexibility matrix, $\fsec$. \\

\indent {\em int commitState (void); } \\
Returns $0$. \\

\indent {\em int revertToLastCommit (void); } \\
Returns $0$. \\

\indent {\em int revertToStart (void); } \\
Returns $0$. \\

\indent {\em SectionForceDeformation *getCopy (void); } \\
Returns a pointer to a new ElasticSection3D object, constructed
using the same values of {\em tag}, $E$, $A$, $I_z$, $I_y$, $J$, $G$, and $\alpha$.
It is up to the caller to ensure that the destructor is invoked. \\

\indent {\em const ID \&getType (void); } \\
Returns the section ID code that indicates the ordering of
section response quantities. For this section, axial response is the
first quantity, bending about the local z-axis is the second, bending about
the local y-axis is the third, shear along the local y-axis is the fourth,
shear along the local z-axis is the fifth, and torsion is the sixth. \\

\begin{equation}
code := \left[
   \begin{array}{c}
       2 \\
       1 \\
       4 \\
       3 \\
       5 \\
       6
   \end{array} 
 \right]
\end{equation} \\

\indent {\em int getOrder (void); } \\
Returns 6. \\

\indent {\em    int sendSelf(int commitTag, Channel \&theChannel); }\\
Creates a Vector of size $8$ into which it places  {\em tag},
$E$, $A$, $I_z$, $I_y$, $J$, $G$, and $\alpha$. Invokes {\em sendVector()} on 
{\em theChannel} using the ElasticSection3D objects {\em dbTag}, the integer
{\em commitTag} and the Vector as arguments. Returns $0$ if successful, a warning
message and a negative number are returned if the Channel object fails to send
the Vector. \\

\indent {\em    int recvSelf(int commitTag, Channel \&theChannel, 
		 FEM\_ObjectBroker \&theBroker); }\\
Creates a Vector of size $8$. Invokes {\em recvVector()} on  {\em
theChannel} using the ElasticSection3D objects {\em dbTag}, the integer
{\em commitTag} and the Vector as arguments. Using the data in the
Vector to set its {\em tag}, $E$, $A$, $I_z$, $I_y$, $J$, $G$, and $\alpha$.
Returns $0$ if successful, a warning message is printed and a negative number
is returned if the Channel object fails to receive the Vector. \\

\indent {\em    void Print(OPS_Stream \&s, int flag =0);} \\
Prints to the stream {\em s} the object's {\em tag}, $E$, $A$, $I_z$, $I_y$, $J$, 
$G$, and $\alpha$ values. \\





\pagebreak \subsubsection{FiberSection2D}
\input{../material/sectionForceDeformation/FiberSection2D}

\pagebreak \subsubsection{FiberSection3D}
\input{../material/sectionForceDeformtion/FiberSection3D}

\pagebreak




%\pagebreak
%\section{Section Classes}

%\pagebreak \input{../section/SECTION}

\section{Data Storage} 
In this work there are two general types of data storage classes provided: 
\begin{enumerate}
\item Classes which can be used to store and provide access to the TaggedObjects
during program execution. The abstract base class for these classes is
TaggedObjectStorage. The concrete subclasses can implement the
interface using the traditional CS data storage techniques, such as
arrays, linked lists, hash tables, etc..
\item Classes which can be used to store and retrieve information from
permanent data archives, i.e. databases. The abstract base class
defining the interface for these classes is FE\_Datastore.
\end{enumerate}

\pagebreak \subsection{{\bf TaggedObject}}
% File: ~/tagged/TaggedObject.tex 
% What: "@(#) TaggedObject.tex, revA"

\noindent {\bf Files}   \\
\indent \#include $<\tilde{ }$/tagged/TaggedObject.h$>$ \\

\noindent {\bf Class Declaration}  \\
\indent class TaggedObject  \\

\noindent {\bf Class Hierarchy} \\
\indent  {\bf TaggedObject} \\
\indent\indent  {DomainComponent} \\
\indent\indent  {Vertex} \\

\noindent {\bf Description}  \\
\indent TaggedObject is used as a base class to represent all classes
that may have a integer identifier, a tag, to identify the object. It
is used in the framework as a base class for many classes, for example
DomainComponent and Vertex. The class is provided so that container
classes can be written to store objects and provide access to
them. This saves us rewriting container classes for each type of 
object. (templates will be able to provide this functionality when
they are provided with all compilers). \\ 

\noindent {\bf Class Interface}  \\
\indent // Constructor  \\
\indent {\em TaggedObject(int tag);}  \\ \\
\indent // Destructor  \\
\indent {\em virtual~ $\tilde{}$TaggedObject();}  \\ \\
\indent // Public Methods  \\
\indent {\em int getTag(void) const;} \\ 
\indent {\em virtual void Print(OPS_Stream \&s) =0;} \\
\indent {\em friend OPS_Stream \&operator$<<$(OPS_Stream \&s,
TaggedObject \&m);} \\  \\ 
\indent // Protected Methods  \\
\indent {\em virtual void setTag(int newTag);} \\

\noindent {\bf Constructors}  \\
\indent {\em TaggedObject(int tag);}  \\
Constructs a TaggedObject with a tag given by {\em tag}. The tag of
a component is some unique means of identifying the component among
like components, i.e. the tag of a node would be its unique node number. \\

\noindent {\bf Destructor}  \\
\indent {\em virtual~ $\tilde{}$TaggedObject();}\\
Does nothing. Provided so the concrete subclasses destructors will be
called. \\

\noindent {\bf Public Member Functions}  \\
\indent {\em int getTag(void) const;} \\
Returns the tag associated with the object. This function is inlined
for performance.\\

\indent {\em virtual void Print(OPS_Stream \&s, int flag =0) =0;} \\
A pure virtual function. The component is to output itself to the output
stream {\em s}. The integer {\em flag} can be used to select just what
should be output, by default $0$ is passed. \\

\indent {\em friend OPS_Stream \&operator$<<$(OPS_Stream \&s, TaggedObject \&m);} \\  
Invokes {\em Print(s)} on the TaggedObject {\em m}. \\

\noindent {\bf Protected Member Functions}  \\
\indent {\em virtual void setTag(int newTag);} \\
Sets the tag of the object to be {\em newTag}. It is provided so that
MovableObjects can set their tag in {\em recvSelf()}.








\pagebreak

\subsection{{\bf TaggedObjectStorage}}
% File: ~/tagged/storage/TaggedObjectStorage.tex 
% What: "@(#) TaggedObjectStorage.tex, revA"

\noindent {\bf Files}   \\
\indent \#include $<\tilde{ }$/tagged/storage/TaggedObjectStorage.h$>$ \\

\noindent {\bf Class Declaration}  \\
\indent class TaggedObjectStorage  \\

\noindent {\bf Class Hierarchy} \\
\indent  {\bf TaggedObjectStorage} \\

\noindent {\bf Description}  \\
\indent TaggedObjectStorage is used as a container object to store and
provide access to objects of type TaggedObject. Each TaggedObject
object stored in a TaggedObjectStorage object must have a unique
integer tag to distinguish it from other the other objects stored. The
TaggedObjectStorage class is an abstract base class, it just defines
the interface all concrete subclasses must provide. The interface
defines methods to add and to remove the components, and methods to
obtain access to the components. \\

\noindent {\bf Class Interface}  \\
\indent // Constructor  \\
\indent {\em TaggedObjectStorage();} \\\\
\indent // Destructor \\
\indent {\em virtual  ~TaggedObjectStorage();} \\  \\
\indent // Pure Virtual Public Methods \\
\indent {\em virtual int  setSize(int newSize) =0;}\\
\indent {\em virtual bool addComponent(TaggedObject *newComponent)=0;}\\
\indent {\em virtual TaggedObject *removeComponent(int tag) =0;}\\
\indent {\em virtual int getNumComponents(void) const =0;}\\
\indent {\em virtual TaggedObject *getComponentPtr(int tag) =0;}\\
\indent {\em virtual TaggedObjectIter \&getComponents(void) =0;}\\
\indent {\em virtual TaggedObjectStorage *getEmptyCopy(void) =0;}\\
\indent {\em virtual void clearAll(bool invokeDestructor = true) =0;}\\
\indent {\em virtual void Print(ostream \&s, int flag =0) =0;}\\

\noindent {\bf Constructor} \\
\indent {\em TaggedObjectStorage();} \\
Does nothing.\\

\noindent {\bf Destructor} \\
\indent {\em virtual ~TaggedObjectStorage();} \\ 
Does nothing. Provided so that the concrete subclasses destructor will
be invoked. The subclasses destructor is NOT to delete the objects
stored in the object. {\em clearAll()} can be invoked by the
programmer if this is required. \\

\noindent {\bf Public Member Functions} \\
\indent {\em virtual int setSize(int newSize) =0;}\\
To provide an indication to the container object that {\em newSize}
components are likely to be added. This is only a hint, it should be
acceptable for more or less objects than {\em newSize} to be added to
the container. \\

\indent {\em virtual bool addComponent(TaggedObject *newComponent);}\\
To add the object {\em newComponent} to the container. To return
{\em true} if the object was added to the container, {\em false}
otherwise. The object should not be added if another object with a
similar tag already exists in the container.\\
 
\indent {\em virtual TaggedObject *removeComponent(int tag) =0;}\\
To remove the component whose tag is given by {\em tag} from the
container. To return a pointer to the removed object if successful,
$0$ if not.\\ 

\indent {\em virtual int  getNumComponents(void) const =0;}\\
To return the number of components currently stored in the container. \\

\indent {\em virtual TaggedObject *getComponentPtr(int tag) =0;}\\
To return a pointer to the TaggedObject whose identifier is given by
{\em tag}. If the object has not been added to the container $0$ is to
be returned. \\

\indent {\em virtual TaggedObjectIter \&getComponents(void) =0;}\\
To return an iter for iterating through the objects that have been
added to the container. \\

\indent {\em virtual TaggedObjectStorage *getEmptyCopy(void) =0;}\\
To return an empty copy of the container. \\

\indent {\em virtual void clearAll(bool invokeDestructor = true) =0;}\\
To remove all objects from the container and {\bf to invoke the
destructor on these objects} if {\em invokeDestructor} is {\em true}. \\

\indent {\em virtual void Print(ostream \&s, int flag =0) =0;}\\
To invoke {\em Print(s,flag)} on all objects which have been added to
the container. 






\pagebreak

\subsubsection{ ArrayOfTaggedObjects}
% File: ~/tagged/storage/ArrayOfTaggedObjects.tex 
% What: "@(#) ArrayOfTaggedObjects.tex, revA"

\noindent {\bf Files}   \\
\indent \#include $<\tilde{ }$/tagged/storage/ArrayOfTaggedObjects.h$>$ \\

\noindent {\bf Class Declaration}  \\
\indent class ArrayOfTaggedObjects  \\

\noindent {\bf Class Hierarchy} \\
\indent  {TaggedObjectStorage} \\
\indent\indent  {\bf ArrayOfTaggedObjects} \\

\noindent {\bf Description}  \\
\indent ArrayOfTaggedObjects is used as a container object to store and
provide access to objects of type TaggedObject. A single one dimensional
array is used to store the pointers to the objects. As a one dimensional
array is used, certain ideas are tried to improve performance: (1) if
the array needs to be larger to hold more components, the array size
is doubled; (2) when adding/retrieving components, the array location
given by the components tag is first checked; and (3) a boolean flag is
used to keep track of whether all objects have been stored at the
location given by the tags.\\

\noindent {\bf Class Interface}  \\
\indent // Constructor  \\
\indent {\em ArrayOfTaggedObjects(int sizeInitialArray);} \\\\
\indent // Destructor \\
\indent {\em  ~ArrayOfTaggedObjects();} \\  \\
\indent // Pure Public Methods \\
\indent {\em int  setSize(int newSize);}\\
\indent {\em bool addComponent(TaggedObject *newComponent)=0;}\\
\indent {\em TaggedObject *removeComponent(int tag);}\\
\indent {\em int getNumComponents(void) const;}\\
\indent {\em TaggedObject *getComponentPtr(int tag);}\\
\indent {\em TaggedObjectIter \&getComponents(void);}\\
\indent {\em TaggedObjectStorage *getEmptyCopy(void);}\\
\indent {\em void clearAll(void);}\\
\indent {\em void Print(OPS_Stream \&s, int flag);}\\

\noindent {\bf Constructor} \\
\indent {\em ArrayOfTaggedObjects(int sizeInitialArray);} \\
Creates an array of size {\em sizeInitialArray} to hold pointers to
the TaggedObject objects that will be added to the container. This
array is created using {\em new()} to allow the array to grow as
needed. Zeros this array and sets a number of integer values: (1) the
size of the array; (2) the last position used in the array is $0$; (3)
the position in the array through which all previous cells are being
used for pointers is $0$; (4) the number of components added to the
container is $0$; and (5) all components have been added at the
position given by their tag. If not enough space is available, the
warning method is invoked on the global ErrorHandler and the integer
outlining the array size is set to $0$. \\ 

\noindent {\bf Destructor} \\
\indent {\em ~ArrayOfTaggedObjects();} \\ 
If memory has been allocated for the array, the object invokes the
destructor on the current array. \\ 

\noindent {\bf Public Member Functions} \\
\indent {\em int setSize(int newSize);}\\
If {\em newSize} is valid, i.e. $>= 1$ AND {\em newSize} $>$ current
size of the array, the object allocates memory for a new array of size
{\em newSize}. It zeros this array and copies the old components to
this array, trying to see if they can be located at the positions
given by their tags. If all placed at ideal location last time and new
size $>=$ position of last entry straight copy, otherwise we reset and
add each again using {\em addComponent()}. Finally invokes the
destructor on the old array. Returns $0$ if successful. If not
successful, the warning method is invoked on the global ErrorHandler
and a negative value is returned.\\ 


\indent {\em bool addComponent(TaggedObject *newComponent);}\\
To add the object {\em newComponent} to the container. First checks to
see that an object with a similar tag does not already exist in the
container, invokes the warning method on the global ErrorHandler and
returns {\em false} if one does. It then checks to ensure that array
is large enough to hold another pointer, if not {\em setSize()} is
invoked with twice the size of the current array. (If {\em setSize()}
fails the warning method is invoked and {\em false} is returned: NOTE
other objects previously added may now be missing!!. The object is
then placed into the array by choosing the first of the following that
is successful: \begin{enumerate} 
\item If the array is large enough, the location given by the objects
tag is first tested to see if being used. If not this location is
used.
\item If it won't go in nicely, we find the first location in the
array that is not being used and use this location. We keep a marker
to this location for subsequent adds so that don't start at $0$
location all the time.
\end{enumerate}
\noindent Finally the integer indicating the numbers of objects in the array is
incremented and {\em true} is returned. \\

 
\indent {\em TaggedObject *removeComponent(int tag);}\\
To remove the component whose tag is given by {\em tag} from the
container and return a pointer to the object. If tag is not too large
and all components have been added nicely, the contents of the array
at location {\em tag} is set to $0$ and its old contents
returned. Otherwise if the tag is not too large, the contents at
position {\em tag} is first checked to see if it is pointing to an
object and if this object has the same tag as {\em tag}, if it does
the contents of the array is set to $0$ and the object returned. If
the object is not at it's nice location, the array is searched from
the start to the position holding the last entry to see if the array
points to the object with the appropriate tag. If it 
does the array location is set to $0$ and the object returned,
otherwise $0$ is returned. If the object has not been found after the
last possible location has been checked, $0$ is returned. \\

\indent {\em int  getNumComponents(void) const;}\\
Returns the number of components currently stored in the container. \\

\indent {\em TaggedObject *getComponentPtr(int tag);}\\
To return a pointer to the TaggedObject whose identifier is given by
{\em tag}. If tag is not too large
and all components have been added nicely, the contents of the array
at location {\em tag} is returned. Otherwise if the tag is not too
large, the contents at position {\em tag} is first checked to see if it
is pointing to an object and if this object has the same tag as {\em
tag}, the object returned. If 
the object is not at it's nice location, the array is searched from
the start to the position holding the last entry to see if the array
points to the object with the appropriate tag. If it does the object
is returned. If the object has not been found after the last possible
location has been checked, $0$ is returned. \\


\indent {\em TaggedObjectIter \&getComponents(void);}\\
To return an iter for iterating through the objects that have been
added to the container. Each container object has its own iter. This
iter() is reset to point to the start and a reference to this iter is
returned.\\ 

\indent {\em TaggedObjectStorage *getEmptyCopy(void);}\\
To return an empty copy of the container. Creates a new
ArrayOfTaggedObjects object using the current size of the array as the
argument for the constructor. It is up to the user of this method to
invoke the destructor on the new object. \\

\indent {\em void clearAll(void);}\\
To remove all objects from the container and {\bf to invoke the
destructor on these objects}. Goes through the array, invoking the
destructor on any object pointed to by a cell and then setting this
cell to $0$. Resets the internal member data to indicate that zero
components have been added to the container.\\

\indent {\em void Print(OPS_Stream \&s, int flag =0);}\\
Invokes {\em Print(s,flag)} on all objects which have been added to
the container. 







\pagebreak

\subsubsection{ MapOfTaggedObjects}
% File: ~/tagged/storage/MapOfTaggedObjects.tex 
% What: "@(#) MapOfTaggedObjects.tex, revA"

\noindent {\bf Files}   \\
\indent \#include $<\tilde{ }$/tagged/storage/MapOfTaggedObjects.h$>$ \\

\noindent {\bf Class Declaration}  \\
\indent class MapOfTaggedObjects  \\

\noindent {\bf Class Hierarchy} \\
\indent  {TaggedObjectStorage} \\
\indent\indent  {\bf MapOfTaggedObjects} \\

\noindent {\bf Description}  \\
\indent A MapOfTaggedObjects object is used as a container to store and
provide access to objects of type TaggedObject. A MapOfTaggedObjects
creates a map object to store the pointers to these objects. A map is
created using a template provided by the standard template
library. The key used to identify the pointers stored in the map
object is the TaggedObjects tag. Each MapOfTaggedObject object also contains
a MapOfTaggedObjectsIter object to iterate through the objects which
have been added.\\ 

\noindent {\bf Class Interface}  \\
\indent // Constructor  \\
\indent {\em MapOfTaggedObjects();} \\\\
\indent // Destructor \\
\indent {\em  ~MapOfTaggedObjects();} \\  \\
\indent // Pure Public Methods \\
\indent {\em int  setSize(int newSize);}\\
\indent {\em bool addComponent(TaggedObject *newComponent, bool
allowMultiple = false);}\\
\indent {\em TaggedObject *removeComponent(int tag);}\\
\indent {\em int getNumComponents(void) const;}\\
\indent {\em TaggedObject *getComponentPtr(int tag);}\\
\indent {\em TaggedObjectIter \&getComponents(void);}\\
\indent {\em TaggedObjectStorage *getEmptyCopy(void)}\\
\indent {\em void clearAll(void);}\\
\indent {\em void Print(OPS_Stream \&s, int flag);}\\

\noindent {\bf Constructor} \\
\indent {\em MapOfTaggedObjects();} \\
Creates the map object and an iter for iterating through the objects
that are added to the map. \\

\noindent {\bf Destructor} \\
\indent {\em ~MapOfTaggedObjects();} \\ 
Does nothing.\\

\noindent {\bf Public Member Functions} \\
\indent {\em int setSize(int newSize);}\\
Checks to see that max size for the map (which is a built in value
defined for the template class) is larger than {\em newSize}. Returns
$0$ if successful. If not successful, the warning method is invoked
on the global ErrorHandler and $-1$ is returned.\\

\indent {\em bool addComponent(TaggedObject *newComponent,
bool allowMultiple);}\\
To add the object {\em newComponent} to the container. First checks to
see if an element with a similar tag already exists in the map. If
not, the pointer to {\em newElement} is added to the map using the
{\em insert()} method. A check is then made to ensure that the object
has been added. (This is done as {\em insert()} returns no error flag).
Returns {\em true} if successful. If not successful, the warning
method is invoked on the global ErrorHandler and {\em false} is
returned. Note that the map template does not allow items with
duplicate keys to be added.\\ 
 
\indent {\em TaggedObject *removeComponent(int tag);}\\
To remove the component whose tag is given by {\em tag} from the
container and return a pointer to the object. Invokes {\em find(tag)}
on the map to first see if the element is there. If it is {\em
erase(tag)} is invoked on the map to remove the item. $0$ is returned
if the component is not in the map, otherwise a pointer to the component
is returned.\\

\indent {\em int  getNumComponents(void) const;}\\
Returns the number of components currently stored in the
container. This is found by invoking {\em size()} on the map object.\\

\indent {\em TaggedObject *getComponentPtr(int tag);}\\
To return a pointer to the TaggedObject whose identifier is given by
{\em tag}. Invokes {\em find(tag)} on the map to determine if the
component is in the container. If it is a pointer to the component is
returned. If it is not in the map $0$ is returned. \\

\indent {\em TaggedObjectIter \&getComponents(void);}\\
To return an iter for iterating through the objects that have been
added to the container. Each MapOfTaggedObjects object has its own iter. This
iter() is first reset and a reference to this iter is then returned.\\ 

\indent {\em TaggedObjectStorage *getEmptyCopy(void)}\\
Returns a pointer to a new MapOfTaggedObjects which was created using
{\em new()}. The new container that is returned is an empty container.
If not enough memory is available to create this object the warning
method on the global ErrorHandler is invoked and $0$ is returned.
Note that it is the responsibility of the caller to invoke the destructor on the
object that is returned. \\


\indent {\em void clearAll(void);}\\
To remove all objects from the container and {\bf to invoke the
destructor on these objects}. Goes through the container, invoking the
destructor on any object in the map. It then invokes {\em clear()} on
the map object to clear it. \\

\indent {\em void Print(OPS_Stream \&s, int flag =0);}\\
Invokes {\em Print(s,flag)} on all objects which have been added to
the container. 







\pagebreak

\subsection{{\bf FE\_Datastore}}
%File: ~/database/FE\_Datastore.tex
%What: "@(#) FE\_Datastore.tex, revA"

\noindent {\bf Files}   \\
\indent \#include $<\tilde{ }$/database/FE\_Datastore.h$>$  \\

\noindent {\bf Class Declaration}  \\
\indent class FE\_Datastore: public ModelBuilder, public Channel \\

\noindent {\bf Class Hierarchy} \\
\indent ModelBuilder \\
\indent Channel \\
\indent\indent {\bf FE\_Datastore} \\

\noindent {\bf Description}  \\
\indent FE\_Datastore is an abstract class. An FE\_Datastore object is
used in the program to store/restore the geometry and state information 
in the domain at particular instances. How, where and how the data is
stored depends on the implementation provided by the concrete
subclasses. \\  

\noindent {\bf Class Interface} \\
\indent // Constructor \\
\indent {\em FE\_Datastore(Domain \&theDomain, FEM\_ObjectBroker \&theBroker);}\\ \\
\indent // Destructor \\
\indent {\em virtual~ $\tilde{}$FE\_Datastore();}\\ \\
\indent // Public Methods  \\
\indent {\em virtual int getDbTag(void) =0;}\\
\indent {\em virtual int commitState(int commitTag);}\\
\indent {\em virtual int restoreState(int commitTag);}\\\\
\indent {\em virtual int validateBaseRelationsWrite(int commitTag)=0;}\\
\indent {\em virtual int validateBaseRelationsRead(int commitTag)=0;}\\ \\
\indent // Protected Methods  \\
\indent {\em FEM\_ObjectBroker *getObjectBroker(void);}\\

\noindent {\bf Constructor}  \\
\indent {\em FE\_Datastore(Domain \&theDomain);}  \\ \\
The Domain object {\em theDomain} is passed to the ModelBuilder
constructor. A pointer is kept to {\em theBroker} object.\\

\noindent {\bf Destructor} \\
\indent {\em virtual~ $\tilde{}$FE\_Datastore();}\\ 
Does nothing. \\

\noindent {\bf Public Methods }  \\
\indent {\em virtual int getDbTag(void) =0;}\\
To return a unique integer identifier at each call. This identifier
will be used by the objects to store/retrieve their information
to/from the database. \\

\indent {\em virtual int commitState(int commitTag) = 0;}\\
Invoked to store the current state of the domain in the database. The
integer {\em commitTag} is used to identify the state for subsequent
calls to restore the information from the database. To return
$0$ if successful, a negative number if not. 

In the implementation for the FE\_Datastore class, the object first
invokes {\em validateBaseRelationsWrite()} on itself. If this method 
returns $0$, the object then loops over all the components of the
Domain object invoking {\em sendSelf(commitTag, this)} on each of
these objects. Returns $0$ if successful, a negative number if
not. For each domain component that could not send itself a warning
message is printed. \\ 

\indent {\em virtual int restoreState(int commitTag) = 0;}\\
Invoked to restore the state of the domain from a database. The state
of the domain at the end of this call is to be the same as the state
of the domain when {\em commitState(commitTag)} was invoked. To return
$0$ if successful, a negative number if not. 

In the implementation for the FE\_Datastore class, the object first
invokes {\em validateBaseRelationsRead()} on itself. If this method 
returns $0$, the object then loops over all the components of the
Domain object invoking {\em recvSelf(commitTag, this)} on each of
these objects. Returns $0$ if successful, a negative number if
not. For each domain component that could not send itself a warning
message is printed. \\ 


\indent {\em virtual int validateBaseRelationsWrite(int commitTag)=0;}\\
This method is invoked before the information can be sent to the
database. It is required to ensure that: \begin{enumerate} 
\item Each Node, Element, SP\_Constraint, MP\_Constraint, NodalLoad and
ElementalLoad which is to save information in the database has a
database tag.
\item That the information in the base tables is up to date so that a
later call to {\em validateBaseRelationsRead(commitTag)} will be successful.
\end{enumerate}
\noindent To return $0$ if the base relations are up to date, to return
$1$ if they are up to date and the component data has been sent to the
database, and a negative number if the method fails. \\

\indent {\em virtual int validateBaseRelationsRead(int commitTag) =0;}\\
This method is invoked before the information can be extracted from the
database. It is required to ensure that the Domain has the same
type of DomainComponent objects and that each of these has the same
database tag as when {\em validateBaseRealationsWrie(commitTag)} was
invoked.  To return $0$ if the base relations are up to date, to return
$1$ if they are up to date and the component data has been received from the
database, and a negative number if the method fails. \\

\indent {\em FEM\_ObjectBroker *getObjectBroker(void);}\\
Returns a pointer to {\em theBroker} object passed in the constructor. 
\pagebreak

\pagebreak \subsubsection{FileDatastore}
%File: ~/database/FileDatastore.tex
%What: "@(#) FileDatastore.tex, revA"

\noindent {\bf Files}   \\
\indent \#include $<\tilde{ }$/database/FileDatastore.h$>$  \\

\noindent {\bf Class Declaration}  \\
\indent class FileDatastore: public FE\_Datastore \\

\noindent {\bf Class Hierarchy} \\
\indent ModelBuilder \\
\indent Channel \\
\indent\indent FE\_Datastore \\
\indent\indent\indent {\bf FileDatastore} \\

\noindent {\bf Description}  \\
\indent FileDatastore is a concrete class. An FileDatastore object is
used in the program to store/restore the geometry and state information 
in the domain at particular instances. This information is stored in
binary form in files. As no standard format is used for the storage of
integers and double values, files used to store the data on one type
of machine, may not be read by a FileDatastore object on another type
of machine where the storage of integers and doubles is different. \\

For each of the base relations, i.e. Domain, Nodes, Elements,
SP\_Constraints, MP\_Constraints, NodalLoads and ElementalLoads, a
separate file is used to store the information. Files are also used
for each size of ID, Vector and Matrix stored. At present, Messages
are not stored, only ID and Vector objects of size $<= 200$ can be
stored, the max $noRows * noCols$ of Matrices that can be stored
is $<= 2000$, and only a single relation is created for Matrices which
have similar sizes but differing dimensions. The data is stored in the
files following the schema outlined previously.\\


\noindent {\bf Class Interface} \\
\indent // Constructor \\
\indent {\em FileDatastore(char *name, Domain \&theDomain, FEM\_ObjectBroker \&theBroker);}  \\ \\
\indent // Destructor \\
\indent {\em $\tilde{ }$FileDatastore();}\\ \\
\indent // Public Methods  inherited from the ModelBuilder Class \\
\indent {\em int buildFE\_Model(void);}\\ \\
\indent // Public Methods  inherited from the FE\_Datastore Class \\
\indent {\em int getDbTag(void);}\\
\indent {\em int validateBaseRelationsWrite(int commitTag); }\\
\indent {\em int validateBaseRelationsRead(int commitTag); }\\ \\
\indent// Public Methods  inherited from the Channel Class \\
\indent {\em char *addToProgram(void); }\\
\indent {\em int setUpShadow(void);}\\
\indent {\em int setUpActor(void);}\\
\indent {\em int setNextAddress(const ChannelAddress \&otherChannelAddress);}\\
\indent {\em ChannelAddress *getLastSendersAddress(void);}\\
\indent {\em int sendObj(int commitTag, \\
\indent\indent\indent\indent\indent		MovableObject \&theObject, \\
\indent\indent\indent\indent\indent		ChannelAddress *theAddress =0);}\\
\indent {\em int recvObj(int commitTag, \\
\indent\indent\indent\indent\indent		MovableObject \&theObject, \\
\indent\indent\indent\indent\indent		FEM\_ObjectBroker \&theBroker, \\
\indent\indent\indent\indent\indent		ChannelAddress *theAddress =0);}\\
		
\indent {\em int sendMsg(int dbTag, int commitTag,  \\
\indent\indent\indent\indent\indent		const Message \&, \\
\indent\indent\indent\indent\indent		ChannelAddress *theAddress =0);}\\
\indent {\em int recvMsg(int dbTag, int commitTag, \\
\indent\indent\indent\indent\indent		Message \&, \\
\indent\indent\indent\indent\indent		ChannelAddress *theAddress =0);}\\
\indent {\em int sendMatrix(int dbTag, int commitTag,  \\
\indent\indent\indent\indent\indent		   const Matrix \&theMatrix, \\
\indent\indent\indent\indent\indent		   ChannelAddress *theAddress =0);}\\
\indent {\em int recvMatrix(int dbTag, int commitTag, \\
\indent\indent\indent\indent\indent		   Matrix \&theMatrix,  \\
\indent\indent\indent\indent\indent		   ChannelAddress *theAddress =0);}\\
\indent {\em int sendVector(int dbTag, int commitTag, \\
\indent\indent\indent\indent\indent const Vector \&theVector, \\
\indent\indent\indent\indent\indent ChannelAddress *theAddress =0);}\\
\indent {\em int recvVector(int dbTag, int commitTag,  \\
\indent\indent\indent\indent\indent		   Vector \&theVector,  \\
\indent\indent\indent\indent\indent		   ChannelAddress *theAddress =0);}\\
\indent {\em int sendID(int dbTag, int commitTag, \\
\indent\indent\indent\indent\indent	   const ID \&theID, \\
\indent\indent\indent\indent\indent	   ChannelAddress *theAddress =0);}\\
\indent {\em int recvID(int dbTag, int commitTag, \\
\indent\indent\indent\indent\indent	   ID \&theID, \\
\indent\indent\indent\indent\indent	   ChannelAddress *theAddress =0);}\\


\noindent {\bf Constructor}  \\
\indent {\em FileDatastore(char *name, Domain \&theDomain, FEM\_ObjectBroker \&theBroker);}  \\ 
Opens the files for the domain and base component relations, files have names {\em
name.relation}, and stores the end of file locations. Creates three arrays
of file pointers for the ID, Vector and Matrix files and then zeros
these arrays. If the files could not be opened, or there is not enough
memory for the arrays an error message is printed and the program
is terminated. \\

\noindent {\bf Destructor} \\
\indent {\em virtual~ $\tilde{}$FileDatastore();}\\ 
Each file that is opened is closed and the arrays of file pointers
obtained from the heap in the constructor are returned to the heap. \\

\noindent {\bf Public Methods }  \\
\indent {\em int buildFE\_Model(void);}\\ 
To build the finite element model from data in the database. It does
this by invoking {\em restor(0)} on itself. \\

\indent {\em int getDbTag(void);}\\
Increments the integer containing the current dbTag and returns this
integer. \\

\indent {\em int validateBaseRelationsWrite(int commitTag); }\\
The object first checks to see if the Domain has already been
committed to the database with a similar {\em commitTag}. If it has, a
check is made to ensure that the current domain stamp and the one at
the time of this last commit are the same. If they are different, an
error message is printed and $-1$ is returned. A check is then made to
see if the component base relations have been updated for this current domain
stamp. If they have not been the base relations are updated, and each
domain component is asked to send its data to the database. Finally
the Domain relation is updated with the current time, load factor and
domain stamp. Returns $1$ if the component base relations needed to be updated
and the component information sent, $0$ if just the Domain relation
needed to be updated. A warning message and a negative number is
printed if an error occurs. \\


\indent {\em int validateBaseRelationsRead(int commitTag); }\\ 
The object first obtains from the Domain relation the information for
{\em commitTag}. If no information exists, an error message is printed
and a $-1$ is returned. A check is then made to see if the domain
stamp for this entity and the Domain objects current stamp are the
same. If different, {\em clearAll()} is invoked on the Domain, and
from the component base relations new domain components are created,
are asked to {\em recvSelf()} from {\em *this} and these new
components are added to the Domain. Finally the current time, domain
stamp and load factor are set using the information in the entity. 
Returns $1$ if the domain needed to be cleared and new component
objects created, $0$ if just the Domain object needed to be updated
with current time and load factor. A warning message and a negative
number is printed if an error occurs. \\

\indent {\em char *addToProgram(void); }\\
Returns $0$. \\

\indent {\em int setUpShadow(void);}\\
Returns $0$. \\

\indent {\em int setUpActor(void);}\\
Returns $0$. \\

\indent {\em int setNextAddress(const ChannelAddress \&otherChannelAddress);}\\
Returns $0$. \\

\indent {\em ChannelAddress *getLastSendersAddress(void);}\\
Returns $0$. \\

\indent {\em int sendObj(int commitTag, \\
\indent\indent\indent\indent\indent		MovableObject \&theObject, \\
\indent\indent\indent\indent\indent		ChannelAddress *theAddress =0);}\\
Returns the result of invoking {\em sendSelf(commitTag, *this)} on
{\em theObject}. \\

\indent {\em int recvObj(int commitTag, \\
\indent\indent\indent\indent\indent		MovableObject \&theObject, \\
\indent\indent\indent\indent\indent		FEM\_ObjectBroker \&theBroker, \\
\indent\indent\indent\indent\indent		ChannelAddress *theAddress =0);}\\
Returns the result of invoking {\em recvSelf(commitTag, *this, theBroker)} on
{\em theObject}. \\
		
\indent {\em int sendMsg(int dbTag, int commitTag,  \\
\indent\indent\indent\indent\indent		const Message \&, \\
\indent\indent\indent\indent\indent		ChannelAddress *theAddress =0);}\\
Prints an error message and returns $-1$ as not yet implemented. \\

\indent {\em int recvMsg(int dbTag, int commitTag, \\
\indent\indent\indent\indent\indent		Message \&, \\
\indent\indent\indent\indent\indent		ChannelAddress *theAddress =0);}\\
Prints an error message and returns $-1$ as not yet implemented. \\

\indent {\em int sendMatrix(int dbTag, int commitTag,  \\
\indent\indent\indent\indent\indent		   const Matrix \&theMatrix, \\
\indent\indent\indent\indent\indent		   ChannelAddress *theAddress =0);}\\
First determines the size of the matrix, $noRows * noCols$. If a files
for matrices of this size has not yet been created, one is created now
and the cell in the array of file pointers is set. If file can not be
created a warning message is printed and program is terminated. A sequential search
is made in the file to see if information is already stored for a Matrix with
this {\em dbTag} and {\em commitTag}. The data is then written at this
location, or eof if no location was found. The end of file location
for Matrices of this size is updated. If successful $0$ is
returned. A warning message and a negative number is returned if the
operation fails: $-1$ if Matrix size is too large. \\

\indent {\em int recvMatrix(int dbTag, int commitTag, \\
\indent\indent\indent\indent\indent Matrix \&theMatrix,  \\
\indent\indent\indent\indent\indent ChannelAddress *theAddress =0);}\\
First determines the size of the matrix, $noRows * noCols$. If a files
for matrices of this size has not yet been created, an error message
is printed and $-1$ is returned.  A sequential search
is made in the file to see if information is already stored for a Matrix with
this {\em dbTag} and {\em commitTag}. If no information is stored a
$-1$ is returned. If information is stored, the information is
retrieved and the data in the Matrix is set. returns $0$ if
successful. \\

\indent {\em int sendVector(int dbTag, int commitTag, \\
\indent\indent\indent\indent\indent const Vector \&theVector, \\
\indent\indent\indent\indent\indent ChannelAddress *theAddress =0);}\\
If a file
for Vectors of this size has not yet been created, one is created now
and the cell in the array of file pointers is set. If file can not be
created a warning message is printed and program is terminated. A sequential search
is made in the file to see if information is already stored for a Vector with
this {\em dbTag} and {\em commitTag}. The data is then written at this
location, or eof if no location was found. The end of file location
for Vectors of this size is updated. If successful $0$ is
returned. A warning message and a negative number is returned if the
operation fails: $-1$ if Vector size is too large. \\

\indent {\em int recvVector(int dbTag, int commitTag,  \\
\indent\indent\indent\indent\indent		   Vector \&theVector,  \\
\indent\indent\indent\indent\indent		   ChannelAddress *theAddress =0);}\\
If a file for Vectors of this size has not yet been created, an error message
is printed and $-1$ is returned.  A sequential search
is made in the file to see if information is already stored for a Vector with
this {\em dbTag} and {\em commitTag}. If no information is stored a
$-1$ is returned. If information is stored, the information is
retrieved and the data in the Vector is set. Returns $0$ if
successful. \\

\indent {\em int sendID(int dbTag, int commitTag, \\
\indent\indent\indent\indent\indent	   const ID \&theID, \\
\indent\indent\indent\indent\indent	   ChannelAddress *theAddress =0);}\\
If a file
for IDs of this size has not yet been created, one is created now
and the cell in the array of file pointers is set. If file can not be
created a warning message is printed and program is terminated. A sequential search
is made in the file to see if information is already stored for a ID with
this {\em dbTag} and {\em commitTag}. The data is then written at this
location, or eof if no location was found. The end of file location
for IDss of this size is updated. If successful $0$ is
returned. A warning message and a negative number is returned if the
operation fails: $-1$ if ID size is too large. \\

\indent {\em int recvID(int dbTag, int commitTag, \\
\indent\indent\indent\indent\indent	   ID \&theID, \\
\indent\indent\indent\indent\indent	   ChannelAddress *theAddress =0);}\\
If a file for IDs of this size has not yet been created, an error message
is printed and $-1$ is returned.  A sequential search
is made in the file to see if information is already stored for a ID with
this {\em dbTag} and {\em commitTag}. If no information is stored a
$-1$ is returned. If information is stored, the information is
retrieved and the data in the ID is set. Returns $0$ if successful. 
\pagebreak

\subsection{{\bf Recorder}}
%File: ~/recorder/Recorder.tex
%What: "@(#) Recorder.tex, revA"

\noindent {\bf Files}   \\
\indent \#include $<\tilde{ }$/recorder/Recorder.h$>$  \\

\noindent {\bf Class Declaration}  \\
\indent class Recorder \\

\noindent {\bf Class Hierarchy} \\
\indent {\bf Recorder} \\

\noindent {\bf Description}  \\
\indent The Recorder class is an abstract class which is introduced to allow
information to be saved during the analysis. The interface defines two
pure virtual methods {\em record()} and {\em playback()}. {\em
record()} is a method which is called by the Domain object during a
{\em commit()}. The {\em playback()} method can be called by the analyst after
the analysis has been performed. \\

\noindent {\bf Class Interface} \\
\indent // Constructor \\
\indent {\em Recorder();}\\ \\
\indent // Destructor \\
\indent {\em virtual $\tilde{ }$Recorder();}\\ \\
\indent // Public Methods  \\
\indent {\em virtual int record(int commitTag) =0;}\\
\indent {\em virtual int playback(int commitTag) =0;}\\
\indent {\em virtual void restart(void) =0;}\\ 

\noindent {\bf Constructor}  \\
\indent {\em Recorder();}  \\ 
Does nothing.\\

\noindent {\bf Destructor} \\
\indent {\em virtual~ $\tilde{}$Recorder();}\\ 
Does nothing. \\

\noindent {\bf Public Methods }  \\
\indent {\em virtual int record(int commitTag) =0;}\\
Invoked by the Domain object after {\em commit()} has been invoked on all the
domain component objects. What the Recorder records depends on the
concrete subtype. \\

\indent {\em virtual int playback(int commitTag) =0;}\\
Invoked by the analyst after the analysis has been performed. What the
Recorder does depends on the concrete subtype. \\

\indent {\em virtual void restart(void) =0;}\\ 
Invoked by the Domain object when {\em revertToStart()} is invoked on
the Domain object. What the Recorder does depends on the concrete subtype. 
\pagebreak

\pagebreak \subsubsection{MaxNodeDispRecorder}
%File: ~/recorder/MaxNodeDispRecorder.tex
%What: "@(#) MaxNodeDispRecorder.tex, revA"

\noindent {\bf Files}   \\
\indent \#include $<\tilde{ }$/recorder/MaxNodeDispRecorder.h$>$  \\

\noindent {\bf Class Declaration}  \\
\indent class MaxNodeDispRecorder: public Recorder\\

\noindent {\bf Class Hierarchy} \\
\indent Recorder \\
\indent\indent {\bf MaxNodeDispRecorder} \\

\noindent {\bf Description}  \\
\indent The MaxNodeDispRecorder class is used to store information
about the absolute maximum nodal displacement at a number of specified
Nodes for a specified degree of freedom during an analysis. \\

\noindent {\bf Class Interface} \\
\indent // Constructor \\
\indent {\em MaxNodeDispRecorder(int dof, const ID \&nodeTags, Domain
\&theDomain);}\\ \\ 
\indent // Destructor \\
\indent {\em $\tilde{ }$MaxNodeDispRecorder();}\\ \\
\indent // Public Methods  \\
\indent {\em int record(int commitTag);}\\
\indent {\em int playback(int commitTag);}\\
\indent {\em void restart(void);}\\ 

\noindent {\bf Constructor}  \\
\indent {\em MaxNodeDispRecorder(int dof, const ID \&nodeTags,
Domain \&theDomain);}\\ 
Creates a $0$ Vector object of size equal to the size of {\em
nodeTags} to store the maximum nodal displacements
and creates an ID object equal to {\em nodeTags} to store the Node
tags. \\

\noindent {\bf Destructor} \\
\indent {\em virtual~ $\tilde{}$MaxNodeDispRecorder();}\\ 
Does nothing. \\

\noindent {\bf Public Methods }  \\
\indent {\em int record(int commitTag);}\\
For each node in {\em nodeTags} the value of the {\em dof}'th committed
displacement is obtained. If the absolute value of this is greater
than the value currently stored in the Vector of max displacements,
the value in the Vector is updated. If no Node exists in {\em
theDomain} with the tag or the Node does not have a {\em dof}'th
degree-of-freedom associated with it $0$ is entered in the
Vector. Returns $0$. \\

\indent {\em int playback(int commitTag);}\\
Prints to {\em opserr} the Vector containing the maximum absolute nodal
displacements. Note, at the end of the analysis, what is printed is
independent of {\em commitTag}. Returns $0$. \\

\indent {\em void restart(void);}\\ 
Zeros the Vector of maximum nodal displacements.
\pagebreak

\pagebreak \subsubsection{FileNodeDispRecorder}
\input{../recorder/FileNodeDispRecorder}
\pagebreak

\pagebreak \subsubsection{DatastoreRecorder}
%File: ~/recorder/DatastoreRecorder.tex
%What: "@(#) DatastoreRecorder.tex, revA"

\noindent {\bf Files}   \\
\indent \#include $<\tilde{ }$/recorder/DatastoreRecorder.h$>$  \\

\noindent {\bf Class Declaration}  \\
\indent class DatastoreRecorder: public Recorder\\

\noindent {\bf Class Hierarchy} \\
\indent Recorder \\
\indent\indent {\bf DatastoreRecorder} \\

\noindent {\bf Description}  \\
\noindent A DatastoreRecorder object is used in the program to invoke
{\em commitState()} on an FE\_Datastore object when {\em commit()} is
invoked on a Domain. The class is included in the framework so that
the Domain class does not have to be modified for FE\_Datastore
objects. \\ 


\noindent {\bf Class Interface} \\
\indent // Constructor \\
\indent {\em DatastoreRecorder(FE\_Datastore \&theDatastore);}\\ \\ 
\indent // Destructor \\
\indent {\em $\tilde{ }$DatastoreRecorder();}\\ \\
\indent // Public Methods  \\
\indent {\em int record(int commitTag);}\\
\indent {\em int playback(int commitTag);}\\
\indent {\em void restart(void);}\\ 

\noindent {\bf Constructor}  \\
\indent {\em DatastoreRecorder(FE\_Datastore \&theDatastore);}\\ 
Saves a pointer to the object {\em theDatastore}. \\

\noindent {\bf Destructor} \\
\indent {\em virtual~ $\tilde{}$DatastoreRecorder();}\\ 
Does nothing. \\

\noindent {\bf Public Methods }  \\
\indent {\em int record(int commitTag);}\\
Returns the result of invoking {\em commitState(commitTag)} on {\em
theDatastore} object. \\ 

\indent {\em int playback(int commitTag);}\\
Returns the result of invoking {\em restoreState(commitTag)} on {\em
theDatastore} object. \\ 

\indent {\em void restart(void);}\\ 
Does nothing.
\pagebreak
\pagebreak

\section {Visualization Classes} 
These are classes used to present images
of the model. These include the abstract classes Renderer and ColorMap.
There are two steps in the creation of any graphical image: (1) create
a model of the object to be displayed; and (2) render the model to an image
that is viewed on the screen. The creation and rendering can be part
of the same application, or can be split into two separate
applications, where a 3d description of the model using a scene
description language is o/p by the modeler and i/p by the renderer.  

In our finite element work {\bf we have the model}, the domain. Though
the model we have is not typical, as it requires the displaying of scalar and
vector quantities and not just rgb values. {\bf We need to develop an
interface for the renderer}. To do this we will introduce an abstract
class {\bf Renderer} which defines this interface. Doing this will
allow full extensibility as it will allow concrete subclasses to be
provided which may render the model to the screen, or generate an SDL
file which can be read in by a rendering application or printed on a
printer at a later stage.

The interface for the Renderer class we introduce will be very
simple. It will only accept very simple primitive objects to be
displayed (line and polygon). As such, the Renderer will not be
required to render the components of the model, rather the components
of the model will be required to display themselves. This will allow
both the introduction of new component types, for example new element
types, without existing Renderer classes being required to be
rewritten and also new Renderer classes will be able to be introduced
with the only requirement that they be able to display the primitive
object types. The drawback of course is that the present design has to
be modified to allow the components to display themselves. On the
otherhand, as the objects can display themselves, it will allow
complex images to be displayed, e.g. 3d beam elements with proper
geometry and deformation along the length or beam with plastic hinges
at ends could indicate amount of rotation at ends. 

\pagebreak \subsection{{\bf Renderer}}
%File: ~/OOP/renderer/Renderer.tex
%What: "@(#) Renderer.tex, revA"

UNDER CONSTRUCTION .. POLYGONS and COLORMAP NEED A RETHINK.

\noindent {\bf Files}   \\
\indent \#include $<\tilde{ }$/earthquake/Renderer.h$>$  \\

\noindent {\bf Class Declaration}  \\
\indent class Renderer   \\

\noindent {\bf Class Hierarchy} \\
\indent {\bf Renderer} \\

\noindent {\bf Description} \\ 
\indent 
The Renderer class is an abstract class which is introduced to allow
the graphical display of the 3d finite element model. The Renderer class
is an abstract class, it defines the interface all concrete subclasses
must provide. In common with most rendering packages, the interface
allows only for the displaying of a few basic primitive objects, lines
and polygons (commercial renderers offer others such as cubes,
circles, cones). By providing only a few basic primitives, the
interface will allow complicated scenes to be created (much like the
ten digits 0-9 allows an infinite number of numbers to be created).

Unlike commercial renderers there is no concept of the optical
properties of materials and light sources in the interface. This is
because the image the analyst is looking for is not a photo realistic
image of the model, rather an image indicating the basic model,
stresses, and nodal displacements.   \\

\noindent {\bf Class Interface} \\
\indent // Constructor \\ 
\indent {\em Renderer(ColorMap \&theMap);}\\ \\
\indent // Destructor \\ 
\indent {\em virtual $\tilde{ }$Renderer();}\\  \\
\indent // Public Methods to display the Model\\ 
\indent // Public Methods invoked at start or end of a rendering\\ 
\indent {\em virtual int clearImage(void) =0;}\\
\indent {\em virtual int startImage(void) =0;}\\
\indent {\em virtual int doneImage(void) =0;}\\ \\
\indent // Public Methods invoked by the Objects to Display Themselves\\
\indent {\em virtual int drawLine(const Vector \&, const Vector \&, 
float V1, float V2) =0;}\\
\indent {\em virtual int drawLine(const Vector \&, const Vector \&, 
donst Vector \&rgb1, const Vector \&rgb2) =0;}\\
\indent {\em virtual int drawPolygon(const Vector *, const Vector \&);} \\
\indent {\em virtual int drawVector(const Vector \&position, 
const Vector \&value);}\\
\indent {\em virtual int drawGText(const Vector \&posGlobal, 
char *string, int length);}\\
\indent {\em virtual int drawLText(const Vector \&posLocal, 
char *string, int length); }\\
\indent {\em void setColorMap(ColorMap \&theMap);}\\ \\
\indent // Public Methods invoked to set up the viewing system \\
\indent {\em virtual int setVRP(float x, float y, float z) =0;}\\
\indent {\em virtual int setVPN(float x, float y, float z) =0;}\\
\indent {\em virtual int setVUP(float x, float y, float z) =0;}\\
\indent {\em virtual int setViewWindow(float uMin, float uMax, float
vMin, float vMax) =0;}\\
\indent {\em    virtual int setPlaneDist(float nearPlane, float farPlane) =0;}\\
\indent {\em    virtual int setProjectionMode(int) =0;}\\ 
\indent {\em    virtual int setFillMode(int) =0;}\\
\indent {\em    virtual int setPRP(float u, float v, float n) =0;}\\
\indent {\em    virtual int setPortWindow(float, float, float, float) =0;}\\


\noindent {\bf Constructor} \\ 
\indent {\em Renderer(Domain \&theDomain, ColorMap \&theMap);}\\ 
Keeps a pointer to Domain object {\em theDomain} and the ColorMap
object {\em theMap}. \\

\noindent {\bf Destructor} \\
\indent {\em virtual $\tilde{ }$Renderer();}\\  
Does Noting. \\

\noindent {\bf Public Methods} \\
\indent {\em virtual int clearImage(void) =0;}\\
Invoked to clear the image. This is required if multiple images can be
displayed at once for example in a graphic window. \\

\indent {\em virtual int startImage(void) =0;}\\
Invoked at the start of {\em displayModel()}, when the image is about
to be displayed. \\

\indent {\em virtual int doneImage(void) =0;}\\
Invoked at the end of {\em displayModel()}, when the image is finished
and ready to be displayed. \\

\indent {\em virtual int drawLine(const Vector \&end1, const Vector \&end2, 
float V1, float V2) =0;}\\
Invoked to draw a line in the image. The two vectors {\em end1} and
{\em end2} define the two end points in the 3d coordinate system, the
values {\em V1} and {\em V2} define the scalar quantity to be
displayed. To return $0$ if successful, a negative number if not.\\



\indent {\em virtual int drawPolygon(const Vector *ends, const Vector
\&values);} \\ 
Invoked to draw a polygon in the image. The Vectors pointed to in {\em
ends} define the end points of the polygon and the Vector {\em values}
define the scalar quantities to be displayed. Note that there must
be at least {\em values.Size()} Vectors pointed to at the start of
{\em ends}, otherwise a segmentation fault will occur. To return $0$
if successful, a negative number if not.\\

\indent {\em virtual int drawVector(const Vector \&position, 
const Vector \&value);}\\
Invoked to draw a vector in the image. The vector is to be drawn
at position {\em position}. The magnitude and direction of the vector
are given in {\em value}. To return $0$ if successful, a negative
number if not.\\ 


\indent {\em virtual int drawGText(const Vector \&posGlobal, 
char *string, int length);}\\
Invoked to draw a string of text in the image at position {\em
posGlobal}. The position is in the 3d model coordinates. The length of
the string is {\em length}. To return $0$ if successful, a negative
number if not.\\ 


\indent {\em virtual int drawLText(const Vector \&posLocal, 
char *string, int length); }\\
Invoked to draw a string of text in the image at position {\em
posGlobal}. The position is in the 3d model coordinates. The length of
the string is {\em length}. To return $0$ if successful, a negative
number if not.\\ 

\indent {\em void setColorMap(ColorMap \&theMap);}\\
To set the ColorMap used in the {\em draw()} routines. The ColorMap
object is used to convert scalar quantities into rgb values. 


\indent {\em virtual int setVRP(float x, float y, float z) =0;}\\
Invoked to set the VRP. The VRP is a point on the viewing plane, which
is that plane in the global coordinate system on which the 3d image is
projected. The point is given in the global coordinate system. \\

\indent {\em virtual int setVPN(float x, float y, float z) =0;}\\
Invoked to set the VPN. The VPN defines the perpendicular to the
viewing plane. \\

\indent {\em virtual int setVUP(float x, float y, float z) =0;}\\
Invoked to set the VUP. The VUP defines the up direction of the
viewing plane. The combination of VRP, VPN and VUP define a new local
coordinate system centered at VRP, with axis u,v,n, where the
direction of u is given by $vuv \% vpn$, v by $vpn \% u$ and n
finishes off the rhs local system. \\

\indent {\em virtual int setViewWindow(float uMin, float uMax, float
vMin, float vMax) =0;}\\
Invoked to set the bounds of the image on the view plane which is then
scaled to fit into the image displayed. The values provided are in the
local coordinate system. \\

\indent {\em virtual int setPlaneDist(float nearPlane, float farPlane)
=0;}\\
Sets the distance to the near and far clipping planes. \\

\indent {\em    virtual int setProjectionMode(int mode) =0;}\\ 
Sets the projection mode. If mode is $0$ a parallel projection of the
image onto the viewing plane will be displayed, if $1$ a perspective
projection. \\

\indent {\em    virtual int setFillMode(int mode) =0;}\\
Sets the fill mode for solid shapes, i.e. polygons. If {\em mode} is
$0$ a wire frame image will be displayed, otherwise solid shapes will
be filled. \\

\indent {\em    virtual int setPRP(float u, float v, float n) =0;}\\
Sets the PRP. The PRP defines the eye location of the viewer. It is
important to note that the eye location is specified in the new local
coordinate system. \\

\indent {\em    virtual int setPortWindow(float, float, float, float)
=0;}\\ 
Sets the portwindow. The portwindow is that region of the display
into which the image is displayed. The range is [-1,1.-1.1]. \\
\pagebreak

\subsubsection{ X11Renderer}
%\input{../renderer/X11Renderer}
\pagebreak

\subsubsection{ OpenGlRenderer}
%\input{../renderer/OpenGlRenderer}
\pagebreak

\subsubsection{ VrmlRenderer}
%\input{../renderer/VrmlRenderer}
\pagebreak

\subsection{{\bf ColorMap}}
%File: ~/OOP/renderer/ColorMap.tex
%What: "@(#) ColorMap.tex, revA"

THIS INTERFACE IS GONNA CHANGE. 1) MOVABLE\_OBJECT SO CAN RENDER IN
PARALLEL, 2) SET\_MINMAX() and 3) ONLY ONE CALL TO GET RGB VALUES, i.e.
getRGB(value, \&r, \&g, \&b) \\

\noindent {\bf Files}   \\
\indent \#include $<\tilde{ }$/earthquake/ColorMap.h$>$  \\

\noindent {\bf Class Declaration}  \\
\indent class ColorMap   \\

\noindent {\bf Class Hierarchy} \\
\indent {\bf ColorMap} \\

\noindent {\bf Description} \\ 
\indent 
The ColorMap is an abstract class, it defines the interface all
concrete subclasses must provide. A ColorMap object is used to
determine the mapping between scalar quantities to be displayed in an
image and the rgb values that are displayed. \\

\noindent {\bf Class Interface} \\
\indent // Constructor \\ 
\indent {\em ColorMap();}\\ \\
\indent // Destructor \\ 
\indent {\em virtual $\tilde{ }$ColorMap();}\\  \\
\indent // Public Methods\\ 
\indent {\em virtual float getRed(float value) =0;}\\
\indent {\em virtual float getGreen(float value) =0;}\\
\indent {\em virtual float getBlue(float value) =0;}\\

\noindent {\bf Constructor} \\ 
\indent {\em ColorMap();}\\ 
Does nothing. \\

\noindent {\bf Destructor} \\
\indent {\em virtual $\tilde{ }$ColorMap();}\\  
Does Noting. \\

\noindent {\bf Public Methods} \\
\indent {\em virtual float getRed(float value) =0;}\\
To return the red intensity of the rgb triple for the scalar quantity
{\em value}. \\

\indent {\em virtual float getGreen(float value) =0;}\\
To return the green intensity of the rgb triple for the scalar quantity
{\em value}. \\

\indent {\em virtual float getBlue(float value) =0;}\\
To return the blue intensity of the rgb triple for the scalar quantity
{\em value}. \\
\pagebreak
\pagebreak

\pagebreak
\section{Utility Classes}
\pagebreak \subsection{{\bf ConvergenceTest}}
%File: ~/OOP/convergenceTest/ConvergenceTest.tex
%What: "@(#) ConvergenceTest.tex, revA"

\noindent {\bf Files}   \\
\indent \#include $<\tilde{}$/convergenceTest/ConvergenceTest.h$>$  \\

\noindent {\bf Class Declaration}  \\
\indent class ConvergenceTest: public MovableObject  \\

\noindent {\bf Class Hierarchy} \\
\indent MovableObject \\
\indent\indent {\bf ConvergenceTest} \\

\noindent {\bf Description}  \\
\indent A ConvergenceTest object is an object which can be used in an
algorithmic class to test if convergence has been achieved for an 
iteration. The ConvergenceTest class is an abstract class, defining
the interface that all subclasses must provide. \\

\noindent {\bf Class Interface} \\
\indent\indent {// Constructors}  \\ 
\indent\indent {\em ConvergenceTest(int classTag);}  \\ \\
\indent\indent {// Destructor}  \\ 
\indent\indent {\em virtual $\tilde{}$ConvergenceTest();} \\ \\
\indent\indent {// Public Methods}  \\ 
\indent\indent {\em virtual int setEquiSolnAlgo(EquiSolnAlgo \&theAlgo) =0;} \\
\indent\indent {\em virtual int start(void) =0;} \\
\indent\indent {\em virtual int test(void) =0;} \\

\noindent {\bf Constructors}  \\
\indent {\em ConvergenceTest();}  \\
The integer {\em classTag} is passed to the MovableObject constructor. \\

\noindent {\bf Destructor} \\
\indent {\em virtual $\tilde{}$ConvergenceTest();} \\ 
Does nothing. \\

\noindent {\bf Public Methods }  \\
\indent {\em virtual int setEquiSolnAlgo(EquiSolnAlgo \&theAlgo) =0;} \\
To set the corresponding EquiSolnAlgo class. \\

\indent {\em virtual int test(void) =0;} \\
To return a positive number if the convergence criteria defined for the
object has been satisfied, the positibe number equal to the number of times 
since {\em start} that {\em test()} has been invoked. Otherwise a negative number 
is to be returned. A {\em -2} 
is returned if the test fails to meet the criteria and no more tests are to be 
performed due to limits set, i.e. the maximum number of iterations, otherwise a
{\em -1} is to be returned.  \\

\indent {\em virtual int start(void) =0;} \\
This is invoked at the start of each iteration. To return {\em 0} if successful, i.e 
that testing can proceed, a negative number if not. \\

\pagebreak
\subsubsection{ CTestNormUnbalance}
%File: ~/OOP/convergenceTest/CTestNormUnbalance.tex
%What: "@(#) CTestNormUnbalance.tex, revA"

\noindent {\bf Files}   \\
\indent \#include $<\tilde{}$/convergenceTest/CTestNormUnbalance.h$>$  \\

\noindent {\bf Class Declaration}  \\
\indent class CTestNormUnbalance: public ConvergenceTest  \\

\noindent {\bf Class Hierarchy} \\
\indent MovableObject \\
\indent\indent ConvergenceTest \\
\indent\indent\indent {\bf CTestNormUnbalance} \\

\noindent {\bf Description}  \\
\indent A CTestNormUnbalance object is an object which can be used in an
algorithmic class to test if convergence has been achieved. The
CTestNormUnbalance class tests using the norm of the right hand side
Vector of a LinearSOE object and a user specified tolerance value. \\


\noindent {\bf Class Interface} \\
\indent\indent {// Constructors}  \\ 
\indent\indent {\em CTestNormUnbalance(double tol, int maxNumIter, int
printFlag);}  \\  
\indent\indent {\em CTestNormUnbalance();}  \\ \\
\indent\indent {// Destructor}  \\ 
\indent\indent {\em $\tilde{ }$CTestNormUnbalance();} \\ \\
\indent\indent {// Public Methods}  \\ 
\indent\indent {\em int setTolerance(double newTol);} \\
\indent\indent {\em int setEquiSolnAlgo(EquiSolnAlgo \&theAlgo);} \\
\indent\indent {\em int start(void);} \\
\indent\indent {\em int test(void);} \\
\indent\indent {\em int sendSelf(Channel \&theChannel, FEM\_ObjectBroker
\&theBroker);}\\ 
\indent\indent {\em int recvSelf(Channel \&theChannel, FEM\_ObjectBroker
\&theBroker);}\\ 

\noindent {\bf Constructors}  \\
\indent {\em CTestNormUnbalance(double tol, int maxNumIter, int printFlag);}  \\ 
\indent {\em CTestNormUnbalance(double tol, int );}  \\
The integer {\em CLASS\_TAGS\_CTestNormUnbalance}, defined in
$<$classTags.h$>$, is passed to the ConvergenceTest constructor. 
Sets the tolerance used in {\em test()} to be {\em tol}, the max
number of iterations to be performed to {\em maxNumIter} and the print
flag used to determine what, if anything, is printed on each test to
{\em printFlag}. \\

\indent {\em CTestNormUnbalance();}  \\
To be used by the FEM\_ObjectBroker object in parallel programs. The
integer {\em CLASS\_TAGS\_CTestNormUnbalance}, defined in
$<$classTags.h$>$, is passed to the ConvergenceTest constructor. Sets
the tolerance used in {\em test()} to be {\em 0.0} and {\em maxNumIter} 
to be $0$. These will be set when {\em recvSelf()} is invoked on the object. \\

\noindent {\bf Destructor} \\
\indent {\em virtual $\tilde{}$CTestNormUnbalance();} \\ 
Does nothing. \\

\noindent {\bf Public Methods }  \\
\indent {\em int setTolerance(double newTol);} \\
Sets the tolerance used in {\em test()} to be {\em newTol}. \\

\indent {\em int setEquiSolnAlgo(EquiSolnAlgo \&theAlgo);} \\
It sets a pointer to {\em theAlgo}'s LinearSOE object. Returns $0$ if
successful, a $-1$ is returned and an error message printed if no
LinearSOE object has been set in {\em theAlgo}. \\

\indent {\em virtual void start(void);} \\
Sets an integer indicating the current number of iterations, {\em
currentNumIter} to $1$. Returns $0$ if successfull, an error message
and $-1$ are returned if no LinearSOE object has been set. \\ 

\indent {\em int test(void);} \\
Returns {currentNumIter} if if the two norm of the LinearSOE objects B
Vector is less than the tolerance {\em tol}. If no LinearSOE has been
set $-2$ is returned. If the {\em currentNumIter} $>=$ {\em
maxNumIter} an error message is printed and $-2$ is returned. If none
of these conditions is met, the {\em currentnumIter} is incremented 
and $-1$ is returned. If the print flag is $0$ nothing is printed to
cerr during the method, if $1$ the current iteration and norm are
printed to cerr, and if $2$ the norm and number of iterations to convergence
are printed to cerr. \\ 


\indent {\em int sendSelf(int commitTag, Channel \&theChannel);}\\ 
Creates a Vector of size 2, puts the tolerance value {\em tol} and {\em numIncr}
in this, and then invokes {\em sendVector()} on {\em
theChannel}. Returns $0$ if successful. A warning message is printed
and a negative number if the Channel object fails to send the Vector.\\ 

\indent {\em int recvSelf(int commitTag, Channel \&theChannel,
FEM\_ObjectBroker \&theBroker);}\\ 
Creates a Vector of size 2, invokes {\em recvVector()} on {\em
theChannel}, and sets the values of {\em tol} and {\em numIncr}.
Returns $0$ if successful. If the Channel object fails to receive the
Vector, {\em tol} is set to $1.0e-8$, {\em numIter} to $25$, a warning
message is printed, and a negative number returned.


\pagebreak
\subsubsection{ CTestNormDispIncr}
%File: ~/OOP/convergenceTest/CTestNormDispIncr.tex
%What: "@(#) CTestNormDispIncr.tex, revA"

\noindent {\bf Files}   \\
\indent \#include $<\tilde{}$/convergenceTest/CTestNormDispIncr.h$>$  \\

\noindent {\bf Class Declaration}  \\
\indent class CTestNormDispIncr: public ConvergenceTest  \\

\noindent {\bf Class Hierarchy} \\
\indent MovableObject \\
\indent\indent ConvergenceTest \\
\indent\indent\indent {\bf CTestNormDispIncr} \\

\noindent {\bf Description}  \\
\indent A CTestNormDispIncr object is an object which can be used in an
algorithmic class to test if convergence has been achieved. The
CTestNormDispIncr class tests using the norm of the solution Vector of
a LinearSOE object and a user specified tolerance value. \\ 


\noindent {\bf Class Interface} \\
\indent\indent {// Constructors}  \\ 
\indent\indent {\em CTestNormDispIncr(double tol, int maxNumIter = 25);}  \\ 
\indent\indent {\em CTestNormDispIncr();}  \\ \\
\indent\indent {// Destructor}  \\ 
\indent\indent {\em $\tilde{ }$CTestNormDispIncr();} \\ \\
\indent\indent {// Public Methods}  \\ 
\indent\indent {\em int setTolerance(double newTol);} \\
\indent\indent {\em int setEquiSolnAlgo(EquiSolnAlgo \&theAlgo);} \\
\indent\indent {\em int test(void);} \\
\indent\indent {\em int start(void);} \\
\indent\indent {\em int sendSelf(int commitTag, Channel \&theChannel);}\\ 
\indent\indent {\em int recvSelf(int commitTag, Channel \&theChannel,
FEM\_ObjectBroker \&theBroker);}\\ 

\noindent {\bf Constructors}  \\
\indent {\em CTestNormDispIncr(double tol, int maxNumIter);}  \\
The integer {\em CLASS\_TAGS\_CTestNormDispIncr}, defined in
$<$classTags.h$>$, is passed to the ConvergenceTest constructor. Sets
the tolerance used in {\em test()} to be {\em tol}, the max number
of iterations to be performed to {\em maxNumIter} and the print
flag used to determine what, if anything, is printed on each test to
{\em printFlag}. \\


\indent {\em CTestNormDispIncr();}  \\
To be used by the FEM\_ObjectBroker object in parallel programs. The
integer {\em CLASS\_TAGS\_CTestNormDispIncr}, defined in
$<$classTags.h$>$, is passed to the ConvergenceTest constructor. Sets
the tolerance used in {\em test()} to be {\em 0.0} and {\em maxNumIter} 
to be $0$. These will be set when {\em recvSelf()} is invoked on the object. \\


\noindent {\bf Destructor} \\
\indent {\em $\tilde{ }$CTestNormDispIncr();} \\ 
Does nothing. \\

\noindent {\bf Public Methods }  \\
\indent {\em int setTolerance(double newTol);} \\
Sets the tolerance used in {\em test()} to be {\em newTol}. \\

\indent {\em int setEquiSolnAlgo(EquiSolnAlgo \&theAlgo);} \\
It sets a pointer to {\em theAlgo}'s LinearSOE object. Returns $0$ if
successful, a $-1$ is returned and an error message printed if no
LinearSOE object has been set in {\em theAlgo}. \\

\indent {\em int start(void);} \\
Sets an integer indicating the current number of iterations, {\em
currentNumIter} to $1$. Returns $0$ if successfull, an error message
and $-1$ are returned if no LinearSOE object has been set. \\

\indent {\em int test(void);} \\
Returns {currentNumIter} if if the two norm of the LinearSOE objects X
Vector is less than the tolerance {\em tol}. If no LinearSOE has been
set $-2$ is returned. If the {\em currentNumIter} $>=$ {\em
maxNumIter} an error message is printed and $-2$ is returned. If none
of these conditions is met, the {\em currentnumIter} is incremented
and $-1$ is returned. If the print flag is $0$ nothing is printed to
opserr during the method, if $1$ the current iteration and norm are
printed to opserr, and if $2$ the norm and number of iterations to convergence
are printed to opserr. \\ 


\indent {\em int sendSelf(int commitTag, Channel \&theChannel);}\\ 
Creates a Vector of size 3, puts the tolerance value {\em tol}, {\em
numIncr} and {\em printFlag}
in this, and then invokes {\em sendVector()} on {\em
theChannel}. Returns $0$ if successful. A warning message is printed
and a negative number if the Channel object fails to send the Vector.\\ 

\indent {\em int recvSelf(int commitTag, Channel \&theChannel,
FEM\_ObjectBroker \&theBroker);}\\ 
Creates a Vector of size 3, invokes {\em recvVector()} on {\em
theChannel}, and sets the values of {\em tol}, {\em numIncr} and {\em printFlag}.
Returns $0$ if successful. If the Channel object fails to receive the
Vector, {\em tol} is set to $1.0e-8$, {\em numIter} to $25$, a warning
message is printed, and a negative number returned.









\pagebreak
\subsubsection{ CTestEnergyIncr}
%File: ~/OOP/convergenceTest/CTestEnergyIncr.tex
%What: "@(#) CTestEnergyIncr.tex, revA"

\noindent {\bf Files}   \\
\indent \#include $<\tilde{}$/convergenceTest/CTestEnergyIncr.h$>$  \\

\noindent {\bf Class Declaration}  \\
\indent class CTestEnergyIncr: public ConvergenceTest  \\

\noindent {\bf Class Hierarchy} \\
\indent MovableObject \\
\indent\indent ConvergenceTest \\
\indent\indent\indent {\bf CTestEnergyIncr} \\

\noindent {\bf Description}  \\
\indent A CTestEnergyIncr object is an object which can be used in an
algorithmic class to test if convergence has been achieved. The
CTestEnergyIncr class tests using the energy increment, 0.5 times the
absolute value of the dot product of the LinearSOE objects solution
and rhs Vectors, and a user specified tolerance value. \\


\noindent {\bf Class Interface} \\
\indent\indent {// Constructors}  \\ 
\indent\indent {\em CTestEnergyIncr(double tol, int maxNumIter, double
printFlag);}  \\  
\indent\indent {\em CTestEnergyIncr();}  \\ \\
\indent\indent {// Destructor}  \\ 
\indent\indent {\em $\tilde{}$CTestEnergyIncr();} \\ \\
\indent\indent {// Public Methods}  \\ 
\indent\indent {\em int setTolerance(double newTol);} \\
\indent\indent {\em int setEquiSolnAlgo(EquiSolnAlgo \&theAlgo);} \\
\indent\indent {\em int start(void);} \\
\indent\indent {\em int test(void);} \\
\indent\indent {\em int sendSelf(int commitTag, Channel \&theChannel);}\\ 
\indent\indent {\em int recvSelf(int commitTag, Channel \&theChannel,
FEM\_ObjectBroker \&theBroker);}\\ 

\noindent {\bf Constructors}  \\
\indent {\em CTestEnergyIncr(double tol, int maxNumIncr = 10);}  \\
The integer {\em CLASS\_TAGS\_CTestEnergyIncr}, defined in
$<$classTags.h$>$, is passed to the ConvergenceTest constructor. Sets
the tolerance used in {\em test()} to be {\em tol} and the max number
of iterations to be performed to {\em maxNumIter}. \\

\indent {\em CTestEnergyIncr();}  \\
To be used by the FEM\_ObjectBroker object in parallel programs. The
integer {\em CLASS\_TAGS\_CTestEnergyIncr}, defined in
$<$classTags.h$>$, is passed to the ConvergenceTest constructor. Sets
the tolerance used in {\em test()} to be {\em 0.0} and {\em maxNumIter} 
to be $0$. These will be set when {\em recvSelf()} is invoked on the object. \\


\noindent {\bf Destructor} \\
\indent {\em $\tilde{ }$CTestEnergyIncry();} \\ 
Does nothing. \\

\noindent {\bf Public Methods }  \\
\indent {\em int setTolerance(double newTol);} \\
Sets the tolerance used in {\em test()} to be {\em newTol}. \\

\indent {\em int setEquiSolnAlgo(EquiSolnAlgo \&theAlgo);} \\
It sets a pointer to {\em theAlgo}'s LinearSOE object. Returns $0$ if
successful, a $-1$ is returned and an error message printed if no
LinearSOE object has been set in {\em theAlgo}. \\


\indent {\em int start(void);} \\
Sets an integer indicating the current number of iterations, {\em
currentNumIter} to $1$. returns $0$ if successfull, an error message
and $-1$ are returned if no LinearSOE object has been set. \\

\indent {\em int test(void);} \\
Returns {currentNumIter} if 0.5 times the absolute value of the dot product of
the LinearSOE objects X and B Vectors is less than the tolerance {\em
tol}. If no LinearSOE has been set $-2$ is returned. If the {\em
currentNumIter} $>=$ {\em maxNumIter} an error message is printed and
$-2$ is returned. If none of these conditions is met, the {\em
currentnumIter} is incremented and $-1$ is returned. If the print flag
is $0$ nothing is printed to 
opserr during the method, if $1$ the current iteration and norm are
printed to opserr, and if $2$ the norm and number of iterations to convergence
are printed to opserr. \\ 


\indent {\em int sendSelf(int commitTag, Channel \&theChannel);}\\ 
Creates a Vector of size 3, puts the tolerance value {\em tol}, {\em
numIncr} and {\em printFlag}
in this, and then invokes {\em sendVector()} on {\em
theChannel}. Returns $0$ if successful. A warning message is printed
and a negative number if the Channel object fails to send the Vector.\\ 

\indent {\em int recvSelf(int commitTag, Channel \&theChannel,
FEM\_ObjectBroker \&theBroker);}\\ 
Creates a Vector of size 3, invokes {\em recvVector()} on {\em
theChannel}, and sets the values of {\em tol}, {\em numIncr} and {\em printFlag}.
Returns $0$ if successful. If the Channel object fails to receive the
Vector, {\em tol} is set to $1.0e-8$, {\em numIter} to $25$, a warning
message is printed, and a negative number returned.y





\pagebreak 
\subsection{Timer}
%File: ~/OOP/timer/Timer.tex
%What: "@(#) Timer.tex, revA"

\noindent {\bf Files}   \\
\indent \#include $<\tilde{}$/Timer/Timer.h$>$  \\

\noindent {\bf Class Declaration}  \\
\indent class Timer: public MovableObject  \\

\noindent {\bf Class Hierarchy} \\
\indent MovableObject \\
\indent\indent {\bf Timer} \\

\noindent {\bf Description}  \\
\indent A Timer object is an object which can be used to measure
system resources, i.e. cpu time and memory usage. Currently for Unix
systems only. COMPILE FLAG NEEDED.\\

\noindent {\bf Class Interface} \\
\indent {// Constructors}  \\ 
\indent {\em Timer(int classTag);}  \\ \\
\indent {// Destructor}  \\ 
\indent {\em virtual $\tilde{}$Timer();} \\ \\
\indent {// Public Methods}  \\ 
\indent {\em  void start(void);}\\
\indent {\em  void pause(void);}\\
\indent {\em  double getReal(void) const;}\\
\indent {\em  double getCPU(void) const;}\\
\indent {\em  int getNumPageFaults(void) const;}\\
\indent {\em  virtual void Print(ostream \&s) const; }\\
\indent {\em  friend ostream \&operator$<<$(ostream \&s, const Timer \&E); }\\

\noindent {\bf Constructors}  \\
\indent {\em Timer();}  \\
Does nothing. \\

\noindent {\bf Destructor} \\
\indent {\em virtual $\tilde{}$Timer();} \\ 
Does nothing. \\

\noindent {\bf Public Methods }  \\
\indent {\em  void start(void);}\\
Sets the accounting variables to mark the start of accounting period
using the unix functions {\em times()} and {\em getrusage}.\\

\indent {\em  void pause(void);}\\
Sets the accounting variables to mark the end of accounting period
using the unix functions {\em times()} and {\em getrusage}.\\


\indent {\em  double getReal(void) const;}\
Uses the difference between the starting and ending accounting
variables to determine the elapsed real time between the last calls to
{\em start()} and {\em pause()}. Returns this value in units of seconds.\\

\indent {\em  double getCPU(void) const;}\\
Uses the difference between the starting and ending accounting
variables to determine the CPU time allocated the process between the
last calls to {\em start()} and {\em pause()}. Returns this value in
units of seconds.\\

\indent {\em  int getNumPageFaults(void) const;}\\
Uses the difference between the starting and ending accounting
variables to determine the number of page faults that required reading
of pages from disk between the last calls to {\em start()} and {\em
pause()}. Returns this value. \\


\indent {\em virtual void Print(ostream \&s) const; }\\
Uses the difference between the starting and ending accounting
variables to determine the real time, CPU time, operating system time
allocate the process, total number of page faults, number of page
faults that required reading of pages from memory, and number of page
faults that required no reading from disk between the last calls to
{\em start()} and {\em pause()}. Send these values to {\em s}. \\

\indent {\em friend ostream \&operator$<<$(ostream \&s, const Timer \&E); }\\
Invokes {\em Print(s)} on the Timer object {\em E}. 


\pagebreak
\pagebreak \subsection{\bf ErrorHandler}
%File: ~/handler/ErrorHandler.tex
%What: "@(#) ErrorHandler.tex, revA"

\noindent {\bf Files}   \\
\indent \#include $<\tilde{ }$/handler/ErrorHandler.h$>$  \\

\noindent {\bf Class Declaration}  \\
\indent class ErrorHandler \\

\noindent {\bf Class Hierarchy} \\
\indent {\bf ErrorHandler} \\

\noindent {\bf Description}  \\
\indent The ErrorHandler class is an abstract class which is
introduced to allow error information to be processed during the
execution of the program. The interface defines two pure virtual
methods {\em warning()} and {\em fatal()}. {\em warning()} is a method
which is called by an object when an error has occured but the program
can continue running after the error has been logged; {\em fatal()} is
invoked when the error is so serious that the program must be
terminated. \\ 

QUESTION: SHOULD THE DOMAIN/RECORDER CLASSES HAVE A shutdown() METHOD? \\

\noindent {\bf Class Interface} \\
\indent // Constructor \\
\indent {\em ErrorHandler();}\\ \\
\indent // Destructor \\
\indent {\em virtual $\tilde{ }$ErrorHandler();}\\ \\
\indent // Public Methods  \\
\indent {\em virtual void warning(const char *msg, ...) =0;}\\
\indent {\em virtual void fatal(const char *msg, ...) =0;}\\ \\
\indent // Protected Methods  \\
\indent {\em void outputMessage(OPS_Stream \&theStream, const char *msg, va\_list args);} \\

\noindent {\bf Constructor}  \\
\indent {\em ErrorHandler();}  \\ 
Does nothing.\\

\noindent {\bf Destructor} \\
\indent {\em virtual~ $\tilde{}$ErrorHandler();}\\ 
Does nothing. \\

\noindent {\bf Public Methods }  \\
\indent {\em virtual void warning(const char *msg, ...) =0;}\\
Invoked by an object when an error has occurred. The program will
continue running after this method has been invoked. The first
argument is a pointer to a format string that specifies how
subsequent (or arguments accessed via the variable-length ellipses
argument are converted for output. As for {\em printf()}, the format
string is composed of ordinary characters (not \%) which are copied to
the log unmodified and conversion specifications, which fetches the
next argument from the ellipses. Conversion specification is
introduced by the \% character and followed by one of the following
flags {\bf d}, {\bf f} and {\bf s}, for integers, doubles and strings
respectively. \\


\indent {\em virtual void playback(const char *msg, ...) =0;}\\ 
Invoked by an object when an error has occurred. The program will be
terminated by this method. The arguments are the same as for the {\em
record()} method.\\ 

\noindent {\bf Protected Methods }  \\
\indent {\em void outputMessage(OPS_Stream \&theStream, const char *msg,
va\_list args);}\\ 
Will print to the OPS_Stream {\em theStream} the error message, where
{\em msg} is the format string  and {\em va\_list} is the list created
from the varying number of arguments passed to the {\em warning()} or
{\em fatal()} methods. It is up to the calling object to both create
{\em va\_start()} and destroy {\em va\_end()} this list. 


\pagebreak

\subsubsection{ConsoleErrorHandler}
%File: ~/handler/ConsoleErrorHandler.tex
%What: "@(#) ConsoleErrorHandler.tex, revA"

\noindent {\bf Files}   \\
\indent \#include $<\tilde{ }$/handler/ConsoleErrorHandler.h$>$  \\

\noindent {\bf Class Declaration}  \\
\indent class ConsoleErrorHandler \\

\noindent {\bf Class Hierarchy} \\
\indent ErrorHandler \\
\indent\indent {\bf ConsoleErrorHandler} \\

\noindent {\bf Description}  \\
\indent The ConsoleErrorHandler class is a concrete subclass of error
handler which sends the error messages to the opserr stream. \\

\noindent {\bf Class Interface} \\
\indent // Constructor \\
\indent {\em ConsoleErrorHandler();}\\ \\
\indent // Destructor \\
\indent {\em $\tilde{ }$ConsoleErrorHandler();}\\ \\
\indent // Public Methods  \\
\indent {\em void warning(const char *msg, ...);}\\
\indent {\em void fatal(const char *msg, ...);}\\ 

\noindent {\bf Constructor}  \\
\indent {\em ConsoleErrorHandler();}  \\ 
Does nothing.\\

\noindent {\bf Destructor} \\
\indent {\em ~ $\tilde{}$ConsoleErrorHandler();}\\ 
Does nothing. \\

\noindent {\bf Public Methods }  \\
\indent {\em void warning(const char *msg, ...) =0;}\\
Creates a va\_list using {\em va\_start()} on the ellipses arguments
and invokes the {\em outputMessage(opserr, msg, va\_list)} routine in
the parent class. It then invokes  {\em va\_end()} on this va\_list
and returns.\\ 

\indent {\em void fatal(const char *msg, ...) =0;}\\
Creates a va\_list using {\em va\_start()} on the ellipses arguments
and invokes the {\em outputMessage(opserr, msg, va\_list)} routine in
the parent class. It then invokes {\em va\_end()} on this va\_list,
and finally terminates the program with an {\em exit()}. 

\pagebreak


\pagebreak
\section{Graph Classes}
In this work Graphs are used for three purposes: \begin{enumerate} 
\item to provide information on the sparsity of the system of equation 
to the SystemofEqn object, 
\item to provide the connectivity of the DOF\_Group objects for determining
a good mapping between degrees-of-freedom and equation numbers.
\item to provide information on the connectivity both the Elements and Nodes
in the Domain, which can be useful for example in partitioning.
\end{enumerate}

The classes provided include Graph, Vertex, GraphNumberer and 
GraphPartitioner. There is no Edge class provided at present. In current 
design each Vertex stores in an ID the tag of all it's adjacent Vertices, 
this may change. For graph numbering and partitioning this has to date 
proved sufficient.

\pagebreak
\subsection{Graph}
%File: ~/OOP/graph/graph/Graph.tex
%What: "@(#) Graph.tex, revA"

\noindent {\bf Files}   \\
\indent \#include $<\tilde{ }$/graph/graph/Graph.h$>$  \\

\noindent {\bf Class Declaration}  \\
\indent class Graph:  \\

\noindent {\bf Class Hierarchy} \\
\indent {\bf Graph} \\

\noindent {\bf Description}  \\
\indent Graph is a base class. A Graph is a container class
responsible for holding the vertex set and edge set. The class is
responsible for: \begin{enumerate} \item providing methods to add
vertices and edges. \item accessing the vertices and
edges. \end{enumerate} All the methods for the class are declared as
virtual to allow subclasses to be introduced. \\



\noindent {\bf Class Interface}  \\
\indent\indent // Constructors  \\
\indent\indent {\em Graph();}  \\
\indent\indent {\em Graph(int numVertices);} \\
\indent\indent {\em Graph(TaggedObjectStorage \&theVerticesStorage);} \\ \\
\indent\indent // Destructor  \\
\indent\indent {\em virtual~$\tilde{}$Graph();}  \\ \\
\indent\indent // Public Methods  \\
\indent\indent {\em virtual bool addVertex(Vertex *vertexPtr, bool
checkAdjacency = true);} \\
\indent\indent {\em virtual int addEdge(int vertexTag, int
otherVertexTag);  } \\ 
\indent\indent {\em virtual Vertex *getVertexPtr(int vertexTag);} \\
\indent\indent {\em virtual VertexIter \&getVertices(void);} \\
\indent\indent {\em virtual int getNumVertex(void) const;} \\
\indent\indent {\em virtual int getNumEdge(void) const;} \\
\indent\indent {\em virtual Vertex *removeVertex(int vertexTag, bool
removeEdgeFlag = true );} \\
\indent\indent {\em virtual void Print(OPS_Stream \&s, int flag = 0);} \\
\indent\indent {\em friend OPS_Stream \&operator<<(OPS_Stream \&s, const Graph 
\&G);} \\

\noindent {\bf Constructors}  \\
\indent {\em Graph();}  \\
To create an empty Graph. Creates an ArrayOfTagged object of initial
size $32$ in which to store the Vertices. The ArrayOfTagged object is
used to store the Vertices.\\

\indent {\em Graph(int numVertices);} \\
To create an empty Graph. Creates an ArrayOfTagged object of initial
size {\em numVertices} in which to store the Vertices. The
ArrayOfTagged object is used to store the Vertices. \\ 

\indent {\em Graph(TaggedObjectStorage \&theVerticesStorage);}\\ 
To create an empty Graph. The {\em theVerticesStorage} object is used to
store the Vertices. {\em clearAll()} is invoked on this object to
ensure it is empty. \\

\noindent {\bf Destructor}  \\
\indent {\em virtual~$\tilde{}$Graph();}  \\
Invokes {\em clearAll()} on the storage object used to store the
Vertices. It then invokes delete on the storage object used, which was
either passed or created in the constructor. \\

\noindent {\bf Public Methods }  \\
\indent {\em virtual bool addVertex(Vertex *vertexPtr, bool
checkAdjacency = true);} \\
Causes the graph to add a vertex to the graph. If {\em checkAdjacency}
is {\em true}, a check is made to ensure that all the Vertices in the
adjacency list of the Vertex are in the Graph. If a vertex in the
adjacency is not in the Graph the vertex is not added, a warning
message is printed and {\em false} is returned. If successful,
returns the result of invoking {\em addComponent()} on the
TaggedStorage object used to store the Vertices. \\



{\em virtual int addEdge(int vertexTag, int otherVertexTag); } \\
Causes the Graph to add an edge {\em (vertexTag,otherVertexTag)} to
the Graph. A check is first made to see if vertices with tags given by
{\em vertexTag} and {\em otherVertexTag} exist in the graph. If they
do not exist a $-1$ is returned, otherwise the method invokes {\em
addEdge()} on each of the corresponding vertices in the 
graph. Increments {\em numEdge} by $1$ and returns $0$ if sucessfull,
a $1$ if the edge already existed, and a $-2$ if one {\em addEdge()}
was successful, but the other was not.\\  

{\em virtual Vertex *getVertexPtr(int vertexTag);} \\
A method which returns a pointer to the vertex whose tag is given by {\em
vertexTag}. If no such vertex exists in the graph $0$ is
returned. Invokes {\em getComponentPtr(vertexTag)} on the vertex
storage object and casts this to a Vertex * if not null. \\

{\em virtual VertexIter \&getVertices(void);} \\
A method which returns a reference to the graphs VertexIter. This iter
can be used for iterating through the vertices of the graph. \\

{\em virtual int getNumVertex(void) const;} \\
A method to return the number of vertices in the graph. Invokes {\em
getNumComponents()} on the Vertex storage object.\\

{\em virtual int getNumEdge(void) const;} \\
A method to return the number of edges in the graph, returns {\em
numEdge}. \\ 

\indent {\em virtual Vertex *removeVertex(int vertexTag, bool
removeEdgeFlag = true );} \\
To remove the Vertex from the Graph whose tag is equal to {\em
vertexTag}. If {\em removeEdgeFlag} is {\em true} will also remove the
Vertex from the remaining Vertices adjacency lists. returns a pointer
to the removed Vertex if successful, $0$ if the Vertex was not in the
Graph. Invokes {\em removeComponent(vertexTag)} on the vertex
storage object and casts this to a Vertex * if not null. DOES NOT YET
DEAL WITH {\em removeEdgeFlag}. \\

{\em virtual void Print(OPS_Stream \&s, int flag =0);} \\
A method to print the graph. Invokes {\em Print(s, flag)} on the vertex
storage object. \\

\indent {\em friend OPS_Stream \&operator<<(OPS_Stream \&s, const Graph 
\&G);} \\
Invokes {\em Print()} on the Graph {\em G}.

\pagebreak \subsection{Vertex}
%File: ~/OOP/graph/graph/Vertex.tex
%What: "@(#) Vertex.tex, revA"

\noindent {\bf Files}   \\
\indent \#include $<\tilde{ }$/graph/graph/Vertex.h$>$  \\

\noindent {\bf Class Declaration}  \\
\indent class Vertex: public TaggedObject  \\

\noindent {\bf Class Hierarchy} \\
\indent TaggedObject \\
\indent\indent {\bf Vertex} \\

\noindent {\bf Description}  \\
\indent Vertex is the abstraction of a vertex in a graph. It has a
color, weight and a temporary integer value associated with it. Also
associated with it is an integer reference, which can be used to identify
an object of some type the vertex is representing and in integer
temporary variable for algorithms which work with graphs. \\

\noindent {\bf Class Interface}  \\
\indent // Constructor  \\
\indent {\em Vertex(int tag, int ref, double weight=0, int color
=0);}  \\\\ 
\indent // Destructor  \\
\indent {\em virtual~$\tilde{}$Vertex();}  \\\\
\indent // Public Methods  \\
\indent {\em virtual void setWeight(double newWeight);} \\
\indent {\em virtual void setColor(int newColor);} \\
\indent {\em virtual void setTmp(int newTmp);} \\
\indent {\em virtual int getTag(void) const;}. \\
\indent {\em virtual int getRef(void) const; } \\
\indent {\em virtual double getWeight(void) const;} \\
\indent {\em virtual int getColor(void) const; } \\
\indent {\em virtual int getTmp(void) const; } \\
\indent {\em virtual int addEdge(int otherTag); } \\
\indent {\em virtual int getDegree(void) const;} \\
\indent {\em virtual const ID \&getAdjacency(void) const;} \\
\indent {\em virtual void printVertex(ostream \&s, int flag=0);} \\ 
\indent {\em virtual void Print(ostream \&s, int flag =0);} \\



\noindent {\bf Constructor}  \\
\indent {\em Vertex(int tag, int ref, double weight=0, int color =0);}  \\
To construct a Vertex whose tag, reference, weight and color are as
given by the arguments. The degree of the vertex is set to $0$. The
integer {\em tag} is passed to the TaggedObject classes constructor.\\

\noindent {\bf Destructor}  \\
\indent {\em virtual~$\tilde{}$Vertex();}  \\
Does nothing. \\

\noindent {\bf Public Methods }  \\
\indent {\em virtual void setWeight(double newWeight);} \\
To set the weight of the vertex to {\em newWeight} \\    

{\em virtual void setColor(int newColor);} \\
To set the color of the vertex to {\em newColor} \\    

{\em virtual void setTmp(int newTmp);} \\
To set the temporary variable of the vertex to {\em newTmp} \\    

{\em virtual int getTag(void) const;}. \\
Returns the vertices tag.\\

{\em virtual int getRef(void) const; } \\
Returns the vertices integer reference.\\

{\em virtual double getWeight(void) const;} \\
Returns the vertices weight. \\

{\em virtual int getColor(void) const; } \\
Returns the vertices color. \\

{\em virtual int getTmp(void) const; } \\
Returns the vertices temporary variable. \\

{\em virtual int addEdge(int otherTag); } \\
If the adjacency list for that vertex does not already contain {\em
otherTag}, {\em otherTag} is added to the adjacency list and the
degree of the vertex is incremented by $1$. Returns a $0$ if
sucessfull, a $1$ if edge already existed and a negative number if
not. Note that no check is done by the vertex to see that a vertex
with {\em otherTag} exists in the graph. The adjacency list for a
Vertex is stored in an ID object containing the adjacent Vertices
tags. A check is made to see if {\em otherTag} is in this ID using
{\em getLocation()}, if it needs to be added the {\em [degree]}
operator is invoked on the ID. \\ 

{\em virtual int getDegree(void) const;} \\
Returns the vertices degree. \\

{\em virtual const ID \&getAdjacency(void) const;} \\
Returns the vertices adjacency list, this is returned as an ID whose
components are tags for vertices which have been successfully added.\\
    
{\em virtual void Print(ostream \&s, int flag=0);} \\
Prints the vertex. If the {\em flag = 0} only the vertex tag and
adjacency list is printed. If the {\em flag =1} the vertex tag, weight
and adjacency are printed. If the {\em flag =2} the vertex tag, color
and adjacency are printed. If the {\em flag =3} the vertex tag,
weight, color and adjacency are printed. \\  








\pagebreak \subsection{{\bf GraphNumberer}}
%File: ~/OOP/graph/numberer/GraphNumberer.tex

%What: "@(#) GraphNumberer.tex, revA"

\noindent {\bf Files}   \\
\indent \#include $<\tilde{ }$/graph/numberer/GraphNumberer.h$>$  \\

\noindent {\bf Class Declaration}  \\
\indent class GraphNumberer: public MovableObject; \\

\noindent {\bf Class Hierarchy} \\
\indent MovableObject \\
\indent\indent {\bf GraphNumberer} \\
\indent\indent\indent RCM \\
\indent\indent\indent MinDegree \\

\noindent {\bf Description}  \\
\indent GraphNumberer is an abstract class. The GraphNumberer
class defines the interface that all programmers must provide when
introducing new GraphNumberer subclasses. A GraphNumberer is an
algorithmic class for numbering the Vertices of a Graph; that is
assigning a unique integer value ($0$ through {\em numVertex} $-1$) to
each Vertex (uses {\em tmp} variable of Vertex) of the Graph. \\


\noindent {\bf Class Interface }  \\
\indent // Constructor  \\
\indent {\em GraphNumberer(int classTag);}  \\ \\
\indent // Destructor  \\
\indent {\em virtual~$\tilde{}$GraphNumberer();}  \\ \\
\indent // Public Methods   \\
\indent {\em virtual const ID \&number(Graph \&theGraph, int
lastVertexTag = -1) =0;}\\
\indent {\em virtual const ID \&number(Graph \&theGraph, const ID
\&lastVertices) =0;}\\


\noindent {\bf Constructor}  \\
\indent {\em GraphNumberer(int classTag);}  \\
The integer {\em classTag} is passed to the MovableObject classes
constructor.\\ 

\noindent {\bf Destructor}  \\
\indent {\em virtual~$\tilde{}$GraphNumberer();}  \\
Does nothing. \\

\noindent {\bf Public Methods }  \\
\indent {\em virtual const ID \&number(Graph \&theGraph, int
lastVertex = -1) =0;}\\
This is the method invoked to perform the graph numbering, that is to
assign a unique integer $1$ through {\em numVertex}, to each Vertex in
the Graph. Returns an ordered ID containing the vertex references in the order
of the numbering, i.e. $ID(0)$ contains the integer reference for the
vertex assigned the number 1, $ID(1)$ contains the integer reference for the
vertex assigned the number 2 and so on. A side effect of the numbering
is that the {\em Tmp} variable of each vertex is set to the number
assigned to that vertex. If {\em lastVertex} is not $-1$ the Vertex
whose tag is given by {\em lastVertex} should be numbered last (it
does not have to be though THIS MAY CHANGE).\\

\indent {\em virtual const ID \&number(Graph \&theGraph, const ID
\&lastVertices) =0;}\\
This is the method invoked to perform the graph numbering, that is to
assign a unique integer $1$ through {\em numVertex}, to each Vertex in
the Graph. Returns an ordered ID containing the vertex references in the order
of the numbering, i.e. $ID(0)$ contains the integer reference for the
vertex assigned the number 1, $ID(1)$ contains the integer reference for the
vertex assigned the number 2 and so on. A side effect of the numbering
is that the {\em Tmp} variable of each vertex is set to the number
assigned to that vertex. {\em lastVertices} is used as a hint to
indicate that these Vertices should be numbered last (they do not have
to be though THIS MAY CHANGE).








\pagebreak \subsubsection{RCM}
%File: ~/OOP/graph/numberer/RCM.tex
%What: "@(#) RCM.tex, revA"


\noindent {\bf Files}   \\
\indent \#include $<\tilde{ }$/graph/numberer/RCM.h$>$  \\

\noindent {\bf Class Declaration}  \\
\indent class RCM: public GraphNumberer; \\

\noindent {\bf Class Hierarchy} \\
\indent MovableObject \\
\indent\indent GraphNumberer \\
\indent\indent\indent {\bf RCM} \\

\noindent {\bf Description}  \\
\indent RCM is a subclass of GraphNumberer which performs the
numbering using the reverse Cuthill-McKee numbering algorithm. \\

\noindent {\bf Class Interface }  \\
\indent\indent // Constructor  \\
\indent\indent {\em RCM(bool GPS = true);}  \\ \\
\indent\indent // Destructor  \\
\indent\indent {\em ~$\tilde{}$RCM();}  \\ \\
\indent\indent // Public Methods   \\
\indent\indent {\em const ID \&number(Graph \&theGraph, int
lastVertexTag = -1) =0;}\\
\indent\indent {\em const ID \&number(Graph \&theGraph, const ID
\&startVertices) =0;}\\
\indent\indent {\em int sendSelf(int commitTag, Channel \&theChannel,
FEM\_ObjectBroker \&theBroker);} \\
\indent\indent {\em int recvSelf(int commitTag, Channel \&theChannel,
FEM\_ObjectBroker \&theBroker); } \\

\noindent {\bf Constructor}  \\
\indent {\em RCM(bool GPS = true);}  \\
The integer {\em classTag} is passed to the MovableObject classes
constructor. The flag {\em GPS} is used to mark whether the
Gibbs-Poole-Stodlmyer algorithm is used to determine a starting vertex
when no starting vertex is given. \\

\noindent {\bf Destructor}  \\
\indent {\em virtual~$\tilde{}$RCM();}  \\
Invokes the destructor on any ID object created when {\em number()} is
invoked. \\

\noindent {\bf Public Methods}  \\
\indent {\em const ID \&number(Graph \&theGraph, int
lastVertex = -1) =0;}\\
If the present ID used for the result is not of size equal to the
number of Vertices in {\em theGraph}, it deletes the old and
constructs a new ID. Starts by iterating through the Vertices of the
graph setting the {\em tmp} variable of each to $-1$. The Vertices are
then numbered using a depth first sort of the Graph, with each
unmarked Vertex in the Graph at a distance $d$ from starting Vertex
being placed in the d'th level set. As this is RCM, the Vertices in
level set $n$ are assigned a higher number than those in level set
$n+1$ with the {\em tmp} variable of the starting Vertex being
assigned {\em numVertices} $-1$. The {\em tags} of the Vertices are
placed into the ID at location given by their {\em tmp} variable. These
are replaced with the {\em ref} variable of each Vertex, which is
returned on successful completion. 


The Vertex chosen as the starting Vertex is the one whose tag is given
by {\em lastVertex}. If this is $-1$ or the Vertex corresponding to
{\em lastVertex} does not exist then another Vertex is chosen. If the
{\em GPS} flag in constructor is {\em false} the first Vertex from the
Graphs VertexIter is used; if {\em true} a RCM numbering using the
first Vertex from the VertexIter is performed and the Vertices in the
last level set are then used to create an ID {\em lastVertices} with
which {\em number(theGraph, lastVertices)} can be invoked to determine
the numbering. \\


\indent {\em const ID \&number(Graph \&theGraph, const ID
\&startVertices) =0;}\\
This method is invoked to determine the best starting Vertex for a RCM
using a Vertex whose tag is in {\em lastVertices}. To do a RCM
numbering is performed using each of the Vertices in {\em
startVertices} as the Vertex in level set $0$. The Vertex which
results in the numbering with the smallest profile is chosen as 
the starting Vertex. The RCM algorithm outlined above is then called
with this starting Vertex. \\

{\em int sendSelf(Channel \&theChannel,
FEM\_ObjectBroker \&theBroker);} \\
Returns $0$. \\

{\em int recvSelf(Channel \&theChannel,
FEM\_ObjectBroker \&theBroker); } \\
Returns $0$.



\pagebreak \subsection{{\bf GraphPartitioner}}
%File: ~/OOP/graph/graph/GraphPartitioner.tex
%What: "@(#) GraphPartitioner.tex, revA"

\noindent {\bf Files}   \\
\indent \#include $<\tilde{ }$/graph/partitioner/GraphPartitioner.h$>$  \\

\noindent {\bf Class Declaration}  \\
\indent class GraphPartitioner:  \\

\noindent {\bf Class Hierarchy} \\
\indent {\bf GraphPartitioner} \\
\indent\indent Metis \\

\noindent {\bf Description}  \\
\indent GraphPartitioner is an abstract class. The GraphPartitioner
class defines the interface that all programmers must provide when
introducing new GraphPartitioner subclasses. A GraphPartitioner is an
algorithm for partitioning (coloring) the vertices of a graph; that is
assigning a color (1 through the number of partitions) to each vertex
of the graph. \\

\noindent {\bf Class Interface}  \\
\indent\indent // Constructor \\
\indent\indent {\em GraphPartitioner();}  \\ \\
\indent\indent // Destructor  \\
\indent\indent {\em virtual~$\tilde{}$GraphPartitioner();}  \\ \\
\indent\indent // Public Methods  \\
\indent\indent {\em virtual int partition(Graph \&theGraph, int numPart) =0;} \\

\noindent {\bf Constructor}  \\
\indent {\em GraphPartitioner();}  \\
To construct a GraphPartitioner. \\

\noindent {\bf Destructor}  \\
\indent {\em virtual~$\tilde{}$GraphPartitioner();}  \\

\noindent {\bf Public Methods }  \\
\indent {\em virtual int partition(Graph \&theGraph, int numPart) =0;} \\
This is the method invoked to partition the graph into {\em numPart}
partitions. On completion of the routine each vertex will be assigned
a color $1$ through {\em numPart}, the color assigned indicating the
partition to which the vertex belongs. Returns a $0$ if successful, a
negative number if not; the value depending on the subclass.  \\





\pagebreak \subsubsection{Metis}
%File: ~/OOP/graph/graph/Metis.tex
%What: "@(#) Metis.tex, revA"

\noindent {\bf Files}   \\
\indent \#include $<\tilde{ }$/graph/partitioner/Metis.h$>$  \\

\noindent {\bf Class Declaration}  \\
\indent class GraphPartitioner:  \\

\noindent {\bf Class Hierarchy} \\
\indent GraphPartitioner \\
\indent\indent {\bf Metis} \\

\noindent {\bf Description}  \\
\indent Metis is a GraphPartitioner. The Metis graph partitioner calls
procedures defined in the METIS library to partition the graph. METIS
is currently being developed by G.~Karypis and V.~Kumar at the
University of Minnesota. At the present time the Graph to be
partitioned MUST have the vertices labeled $0$ through $numVertex-1$. \\

The METIS library uses two integer arrays to represent the graph, {\em
xadj} and {\em adjncy}. $xadj(i)$ stores the location in {\em adjncy}
of the start of the $i$'th Vertices adjacent Vertices. {\em adjncy}
contains the tags of all the adjacent vertices. For example, the graph
which is represented by the following matrix $A$:


$$ A =
\left[
\begin{array}{ccccc}
1 & 0 & 1 & 1 & 0  \\
1 & 1 & 0 & 0 & 0  \\
0 & 1 & 1 & 0 & 0 \\
0 & 0 & 0 & 1 & 1 \\
1 & 1 & 0 & 0 & 1
\end{array}
\right] 
$$

is represented by:

$$
xadj =
\left[
\begin{array}{cccccccccccccc}
0 & 2 & 3 & 4 & 5 & 7
\end{array}
\right] 
$$

and

$$
adjncy =
\left[
\begin{array}{cccccccccccccc}
2 & 3 & 0 & 1 & 4 & 0 & 1
\end{array}
\right] 
$$

\noindent note that there is no space allocated for the diagonal
components. \\

\noindent {\bf Class Interface}  \\
\indent\indent // Constructors  \\
\indent\indent {\em Metis();}  \\ \\
\indent\indent {\em Metis(int pType, int mType, int coarsenTo, int 
rType, int ipType);}  \\ \\
\indent\indent // Destructor  \\
\indent\indent {\em virtual~$\tilde{}$Metis();}  \\ \\
\indent\indent // Public Methods  \\
\indent\indent {\em virtual int partition(Graph \&theGraph, int numPart) =0;} \\
\indent\indent {\em bool setDefaultOptions(void);}\\
\indent\indent {\em bool setOptions(int pType, int mType, int coarsenTo, int 
rTypem, int ipType);}  \\ \\
\indent\indent // Private Method  \\
\indent\indent {\em bool checkOptions(void) const;} \\


\noindent {\bf Constructors}  \\
\indent {\em Metis();}  \\
To construct a Metis object which will use the default settings when
partitioning. \\ 

\indent {\em Metis(int pType, int mType, int coarsenTo, int 
rTypem, int ipType);}  \\
To construct a Metis object which will use the setting passed into the
constructor as options to metis's {\em PMETIS()} routine. {\em
checkOptions()} is invoked to ensure the settings are valid. \\

\noindent {\bf Destructor}  \\
\indent {\em virtual~$\tilde{}$Metis();}  \\

\noindent {\bf Public Methods }  \\
\indent {\em virtual int partition(Graph \&theGraph, int numPart) =0;} \\
This is the method invoked to partition the graph into {\em numPart}
partitions. On completion of the routine each vertex will be assigned
a color $1$ through {\em numPart}, the color assigned indicating the
partition to which the vertex belongs. 

To partition a number of integer arrays are created, {\em options[5]},
{\em partition[numVertex+1]}, {\em xadj[numVertex+1]} and {\em
adjncy[2*numEdge]} (CURRENTLY ASSUMING GRAPH IS SYMMETRIC - THIS MAY
CHANGE \& xadj and partition 1 LARGER THAN REQUIRED). If not enough
memory is available for the arrays, a warning message is printed and
$-2$ is returned. The data for {\em xadj} and {\em adjncy} are
determined from the Vertices of the Graph by iterating over each
Vertex from $0$ through {\em numVertex} $-1$. If default options are
specified {\em options[0]} is set to $0$, otherwise $1$ with {\em
options[1:4] = coarsenTo, mType, ipType, rType}. if {\em pType} equals
$1$ {\em PMETIS} is called, otherwise {\em KMETIS} is called. Both are
called with the following arguments: {\em numVertex, xadj,adjncy, 0,
0, \&weightFlag, options, numPart, \&numbering, \&edgecut, partition} 
The colors of the partitions are then set equal to the color indicated
in {\em partition}.  The integer arrays are destroyed and $0$
returned. \\

\indent {\em bool setDefaultOptions(void);}\\
Sets the default options. \\

\indent {\em bool setOptions(int pType, int mType, int coarsenTo, int 
rType, int ipType);}  \\ 
Sets the options for the partitioning to those passed as
arguments. Then invokes {\em checkOptions()} to see if the options are
valid. HOW ABOUT REFERRINGR TO MANUAL TO SEE WHAT OPTIONS MEAN. \\

\noindent {\bf Private Method}  \\
\indent {\em bool checkOptions(void) const;}
If options are not valid sets the default options. EXPAND ON VALID
OPTIONS OR REFER TO METIS MANUAL. \\



\pagebreak
\section{Parallel Classes}
To facilitate the development of parallel object-oriented finite
element programs, a new framework is presented in this section. The
classes in the framework support the aggregate programming model. The
new classes are:

\begin{itemize} 
\item {\bf Shadow} - A Shadow object represents a remote actor object
in the local actor process.

\item {\bf Actor} - An { Actor} object is a local object in the remote
actor process. It performs the operations requested of it by the
{ Shadow} object. The actor objects in an aggregation collectively
perform the analysis operations by communicating between themselves.

\item {\bf Channel} - The { Shadow} and { Actor} objects
communicate with 
each other through { Channel} objects. A { Channel} object represents a
point in a local actor process through which a local object can send
and receive information.

\item {\bf Address} - An { Address} object represents the location of a
{ Channel} object in the machine space. { Channel} objects send
information to other { Channel} objects, whose locations are given by
an { Address} object. { Channel} objects also receive
information from other { Channel} objects, whose locations are
given by an { Address} object.  

\item {\bf MovableObject} - A { MovableObject} is an object which can
send its state from one actor process to another.

\item {\bf ObjectBroker} - An { ObjectBroker} is an object in a local
actor process for creating new objects.

\item {\bf MachineBroker} - A { MachineBroker} is an object in a local
actor process that is responsible for creating remote actor processes
at the request of { Shadow} objects in the same local process.
\end{itemize}

\pagebreak
\subsection{{\bf Actor}}
%File: ~/OOP/actor/Actor.tex
%What: "@(#) Actor.tex, revA"

\noindent {\bf Files}   \\
\indent \#include $<\tilde{ }$/actor/actor/Actor.h$>$  \\

\noindent {\bf Class Declaration}  \\
\indent class Actor \\

\noindent {\bf Class Hierarchy} \\
\indent {\bf Actor} \\

\noindent {\bf Description}  \\
\indent Actor is meant as an abstract class, i.e. no instances of Actor
should exist. An actor is associated with a shadow object. The shadow
acts like a normal object in the users address space, data and
processing that is done by the shadow may be stored and processed in a
remote process, the actor resides in this remote address space. The
actor and the shadow both have a channel, a communication port. This
allows the two to communicate with each other.\\

\noindent {\bf Class Interface }  \\
\indent // Constructor  \\
\indent {\em Actor(Channel \&theChannel, FEM\_ObjectBroker
\&theBroker, int numActorMethods);}  \\  \\
\indent // Destructor \\
\indent {\em virtual $\tilde{ }$Actor();}\\ \\
\indent // Public Methods for Processing Functions \\
\indent {\em virtual int addMethod(int tag, int (*fp)());}\\
\indent {\em virtual int  getMethod();} \\
\indent {\em virtual int  processMethod(int tag);}\\ \\
\indent // Public Methods for Sending/Receiving Objects\\
\indent {\em virtual int sendObject(MovableObject \&theObject, ChannelAddress
*theAddress =0);}\\
\indent {\em virtual int recvObject(MovableObject \&theObject, ChannelAddress
*theAddress =0);}\\
\indent {\em virtual int sendMessage(Message \&theMessage, ChannelAddress
*theAddress =0);}\\
\indent {\em virtual int recvMessage(Message \&theMessage);} \\
\indent {\em virtual int sendMatrix(Matrix \&theMatrix, ChannelAddress
*theAddress =0);}\\
\indent {\em virtual int recvMatrix(Matrix \&theMatrix);} \\
\indent {\em virtual int sendVector(Vector \&theVector, ChannelAddress
*theAddress =0);}\\
\indent {\em virtual int recvVector(Vector \&theVector);} \\
\indent {\em virtual int sendID(ID \&theID, ChannelAddress *theAddress =0);}\\
\indent {\em virtual int recvID(ID \&theID);} \\
\indent {\em void Channel *getChannelPtr(void) const;}\\
\indent {\em void FEM\_ObjectBroker *getObjectBrokerPtr(void) const;}\\
\indent {\em void ChannelAddress *getChannelAddressPtr(void) const;}\\


\noindent {\bf Constructor}  \\
\indent {\em Actor(Channel \&theChannel, FEM\_ObjectBroker
\&theBroker, int numActorMethods);}  \\ 
This is called by the remote process upon initialization to construct
the local actor object. It is used to create an Actor object in that
remote address space which will communicate with objects in other
processes through a channel object, {\em theChannel} and which uses
{\em theBroker} to receive movable objects sent from other
processes. The subclass will be able to add {\em numMethods} actor
methods using {\em addMethod()} call. \\   

The base classes constructor invokes {\em setUpActor()} on {\em
theChannel} object. It then sets the Address of the remote shadow
object which created the actor process by invoking {\em
getLastSendersAddress()} on {\em theChannel}. \\


\noindent {\bf Destructor} \\
\indent {\em virtual $\tilde{ }$Actor();}\\ 
Provided so subclass destructor will be called. \\

\noindent {\bf Public Methods }  \\
\indent {\em virtual int addMethod(int tag, int (*fp)());}\\
A method to add as a function to the actor object the function {\em
fp}, this function is identified by the {\em tag} value. This
function will be invoked by the actor on invocation of {\em 
processMethod()} with {\em tag} as the argument. The object checks to
see that the {\em tag} has not been used previously. If it has not and
the number of functions so far added is less than {\em numActorMethods}
the function is added and $0$ is returned, otherwise a $-1$ (if {\em
tag} was already used) or $-2$ (if {\em numActorMethods} already
added) or a $-3$ (if running out of space) is returned to indicate the
function was not added. \\  

{\em virtual int  getMethod();} \\
A method which returns the next integer value sitting in the actors
channel. This int value corresponds to the {\em tag} of the next
method that the shadow object wants the actor to perform. If an error
occurs $-1$ will be returned. \\

{\em virtual int  processMethod(int tag);}\\
This causes the actor object to invoke the function that was added to
the actor with the {\em tag} identifier. If no method with {\em tag}
exists a $-1$ is returned. \\


{\em virtual int sendObject(MovableObject \&theObject, ChannelAddress
*theAddress =0);}\\
A method which will send {\em theObject}
through the actors channel either to the address given by {\em
theAddress} or to the address of the shadow object that created the
actor if no address is specified.

Returns the result of invoking {\em sendObj(0, theObject,theBroker,theAddress)} on
the actors channel object if an address is specified, otherwise
the result of invoking {\em
sendObj(theMessage,theBroker,this->getShadowAdressPtr())} on the
actors channel object is returned. \\ 

{\em virtual int recvObject(MovableObject \&theObject, ChannelAddress
*theAddress =0);}\\
A method which will receive {\em theObject}
from the actors channel either from the address given by {\em
theAddress} or from the address of the shadow object that created the
actor if no address is specified.

Returns the result of invoking {\em recvObj(0, theObject,theBroker,theAddress)} on
the actors channel object if an address is specified, otherwise
the result of invoking {\em
recvObj(theMessage,theBroker,this->getShadowAdressPtr())} on the actors channel
object is returned. \\ 



{\em virtual int sendMessage(Message \&theMessage, ChannelAddress
*theAddress =0);}\\
A method which will send the data in the message {\em theMessage} 
through the actors channel either to the address given by {\em
theAddress} or to the address of the shadow object that created the
actor if no address is specified.

Returns the result of invoking {\em sendMsg(0,0,theMessage,theAddress)} on
the actors channel object if an address is specified, otherwise
the result of invoking {\em
sendMsg(theMessage,this->getShadowAdressPtr())} on the actors channel
object is returned. \\ 


{\em virtual int recvMessage(Message \&theMessage);} \\
A method which will receive the data in the message {\em theMessage} from
the actors channel either from the address given by {\em
theAddress} or from the address of the shadow object that created the
actor if no address is specified.

Returns the result of invoking {\em recvMsg(0,0,theMessage,theAddress)} on
the actors channel object if an address is specified, otherwise
the result of invoking {\em
recvMsg(0,0,theMessage,this->getShadowAdressPtr())} on the actors channel
object is returned. \\ 

{\em virtual int sendMatrix(Matrix \&theMatrix, ChannelAddress
*theAddress =0);}\\
A method which will send {\em theMatrix} 
through the actors channel either to the address given by {\em
theAddress} or to the address of the shadow object that created the
actor if no address is specified.

Returns the result of invoking {\em sendMatrix(0,0,theMatrix,theAddress)} on
the actors channel object if an address is specified, otherwise
the result of invoking {\em
sendMatrix(0,0,theMatrix,this->getShadowAdressPtr())} on the actors channel
object is returned. \\ 


{\em virtual int recvMatrix(Matrix \&theMatrix);} \\
A method which will receive {\em theMatrix} from
the actors channel either from the address given by {\em
theAddress} or from the address of the shadow object that created the
actor if no address is specified.

Returns the result of invoking {\em recvMatrix(0,0,theMatrix,theAddress)} on
the actors channel object if an address is specified, otherwise
the result of invoking {\em
recvMatrix(0,0,theMatrix,this->getShadowAdressPtr())} on the actors channel
object is returned. \\ 

{\em virtual int sendVector(Vector \&theVector, ChannelAddress
*theAddress =0);}\\
A method which will send {\em theVector} 
through the actors channel either to the address given by {\em
theAddress} or to the address of the shadow object that created the
actor if no address is specified.

Returns the result of invoking {\em sendVector(0,0,theVector,theAddress)} on
the actors channel object if an address is specified, otherwise
the result of invoking {\em
sendVector(0,0,theVector,this->getShadowAdressPtr())} on the actors channel
object is returned. \\ 


{\em virtual int recvVector(Vector \&theVector);} \\
A method which will receive {\em theVector} from
the actors channel either from the address given by {\em
theAddress} or from the address of the shadow object that created the
actor if no address is specified.

Returns the result of invoking {\em recvVector(0,0,theVector,theAddress)} on
the actors channel object if an address is specified, otherwise
the result of invoking {\em
recvVector(0,0,theVector,this->getShadowAdressPtr())} on the actors channel
object is returned. \\ 


{\em virtual int sendID(ID \&theID, ChannelAddress
*theAddress =0);}\\
A method which will send {\em theID} 
through the actors channel either to the address given by {\em
theAddress} or to the address of the shadow object that created the
actor if no address is specified.

Returns the result of invoking {\em sendID(0,0,theID,theAddress)} on
the actors channel object if an address is specified, otherwise
the result of invoking {\em
sendID(0,0,theID,this->getShadowAdressPtr())} on the actors channel
object is returned. \\ 


{\em virtual int recvID(ID \&theID);} \\
A method which will receive {\em theID} from
the actors channel either from the address given by {\em
theAddress} or from the address of the shadow object that created the
actor if no address is specified.

Returns the result of invoking {\em recvID(0,0,theID,theAddress)} on
the actors channel object if an address is specified, otherwise
the result of invoking {\em
recvID(0,0,theID,this->getShadowAdressPtr())} on the actors channel
object is returned. \\ 

{\em void Channel *getChannelPtr(void) const;}\\
A method which returns a pointer to the channel passed in the
constructor. \\

{\em void FEM\_ObjectBroker *getObjectBrokerPtr(void) const;}\\
A method which returns a pointer to the FEM\_ObjectBroker passed in the
constructor. \\

{\em void ChannelAddress *getChannelAddressPtr(void) const;}\\
A method which returns a pointer to the channel address for the shadow
object that created the actor. \\



\pagebreak
\subsection{{\bf Shadow}}
%File: ~/OOP/actor/Shadow.tex
%What: "@(#) Shadow.tex, revA"

\noindent {\bf Files}   \\
\indent \#include $<\tilde{ }$/actor/shadow/Shadow.h$>$  \\

\noindent {\bf Class Declaration}  \\
\indent class Shadow \\

\noindent {\bf Class Hierarchy} \\
\indent {\bf Shadow} \\

\noindent {\bf Description}  \\
\indent Shadow is meant as an abstract class, i.e. no instances of Shadow
should exist. A Shadow object is associated with an Actor object which
may exist in another process. The Shadow acts like a normal object in
the users address space, data and processing that is done by the
Shadow may be stored and processed in a remote process, the Actor object
resides in this remote address space. The Actor and the Shadow both
have a Channel, a communication port. This allows the two to
communicate with each other.\\ 

\noindent {\bf Class Interface}  \\
\indent // Constructors  \\
\indent\indent {\em Shadow(Channel \&theChannel, \\
\indent\indent\indent\indent\indent FEM\_ObjectBroker \&theBroker, \\
\indent\indent\indent\indent\indent ChannelAddress \&theActorsAddress);}\\
\indent\indent {\em Shadow(char *program, \\
\indent\indent\indent\indent\indent Channel \&theChannel, \\
\indent\indent\indent\indent\indent FEM\_ObjectBroker \&theBroker, \\
\indent\indent\indent\indent\indent MachineBroker \&theMachineBroker, \\
\indent\indent\indent\indent\indent int compDemand, \\
\indent\indent\indent\indent\indent bool startShadow);}  \\\\
\indent\indent // Destructor \\
\indent\indent {\em virtual $\tilde{ }$Shadow();}\\  \\
\indent\indent // Protected Methods for Sending Objects to Remote Actor \\
\indent\indent {\em virtual void sendObject(MovableObject \&theObject);}\\
\indent\indent {\em virtual void recvObject(MovableObject \&theObject);} \\
\indent\indent {\em virtual void sendMessage(Message \&theMessage);}\\
\indent\indent {\em virtual void recvMessage(Message \&theMessage);} \\
\indent\indent {\em virtual void sendMatrix(Matrix \&theMatrix);}\\
\indent\indent {\em virtual void recvMatrix(Matrix \&theMatrix);} \\
\indent\indent {\em virtual void sendVector(Vector \&theVector);}\\
\indent\indent {\em virtual void recvVector(Vector \&theVector);} \\
\indent\indent {\em virtual void sendID(ID \&theID);}\\
\indent\indent {\em virtual void recvID(ID \&theID);} \\
\indent\indent {\em void Channel *getChannelPtr(void) const;}\\
\indent\indent {\em void FEM\_ObjectBroker *getObjectBrokerPtr(void) const;}\\
\indent\indent {\em void ObjectBroker *getActorAddressPtr(void) const;}\\


\noindent {\bf Constructors}  \\
\indent {\em Shadow(Channel \&theChannel, \\
\indent\indent\indent\indent FEM\_ObjectBroker \&theBroker, \\
\indent\indent\indent\indent ChannelAddress \&theActorsAddress);}\\
This constructor is used when the actor process is already up and
running. The constructor sets its channel to be {\em theChannel}, its 
associated object broker to be {\em theBroker}, and the address to
which it will send data to be {\em thaActorsAddress}. \\ 

\indent {\em Shadow(char *program, \\
\indent\indent\indent\indent Channel \&theChannel, \\
\indent\indent\indent\indent FEM\_ObjectBroker \&theBroker, \\
\indent\indent\indent\indent MachineBroker \&theMachineBroker, \\
\indent\indent\indent\indent int compDemand, \\
\indent\indent\indent\indent bool startShadow);}  \\
This constructor is used to get an actor process up and running. If 
{\em startShadow} is {\em true}, the constructor invokes 
{\em startActor(program,theChannel,compDemand)} on  
{\em theMachineBroker} object. It then invokes {\em setUpShadow()} on
{\em theChannel} and {\em getLastSendersAddress()} on {\em theChannel}
to set up store the address of the actors channel, {\em
theActorsAddress}; this is done in case multiple shadow objects use a
single channel. 
If an error occurs an error message is printed and the program terminated.\\ 

\noindent {\bf Destructor} \\
\indent {\em virtual~ $\tilde{}$Shadow();}\\ 
Does nothing. \\

\noindent {\bf Public Member Functions }  \\

\noindent {\bf Protected Member Functions }  \\
\indent {\em virtual void sendObject(MovableObject \&theObject);}\\
A method which will send the MovableObject {\em theObject} to
the actor object through the shadows channel. It returns the
result of invoking {\em sendObj(0,0,theObject, theBroker, theActorsAddress)} on the
shadow's associated channel {\em theChannel}. \\ 


{\em virtual void recvObject(MovableObject \&theObject);} \\
A method which will cause the object to read the MovableObject {\em
theObject} from the channel. It will return the
result of invoking {\em recvObj(0,0,theObject, theBroker, theActorsAddress)} on the
shadows associated channel {\em theChannel}. \\ 

\indent {\em virtual void sendMessage(Message \&theMessage);}\\
A method which will send the message {\em theMessage} to
the actor object through the shadows channel. It will return the
result of invoking {\em sendMsg(0,0,theMessage, theActorsAddress)} on the
shadows associated channel {\em theChannel}. \\ 


{\em virtual void recvMessage(Message \&theMessage);} \\
A method which will cause the object to read the message {\em
theMessage} from the channel. It will return the
result of invoking {\em recvMsg(0,0,theMessage, theActorsAddress)} on the
shadows associated channel {\em theChannel}. \\ 


\indent {\em virtual void sendMatrix(Matrix \&theMatrix);}\\
A method which will send the Matrix {\em theMatrix} to
the actor object through the shadows channel. It will return the
result of invoking {\em sendMatrix(0,0,theMatrix, theActorsAddress)} on the
shadows associated channel {\em theChannel}. \\ 


{\em virtual void recvMatrix(Matrix \&theMatrix);} \\
A method which will cause the object to read the Matrix {\em
theMatrix} from the channel. It will return the
result of invoking {\em recvMatrix(0,0,theMatrix, theActorsAddress)} on the
shadows associated channel {\em theChannel}. \\ 

\indent {\em virtual void sendVector(Vector \&theVector);}\\
A method which will send the Vector {\em theVector} to
the actor object through the shadows channel. It will return the
result of invoking {\em sendVector(0,0,theVector, theActorsAddress)} on the
shadows associated channel {\em theChannel}. \\ 


{\em virtual void recvVector(Vector \&theVector);} \\
A method which will cause the object to read the Vector {\em
theVector} from the channel. It will return the
result of invoking {\em recvVector(0,0,theVector, theActorsAddress)} on the
shadows associated channel {\em theChannel}. \\ 


\indent {\em virtual void sendID(ID \&theID);}\\
A method which will send the ID {\em theID} to
the actor object through the shadows channel. It will return the
result of invoking {\em sendID(0,0,theID, theActorsAddress)} on the
shadows associated channel {\em theChannel}. \\ 


{\em virtual void recvID(ID \&theID);} \\
A method which will cause the object to read the ID {\em
theID} from the channel. It will return the
result of invoking {\em recvID(0,0,theID, theActorsAddress)} on the
shadows associated channel {\em theChannel}. \\ 


{\em void Channel *getChannelPtr(void) const;}\\
A method which returns a pointer to the channel currently set. \\

{\em void FEM\_ObjectBroker *getObjectBrokerPtr(void) const;}\\
A method which returns a pointer to the FEM\_ObjectBroker passed in the
constructor. \\

{\em void ObjectBroker *getActorAddressPtr(void) const;}\\
A method which returns a pointer to the channel in the actors address
space. \\



\pagebreak
\subsection{{\bf Channel}}
%File: ~/OOP/actor/Channel.tex
%What: "@(#) Channel.tex, revA"

\noindent {\bf Files}   \\
\indent \#include $<\tilde{ }$/actor/channel/Channel.h$>$  \\

\noindent {\bf Class Declaration}  \\
\indent class Channel \\

\noindent {\bf Class Hierarchy} \\
\indent {\bf Channel} \\

\noindent {\bf Description}  \\
\indent Channel is an abstract class, i.e. no instances of Channel
should exist. A Channel is a point of communication in a program, a
mailbox to/from which data enters/leaves a program. Channels are
objects through which the objects in the current processes address
space can interact with objects in another processes address space. A
channel in one process space is associated with a channel in the
address space of another process space. The interaction is in the form
of data sent between the two processes along the connection line. \\ 

\indent // Constructor  \\
\indent {\em Channel();}  \\ \\
\indent // Destructor \\
\indent {\em virtual~ $\tilde{}$Channel();}\\  \\
\indent // Public Methods  \\
\indent {\em char *addToProgram(void) =0;} \\
\indent {\em virtual int setUpShadow(void) =0;} \\
\indent {\em virtual int setUpActor(void) =0;} \\
\indent {\em virtual int setNextAddress(ChannelAddress \&theNextAddress) =0;}\\
\indent {\em virtual ChannelAddress *getLastSendersAddress(void) =0;}\\
\indent {\em virtual int getDbTag(void);} \\

\indent {\em int sendObj(int commitTag, \\
\indent\indent\indent\indent\indent MovableObject \&theObject, \\
\indent\indent\indent\indent\indent ChannelAddress *theAddress =0) =0;}\\
\indent {\em int recvObj(int commitTag, \\
\indent\indent\indent\indent\indent MovableObject \&theObject, \\
\indent\indent\indent\indent\indent FEM\_ObjectBroker \&theBroker, \\
\indent\indent\indent\indent\indent ChannelAddress *theAddress =0) =0;}\\
 
\indent {\em int sendMsg(int dbTag, int commitTag,  \\
\indent\indent\indent\indent\indent const Message \&, \\
\indent\indent\indent\indent\indent ChannelAddress *theAddress =0) =0;}\\
\indent {\em int recvMsg(int dbTag, int commitTag, \\
\indent\indent\indent\indent\indent Message \&, \\
\indent\indent\indent\indent\indent ChannelAddress *theAddress =0) =0;}\\
\indent {\em int sendMatrix(int dbTag, int commitTag,  \\
\indent\indent\indent\indent\indent    const Matrix \&theMatrix, \\
\indent\indent\indent\indent\indent    ChannelAddress *theAddress =0) =0;}\\
\indent {\em int recvMatrix(int dbTag, int commitTag, \\
\indent\indent\indent\indent\indent    Matrix \&theMatrix,  \\
\indent\indent\indent\indent\indent    ChannelAddress *theAddress =0) =0;}\\
\indent {\em int sendVector(int dbTag, int commitTag, \\
\indent\indent\indent\indent\indent const Vector \&theVector, \\
\indent\indent\indent\indent\indent ChannelAddress *theAddress =0) =0;}\\
\indent {\em int recvVector(int dbTag, int commitTag,  \\
\indent\indent\indent\indent\indent    Vector \&theVector,  \\
\indent\indent\indent\indent\indent    ChannelAddress *theAddress =0) =0;}\\
\indent {\em int sendID(int dbTag, int commitTag, \\
\indent\indent\indent\indent\indent const ID \&theID, \\
\indent\indent\indent\indent\indent ChannelAddress *theAddress =0) =0;}\\
\indent {\em int recvID(int dbTag, int commitTag, \\
\indent\indent\indent\indent\indent ID \&theID, \\
\indent\indent\indent\indent\indent ChannelAddress *theAddress =0) =0;}\\



\noindent {\bf Constructor}  \\
\indent {\em Channel();}  \\
Does nothing. \\

\noindent {\bf Destructor} \\
\indent {\em virtual~ $\tilde{}$Channel();}\\ 
Does nothing. Provided so that a subclasses destructor will be invoked. \\

\noindent {\bf Public Methods}  \\
\indent {\em char *addToProgram(void) =0;} \\
When creating remote actors the channels created in the actor space
need to know how to contact the shadows channels. This information is
provided in the string returned from this method. It is used by the
machine broker when starting the remote process. It places this
information as the last arguments to the program. \\

\indent {\em virtual int setUpShadow(void) =0;} \\
A method invoked in the local address space by a shadow object. The
method is to be invoked concurrently with a {\em setUpShadow()}
invocation on a channel object in all the remote actor processes.\\

{\em virtual int setUpActor(void) =0;} \\
A method invoked in the remote address space by the actor. The method
is invoked concurrently with a corresponding {\em setUpShadow()}
invocation on a channel in a local actor process by the shadow object
that created the running actor process. If the method fails returns a
negative number. For actors with only one Channel this should cause
the termination of the actor. \\

{\em virtual int setNextAddress(ChannelAddress \&theNextAddress) =0;}\\
A method invoked to set specify the next address that the next
messages to be sent if {\em sendMessage()} or received if {\em
recvMessage()} is invoked with a null pointer. \\

\indent {\em virtual int getDbTag(void);} \\
To return the next available database tag, must be an integer value
greater than $0$, $0$ is used my the objects to check if they have yet
been assigned a database tag. The method defined for the Channel base
class always returns $0$, only database channel objects need worry
about assigning unique integer values. \\

\indent {\em int sendObj(int commitTag, \\
\indent\indent\indent\indent\indent MovableObject \&theObject, \\
\indent\indent\indent\indent\indent ChannelAddress
*theAddress =0) =0;}\\
To send the object {\em theObject} and the commit tag {\em commitTag}
to a remote Channel whose address is given by {\em theAddress}. If
{\em theAddress} is $0$, the Channel sends to the Channel with the
address last set in a {\em send..()}, {\em recv..()}, or {\em
setNextAddress()} operation. To return $0$ if successful, a negative
number if not. \\  

\indent {\em int recvObj(int commitTag, \\
\indent\indent\indent\indent\indent MovableObject \&theObject, \\
\indent\indent\indent\indent\indent FEM\_ObjectBroker \&theBroker, \\
\indent\indent\indent\indent\indent ChannelAddress
*theAddress =0) =0;}\\
To receive the object {\em theObject} with the commit tag {\em commitTag}
from a remote Channel whose address is given by {\em theAddress}. If
{\em theAddress} is $0$, the Channel receives from the Channel with the
address last set in a {\em send..()}, {\em recv..()}, or {\em
setNextAddress()} operation. To return $0$ if successful, a negative
number if not. \\  



\indent {\em int sendMsg(int dbTag, int commitTag,  \\
\indent\indent\indent\indent\indent const Message \&theMessage, \\
\indent\indent\indent\indent\indent ChannelAddress *theAddress =0) =0;}\\
A method which is invoked to send the data in the Message object {\em
theMessage} to another Channel object. The object will obtain the
data and size of the data to be sent by invoking {\em getData()} and
{\em getSize()} on {\em theMessage}. The channel object is then
responsible for sending that data to the remote channel address given
by {\em theAddress}. If {\em theAddress} is $0$, the Channel sends to
the Channel with the address last set in a {\em send..()}, {\em
recv..()}, or {\em setNextAddress()} operation. To return $0$ if
successful, a negative number if not. \\  


\indent {\em int recvMsg(int dbTag, int commitTag, \\
\indent\indent\indent\indent\indent Message \& theMessage, \\
\indent\indent\indent\indent\indent ChannelAddress *theAddress =0) =0;}\\
A method which is invoked to send the data in the Message object {\em
theMessage} to another Channel object. The object will obtain the
the size of the data that is being received by invoking {\em getSize()}
on {\em theMessage}. The channel object is then responsible for
receiving that amount of data from the channel whose address is given
by {\em theAddress}. If {\em theAddress} is $0$, the Channel receives from
the Channel with the address last set in a {\em send..()}, {\em
recv..()}, or {\em setNextAddress()} operation. To return $0$ if
successful, a negative number if not. \\  


\indent {\em int sendMatrix(int dbTag, int commitTag,  \\
\indent\indent\indent\indent\indent const Matrix \&theMatrix, \\
\indent\indent\indent\indent\indent ChannelAddress *theAddress =0) =0;}\\
A method for sending a Matrix {\em theMatrix} to a
remote Channel, whose address is given by {\em theAddress}, with the
integer identifiers {\em dbTag} and {\em commitTag}. If {\em theAddress} 
is $0$, the Channel sends to the Channel with the address last set in
a {\em send..()}, {\em recv..()}, or {\em setNextAddress()}
operation. To return $0$ if successful, a negative number if not. \\ 


\indent {\em int recvMatrix(int dbTag, int commitTag, \\
\indent\indent\indent\indent\indent Matrix \&theMatrix,  \\
\indent\indent\indent\indent\indent ChannelAddress *theAddress =0) =0;}\\
A method for receiving a Matrix {\em theMatrix} from a
remote Channel, whose address is given by {\em theAddress}, with the
integer identifiers {\em dbTag} and {\em commitTag}. If {\em theAddress} 
is $0$, the Channel receives from the Channel at the address last set
in a {\em send..()}, {\em recv..()}, or {\em setNextAddress()}
operation. To return $0$ if successful, a negative number if not. \\ 

\indent {\em int sendVector(int dbTag, int commitTag,  \\
\indent\indent\indent\indent\indent const Vector \&theVector, \\
\indent\indent\indent\indent\indent ChannelAddress *theAddress =0) =0;}\\
A method for sending a Vector {\em theVector} to a
remote Channel, whose address is given by {\em theAddress}, with the
integer identifiers {\em dbTag} and {\em commitTag}. If {\em theAddress} 
is $0$, the Channel sends to the Channel with the address last set in
a {\em send..()}, {\em recv..()}, or {\em setNextAddress()}
operation. To return $0$ if successful, a negative number if not. \\ 


\indent {\em int recvVector(int dbTag, int commitTag, \\
\indent\indent\indent\indent\indent Vector \&theVector,  \\
\indent\indent\indent\indent\indent ChannelAddress *theAddress =0) =0;}\\
A method for receiving a Vector {\em theVector} from a
remote Channel, whose address is given by {\em theAddress}, with the
integer identifiers {\em dbTag} and {\em commitTag}. If {\em theAddress} 
is $0$, the Channel receives from the Channel at the address last set
in a {\em send..()}, {\em recv..()}, or {\em setNextAddress()}
operation. To return $0$ if successful, a negative number if not. \\ 

\indent {\em int sendID(int dbTag, int commitTag,  \\
\indent\indent\indent\indent\indent const ID \&theID, \\
\indent\indent\indent\indent\indent ChannelAddress *theAddress =0) =0;}\\
A method for sending a ID {\em theID} to a
remote Channel, whose address is given by {\em theAddress}, with the
integer identifiers {\em dbTag} and {\em commitTag}. If {\em theAddress} 
is $0$, the Channel sends to the Channel with the address last set in
a {\em send..()}, {\em recv..()}, or {\em setNextAddress()}
operation. To return $0$ if successful, a negative number if not. \\ 


\indent {\em int recvID(int dbTag, int commitTag, \\
\indent\indent\indent\indent\indent ID \&theID,  \\
\indent\indent\indent\indent\indent ChannelAddress *theAddress =0) =0;}\\
A method for receiving a ID {\em theID} from a
remote Channel, whose address is given by {\em theAddress}, with the
integer identifiers {\em dbTag} and {\em commitTag}. If {\em theAddress} 
is $0$, the Channel receives from the Channel at the address last set
in a {\em send..()}, {\em recv..()}, or {\em setNextAddress()}
operation. To return $0$ if successful, a negative number if not.




\pagebreak
\subsubsection{TCP\_Socket}
% \input{../actor/channel/TCP_Socket}

\pagebreak
\subsubsection{MPI\_Channel}
% \input{../actor/channel/MPI_Channel}


\pagebreak
\subsection{{\bf MovableObject}}
%File: ~/OOP/actor/MovableObject.tex
%What: "@(#) MovableObject.tex, revA"

\noindent {\bf Files}   \\
\indent \#include $<$/actor/actor/MovableObject.h$>$  \\

\noindent {\bf Class Declaration}  \\
\indent class MovableObject \\

\noindent {\bf Class Hierarchy} \\
\indent {\bf MovableObject} \\

\noindent {\bf Description}  \\
\indent MovableObject is an abstract class, i.e. no instances of MovableObject
should exist. MovableObjects are objects which are able to
send/receive themselves to/from Channel objects. With each movable
object is associated a unique class identifier, it is this id which
will allow object brokers in remote processes to create an object of
the correct type. In addition when databases are being used, each
MovableObject will have a unique database tag, it is this integer
which will allow the objects to retrieve their own data from the database. \\ 


\noindent {\bf Class Interface }  \\
\indent\indent // Constructor  \\
\indent\indent {\em MovableObject(int classTag, int dbTag);}  \\
\indent\indent {\em MovableObject(int classTag);}  \\ \\
\indent\indent // Destructor \\
\indent\indent {\em virtual~ $\tilde{}$MovableObject();}\\  \\
\indent\indent // Public Methods  \\
\indent\indent {\em int getClassTag(void) const;}\\
\indent\indent {\em int getDbTag(void) const;}\\
\indent\indent {\em void setDbTag(int dbTag);}\\
\indent\indent {\em virtual int sendSelf(int commitTag, Channel \&theChannel) =0;} \\ 
\indent\indent {\em virtual int recvSelf(int commitTag, Channel
\&theChannel, FEM\_ObjectBroker \&theBroker) =0;} \\ 


\noindent {\bf Constructor}  \\
\indent {\em MovableObject(int classTag, int dbTag);}  \\
The constructor sets the objects class identifier to {\em classTag}:
this is a unique id for each class of instantiable movable
objects. The identifier will allow an object broker to recognize the
object type to be instantiated. Sets the objects database tag to {\em
dbTag}: this is a unique id for identifying the object in a database. \\


\indent {\em MovableObject(int classTag);}  \\
The constructor sets the objects class identifier to {\em classTag} 
and sets the objects database tag to {\em 0}. \\

\noindent {\bf Destructor} \\
\indent {\em virtual~ $\tilde{}$MovableObject();}\\ 

\noindent {\bf Public Methods }  \\
\indent {\em int getClassTag(void) const;}\\
A method which returns the class id, {\em classTag}, provided in
the constructor. \\

{\em int getDbTag(void) const;}\\
A method which returns the database tag, {\em dbTag}, provided in
the constructor or last set in {\em setDbTag()}. \\

{\em void setDbTag(int dbTag);}\\
A method to set the database tag to be equal to {\em dbTag}. \\

{\em virtual int sendSelf(int commitTag, Channel \&theChannel) =0;} \\ 
This is a pure virtual method, one must be written for each
instantiable subclass of MovableObject. Each object has to send the
data needed to be able to reproduce that object in a remote
process. The object uses the methods provided by {\em theChannel}
object to send the data to another channel at the remote actor, the
address of the channel is set before this method is called.
An object of similar type at the remote actor is invoked with a  {\em
receiveSelf()} to receive the data. Returns $0$ if successful
(successful in that the data got to the channel), or a $-1$ if no
data was sent. \\  

{\em virtual int recvSelf(int commitTag, Channel \&theChannel, FEM\_ObjectBroker
\&theBroker) =0;} \\ 
This is a pure virtual method, one must be written for each
instantiable subclass of MovableObject. Each object has to receive the
data needed to be able to recreate itself in the new process after it
has been sent through {\em theChannel}. If the object is an
aggregation containing other objects, new objects of the correct type
can be constructed using {\em theBroker}. To return $0$ if successful
or a $-1$ if not. \\






\pagebreak
\subsection{Message}
%File: ~/OOP/actor/Message.tex
%What: "@(#) Message.tex, revA"

\noindent {\bf Files}   \\
\indent \#include $<\tilde{ }$/actor/message/Message.h$>$  \\

\noindent {\bf Class Declaration}  \\
\indent class Message \\

\noindent {\bf Class Hierarchy} \\
\indent {\bf Message} \\

\noindent {\bf Description}  \\
\indent Messages are objects that can be sent between Channels. They
are provided to allow data of arbitrary length and type, e.g. structs,
to be sent between processes running on similar machine
architectures. WARNING Sending Messages between machines with different
architectures can result in erroniuos data being received. Each
Message object keeps a pointer to the data it represents and an integer
outlining the data size. There is no copy of the data kept by the
Message. \\


\noindent {\bf Constructors}  \\
\indent // Constructors  \\
\indent {\em Message();}  \\
\indent {\em Message(double *, int num);}\\
\indent {\em Message(int *, int num);}\\
\indent {\em Message(char *, int num);} \\ \\
\indent // Destructor \\
\indent {\em virtual~ $\tilde{}$Message();}\\  \\
\indent // Public Member Functions  \\
\indent {\em virtual int putData(char *theData, int startLoc, int endLoc);}; \\  
\indent {\em virtual const char *getData(void);}\\
\indent {\em virtual int   getSize(void);} \\

\noindent {\bf Constructors}  \\
\indent {\em Message();}  \\
To construct an empty message. \\

\indent {\em Message(double *data, int num);}\\
To construct a message for sending/receiving an array containing {\em
num} doubles. \\

\indent {\em Message(int *data, int num);}\\
To construct a message for sending/receiving an array containing {\em
num} ints. \\

\indent {\em Message(char *data, int num);} \\
To construct a message for sending/receiving a string of {\em num}
characters or a struct. \\ 

\noindent {\bf Destructor} \\
\indent {\em virtual~ $\tilde{}$Message();}\\ 
Does nothing. \\

\noindent {\bf Public Methods }  \\
\indent {\em    virtual int putData(char *theData, int startLoc, int
endLoc);}; \\ 
A method which will put the data given by the character pointer {\em
theData} of size {\em endLoc -startLoc} into the data array pointed to
by the Message starting at location $startLoc$ in this array. Returns $0$ if
successful; an error message is printed and a $-1$ is returned if
not. The routine {\em bcopy()} is used to copy the data. \\ 

{\em  virtual const char *getData(void);}\\
A method which returns a const char * pointer to the messages data. \\

{\em  virtual int   getSize(void);} \\
A method to get the size of the array. The unit of size is that of a
character. 



\pagebreak
\subsection{FEM\_ObjectBroker}
%File: ~/OOP/actor/objectBroker/FEM\_ObjectBroker.tex
%What: "@(#) FEM\_ObjectBroker.tex, revA"

\noindent {\bf Files}   \\
\indent \#include $<\tilde{ }$/actor/objectBroker/FEM\_ObjectBroker.h$>$  \\

\noindent {\bf Class Declaration}  \\
\indent class FEM\_ObjectBroker \\

\noindent {\bf Class Hierarchy} \\
\indent {\bf FEM\_ObjectBroker} \\

\noindent {\bf Description}  \\
\indent FEM\_ObjectBrokers is an object used to create a new blank 
of a certain type in a process. The explicit type of object
created depends on the method invoked and the integer {\em classTag}
passed as an argument to the method. Once the object has been created, {\em
recvSelf()} can be invoked on the object to instantiate the object
with it's data. \\


\noindent {\bf Constructor}  \\
\indent // Constructor  \\
\indent {\em FEM\_ObjectBroker();}  \\\\
\indent // Destructor \\
\indent {\em virtual $\tilde{ }$FEM\_ObjectBroker();}\\  \\
\indent // Public Methods to get new Domain objects \\
\indent {\em virtual Element       *getNewElement(int classTag); } \\
\indent {\em virtual Node          *getNewNode(int classTag); } \\
\indent {\em virtual MP\_Constraint *getNewMP(int classTag); } \\
\indent {\em virtual SP\_Constraint *getNewSP(int classTag); } \\
\indent {\em virtual NodalLoad     *getNewNodalLoad(int classTag); } \\
\indent {\em virtual ElementalLoad *getNewElementalLoad(int classTag); } \\
\indent {\em virtual UniaxialMaterial
*getNewUniaxialMaterial(int classTag); } \\  \\
\indent // Public Methods to get New Matrix,Vector and ID objects -
NOT USED \\
\indent {\em virtual Matrix	  *getPtrNewMatrix(int classTag, int
noRows, int noCols); } \\ 
\indent {\em virtual Vector	  *getPtrNewVector(int classTag, int size); } \\
\indent {\em virtual ID	          *getPtrNewID(int classTag, int size); } \\\\
\indent // Public Methods to get new Analysis objects - NOT NEEDED SEQUENTIAL\\
\indent {\em virtual ConvergenceTest *getNewConvergenceTest(int classTag); } \\ 
\indent {\em virtual ConstraintHandler   *getNewConstraintHandler(int classTag); } \\
\indent {\em virtual DOF\_Numberer        *getNewNumberer(int classTag); } \\
\indent {\em virtual AnalysisModel       *getNewAnalysisModel(int
classTag); } \\ 
\indent {\em virtual EquiSolnAlgo        *getNewEquiSolnAlgo(int
classTag); } \\ 
\indent {\em virtual DomainDecompAlgo
*getNewDomainDecompAlgo(int classTag); } \\ 
\indent {\em virtual StaticIntegrator
*getNewStaticIntegrator(int classTag); } \\ 
\indent {\em virtual TransientIntegrator
*getNewTransientIntegrator(int classTag); } \\ 
\indent {\em virtual LinearSOE *getNewLinearSOE(int classTagSOE,
int classTagSolver); } \\ 
\indent {\em virtual LinearSOESolver *getNewLinearSolver(void); } \\
\indent {\em virtual LinearSOE *getPtrNewDDLinearSOE(int classTagSOE, 
int classTagDDSolver); } \\
\indent {\em virtual DomainSolver *getNewDomainSolver(void); } \\
\indent {\em virtual DomainDecompositionAnalysis *
getNewDomainDecompAnalysis(int classTag, Subdomain \&theDomain); } \\
\indent {\em virtual Subdomain 	  *getSubdomainPtr(int classTag); } \\ \\
\indent // Public Methods for Parallel Model Generation  \\
\indent {\em virtual PartitionedModelBuilder *
getPtrNewPartitionedModelBuilder(Subdomain \&theSub, int classTag); } \\
\indent {\em virtual GraphNumberer *getPtrNewGraphNumberer(int classTag); } \\



\noindent {\bf Constructor}  \\
\indent {\em FEM\_ObjectBroker();}  \\
Does nothing. \\

\noindent {\bf Destructor} \\
\indent {\em virtual $\tilde{ }$FEM\_ObjectBroker();}\\ 
Does nothing. \\

\noindent {\bf Public Methods }  \\


\pagebreak
\subsection{{\bf MachineBroker}}
%File: ~/OOP/actor/machineBroker/MachineBroker.tex
%What: "@(#) MachineBroker.tex, revA"

\noindent {\bf Files}   \\
\indent \#include $<\tilde{ }$/actor/machineBroker/MachineBroker.h$>$  \\

\noindent {\bf Class Declaration}  \\
\indent class MachineBroker \\

\noindent {\bf Class Hierarchy} \\
\indent {\bf MachineBroker} \\

\noindent {\bf Description}  \\
\indent MachineBrokers are objects that are used to start remote
processes running on the parallel machine. \\


\noindent {\bf Constructor}  \\
\indent // Constructor  \\
\indent {\em MachineBroker();}  \\\\
\indent // Destructor \\
\indent {\em virtual $\tilde{ }$MachineBroker();}\\  \\
\indent // Public Member Functions  \\
\indent {\em virtual int startActor(char *actorProgram, Channel \&theChannel,
int compDemand =0) =0;}\\

\noindent {\bf Constructor}  \\
\indent {\em MachineBroker();}  \\

\noindent {\bf Destructor} \\
\indent {\em virtual~ $\tilde{}$MachineBroker();}\\ 
Does nothing. \\

\noindent {\bf Public Methods }  \\
\indent {\em virtual int startActor(char *actorProgram, Channel \&theChannel,
int compDemand =0) =0;}\\
Invoked to start the program, {\em actorProgram}, on the parallel
machine. The remote actor process uses information supplied by {\em
theChannel} to allow the remote process to connect to the local
process. The integer {\em compDemand} provides an indication of the
computational demands of the remote process in a heterogeneous
environment. 







\end{document}


