% Boris Jeremic (@ucdavis.edu)
\section{OpenSees Commands to Create Brick Elements }

%\paragraph{The Eight Node Brick Element} \
\subsection{The Eight Node Brick Element} \
\begin{verbatim}
element Brick8N  eletag? node1? node2? node3? node4? node5? node6? node7?
                 node8? matTag?  bf1? bf2? bf3? massDens?
\end{verbatim}
The   Brick8N   element   is  the  standard  eight  node  three
dimensional  element implemented based on tensor operation. The
arguments  to  construct the element are its tag, eletag, eight
nodes  ordered  according  to  Figure \ref{8node}, the material
tag,  matTag,  the  body  forces,  bf1,  bf2, bf3, and the mass
density, massDens. By default, $3\times 3 \times 3$ integration
points  are  used.  Users  will  be  able  to specify number of
integration points very soon.

The  valid  queries  to  a  Brick8N  element  when  creating an
ElementRecorder   are   ``force'',  ``stiffness'',  ``stress'',
``pq'',     ``pqall'',     ``gausspoint'',    ``plastic''    or
``plasticGPC''.   
%
For   ``stress''   output,   the  six  stress
components  from  each  Gauss  point  are  output by the order:
$\sigma_x$, $\sigma_y$, $\sigma_z$, $\tau_{xy}$,
$\tau_{xz}$,  $\tau_{yz}$. 
%
The  stresses can also be output in $p$ and $q$ format by using
query  ``pq'', where $p$ is the hydrostatic pressure, while $q$
is  the  equivalent deviatoric stress. In this case, the stress
state  at one gauss point is printed in the $pq$ format. If the
stress  states  at all gauss points need to be printed, use the
query ``pqall''.
%
For  ``gausspoint'',  the  coordinates  of all Gauss points are
printed out.
%
For  ``plastic'', the equivalent deviatoric plastic strain from
each Gauss point is output in the same order as the coordinates
are printed. But the coordinates have to be output separated.
%
If  one  needs  to  output the gauss point coordinates together
with the plastic strain, the query ``plasticGPC'' needs to be used.




%
\begin{figure}[!htbp]
\begin{center}
\includegraphics[width=9.0cm]{/home/jeremic/tex/works/Thesis/ZhaohuiYang/Appendex/brick8.eps}
\caption{\label{8node}
Node numbering for 8 node three dimensional element.}
\end{center}
\end{figure}


%\paragraph{The Twenty Node Brick Element} \
\subsection{The Twenty Node Brick Element} \

\begin{verbatim}
element Brick20N eletag? node1? node2? node3? node4? node5? node6? node7?
        node8?  node9?  node10?  node11?  node12? node13? node14? node15? 
        node16? node17? node18?  node19?  node20? matTag? bf1?  bf2? bf3? 
        massDens?
\end{verbatim}
The   Brick8N   element   is  the  standard  eight  node  three
dimensional  element implemented based on tensor operation. The
arguments  to  construct the element are its tag, eletag, twenty
nodes  ordered  according  to  Figure \ref{20node}, the material
tag,  matTag,  the  body  forces,  bf1,  bf2, bf3, and the mass
density, massDens. By default, $3\times 3 \times 3$ integration
points  are  used.  Users  will  be  able  to specify number of
integration points very soon.

The  valid  queries  to  a  Brick20N  element  when creating an
ElementRecorder   are  ``force'',  ``stiffness'',  ``stress'',  ``pq'',
``gausspoint'', or ``plastic''.
%
For  ``stress''  output,  the  six  stress components from each
Gauss  point  are  output by the order: $\sigma_x$, $\sigma_y$,
$\sigma_z$, $\tau_{xy}$, $\tau_{xz}$, $\tau_{yz}$.
%
The  stresses can also be output in $p$ and $q$ format by using
query ``pq'', where $p$ is the hydrostatic pressure, while $q$ is
the  equivalent  deviatoric  stress.  In  this case, the stress
state at one gauss point is printed in the $pq$ format.
%
For  ``gausspoint'',  the  coordinates  of all Gauss points are
printed out. For ``plastic'', the equivalent deviatoric plastic
strain from each Gauss point is output in the same order as the
coordinates are printed.
%
\begin{figure}[!htbp]
\begin{center}
%\vspace{-1cm}
\includegraphics[width=9cm]{/home/jeremic/tex/works/Thesis/ZhaohuiYang/Appendex/brick20.eps}
\caption{\label{20node} Node numbering for 20 node three dimensional element.}
\end{center}
\end{figure}
%
