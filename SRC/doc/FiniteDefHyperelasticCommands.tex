% Boris Jeremic (@ucdavis.edu)

\section{Large Deformation Hyperelastic Material Model Commands}


%%%%%%%%%%%%%%%%%%%%%%%%%%%%%%%%%%%%%%%%%%%%%%%%%%%%%%%%%%%%%%%%%%%%%%%%%%%%%%%%%%%%
\subsection{Compressible Neo-Hookean Material Commands}
\label{CompNH}

\begin{verbatim}
nDMaterial NeoHookean3D matTag? K? G? <rho?> 
\end{verbatim}
or
\begin{verbatim}
nDMaterial NeoHookeanCompressible3D matTag? K? G? <rho?> 
\end{verbatim}

This model is the traditional compressible Neo-Hookean material model.
A compressible Neo-Hookean material is constructed using \texttt{nDMaterial} command.
The argument \texttt{matTag} is used to uniquely identify this nDMaterial object among 
nDMaterial objects in the BasicBuilder object. 
The parameter \texttt{K} defines the material reference bulk modulus. 
The parameter \texttt{G} defines the material reference shear modulus. 
The optional parameter \texttt{rho} defines the material reference density, 
the default value of rho is zero.  
  
Although this is named compressible Neo-Hookean material, it can also be used for 
near-compressible or incompressible material. In this case, 
make \texttt{K} a value far bigger than \texttt{G}.  


%%%%%%%%%%%%%%%%%%%%%%%%%%%%%%%%%%%%%%%%%%%%%%%%%%%%%%%%%%%%%%%%%%%%%%%%%%%%%%%%%%%%
\subsection{Decoupled Neo-Hookean Material Commands}
\label{DNH}

\begin{verbatim}
nDMaterial DecoupledNH3D matTag? K? G? <rho?> 
\end{verbatim}
or
\begin{verbatim}
nDMaterial DecoupledNeoHookean3D matTag? K? G? <rho?> 
\end{verbatim}

This model is a decoupled model for its volumetric and isochoric parts.
A decoupled Neo-Hookean material is constructed using \texttt{nDMaterial} command.
The argument \texttt{matTag} is used to uniquely identify this nDMaterial object among 
nDMaterial objects in the BasicBuilder object. 
The parameter \texttt{K} defines the material reference bulk modulus. 
The parameter \texttt{G} defines the material reference shear modulus. 
The optional parameter \texttt{rho} defines the material reference density, 
the default value of rho is zero.  


%%%%%%%%%%%%%%%%%%%%%%%%%%%%%%%%%%%%%%%%%%%%%%%%%%%%%%%%%%%%%%%%%%%%%%%%%%%%%%%%%%%%
\subsection{Decoupled Logarithmic Material Commands}
\label{DLog}

\begin{verbatim}
nDMaterial DecoupledLog3D matTag? K? G? <rho?> 
\end{verbatim}
or
\begin{verbatim}
nDMaterial DecoupledLogarithmic3D matTag? K? G? <rho?> 
\end{verbatim}

This model is a decoupled model for its volumetric and isochoric parts.
A decoupled logarithmic material is constructed using \texttt{nDMaterial} command.
The argument \texttt{matTag} is used to uniquely identify this nDMaterial object among 
nDMaterial objects in the BasicBuilder object. 
The parameter \texttt{K} defines the material reference bulk modulus. 
The parameter \texttt{G} defines the material reference shear modulus. 
The optional parameter \texttt{rho} defines the material reference density, 
the default value of rho is zero. 


%%%%%%%%%%%%%%%%%%%%%%%%%%%%%%%%%%%%%%%%%%%%%%%%%%%%%%%%%%%%%%%%%%%%%%%%%%%%%%%%%%%%
\subsection{Decoupled Mooney-Rivlin-Simo Material Commands}
\label{DMRS}

\begin{verbatim}
nDMaterial DecoupledMRS3D matTag? c1? c2? K? <rho?> 
\end{verbatim}
or
\begin{verbatim}
nDMaterial DecoupledMooneyRivlinSimo3D matTag? c1? c2? K? <rho?> 
\end{verbatim}

This model is a decoupled model for its volumetric and isochoric parts.
Its isochoric part is Mooney-Rivlin model and its volumetric part is 
Simo-Pister model. 
A decoupled Mooney-Rivlin-Simo material is constructed using \texttt{nDMaterial} command.
The argument \texttt{matTag} is used to uniquely identify this nDMaterial object among 
nDMaterial objects in the BasicBuilder object. The parameter \texttt{c1} and \texttt{c2}
define the material constants. 
The parameter \texttt{K} defines the material reference bulk modulus. 
The optional parameter \texttt{rho} defines the material reference density, 
the default value of rho is zero. 


%%%%%%%%%%%%%%%%%%%%%%%%%%%%%%%%%%%%%%%%%%%%%%%%%%%%%%%%%%%%%%%%%%%%%%%%%%%%%%%%%%%%
\subsection{Decoupled Ogden-Simo Material Commands}
\label{DOS}

\begin{verbatim}
nDMaterial DecoupledOS3D matTag? N? c1? ... cN? m1? ... mN? K? <rho?> 
\end{verbatim}
or
\begin{verbatim}
nDMaterial DecoupledOgdenSimo3D matTag? N? c1? ... cN? m1? ... mN? K? <rho?> 
\end{verbatim}

This model is a decoupled model for its volumetric and isochoric parts.
Its isochoric part is Ogden model and its volumetric part is Simo-Pister model. 
A decoupled Ogden-Simo material is constructed using \texttt{nDMaterial} command.
The argument \texttt{matTag} is used to uniquely identify this nDMaterial object among 
nDMaterial objects in the BasicBuilder object. The parameter \texttt{N},  
\texttt{c1}, ..., \texttt{cN}, and \texttt{m1}, ..., \texttt{mN} define the material constants. 
The parameter \texttt{K} defines the material reference bulk modulus. 
The optional parameter \texttt{rho} defines the material reference density, 
the default value of rho is zero. 


%%%%%%%%%%%%%%%%%%%%%%%%%%%%%%%%%%%%%%%%%%%%%%%%%%%%%%%%%%%%%%%%%%%%%%%%%%%%%%%%%%%%
\subsection{Decoupled Mooney-Rivlin Material Commands}
\label{DMR}

\begin{verbatim}
nDMaterial DecoupledMR3D matTag? c1? c2? <rho?> 
\end{verbatim}
or
\begin{verbatim}
nDMaterial DecoupledMooneyRivlin3D matTag? c1? c2? <rho?> 
\end{verbatim}

This model is a decoupled isochoric model. 
A decoupled Mooney-Rivlin-Simo material is constructed using \texttt{nDMaterial} command.
The argument \texttt{matTag} is used to uniquely identify this nDMaterial object among 
nDMaterial objects in the BasicBuilder object. The parameter \texttt{c1} and \texttt{c2}
define the material constants.  
The optional parameter \texttt{rho} defines the material reference density, 
the default value of rho is zero. 


%%%%%%%%%%%%%%%%%%%%%%%%%%%%%%%%%%%%%%%%%%%%%%%%%%%%%%%%%%%%%%%%%%%%%%%%%%%%%%%%%%%%
\subsection{Decoupled Ogden Material Commands}
\label{DO}

\begin{verbatim}
nDMaterial DecoupledOgden3D matTag? N? c1? ... cN? m1? ... mN? <rho?> 
\end{verbatim}

This model is a decoupled isochoric model. 
A decoupled Ogden-Simo material is constructed using \texttt{nDMaterial} command.
The argument \texttt{matTag} is used to uniquely identify this nDMaterial object among 
nDMaterial objects in the BasicBuilder object. The parameter \texttt{N},  
\texttt{c1}, ..., \texttt{cN}, and \texttt{m1}, ..., \texttt{mN} define the material constants. 
The optional parameter \texttt{rho} defines the material reference density, 
the default value of rho is zero. 


%%%%%%%%%%%%%%%%%%%%%%%%%%%%%%%%%%%%%%%%%%%%%%%%%%%%%%%%%%%%%%%%%%%%%%%%%%%%%%%%%%%%
\subsection{Decoupled Simo-Pister Material Commands}
\label{DSP}

\begin{verbatim}
nDMaterial DecoupledSP3D matTag? K? <rho?> 
\end{verbatim}
or
\begin{verbatim}
nDMaterial DecoupledSimoPister3D matTag? K? <rho?> 
\end{verbatim}


This model is a decoupled volumetric model. 
A decoupled Simo-Pister material is constructed using \texttt{nDMaterial} command.
The argument \texttt{matTag} is used to uniquely identify this nDMaterial object among 
nDMaterial objects in the BasicBuilder object.
The parameter \texttt{K} defines the material reference bulk modulus.   
The optional parameter \texttt{rho} defines the material reference density, 
the default value of rho is zero. 

