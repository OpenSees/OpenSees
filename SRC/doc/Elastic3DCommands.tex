% Boris Jeremic (@ucdavis.edu)
\section{OpenSees Command to Create Elastic Isotropic/Anisotropic 3D Material }


There  are  two  types  of  3D  elastic  isotropic material models
(i.e.  linear  elastic  and  nonlinear or pressure
sensitive elastic materials)
 and one cross
anisotropic elastic material model that  you
can  create.
% Section \ref{LinEla} discusses the
%linear  elastic  material  command, while Section \ref{NonlEla}
%examines the nonlinear elastic material command.


\subsection{ElasticIsotropic3D command}
\label{LinEla}
%\texttt{nDMaterial ElasticIsotropic3D MatTag? $E_o$? $\nu?$ $\rho?$}
\begin{verbatim}
nDMaterial ElasticIsotropic3D matTag? E0? nu? rho?
\end{verbatim}

The  \texttt{ElasticIsotropic3D}  material  is the standard linear elastic
isotropic  three  dimensional  material  implemented  based  on
tensor  operation.  The arguments to construct the material are
its tag, \texttt{matTag}, Young's Modulus at atmospheric pressure \texttt{E0},
Poisson's ratio \texttt{nu}, and mass density \texttt{rho}.




\subsection{PressureDependentElastic3D command}
\label{NonlEla}
%\texttt{nDMaterial PressureDependentElastic3D MatTag? $E_o$? $\nu?$ 
%$\rho?$ $n?$ $p_{ref}?$ $p_{cutoff}?$}
\begin{verbatim}
nDMaterial PressureDependentElastic3D matTag? E0? nu? rho? n? pr? pc? 
\end{verbatim}

The  \texttt{PressureDependentElastic3D}  material  is the standard
nonlinear   elastic   isotropic   three   dimensional  material
implemented based on tensor operation. The first four arguments
are  the  same  as  linear elastic command described above. 
The pressure  dependent  elastic  modulus is to be determined using
the following formula \ref{NonLineEl011}
% ( Manzari and Dafalias
%\cite{Manzari97})
.  
There  are  three  more  arguments for this command. 
\texttt{n} is  the  exponent, 
\texttt{pr} ($p_{ref}$) is  the  atmospheric pressure, 
while \texttt{pc} ($p_{cut-off}$) is the cut-off pressure. 
When $p'$ ( the mean effective normal stress) is less than $p_{cut-off}$,
then $p'~=~p_{cut-off}$.
%
\begin{equation}
E = E_o \left(\frac{p'}{p_{ref}}\right)^{n}
\label{NonLineEl011}
\end{equation}
%

\subsection{Elastic Cross Anisotropic 3D command}
\label{ECA3D}
\begin{verbatim}
nDMaterial ElasticCrossAnisotropic matTag? Eh? Ev? nuhv? nuhh? Ghv?
\end{verbatim}

The  \texttt{ElasticCrossAnisotropic}  material  is the standard linear elastic
cross anisotropic three dimensional material implemented  based  on
tensor  operation.  
The arguments to construct the material are 
its tag, \texttt{matTag}, 
the elastic modulus in the cross plane, \texttt{Eh}, 
the elastic modulus in the plane vertical to the cross plane, \texttt{Ev},
Poisson's ratio between the cross plane and its vertical plane, \texttt{nuhv},
Poisson's ratio between in the cross plane, \texttt{nuhh},
and the shear modulus between the cross plane and its vertical plane, \texttt{Ghv}.


















