% Boris Jeremic (@ucdavis.edu)
\section{OpenSees Command for the Domain Reduction Method} 

The syntax of the DRM command is : 

{\texttt{pattern PBowlLoading}} loadpattrnTag? {\texttt{-pbele}} PBElmentFile? 
\texttt{-acce}  acceFile? \texttt{-disp} dispFile? \texttt{-dt}
dt?  $<$\texttt{-factor} factor?$>$ \texttt{-xp} +X? \texttt{-xm} -X?
\texttt{-yp}  -Y? \texttt{-ym} -Y? \texttt{-zp} +Z? \texttt{-zm} 
-Z? \\
%
where pattern and PBowlLoading are key words. loadpattrnTag? is
the tag for this loadPattern. A user needs to supply an element
file,   acceleration  file,  displacement  file  following  the
identifiers      {\texttt{-pbele}},      \texttt{-acce}     and
\texttt{-disp},  respectively. dt? is the time increment in the
input data and is a parameter following identifier \texttt{-dt}.
factor?  is  an  optional  parameter for scaling the equivalent
forces.  If  not  supplied,  a  default value 1.0 will be used.
In  particular,  the interior boundary of the boundary layer is
input  by  a set of coordinates, i.e. +X, -X, +Y, -Y, +Z and -Z,
as shown in Figure \ref{PBDefinition}.
%
\begin{figure}[!htbp]
\begin{center}
\vspace{0.5cm}
\includegraphics[width=8.5cm]{/home/jeremic/tex/works/Thesis/JinxiuLiao/newthesis/figures/PB_Command.eps}
\caption{\label{PBDefinition}  Sketch  showing  coordinates of the
interior boundary.}
\end{center}
\end{figure}





