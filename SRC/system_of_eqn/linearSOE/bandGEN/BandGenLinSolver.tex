%File: ~/OOP/system_of_eqn/linearSOE/bandGEN/BandGenLinSolver.tex
%What: "@(#) BandGenLinSolver.tex, revA"

\noindent {\bf Files}   \\
\indent \#include $<\tilde{ }$/system\_of\_eqn/linearSOE/bandGEN/BandGenLinSolver.h$>$  \\

\noindent {\bf Class Declaration}  \\
\indent class BandGenLinSolver: public LinearSOESolver  \\

\noindent {\bf Class Hierarchy} \\
\indent MovableObject \\
\indent\indent  Solver \\
\indent\indent\indent LinearSOESolver \\
\indent\indent\indent\indent {\bf BandGenLinSolver} \\

\noindent {\bf Description}  \\
\indent BandGenLinSolver is an abstract class.  The BandGenLinSolver
class provides access for each subclass to the BandGenLinSOE object
through the pointer {\em theSOE}, which is a protected pointer. \\


\noindent {\bf Interface}  \\
\indent\indent // Constructor \\
\indent\indent {\em BandGenLinSolver(int classTag);}  \\ \\
\indent\indent // Destructor \\
\indent\indent {\em virtual~ $\tilde{}$BandGenLinSolver();}\\  \\
\indent\indent // Public Methods \\
\indent\indent {\em virtual int setLinearSOE(BandGenLinSOE \&theSOE);} \\

\noindent {\bf Constructor}  \\
\indent {\em BandGenLinSolver(int classTag);}  \\
The integer {\em classTag} is passed to the LinearSOESolver classes
constructor. \\ 

\noindent {\bf Destructor} \\
\indent {\em virtual~ $\tilde{}$BandGenLinSolver();}\\ 
Does nothing, provided so the subclasses destructor will be called. \\

\noindent {\bf Public Methods }  \\
\indent {\em virtual int setLinearSOE(BandGenLinSOE \&theSOE);} \\
The method sets up the link between the BandGenLinSOE object and the
BandGenLinSolver, that it is sets the pointer the subclasses use.  \\




